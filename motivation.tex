\section{Motivation}
\label{sec:motivation}

% Examples, examples, examples from simple to more advanced showing off
% the (difficult) features of inference.

In this section, we present a sequence of more and more challenging
examples for global type inference (GTI). To spice up our examples
somewhat, we assume some predefined utility classes with the following
interfaces.
\begin{lstlisting}
class Bool {
  Bool not(); 
}
class Int {
  Int negate ();
  Int add (Int that);
  Int mult (Int that);
}
class Double {
  Double negate ();
  Double add (Double that);
  Double multi (Double that);
}
\end{lstlisting}

\subsection{Multiplication}
\label{sec:multiplication}

Here is the FGJ-GT code  for multiplying the components of a pair.
\begin{lstlisting}
class MultPair {
  mult (p) { return p.fst.mult(p.snd); }
}
\end{lstlisting}
Assuming the parameter typing $\mv p:P$, result type $\mv R$, and that
\texttt{mult} in the body refers to \texttt{Int.mult}, we
obtain the following constraints.
\begin{itemize}
\item From \texttt{p.fst}: $\mv P \le \mathtt{Pair}\Angle{X,Y}$ and
  $\texttt{p.fst} : X$
\item From \texttt{p.snd}: $\mv P \le \mathtt{Pair}\Angle{X,Y}$ and
  $\texttt{p.snd} : Y$
\item From \texttt{.mult (p.snd)}: $X \le \mathtt{Int}$, $Y \le
  \mathtt{Int}$, and $\mathtt{Int} \le \mv R$.
\end{itemize}
As the return type only occurs positively in the constraints, we can
set it to $\mathtt{Int}$.
% \begin{lstlisting}
% class MultPair {
%   Int mult (Pair<Int,Int> p) { return p.fst.mult(p.snd); }
% }
% \end{lstlisting}
\begin{lstlisting}
class MultPair {
  <X extends Int, Y extends Int>
  Int mult (Pair<X,Y> p) { return p.fst.mult(p.snd); }
}
\end{lstlisting}
We obtain a second version if we assume that \texttt{mult} refers to
\texttt{Double.mult}.
\begin{lstlisting}
class MultPair {
  <X extends Double, Y extends Double>
  Double mult (Pair<X,Y> p) { return p.fst.mult(p.snd); }
}
\end{lstlisting}

Finally, \texttt{mult} might be recursive, which results different constraints.
\begin{itemize}
\item From \texttt{.mult (p.snd)}: $X \le \mathtt{MultPair}$, $Y \le
  \mathtt{P}$, and $\mathtt{R} \le \mv R$.
\end{itemize}
Combining $\mv P \le \mathtt{Pair}\Angle{X,Y}$ and $Y \le \mathtt{P}$
yields $Y \le \mathtt{Pair}\Angle{X,Y}$, which is not solvable and
hence rejected. 

The corresponding example in full Java would be the dot product. In
full Java, we also have to deal with overloading of the multiplication
operator, which amounts to considering \texttt{Int.mult} and
\texttt{Double.mult}. Overloading is not allowed in FGJ, but method
names in different classes may overlap.


\subsection{Inheritance}
\label{sec:inheritance}

\begin{lstlisting}[float,caption={Method overriding},label={lst:method-overriding}]
class A {
  m(x) { return x.add(x); }
}
class B extends A {
  m(x) { return x; }
}
\end{lstlisting}
Let's start with an artificial example in Listing~\ref{lst:method-overriding}. Class \texttt{A} has
a method \texttt{m} of inferred type \texttt{Int -> Int}. Class \texttt{B} is a
subclass of \texttt{A} which ought to override \texttt{m}, but its inferred
type is \texttt{<T> T -> T}.
However, a generic type for \texttt{B.m()} is is not a correct
method override for \texttt{A.m()}. In Java proper, we would either
have two different overloaded methods in \texttt{B} or we would have
to impose
the typing \texttt{Int B.m(Int)} to enforce overriding of
\texttt{A.m()}. As FGJ does not support overloading, the type of
\texttt{B.m()} has to be restricted accordingly.
\todo[inline]{This example shows that GTI must first infer the type
  for \texttt{A.m{}} in the superclass. Later, when it infers the types for a subclass
  like \texttt{B}, GTI must constrain the type of \texttt{B.m()} by
  the type of \texttt{A.m()} transported to the subclass.}

Otherwise, we would get strange results. Consider extending the
program with \texttt{A} and \texttt{B} by the following class.
\begin{lstlisting}
class C {
  mbool(x) { return x.m(new Bool()); }
}
\end{lstlisting}
Inferring the type of \texttt{C.mbool()} using \texttt{A.m()} would
fail because \texttt{Int} and \texttt{Bool} are not
unifiable. However, inferring the type of \texttt{C.mbool()} using
\texttt{B.m()} (with the generic type) would succeed and yield the
typing
\begin{lstlisting}
class C {
  Bool mbool(B x) { return x.m(new Bool()); }
}
\end{lstlisting}

\subsection{Inheritance and Generics}
\label{sec:inheritance-generics}


\begin{lstlisting}[float,caption={Function class}, label={lst:function-class}]
class Function<S,T> {
  T apply(S arg) { return this.apply (arg); }
}
\end{lstlisting}
Suppose we are given the generic FGJ class for modeling functions in Listing~\ref{lst:function-class}.
This code is constructed to serve as an ``abstract'' super class to derive more
interesting subclasses.
The \texttt{Function<S,T>} class must be presented in this explicit
way. Its type annotations \textbf{cannot} be inferred by GTI because
the connection to the generic parameters cannot be inferred from the
implementation.

If we apply GTI to the type-erased version of
Listing~\ref{lst:function-class}, the \texttt{apply} method is considered a generic method: 
\begin{lstlisting}
  apply (arg) { ... }  --GTI-->    <A,B> B apply (A arg) { ... }
\end{lstlisting}
The typing of \texttt{apply} in Listing~\ref{lst:function-class} is an
instance of this result, so that completeness of GTI is preserved!

While it is possible to  force GTI to infer the intended typing, it requires some awkward coding.
\begin{lstlisting}
class Function<S,T> {
  S in;
  T out;
  dummy (arg) { return this.dummy(in); }
  apply (arg) { return new Pair<>(this.out, this.dummy (arg)).fst; }
}
\end{lstlisting}

Now that we have the abstract class \texttt{Function<S,T>} at our
disposal, let's apply GTI to a class of boxed values with a
\texttt{map} function:
\begin{lstlisting}
class Box<S> {
  S val;
  map(f) {
    return new Box<>(f.apply(this.val));
} }
\end{lstlisting}
GTI finds the following constraints
\begin{itemize}
\item the return value must be of type \texttt{Box<T>}, for some type
  \texttt{T},
\item \texttt{T} is a supertype of the type returned by
  \texttt{f.apply},
\item \texttt{apply} is defined in class \texttt{Function<S1,T1>} with
  type \texttt{T1 apply(S1 arg)}, 
\item hence \texttt{T1 <: T} and \texttt{S <: S1} (because
  \texttt{this.val : S}),
\end{itemize}
and resolves them to the desired outcome where \texttt{T1=T} and
\texttt{S=S1} using the methods of
Simonet~\cite{DBLP:conf/aplas/Simonet03}. 
\begin{lstlisting}
class Box<S> {
  S val;
  <T> Box<T> map(Function<S,T> f) {
    return new Box<T>(f.apply<S,T>(this.val));
} }
\end{lstlisting}
But what happens if we add useful subclasses of \texttt{Function} as in
\begin{lstlisting}
class Not extends Function<Bool,Bool> {
  apply(b) { return b.not(); }
}
class Negate extends Function<Int,Int> {
  apply(x) { return x.negate(); }
}
\end{lstlisting}
\todo[inline]{Does this approach really make sense? If we have a
  subclass that overrides a method of the superclass, then the public
  interface should be the one of the superclass. Subsequently, the
  implementation of the method in the subclass should be checked
  against the type inferred for the superclass.}
If we rerun GTI with these classes, we now have additional
possibilities to invoke the \texttt{apply} method. With \texttt{Not}
we obtain the constraints \texttt{Bool <: T} and \texttt{S <: Bool}, but the latter cannot
be satisfied because \texttt{S} is a generic type variable. With
\texttt{Negate} we run into the same situation with the constraints
\texttt{Int <: T} and \texttt{S <: Int}. That means there is no way to
\emph{directly} invoke \texttt{Not.apply} or \texttt{Negate.apply}
from \texttt{Box.map}, but it may happen indirectly via inheritance.

Here is another subclass of \texttt{Function<S,T>} that we might want
to consider.
\begin{lstlisting}
class Identity<S> extends Function<S,S> {
  S apply(S arg) { return arg; }
}
\end{lstlisting}
Here, we obtain the following type constraints
\begin{itemize}
\item \texttt{apply} is defined in class \texttt{Identity<S1>} with
  type \texttt{S1 apply (S1 arg)},
\item hence \texttt{S1 <: T} and \texttt{S <: S1}.
\end{itemize}
Resolving the constraints yields \texttt{S = T} thus the typing
\begin{lstlisting}
Box<S> map(Identity<S> f);
\end{lstlisting}
which is an instance of the previous typing.

\subsection{Multiple results}
\label{sec:multiple-results}
% Our global type inference algorithm is able to infer every type annotation.
% For the sake of simplicity we will only consider method types in this
% paper and assume that field types are specified.
\begin{figure}[tp]
  \begin{minipage}{0.49\linewidth}
\begin{lstlisting}
class List<A> {
  List<A> add(A item) {...}
  A get() { ... }
}
\end{lstlisting}
  \end{minipage}
  ~$\left|
  \begin{minipage}{0.49\linewidth}
\begin{lstlisting}
class Global{
  m(a){
    return a.add(this).get();
} }
\end{lstlisting}
  \end{minipage}\right.$
  \caption{Example for multiple inferred types}
  \label{fig:example-types-not-unique}
\end{figure}
Consider the classes \texttt{List<A>} and \texttt{Global} in Figure~\ref{fig:example-types-not-unique}.
Global type inference processes classes in order of
dependency. Class \texttt{Global} may depend on
class \texttt{List} because \texttt{List} defines methods \texttt{add} and
\texttt{get} which may be used in \texttt{Global}. The dependency is
not certain because, in general, there may be additional classes
providing methods \texttt{add} and \texttt{get}.

In the example, it is safe to assume that the types for the methods of class \texttt{List}
are already available, either because they are given (as in the code
fragment) or because they were inferred before considering class \texttt{Global}.

The method \texttt{m} in class \texttt{Global} first invokes
\texttt{add} on \texttt{a}, so the type of \texttt{a} as well as the
return type of \texttt{a.add(this)} must be
\texttt{List<T>}, for some \texttt{T}. As \texttt{this} has
type \texttt{Global}, it must be that \texttt{Global} is a
subtype of \texttt{T}. By the typing of \texttt{get()} we find that
the return type of method \texttt{m} is also \texttt{T}.

But now we are in a dilemma because \texttt{Global <: T} is not a
valid constraint in FGJ: FGJ only supports \emph{upper bounds} for
type variables!\footnote{Java has the same restriction. Lower bounds
  are only allowed for wildcards.}
To stay compatible with this restriction, global type inference
expands the lower bound into two possibilities for
\texttt{T} in this program, \texttt{Global} and
\texttt{Object}.  They give rise to two incomparable types for
\texttt{m}, \texttt{List<Global> -> Global} and
\texttt{List<Object> -> Object}. So there are two different FGJ
programs that are completions of the \texttt{Global} class.

For that reason, global type inference returns the \emph{intersection type}
\texttt{List<Global> -> Global} \verb!&! \texttt{List<Object> -> Object}
for method \texttt{m} in this situation. Thus, there are two
FGJ-completions of the \texttt{Global} class for the two different
instances of the intersection type:
\begin{center}
  \begin{minipage}{0.49\linewidth}
\begin{lstlisting}
class Global {
  Global m(List<Global> a) {
    return a.add(this).get();
}
\end{lstlisting}
  \end{minipage}
  \begin{minipage}{0.49\linewidth}
\begin{lstlisting}
class Global {
  Object m(List<Object> a) {
    return a.add(this).get();
}
\end{lstlisting}
  \end{minipage}
\end{center}
% \begin{lstlisting}
% class Global {
%   <Global <: T> T m( List<T> a ) {
%     return a.add(this).get();
% }
% \end{lstlisting}
Additional classes in the program may further restrict the number of
viable types. For instance, if we define a class \texttt{UseGlobal} as
follows:
\begin{lstlisting}
class UseGlobal {
  main() {
    return new Global().m(new List<Object>());
} }
\end{lstlisting}
Due to the dependency type inference considers this class after class
\texttt{Global}. As it uses \texttt{m} at type
\texttt{List<Object> -> Object}, global type inference narrows the
type of \texttt{m} to just this alternative.

\subsection{Polymorphic recursion}
\label{sec:polym-recurs}
\begin{figure}[tp]
  \begin{minipage}{0.49\linewidth}
\begin{lstlisting}
class UsePair {
  <X,Y> Object prc(Pair<X,Y> p) {
    return this.prc<Y,X> (p.swap<X,Y>());
} }
\end{lstlisting}
  \end{minipage}
  ~$\left|
  \begin{minipage}{0.49\linewidth}
\begin{lstlisting}
class UsePair {
  prc(p) {
    return this.prc (p.swap());

} }
\end{lstlisting}
  \end{minipage}\right.$
  \caption{Example for polymorphic recursion}
  \label{fig:examples-poly-rec}
\end{figure}

As a toy example for polymorphic recursion consider a class in FGJ with a
generic method \texttt{prc} that invokes itself
recursively on a swapped version of its argument pair
(Figure~\ref{fig:examples-poly-rec}, left).
This method makes use of polymorphic recursion because the type of the
recursive call is different from the type of the original call. More
precisely, the original argument has type \texttt{Pair<X,Y>} whereas
the argument of the recursive call has type
\texttt{Pair<Y,X>}---an instance of the original type.

For this particular example, global type inference succeeds on the
corresponding stripped program shown in
Figure~\ref{fig:examples-poly-rec}, right, but it yields a more restrictive
typing of \texttt{<X> Object prc (Pair<X,X> p)} for the method. 
A variation of the program with a non-variable instantiation makes type inference fail entirely:
\begin{lstlisting}
class UsePair2 {
  <X,Y> Object prc(Pair<X,Y> p) {
    return this.prc<Y,Pair<X,Y>> (new Pair (p.snd, p));
  }
}
\end{lstlisting}



Polymorphic recursion is known to make type inference intractable
\cite{DBLP:journals/toplas/Henglein93,DBLP:journals/toplas/KfouryTU93}
because it can be reduced to an undecidable semi-unification problem
\cite{DBLP:journals/iandc/KfouryTU93}. However, type checking with
polymorphic is tractable and routinely used in languages like Haskell
and Java.


% Features:
% \begin{itemize}
% \item overloaded methods
% \item class header is complete in the form
%   $\mathtt{class}\ C\langle\overline X <: \overline N\rangle <: N \{
%   \overline T\ \overline f;\ K\ \overline M \}$
% \item constructor fully typed (as it is determined by the types of the
%   fields and the supertype)
% \item method signatures may be omitted, i.e., 
%   $M\ ::=\ m(\overline x) \{ \ \mathtt{return}\ e;\ \}$
% \item polymorphic recursive methods must be annotated.
% \end{itemize}

%%% Local Variables:
%%% mode: latex
%%% TeX-master: "TIforGFJ"
%%% End:
