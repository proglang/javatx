\section{Motivation}
\label{sec:motivation}

Examples, examples, examples from simple to more advanced showing off
the (difficult) features of inference.

\begin{lstlisting}
// given this (abstract) class
class Function<S,T> {
  T apply(S arg) { return this.apply (arg); }
}
// I want to infer the type for this class
class Box<S> {
  S val;
  map(f) {
    return new Box(f.apply(this.val));
} }
// the desired outcome is
class Box<S> {
  S val;
  <T> Box<T> map(Function<S,T> f) {
    return new Box<T>(f.apply<S,T>(this.val));
} }
\end{lstlisting}

\subsection{Multiple results}
\label{sec:multiple-results}
% Our global type inference algorithm is able to infer every type annotation.
% For the sake of simplicity we will only consider method types in this
% paper and assume that field types are specified.
\begin{figure}[tp]
  \begin{minipage}{0.49\linewidth}
\begin{lstlisting}
class List<A> {
  List<A> add(A item) {...}
  A get() { ... }
}
\end{lstlisting}
  \end{minipage}
  ~$\left|
  \begin{minipage}{0.49\linewidth}
\begin{lstlisting}
class Global{
  m(a){
    return a.add(this).get();
} }
\end{lstlisting}
  \end{minipage}\right.$
  \caption{Example for multiple inferred types}
  \label{fig:example-types-not-unique}
\end{figure}
Consider the classes \texttt{List<A>} and \texttt{Global} in Figure~\ref{fig:example-types-not-unique}.
Global type inference processes classes in order of
dependency. Class \texttt{Global} may depend on
class \texttt{List} because \texttt{List} defines methods \texttt{add} and
\texttt{get} which may be used in \texttt{Global}. The dependency is
not certain because, in general, there may be additional classes
providing methods \texttt{add} and \texttt{get}.

In the example, it is safe to assume that the types for the methods of class \texttt{List}
are already available, either because they are given (as in the code
fragment) or because they were inferred before considering class \texttt{Global}.

The method \texttt{m} in class \texttt{Global} first invokes
\texttt{add} on \texttt{a}, so the type of \texttt{a} as well as the
return type of \texttt{a.add(this)} must be
\texttt{List<T>}, for some \texttt{T}. As \texttt{this} has
type \texttt{Global}, it must be that \texttt{Global} is a
subtype of \texttt{T}. By the typing of \texttt{get()} we find that
the return type of method \texttt{m} is also \texttt{T}.

But now we are in a dilemma because \texttt{Global <: T} is not a
valid constraint in FGJ: FGJ only supports \emph{upper bounds} for
type variables!\footnote{Java has the same restriction. Lower bounds
  are only allowed for wildcards.}
To stay compatible with this restriction, global type inference
expands the lower bound into two possibilities for
\texttt{T} in this program, \texttt{Global} and
\texttt{Object}.  They give rise to two incomparable types for
\texttt{m}, \texttt{List<Global> -> Global} and
\texttt{List<Object> -> Object}. So there are two different FGJ
programs that are completions of the \texttt{Global} class.

For that reason, global type inference returns the \emph{intersection type}
\texttt{List<Global> -> Global} \verb!&! \texttt{List<Object> -> Object}
for method \texttt{m} in this situation. Thus, there are two
FGJ-completions of the \texttt{Global} class for the two different
instances of the intersection type:
\begin{center}
  \begin{minipage}{0.49\linewidth}
\begin{lstlisting}
class Global {
  Global m(List<Global> a) {
    return a.add(this).get();
}
\end{lstlisting}
  \end{minipage}
  \begin{minipage}{0.49\linewidth}
\begin{lstlisting}
class Global {
  Object m(List<Object> a) {
    return a.add(this).get();
}
\end{lstlisting}
  \end{minipage}
\end{center}
% \begin{lstlisting}
% class Global {
%   <Global <: T> T m( List<T> a ) {
%     return a.add(this).get();
% }
% \end{lstlisting}
Additional classes in the program may further restrict the number of
viable types. For instance, if we define a class \texttt{UseGlobal} as
follows:
\begin{lstlisting}
class UseGlobal {
  main() {
    return new Global().m(new List<Object>());
} }
\end{lstlisting}
Due to the dependency type inference considers this class after class
\texttt{Global}. As it uses \texttt{m} at type
\texttt{List<Object> -> Object}, global type inference narrows the
type of \texttt{m} to just this alternative.

\subsection{Polymorphic recursion}
\label{sec:polym-recurs}
\begin{figure}[tp]
  \begin{minipage}{0.49\linewidth}
\begin{lstlisting}
class UsePair {
  <X,Y> Object prc(Pair<X,Y> p) {
    return this.prc<Y,X> (p.swap<X,Y>());
} }
\end{lstlisting}
  \end{minipage}
  ~$\left|
  \begin{minipage}{0.49\linewidth}
\begin{lstlisting}
class UsePair {
  prc(p) {
    return this.prc (p.swap());

} }
\end{lstlisting}
  \end{minipage}\right.$
  \caption{Example for polymorphic recursion}
  \label{fig:examples-poly-rec}
\end{figure}

As a toy example for polymorphic recursion consider a class in FGJ with a
generic method \texttt{prc} that invokes itself
recursively on a swapped version of its argument pair
(Figure~\ref{fig:examples-poly-rec}, left).
This method makes use of polymorphic recursion because the type of the
recursive call is different from the type of the original call. More
precisely, the original argument has type \texttt{Pair<X,Y>} whereas
the argument of the recursive call has type
\texttt{Pair<Y,X>}---an instance of the original type.

For this particular example, global type inference succeeds on the
corresponding stripped program shown in
Figure~\ref{fig:examples-poly-rec}, right, but it yields a more restrictive
typing of \texttt{<X> Object prc (Pair<X,X> p)} for the method. 
A variation of the program with a non-variable instantiation makes type inference fail entirely:
\begin{lstlisting}
class UsePair2 {
  <X,Y> Object prc(Pair<X,Y> p) {
    return this.prc<Y,Pair<X,Y>> (new Pair (p.snd, p));
  }
}
\end{lstlisting}



Polymorphic recursion is known to make type inference intractable
\cite{DBLP:journals/toplas/Henglein93,DBLP:journals/toplas/KfouryTU93}
because it can be reduced to an undecidable semi-unification problem
\cite{DBLP:journals/iandc/KfouryTU93}. However, type checking with
polymorphic is tractable and routinely used in languages like Haskell
and Java.


% Features:
% \begin{itemize}
% \item overloaded methods
% \item class header is complete in the form
%   $\mathtt{class}\ C\langle\overline X <: \overline N\rangle <: N \{
%   \overline T\ \overline f;\ K\ \overline M \}$
% \item constructor fully typed (as it is determined by the types of the
%   fields and the supertype)
% \item method signatures may be omitted, i.e., 
%   $M\ ::=\ m(\overline x) \{ \ \mathtt{return}\ e;\ \}$
% \item polymorphic recursive methods must be annotated.
% \end{itemize}

%%% Local Variables:
%%% mode: latex
%%% TeX-master: "TIforGFJ"
%%% End:
