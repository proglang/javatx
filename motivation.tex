\section{Motivation}
\label{sec:motivation}

% Examples, examples, examples from simple to more advanced showing off
% the (difficult) features of inference.

In this section, we present a sequence of more and more challenging
examples for global type inference (GTI). To spice up our examples
somewhat, we assume some predefined utility classes with the following
interfaces.
\begin{lstlisting}[style=fgj]
class Bool {
  Bool not(); 
}
class Int {
  Int negate ();
  Int add (Int that);
  Int mult (Int that);
}
class Double {
  Double negate ();
  Double add (Double that);
  Double mult (Double that);
}
\end{lstlisting}

We generally use upper case single-letter identifiers like $\TVX,
\TVY, \dots$ for type variables.
Given a FGJ-GT class \CL 0, 
we call any FGJ class \CL i that can be transformed to \CL 0 by
erasing type annotations a \emph{completion of  \CL 0}.

\subsection{Multiplication}
\label{sec:multiplication}

Here is the \TFGJ code  for multiplying the components of a
pair.\footnote{We indicate \TFGJ code fragments by using a
  {gray background}.}
\begin{lstlisting}[style=tfgj]
class MultPair {
  mult (p) { return p.fst.mult(p.snd); }
}
\end{lstlisting}
Assuming the parameter typing $\mv p:\mv P$, result type $\mv R$, and that
\texttt{mult} in the body refers to \texttt{Int.mult}, we
obtain the following constraints.
\begin{itemize}
\item From \texttt{p.fst}: $\mv P \subconstr \mathtt{Pair}\Angle{\TVX,\TVY}$ and
  $\texttt{p.fst} : \TVX$.
\item From \texttt{p.snd}: $\mv P \subconstr \mathtt{Pair}\Angle{\TVZ,\TVW}$ and
  $\texttt{p.snd} : \TVW$.
\item The two constraints on $\mv P$ imply that $\TVX \eqconstr \TVZ$ and
  $\TVY \eqconstr \TVW$.
\item From \texttt{.mult (p.snd)}: $\TVX \subconstr \mathtt{Int}$, $\TVY \subconstr
  \mathtt{Int}$, and $\mathtt{Int} \subconstr \mv R$.
\end{itemize}
The return type $\mv R$ only occurs positively in the constraints, so we can
set $\mv R = \mathtt{Int}$.
The argument type $\mv P$ only occurs negatively in the constraints,
so $\mv P = \mathtt{Pair} \Angle{\TVX,\TVY}$.
This reasoning gives rise to the following completion.
% \begin{lstlisting}[style=fgj]
% class MultPair {
%   Int mult (Pair<Int,Int> p) { return p.fst.mult(p.snd); }
% }
% \end{lstlisting}
\begin{lstlisting}[style=fgj]
class MultPair {
  <X extends Int, Y extends Int>
  Int mult (Pair<X,Y> p) { return p.fst.mult(p.snd); }
}
\end{lstlisting}
We obtain a second completion if we assume that \texttt{mult} refers to
\texttt{Double.mult}.
\begin{lstlisting}[style=fgj]
class MultPair {
  <X extends Double, Y extends Double>
  Double mult (Pair<X,Y> p) { return p.fst.mult(p.snd); }
}
\end{lstlisting}

Finally, the definition of \texttt{mult} might be recursive, which
generates different constraints for the method invocation of \texttt{mult}.
\begin{itemize}
\item From \texttt{.mult (p.snd)}: $\TVX \subconstr \mathtt{MultPair}$, $\TVY \subconstr
  \mathtt{P}$, and $\mathtt{R} \subconstr \mv R$.
\end{itemize}
Transitivity of subtyping applied to $\TVY \subconstr \mathtt{P}$ and $\mv P \subconstr \mathtt{Pair}\Angle{\TVX,\TVY}$
yields the constraint $\TVY \subconstr \mathtt{Pair}\Angle{\TVX,\TVY}$, which triggers the
occurs-check in unification and is hence rejected. 

% The corresponding example in full Java would be the dot product. In
% full Java, we also have to deal with overloading of the multiplication
% operator, which amounts to considering \texttt{Int.mult} and
% \texttt{Double.mult}. Overloading is not allowed in FGJ, but method
% names in different classes may overlap.

The two solutions can be combined to
\begin{lstlisting}[style=fgj]
class MultPair {
  <X extends T1, Y extends T2>
  T0 mult (Pair<X,Y> p) { return p.fst.mult(p.snd); }
}
\end{lstlisting}
where
\begin{align*}
  (T_0, T_1, T_2) & \in \{ (\mv{Int}, \mv{Int}, \mv{Int}), (\mv{Double}, \mv{Double}, \mv{Double}) \}
\end{align*}

\subsection{Inheritance}
\label{sec:inheritance}

\begin{figure}[tp]
  \begin{subfigure}[t]{0.49\linewidth}
\begin{lstlisting}[style=tfgj]
class A1 {
  m(x) { return x.add(x); }
}
class B1 extends A1 {
  m(x) { return x; }
}
\end{lstlisting}
  \end{subfigure}
\begin{subfigure}[t]{0.49\linewidth}
\begin{lstlisting}[style=tfgj]
class A2 {
  m(x) { return x; }
}
class B2 extends A2 {
  m(x) { return x.add(x); }
}
\end{lstlisting}
  \end{subfigure}
  \caption{Method overriding}
  \label{fig:method-overriding}
\end{figure}

Let's start with the artificial example in the left
listing of Figure~\ref{fig:method-overriding} and ignore the \texttt{Double} class. Type
inference proceeds according 
to the inheritance hierarchy starting from the superclasses. In class
\texttt{A1}, the inferred method type is \texttt{Int A1.m (Int)}. Class \texttt{B1} is a
subclass of \texttt{A1} which must override \texttt{m} as there is no
overloading in FGJ. However, the inferred method 
type is \texttt{<T> T B1.m(T)}, which is not a correct
method override for \texttt{A1.m()}.
Hence, GTI must instantiate the method type in the subclass \texttt{B1} to
\texttt{Int B1.m(Int)}.

Conversely, for the right listing of
Figure~\ref{fig:method-overriding}, GTI infers the types
\texttt{<T> T A2.m (T)} and \texttt{Int B2.m (Int)}. Again, these
types do not give rise to a correct method override and 
GTI is now forced to instantiate the type in the superclass to
\texttt{Int A2.m (Int)}.

\todo[inline]{This example shows that GTI must first collect the constraints 
  for all methods \texttt{m{}} in a class hierarchy. Generalization can only happen after
  all constraints on \texttt{m}'s type have been considered.}

% Otherwise, we would get strange results. Consider extending the
% program with classes \texttt{A} and \texttt{B} by the following class.
% \begin{lstlisting}[style=tfgj]
% class C {
%   mbool(x) { return x.m(new Bool()); }
% }
% \end{lstlisting}
% Inferring the type of \texttt{C.mbool()} using \texttt{A.m()} would
% fail because \texttt{Int} and \texttt{Bool} are not
% unifiable. However, inferring the type of \texttt{C.mbool()} using
% \texttt{B.m()} (with the generic type) would succeed and yield the
% typing
% \begin{lstlisting}[style=fgj]
% class C {
%   Bool mbool(B x) { return x.m(new Bool()); }
% }
% \end{lstlisting}


In full Java, type inference would have to offer two alternative
results: either two different
overloaded methods (one inherited and one local) in \texttt{B1}/\texttt{B2} or
impose the typing \texttt{Int B1.m(Int)} or \texttt{Int A2.m(Int)} to enforce correct overriding. 


\subsection{Inheritance and Generics}
\label{sec:inheritance-generics}


\begin{lstlisting}[float,caption={Function class}, label={lst:function-class},style=fgj]
class Function<S,T> {
  T apply(S arg) { return this.apply (arg); }
}
\end{lstlisting}
Suppose we are given a generic class for modeling functions in  FGJ (Listing~\ref{lst:function-class}).
This code is constructed to serve as an ``abstract'' super class to derive more
interesting subclasses.
The class \texttt{Function<S,T>} must be presented in this explicit
way. Its type annotations \textbf{cannot} be inferred by GTI because
the use to the generic class parameters in the method type cannot be inferred from the
implementation.

If we applied GTI to the type-erased version of
Listing~\ref{lst:function-class}, the \texttt{apply} method would be considered a generic method: 
\begin{center}
  \begin{minipage}{0.3\linewidth}
\begin{lstlisting}[style=tfgj]
apply (arg) { ... }
\end{lstlisting}
  \end{minipage}
  \hfill\texttt{ --GTI--> }\hfill
  \begin{minipage}{0.45\linewidth}
\begin{lstlisting}[style=fgj]
<A,B> B apply (A arg) { ... }
\end{lstlisting}
  \end{minipage}
\end{center}
The typing of \texttt{apply} in Listing~\ref{lst:function-class} is an
instance of this result, so that completeness of GTI is preserved!

% While it is possible to  force GTI to infer the intended typing, it requires some awkward coding.
% \begin{lstlisting}[style=tfgj]
% class Function<S,T> {
%   S in;
%   T out;
%   dummy (arg) { return this.dummy(in); }
%   apply (arg) { return new Pair<>(this.out, this.dummy (arg)).fst; }
% }
% \end{lstlisting}

Now that we have the abstract class \texttt{Function<S,T>} at our
disposal, let us apply GTI to a class of boxed values with a
\texttt{map} function:
\begin{lstlisting}[style=tfgj]
class Box<S> {
  S val;
  map(f) {
    return new Box<>(f.apply(this.val));
} }
\end{lstlisting}
GTI finds the following constraints
\begin{itemize}
\item the return value must be of type \texttt{Box<T>}, for some type
  \texttt{T},
\item \texttt{T} is a supertype of the type returned by
  \texttt{f.apply},
\item \texttt{apply} is defined in class \texttt{Function<S1,T1>} with
  type \texttt{T1 apply(S1 arg)}, 
\item hence \texttt{T1 <: T} and \texttt{S <: S1} (because
  \texttt{this.val : S}),
\end{itemize}
and resolves them to the desired outcome where \texttt{T1=T} and
\texttt{S=S1} using the methods of
Simonet~\cite{DBLP:conf/aplas/Simonet03}. 
\begin{lstlisting}[style=fgj]
class Box<S> {
  S val;
  <T> Box<T> map(Function<S,T> f) {
    return new Box<T>(f.apply<S,T>(this.val));
} }
\end{lstlisting}
But what happens if we add subclasses of \texttt{Function}?
For example:
\begin{lstlisting}[style=tfgj]
class Not extends Function<Bool,Bool> {
  apply(b) { return b.not(); }
}
class Negate extends Function<Int,Int> {
  apply(x) { return x.negate(); }
}
\end{lstlisting}
\todo[inline]{Does this approach really make sense? If we have a
  subclass that overrides a method of the superclass, then the public
  interface should be the one of the superclass. Subsequently, the
  implementation of the method in the subclass should be checked
  against the type inferred for the superclass.}
If we rerun GTI with these classes, we now have additional
possibilities to invoke the \texttt{apply} method. With \texttt{Not}, we need to use the
generic type of \texttt{Function.apply()}, but instantiate it according to
\texttt{Function<Bool,Bool>}. Thus,
we obtain the constraints \texttt{Bool $\subconstr$ T} and \texttt{S $\subconstr$ Bool} for \texttt{T = Bool}
and \texttt{S = Bool}, which are both satisfiable. With
\texttt{Negate} we run into the same situation with the constraints
\texttt{Int $\subconstr$ Int} and \texttt{Int $\subconstr$ Int}.
% That means there is no way to
% \emph{directly} invoke \texttt{Not.apply} or \texttt{Negate.apply}
% from \texttt{Box.map}, but it may happen indirectly via inheritance.

Here is another subclass of \texttt{Function<S,T>} that we want
to consider.
\begin{lstlisting}[style=fgj]
class Identity<S> extends Function<S,S> {
  S apply(S arg) { return arg; }
}
\end{lstlisting}
Here, we obtain the following type constraints
\begin{itemize}
\item \texttt{apply} is defined in class \texttt{Identity<S1>} with
  type \texttt{S1 apply (S1 arg)},
\item hence \texttt{S1 $\subconstr$ T} and \texttt{S $\subconstr$ S1}.
\end{itemize}
Resolving the constraints yields \texttt{S = T} thus the typing
\begin{lstlisting}[style=fgj]
Box<S> map(Identity<S> f);
\end{lstlisting}
which is an instance of the previous typing.

\subsection{Multiple typings}
\label{sec:multiple-results}
% Our global type inference algorithm is able to infer every type annotation.
% For the sake of simplicity we will only consider method types in this
% paper and assume that field types are specified.
\begin{figure}[tp]
  \begin{minipage}{0.49\linewidth}
\begin{lstlisting}[style=fgj]
class List<A> {
  List<A> add(A item) {...}
  A get() { ... }
}
\end{lstlisting}
  \end{minipage}
  ~$\left|
  \begin{minipage}{0.49\linewidth}
\begin{lstlisting}[style=tfgj]
class Global{
  m(a){
    return a.add(this).get();
} }
\end{lstlisting}
  \end{minipage}\right.$
  \caption{Example for multiple inferred types}
  \label{fig:example-types-not-unique}
\end{figure}
Global type inference processes classes in order of
dependency.
To see why, consider the classes \texttt{List<A>} and \texttt{Global}
in Figure~\ref{fig:example-types-not-unique}. 
Class \texttt{Global} may depend on
class \texttt{List} because \texttt{Global} uses methods \texttt{add} and
\texttt{get} and \texttt{List} defines methods with the same names.
The dependency is only approximate because, in general, there may be additional classes
providing methods \texttt{add} and \texttt{get}.

In the example, it is safe to assume that the types for the methods of class \texttt{List}
are already available, either because they are given (as in the code
fragment) or because they were inferred before considering class \texttt{Global}.

The method \texttt{m} in class \texttt{Global} first invokes
\texttt{add} on \texttt{a}, so the type of \texttt{a} as well as the
return type of \texttt{a.add(this)} must be
\texttt{List<T>}, for some \texttt{T}. As \texttt{this} has
type \texttt{Global}, it must be that \texttt{Global} is a
subtype of \texttt{T}, which gives rise to the constraint
\texttt{Global $\subconstr$ T}. By the typing of \texttt{get()} we
find that the return type of method \texttt{m} is also \texttt{T}.

But now we are in a dilemma because FGJ only supports \emph{upper bounds} for
type variables,\footnote{Java has the same restriction. Lower bounds
  are only allowed for wildcards.} so that 
\texttt{Global $\subconstr$ T} is not a valid constraint in FGJ.
To stay compatible with this restriction, global type inference
expands the constraint by instantiating \texttt{T} with the (two) superclasses fulfilling
the constraint, \texttt{Global} and \texttt{Object}.  They give rise to two incomparable
types for 
\texttt{m}, \texttt{List<Global> -> Global} and
\texttt{List<Object> -> Object}. So there are two different FGJ
programs that are completions of the \texttt{Global} class.

GTI models these instances by inferring an \emph{intersection type}
\texttt{List<Global> -> Global \& List<Object> -> Object}
for method \texttt{m} and the different FGJ-completions of class \texttt{Global} are
instances of the intersection type:\footnote{The cognoscenti will be
  reminded of overloading. However, FGJ does not support overloading,
  so we rely on resolution by subsequent uses of the method. Moreover,
  this intersection type cannot be realized by overloading in a Java
  source program because it is resolved according to the raw classes
  of the arguments, in this case \lstinline{List}. It can be realized
  in bytecode which supports overloading on the return type, too.% 
}
\begin{center}
  \begin{minipage}{0.49\linewidth}
\begin{lstlisting}[style=fgj]
class Global {
  Global m(List<Global> a) {
    return a.add(this).get();
}
\end{lstlisting}
  \end{minipage}
  \begin{minipage}{0.49\linewidth}
\begin{lstlisting}[style=fgj]
class Global {
  Object m(List<Object> a) {
    return a.add(this).get();
}
\end{lstlisting}
  \end{minipage}
\end{center}
In this sense, the inferred intersection type represents a principal
typing for the class.
% \begin{lstlisting}[style=fgj]
% class Global {
%   <Global <: T> T m( List<T> a ) {
%     return a.add(this).get();
% }
% \end{lstlisting}
Additional classes in the program may further restrict the number of
viable types. Suppose we define a class \texttt{UseGlobal} as
follows:
\begin{lstlisting}[style=tfgj]
class UseGlobal {
  main() {
    return new Global().m(new List<Object>());
} }
\end{lstlisting}
Due to the dependency on \texttt{Global.m()}, type inference considers this class after class
\texttt{Global}. As it uses \texttt{m} at type
\texttt{List<Object> -> Object}, global type inference narrows the
type of \texttt{m} to just this alternative.

% \todo[inline]{If there are conflicting uses of a method with an
%   intersection type \texttt{T1\&T2}, one might consider splitting the
%   class into one providing the method with type \texttt{T1} and
%   another providing it with \texttt{T2}.}

\subsection{Polymorphic recursion}
\label{sec:polym-recurs}
\begin{figure}[tp]
  \begin{minipage}{0.49\linewidth}
\begin{lstlisting}[style=fgj]
class UsePair {
  <X,Y> Object prc(Pair<X,Y> p) {
    return this.prc<Y,X> (p.swap<X,Y>());
} }
\end{lstlisting}
  \end{minipage}
  ~$\left|
  \begin{minipage}{0.49\linewidth}
\begin{lstlisting}[style=tfgj]
class UsePair {
  prc(p) {
    return this.prc (p.swap());

} }
\end{lstlisting}
  \end{minipage}\right.$
  \caption{Example for polymorphic recursion}
  \label{fig:examples-poly-rec}
\end{figure}
A program uses \emph{polymorphic recursion} if there is a generic method that is invoked
recursively at a more specific type than its definition.
As a toy example for polymorphic recursion consider the FGJ class \texttt{UsePair} with a
generic method \texttt{prc} that invokes itself
recursively on a swapped version of its argument pair
(Figure~\ref{fig:examples-poly-rec}, left).
This method makes use of polymorphic recursion because the type of the
recursive call is different from the declared type of the method. More
precisely, the declared argument type is \texttt{Pair<X,Y>} whereas
the argument of the recursive call has type
\texttt{Pair<Y,X>}---an instance of the declared type.

For this particular example, global type inference succeeds on the
corresponding stripped program shown in
Figure~\ref{fig:examples-poly-rec}, right, but it yields a more restrictive
typing of \texttt{<X> Object prc (Pair<X,X> p)} for the method. 
A minor variation of the FGJ program with a non-variable instantiation makes type inference fail entirely:
\begin{lstlisting}[style=fgj]
class UsePair2 {
  <X,Y> Object prc(Pair<X,Y> p) {
    return this.prc<Y,Pair<X,Y>> (new Pair (p.snd, p));
  }
}
\end{lstlisting}





Polymorphic recursion is known to make type inference intractable
\cite{DBLP:journals/toplas/Henglein93,DBLP:journals/toplas/KfouryTU93}
because it can be reduced to an undecidable semi-unification problem
\cite{DBLP:journals/iandc/KfouryTU93}. However, \emph{type checking} with
polymorphic recursion is tractable and routinely used in languages like Haskell
and Java.

GTI does not infer method types with polymorphic recursion. Inference either fails or
returns a more restrictive type. Classes making use of polymorphic recursion need to supply
explicit typings for methods in question. 


\subsection{Polymorphic Recursion, Formally}
\label{sec:polym-recurs-form}

Consider a class $\mv{C}$ with $n$ mutually recursive methods $\mv{m}_i :
\forall\ol{X}_i. \ol{A}_i \to \ol{A}_i$, for $1\le i\le n$. Define the \emph{instantiation
  multigraph $IG(\mv{C})$} as a directed multigraph with vertices $\{1,
\dots, n\}$.
Edges between $i$ and $j$ in this graph are labeled with a
substitution from $\ol{X}_j$ to types in $\mv{m}_i$, which may contain
type variables from $\ol{X}_i$.
In particular, if $\mv{m}_i$ invokes $\mv{m}_j$ where the generic type variables in
the type of $\mv{m}_j$ are instantiated with substitution
$\ol{U}/\ol{X}_j$ (see rule GT-INVK), then 
$
i \stackrel{\ol{U}/\ol{X}_j}{\longrightarrow} j
$
is an edge of $IG (\mv{C})$.

Define the \emph{closure of the instantiation multigraph $IG^*(\mv{C})$} as the multigraph
obtained from $IG(\mv{C})$ by applying the following rule, which
composes the instantiating substitutions, exhaustively:
\begin{gather}\label{eq:1}
  i \stackrel{\ol{U}/\ol{X}_j}{\longrightarrow} j
  \quad\wedge\quad
  j \stackrel{\ol{V}/\ol{X}_k}{\longrightarrow} k
  \qquad\Rightarrow\qquad
  i \stackrel{[\ol{U}/\ol{X}_j]\ol{V}/\ol{X}_k}{\longrightarrow} k
\end{gather}

\begin{definition}
  We say that method $\mv{m}_i$ is involved in polymorphic recursion
  if there is an edge
  \begin{gather}\label{eq:2}
    i \stackrel{\ol{W}/\ol{X}_i}{\longrightarrow} i \quad \in IG^*
    (\mv{C}) \qquad \text{such that} \qquad \ol{W} \ne \ol{X}_i
  \end{gather}
\end{definition}
For the toy example in Figure~\ref{fig:examples-poly-rec}, we obtain
the multigraph $IG^* (\mv{UsePair})$ which indicates that \mv{prc} is
involved in polymorphic recursion:
\begin{gather*}
  \begin{array}{l@{\quad}|@{\quad}l}
    IG(\mv{UsePair})& IG^* (\mv{UsePair}) \\\hline
    \mv{prc} \stackrel{\mv{Y,X}/\mv{X,Y}}{\longrightarrow} \mv{prc} &
    \mv{prc} \stackrel{\mv{Y,X}/\mv{X,Y}}{\longrightarrow} \mv{prc} \qquad
    \mv{prc} \stackrel{\mv{X,Y}/\mv{X,Y}}{\longrightarrow} \mv{prc}
  \end{array}
\end{gather*}
The method call to \mv{swap} does not appear in the graph because
\mv{swap} is defined in a different class.

For \mv{UsePair2}, we obtain a multigraph $IG^* (\mv{UsePair2})$ with
infinitely many edges which is also clear indication for polymorphic recursion:
\begin{gather*}
  \begin{array}{l@{\quad}|@{\quad}l}
    IG(\mv{UsePair2})& IG^* (\mv{UsePair2}) \\\hline
    \mv{prc} \stackrel{\mv{Y,Pair<X,Y>}/\mv{XY}}{\longrightarrow} \mv{prc}
    &
    \mv{prc} \stackrel{\mv{Y,Pair<X,Y>}/\mv{XY}}{\longrightarrow}
      \mv{prc} \\
    &
      \mv{prc}
      \stackrel{\mv{Pair<X,Y>,Pair<Y,Pair<X,Y>>}/\mv{XY}}{\longrightarrow}
      \mv{prc}
      \\
                     & \dots
  \end{array}
\end{gather*}


Clearly, $IG (\mv{C})$ is finite and can be constructed effectively by
collecting the instantiating substitutions from all method call
sites.
Repeated application of the propagation rule~\eqref{eq:1} either
results in saturation where no edge of the resulting multigraph satisfies~\eqref{eq:2} or it
detects an instantiating edge as in condition~\eqref{eq:2}. 

The following condition is necessary for the absence of
polymorphic recursion.

\begin{proposition}\label{prop:polymorphi-recursion}
  Suppose an FGJ class \mv{C} has $n$ methods, which are mutually
  recursive.
  If \mv{C} does not exhibit
  polymorphic recursion, then
  \begin{itemize}
  \item all methods quantify over the same number of generic
    variables;
  \item if a method has generic variables $\ol{X}$, then each call to
    a method of \mv{C} instantiates with a permutation of the
    $\ol{X}$;
  \item $IG^* (\mv{C})$ is finite.
  \end{itemize}
\end{proposition}
\begin{proof}
  Suppose for a contradiction that there are two distinct methods $\mv{m}_i$ and
  $\mv{m}_j$ with generic variables $\ol{X}_i$ and $\ol{X}_j$, respectively,
  where $|\ol{X}_i| < |\ol{X}_j|$. By mutual recursion, $\mv{m}_i$ invokes
  $\mv{m}_j$ directly or indirectly and vice versa. Hence, $IG^* (\mv{C})$
  contains edges from $i$ to $j$ and back:
  \begin{gather*}
    i \stackrel{\ol{U}/\ol{X}_j}{\longrightarrow} j
    \qquad
    j \stackrel{\ol{V}/\ol{X}_i}{\longrightarrow} i
  \end{gather*}
  As $IG^*(\mv{C})$ is closed under composition, it must also contain
  the edge
  \begin{gather*}
    j \stackrel{[\ol{V}/\ol{X}_i]\ol{U}/\ol{X}_j}{\longrightarrow} j
    \text{.}
  \end{gather*}
  By assumption $\mv{C}$ does not use polymorphic recursion, so it
  must be that $[\ol{V}/\ol{X}_i]\ol{U}/\ol{X}_j =
  \ol{X}_j/\ol{X}_j$. To fulfill this condition, all components of
  $\ol{U}$ must be variables $\in \ol{X}_i$. As  $|\ol{X}_i| <
  |\ol{X}_j| = |\ol{U}|$, there must be some variable $\mv{X} \in \ol{X}_i$ that
  occurs more than once in $\ol{U}$, say, at positions $j_1$ and $j_2$. 
  But that means the variables at positions $j_1$ and $j_2$ in
  $\ol{X}_j$ are mapped to the same component of $\ol{V}$. This is a
  contradiction because this substitution cannot be the identity
  substitution $\ol{X}_j/\ol{X}_j$.

  Hence, all methods have the same number of generic variables and all
  instantiations must use variables.

  Suppose now that there is a direct call from $\mv{m}_i$ to
  $\mv{m}_j$ where the instantiation $\ol{U}/\ol{X}_j$ is not a permutation. Hence,
  there is a variable that appears more than once in $\ol{U}$, which
  leads to a contradiction using similar reasoning as before.

  Hence, all instantiations must be permutations over a finite set of
  variables, so that $IG^*(\mv{C})$ is finite! 
\end{proof}

Moreover, if a class has only mutually recursive methods without
polymorphic recursion, we can assume that each method uses the same
generic variables, say $\ol{X}$, and each instantiation for
class-internal method calls is the identity $\ol X/\ol X$.

Using the same generic variables is achieved by $\alpha$ conversion.
By Proposition~\ref{prop:polymorphi-recursion}, we already know that
each instantiation is a permutation. Each self-recursive call must use
an identity instantiation already, otherwise it would constitute an
instance of polymorphic recursion. Suppose that method \mv{m} calls
method \mv{n} instantiated with a non-identity permutation, say
$\pi$ so that parameter $\mv{X_i}$ of \mv{n} gets instantiated with
$\mv{X_{\pi(i)}}$ of \mv{m}. In this case, we reorder the generic
parameters of \mv{n}  according to the inverse permutation
$\pi^{-1}$ and propagate this permutation to all call sites of
\mv{n}. For the call in \mv{m}, we obtain the identity permutation
$\pi \cdot \pi^{-1}$, for self-recursive calls inside \mv{n}, the
instantiation remains the identity (for the same reason), for a
call site in another method which instantiates \mv{n} with permutation
$\sigma$, we change that permutation to $\sigma \cdot \pi^{-1}$, which
is again a permutation.
This way, we can eliminate all non-identity instantiations from calls
inside \mv{m}.

We move our attention to \mv{n}. Each self-recursive call and each call to \mv{m} uses the
identity instantiation, the latter by construction. So we only need to
consider calls to $\mv{n'}\notin\{\mv{n}, \mv{m}\}$ with an
instantiation which is not the identity permutation. We can also
assume that $\mv{n'}$ is not called from $\mv{m}$: otherwise, \mv{n'}
would have the generic variables in the same order as \mv{m} and hence
as \mv{n}! But that means we can fix all calls to $\mv{n'}$ by
applying the inverse permutations as for \mv{n} \emph{without disturbing the already
  established identity instantiations}!

Each such step eliminates all non-identity instantiations for at least
one method without disturbing previous identity instantiations. Hence,
the procedure terminates after finitely many steps with a class with
all instantiations being identity permutations.

% Features:
% \begin{itemize}
% \item overloaded methods
% \item class header is complete in the form
%   $\mathtt{class}\ C\langle\overline X <: \overline N\rangle <: N \{
%   \overline T\ \overline f;\ K\ \overline M \}$
% \item constructor fully typed (as it is determined by the types of the
%   fields and the supertype)
% \item method signatures may be omitted, i.e., 
%   $M\ ::=\ m(\overline x) \{ \ \mathtt{return}\ e;\ \}$
% \item polymorphic recursive methods must be annotated.
% \end{itemize}

%%% Local Variables:
%%% mode: latex
%%% TeX-master: "TIforGFJ"
%%% End:
