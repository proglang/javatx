\label{sec:unify}
This chapter describes the \textbf{GenericUnify} algorithm
which is used to find type solutions for the constraints generated by \textbf{FGJType}.

\begin{description}
\item[input] A set of type constraints $Cons_{in}$ %and a set of subtype relationships $S_\leq$
\item[output] A set of type unifiers $Uni$
or fail $Uni = \emptyset$.
%The unifiers have the form of $Uni = \{ \sigma_1, \ldots , \sigma_n \}$.
\end{description}

The unify algorithm first has to build the cartesian product of all the \textbf{OrConstraint}s and the remaining normal constraints:
\begin{align*}
\Omega &= \text{All }\mathbf{OrConstraints} \text{ in } {Cons}_{in}\\
C &= {Cons}_{in} \setminus \Omega \\
Eq_{set} &= \Omega_1 \times \ldots \times \Omega_n \times C \quad \text{for all }\Omega_i \in \Omega
\end{align*}

%The algorithm starts by setting $Eq_{set} = \set{ Cons_{in} }$.
Afterwards the following steps are repeatedly executed on $Eq_{set}$ until the algorithm terminates:
%For every $Eq \in Eq_{set}$ the following steps are applied.
%The resulting unifiers $\sigma$ from each $Eq$ merged together form the result of the \textbf{Unify} algorithm.

%$\textbf{SubUnify} :: [Constraint] \to [Unifier]$
\begin{enumerate}
\item Repeated application of the rules depicted in figure \ref{fig:fgjreduce-rules} and \ref{fig:fgjerase-rules}.
The end configuration of $Eq$ is reached if for each element
no rule is applicable.

\item
(The function $\textbf{fresh}(i)$ returns an array of $i$ fresh type variables.)

\begin{align*}
Eq_1 =& \text{Subset of pairs where both type terms are type variables}\\
Eq_2 =& Eq / Eq_1 \\
Eq_{set}\\ 
    = 
     %& \begin{array}[t]{l@{\,}ll}
     % \times \, (\displaystyle{\bigotimes_{(a \lessdot \exptype{C}{\ol{X}}) \in Eq'_2}}
     % \{\,(a \doteq \exptype{D}{\ol{A}}) \, \cup \, (\ol{X} \doteq \ol{Y'}) \ | \ \sarray{@{}l}{
     %   \exptype{D}{\ol{Z}} <: \exptype{C}{\ol{Y}}, \\
     %   \ol{A} = \textbf{fresh}(\#(\ol{Z})), \\
     %   \ol{Y'} = [\ol{A}/\ol{Z}]\ol{Y}
     %   \,\})}\\ 
     % \end{array}\\
   %& \times\, 
   %   (\displaystyle{\bigotimes_{(\exptype{C}{\overline{T}} \lessdot a) \in Eq'_2}}\!\!
   %   \set{(a \doteq [\ol{T}/\ol{X}]N ) \ | \ (\exptype{C}{\overline{X}} \leq N) \in
   %     S_\leq})\\
    & %\times\, 
    (\displaystyle{\bigotimes_{(\exptype{C}{\overline{T}} \lessdot a) \in Eq'_2}}\!\!
    \set{a \doteq [\ol{T}/\ol{X}]N ) \ | \ (\exptype{C}{\overline{X} \triangleleft \ol{N}} <: N})\\
    & \times\, 
      (\displaystyle{\bigotimes_{((X \triangleleft M)\lessdot a) \in Eq'_2}}\!\!
      \set{a \doteq (X \triangleleft M)} \cup
      \set{a \doteq N ) \ | \ M <: N } \cup \set{a \doteq (X \triangleleft M)})\\
      & \times\, \set{[a \doteq N \ | \  (a \doteq N) \in Eq'_2]} \\
      & \times\, \set{[a \lessdot N \ | \  (a \lessdot N) \in Eq'_2]} \times Eq_1 \\
\end{align*}

\item \label{subst-step}  Application of the following \emph{subst} rule
    %\begin{enumerate}
    %\item Apply the following subst\_eq rule
    
      $$\begin{array}[c]{lll}
        (\mathrm{subst}) &
        \begin{array}[c]{l}
          Eq'' \cup \set{a \doteq \theta}\\
          \hline
          Eq''[a \mapsto \theta] \cup \set{a \doteq \theta}
        \end{array}
        & a \textrm{ occurs in } Eq'' \textrm{ but not in } \theta 
      \end{array}$$
      
      for each $a \doteq \theta$ in each element of $Eq' \in Eq'_{set}$.

\item 
    \begin{enumerate}
    \item Foreach $Eq \in Eq_{set}$ which has changed in the last step
      start again with the first step.
    \item Build the union $Eq_{set}$ of all results of (a) and all $Eq' \in
      Eq'_{set}$ which has not changed in the last step.
    \end{enumerate}
\item
\begin{enumerate}
\item Filter all constraint sets which are in solved form:\\
$Eq_{solved} = \set{ Eq \ | \ Eq \in Eq_{set}, Eq \ \text{is in solved form}}$
\item We apply the following rule to every constraint set in $Eq_{solved}$:
\begin{align*}
\ddfrac{
  Eq \cup \set{ a \lessdot b } %There are only Type variables left at this point
}{
  Eq \cup \set{ a \doteq b }
}
\end{align*}
\item $\emph{Uni} = \set{\sigma \ | \ Eq \in Eq_{solved},\ \sigma = \set{a \mapsto \theta \ | \ (a \doteq \texttt{T}) \in Eq} }$
%\item $\emph{Uni} = \sarray{l@{\ }l}{\set{\sigma \ | & Eq'' \in Eq''_{set},
%        Eq'' \textrm{ is in solved form,}\\ 
%        & \sigma = \set{a \mapsto \theta \ | \ (a \doteq \theta) \in Eq''}
%        \\ & \quad \cup \ \set{a \mapsto \texttt{A}, b \mapsto \texttt{B} \ | \ (a \lessdot b) \in Eq \ \text{and a is an isolated type variable}}
%        }}$
\end{enumerate}
\end{enumerate}

\begin{definition}[Isolated type variable] \label{def:isolated-type-variable}
  \rm
An isolated type variable does only occur in constraints together with another isolated type variable.
\end{definition}

\begin{definition}[Solved form]\label{def:solved-form}
  \rm
  A set of constraints $Eq$ has reached solved form if it contains only the following kind of constraints:
  \begin{enumerate}
    \item $a \lessdot b$, with $a$ and $b$ both isolated type variables
    \item $a \lessdot \exptype{C}{\ol{X}}$
    \item $a \doteq \exptype{C}{\ol{X}}$
  \end{enumerate}
\end{definition}

\begin{figure}
\begin{center}
    \leavevmode
    \fbox{
    \begin{tabular}[t]{ll}
      (match)
      & $
      \begin{array}[c]{ll}
      \begin{array}[c]{l}
         Eq \cup \, \set{a \lessdot
         \exptype{C}{\ol{X}},
         a \lessdot
          \exptype{D}{\ol{Y}}} \\ 
        \hline
        \vspace*{-0.4cm}\\
        Eq \cup \set{a \lessdot \exptype{C}{\ol{X}}
        , \exptype{C}{\ol{X}} \lessdot \exptype{D}{\ol{Y}}}
      %Eq \cup \set{\theta_1 \doteq \lambda'_1 \ldo \theta_n \doteq \lambda'_n}
      \end{array}
      & \exptype{C}{\ol{Z}} <: \exptype{D}{\ol{N}} 
      \end{array}
      $
    \\\\
    (adopt)
    & $
    \begin{array}[c]{ll}
    \begin{array}[c]{l}
       Eq \cup \, \set{a \lessdot
       \exptype{C}{\ol{X}},
       b \lessdot^* a, b \lessdot \exptype{D}{\ol{Y}}} \\ 
      \hline
      \vspace*{-0.4cm}\\
      Eq \cup \set{
        a \lessdot
       \exptype{C}{\ol{X}},
       b \lessdot
        a
      , b \lessdot \exptype{C}{\ol{X}}}
    %Eq \cup \set{\theta_1 \doteq \lambda'_1 \ldo \theta_n \doteq \lambda'_n}
    \end{array}
    \end{array}
    $
  \\\\
      (adapt)
      & $
      \begin{array}[c]{ll}
      \begin{array}[c]{l}
         Eq \cup \, \set{\exptype{C}{\ol{A}} \lessdot
          \exptype{D}{\ol{B}}} \\ 
        \hline
        \vspace*{-0.4cm}\\
        Eq \cup \set{\exptype{D}{[ \ol{A} / \ol{X} ]\ol{Y}}
        \doteq \exptype{D}{\ol{B}}}
      %Eq \cup \set{\theta_1 \doteq \lambda'_1 \ldo \theta_n \doteq \lambda'_n}
      \end{array}
      & \exptype{C}{\ol{X}} <:\ \exptype{D}{\ol{Y}}
      \end{array}
      $
    \\\\
(reduce1) & $
\begin{array}[c]{l}
  Eq \cup \set{\exptype{D}{\ol{A}} \lessdot
    \exptype{D}{\ol{B}}}\\
  \hline
  Eq \cup \set{\ol{A} \doteq \ol{B}}
\end{array}
      $ \\\\
(reduce2) & $
\begin{array}[c]{l}
  Eq \cup \set{\exptype{D}{\ol{A}} \doteq
    \exptype{D}{\ol{B}}}\\
  \hline
  Eq \cup \set{\ol{A} \doteq \ol{B}}
\end{array}
      $ \\\\
(equals) & $
\begin{array}[c]{l}
  Eq \cup \set{a \lessdot
    b, b \lessdot c, \ldots, m \lessdot n, n \lessdot a}\\
  \hline
  Eq \cup \set{a \doteq b, a \doteq c, \ldots}
\end{array}
      $ \\\\
    \end{tabular}}
  \end{center}
\caption{Reduce and adapt rules}\label{fig:fgjreduce-rules}
\end{figure}

The $\lessdot^*$ used in the \texttt{adopt} rule means,
that this can either be a single constraint $b \lessdot a$ or a series of constraints
$n \lessdot a, \ldots, m \lessdot n, b \lessdot m$.

\begin{figure}
  \begin{center}
      \leavevmode
      \fbox{
      \begin{tabular}[t]{ll}
        (matchG)
        & $
        \begin{array}[c]{ll}
        \begin{array}[c]{l}
           Eq \cup \, \set{a \lessdot
           \exptype{C}{\ol{X}},
           a \lessdot
            (X \triangleleft \exptype{D}{\ol{Y}})} \\ 
          \hline
          \vspace*{-0.4cm}\\
          Eq \cup \set{a \lessdot \exptype{C}{\ol{X}}
          , \exptype{C}{\ol{X}} \lessdot \exptype{D}{\ol{Y}}}
        %Eq \cup \set{\theta_1 \doteq \lambda'_1 \ldo \theta_n \doteq \lambda'_n}
        \end{array}
        & \exptype{C}{\ol{Z}} <: \exptype{D}{\ol{N}} 
        \end{array}
        $
      \\\\
      (adoptG)
      & (the same as adopt)
    \\\\
    (adaptG)
    & $
    \begin{array}[c]{ll}
    \begin{array}[c]{l}
       Eq \cup \, \set{(X \triangleleft \exptype{D}{\ol{A}}) \lessdot
        \exptype{C}{\ol{B}}} \\ 
      \hline
      \vspace*{-0.4cm}\\
      Eq \cup \set{\exptype{C}{[ \ol{A} / \ol{X} ]\ol{Y}}
      \doteq \exptype{C}{\ol{B}}}
    %Eq \cup \set{\theta_1 \doteq \lambda'_1 \ldo \theta_n \doteq \lambda'_n}
    \end{array}
    & \exptype{D}{\ol{X}} <:\ \exptype{C}{\ol{Y}}
    \end{array}
    $
  \\\\
  (adaptG2)
  & $
  \begin{array}[c]{ll}
  \begin{array}[c]{l}
     Eq \cup \, \set{(X \triangleleft \exptype{C}{\ol{A}}) \lessdot
      \exptype{D}{\ol{B}}} \\ 
    \hline
    \vspace*{-0.4cm}\\
    Eq \cup \set{\exptype{C}{\ol{A}}
    \doteq \exptype{C}{[ \ol{B} / \ol{X} ]\ol{Y}}}
  %Eq \cup \set{\theta_1 \doteq \lambda'_1 \ldo \theta_n \doteq \lambda'_n}
  \end{array}
  & \exptype{D}{\ol{X}} <:\ \exptype{C}{\ol{Y}}
  \end{array}
  $
\\\\
  (reduce1) & $
  \begin{array}[c]{l}
    Eq \cup \set{\exptype{D}{\ol{A}} \lessdot
      \exptype{D}{\ol{B}}}\\
    \hline
    Eq \cup \set{\ol{A} \doteq \ol{B}}
  \end{array}
        $ \\\\
  (reduce2) & $
  \begin{array}[c]{l}
    Eq \cup \set{\exptype{D}{\ol{A}} \doteq
      \exptype{D}{\ol{B}}}\\
    \hline
    Eq \cup \set{\ol{A} \doteq \ol{B}}
  \end{array}
        $ \\\\
      \end{tabular}}
    \end{center}
  \caption{Reduce and adapt rules with generic variables}\label{fig:fgjreduce-rules-generic}
  \end{figure}

\begin{figure}
\begin{align*}
&\begin{tabular}[t]{ll}
      (erase1)  & $ 
      \begin{array}[c]{ll}
        \begin{array}[c]{l}
          Eq \cup \set{C \lessdot D}\\
          \hline
          Eq
        \end{array}
        & C \leq^* D \in S_\leq
      \end{array}$\\
          \end{tabular}\\
&\begin{tabular}[t]{ll}
      (erase2)  & $ 
      \begin{array}[c]{ll}
        \begin{array}[c]{l}
          Eq \cup \set{C \doteq C}\\
          \hline
          Eq
        \end{array}
      \end{array}$\\
          \end{tabular}\\
    &      \begin{tabular}[t]{ll}
       (swap) & $
            \begin{array}[c]{ll}
              \begin{array}[c]{l}
                Eq \cup \set{C \doteq a}\\
                \hline
                Eq \cup \set{a \doteq C}
              \end{array}
            \end{array}$
          \end{tabular}
\end{align*}
\caption{Erase and swap rules}\label{fig:fgjerase-rules}
\end{figure}