\subsection{Properties of \unify{}}
\label{sec:properties-unify}

First we give some definitions and results.
For the complete proofs see appendix \ref{chapter:unifySoundnessProof}, \ref{chapter:unifyCompletenessProof} and \ref{chapter:unifyTerminationProof}.
%We proof Soundness and Completeness in appendix \ref{chapter:unifySoundnessProof} and \ref{chapter:unifyCompletenessProof}.
%We proof termination of the \unify{} algorithm in appendix \ref{chapter:unifyTerminationProof}.

\begin{definition}[Unifier]
  Let $C$ be a set of constraints and $\Delta$ a type environment.
  A substitution $\sigma$  is a \emph{unifier} of $(C,\Delta)$ if
  \begin{itemize}
  \item for each $(\itype{T} \lessdot \itype{U}) \in C$ it holds that
    $\Delta \vdash \exp{\sigma}{T} \olsub \exp{\sigma}{U}$;
  \item  for each $(T \doteq U) \in C$ it holds that
    $\exp{\sigma}{T} = \exp{\sigma}{U}$; and
  \item for each or-constraint $\set{\set{\overline{\simpleCons_1}},
      \ldots, \set{\overline{\simpleCons_n}}} \in C$, there exists $1
    \le i \le n$ such that $\sigma$ is a unifier of
    $(\set{\overline{\simpleCons_i}}, \Delta)$. 
  \end{itemize}
\end{definition}

A set of general unifiers can provide any unifier as a substitution
instance of one of its members.
\begin{definition}[Set of general unifiers]
  Let $C$ be a set of constraints and $\Delta$ a type environment. 

  A set of unifiers $M$ for $(C, \Delta)$
  is called \emph{set of general unifiers} if for any unifier $\omega$
  for $(C, \Delta)$ there is some unifier $\sigma \in M$ and a substitution
  $\lambda$ such that $\omega = \lambda   \circ \sigma$.
\end{definition}

A unification problem is \emph{finitary} if there is a finite set of
general unifiers for each constraint set $C$ and type environment $\Delta$.

\begin{theorem}[Soundness]
  \label{theo:unifySoundness}
  % \replaced{If the \unify{} algorithm finds a solution it does not contradict any of the input constraints:
  % $\nexists (a \lessdot b) \in \consSet_{in}$ where $\sigma(a) \nless :
  % \sigma(b)$}{}
   If $\unify{}(\consSet, \Delta) = {(\sigma,  \ol{Y} \triangleleft
     \ol{P})}$, then $\sigma$ is a unifier of $(\consSet,\Delta \cup \set{\ol{Y} <: \ol{P}})$. 
\end{theorem}

\begin{theorem}[Completeness]\label{theo:unifyCompleteness}
  $\unify{} (\consSet, \Delta)$ calculates {the set} of general
  unifiers for $(\consSet, \Delta)$. 
\end{theorem}
% A unifier $\sigma$ is a general unifier for $\consSet_{in}$ if it unifies $\consSet_{in}$
% and for every other unifier $\omega$ there is a substitution $\lambda$ so
% that $\omega(x) = \lambda(\sigma(x))$.


\begin{theorem}[Termination]\label{theo:unifyTermination}
  The \unify{} algorithm terminates on every finite input set.
\end{theorem}


%%% Local Variables:
%%% mode: latex
%%% TeX-master: "TIforGFJ"
%%% End:
