
\subsection{Unify proof}

%Soundness: We have to prove that each calculated result of the algorithm is a general unifier of the corresponding input
\begin{theoremAndi}
  \label{theo:unifySoundness}
  \textbf{(Soundness):}
  If the \textbf{Unify} algorithm finds a solution it does not contradict any of the input constraints:
  $\nexists (a \lessdot b) \in {Cons}_{in}$ where $\sigma(a) \nleq \sigma(b)$  
\end{theoremAndi}
%We show this by induction.
%No step alters the constraint set in a way that would make a wrong solution possible.
\textit{Proof:}
We show theorem \ref{theo:unifySoundness} by going backwards over every step of the algorithm.
We assume there exists a unifier $\sigma = \set {a_1 \mapsto \theta_1, \ldots , a_n \mapsto \theta_n}$ for the input constraints,
which is the result of the \textbf{Unify} algorithm.
This means for every constraint in the input set $(a \lessdot b) \in {Cons}_{in}$ and $(c \doteq d) \in {Cons}_{in}$
this unifier will substitute all variables in a way that all constraints are satisfied:
$\sigma(a) \leq \sigma(b)$, $\sigma(c) = \sigma(d)$\\

We now look at each step of the \textbf{Unify} algorithm
which transforms the input set of constraints $Eq$ to a set $Eq'$.
If we assume the unifier $\sigma$ is correct for the set $Eq'$,
then we can show that it will also be correct for the constraints $Eq$. 


%TODO:
%Assume we find a unifier. Then do every step backwards.
%For each step the constraint set before and after the step have the same unifier.
% -> This means no step adds an incorrect unifier / makes a incorrecto unifier possible
%At the end we must end at the correct unifier.

\begin{description}
\item[Step 5 c)]
The last step of the algorithm transforms a set of constraints $Eq$ of the the form

$Eq = \set{a_1 \doteq \theta_1, \ldots , a_n \doteq \theta_n}$

to the unifier $\sigma$.
Trivial.

\item[Step 5 b)]
A unifier which is correct for $a \doteq b$ is also correct for $a \lessdot b$.

\item[Step 5 a)]
Trivial, we do not alter the constraint set which lateron leads to the unifier.

\item[Step 4]
Trivial, the constraint sets are not altered here.

\item[Step 3]
An unifier $\sigma$ that is correct for a constraint set
$Eq[a \to \theta] \cup (a \doteq \theta)$ is also correct for
the set $Eq \cup (a \doteq \theta)$.
From the constraint $(a \doteq \theta)$ it follows that $\sigma(a) = \theta$.
This means that $\sigma(Eq) = \sigma(Eq[a \to \theta])$,
because every occurence of $a$ in $Eq$ will be raplaced by $\theta$ anyways when using the unifier $\sigma$.

\item[Step 2]
This step transforms constraints of the form $(\exptype{C}{\ol{X}} \lessdot a)$ and $(a \lessdot \exptype{C}{\ol{X}})$
into sets of constraints and builds the cartesian product with the remaining constraints.
We can show that if there is a resulting set of constraints which has $\sigma$ as its correct unifier
then $\sigma$ also has to be a correct unifier for the constraints before this transformation.
Proof:
%normal version:
%\begin{description}
%\item[$(a \lessdot C)$] If $\sigma$ is a correct unifier for a set containing $(a \doteq \theta)$
%and $\theta \leq C$, then $\sigma$ is also a correct unifier for the set containing $(a \lessdot C)$.
%\item[$(C \lessdot a)$] Same goes the other way. If $C leq \theta$ and $\sigma$ is correct for $(C \lessdot \theta)$
%then $\sigma$ is also correct for $(a \doteq \theta)$
%\end{description}

\begin{description}
\item[$(a \lessdot \exptype{C}{\ol{X}})$] If $\sigma$ is a correct unifier for a set containing $(a \doteq \exptype{D}{\ol{A}})$
and $(\ol{X} \doteq \ol{Y'})$
, then $\sigma$ is also a correct unifier for the set containing $(a \lessdot \exptype{C}{\ol{X}})$.
The reason is the \texttt{S-CLASS} subtype rule of FGJ. %TODO: Hier etwas ausführlicher?
%TODO:
%\item[$(\exptype{C}{\ol{T}} \lessdot a)$] Same goes the other way. If $\exptype{C}{\ol{X} leq \theta$ and $\sigma$ is correct for $(C \lessdot \theta)$
%then $\sigma$ is also correct for $(a \doteq \theta)$

%If $\sigma$ is a correct unifier for a set \\
%$Eq = Eq' \cup (a \doteq \exptype{D}{\ol{A}}) \, \cup \, (\ol{X} \doteq \ol{Y'})$ \\
%then it is also a correct unifier for a set $Eq = Eq' \cup (a \lessdot \exptype{C}{\ol{X}})$: \\
When substituting $(a \to \exptype{D}{\ol{A}})$ and $(\ol{X} \to \ol{Y'})$
and finally $ (\ol{Y'} \to [\ol{A}/\ol{Z}]\ol{Y})$ in the constraint $(a \lessdot \exptype{C}{\ol{X}})$
we get: $(\exptype{D}{\ol{A}} \lessdot [\ol{A}/\ol{Z}]\exptype{C}{\ol{Y}}$
which is correct under the \texttt{S-CLASS} rule (see chapter \ref{chapter:type-rules}).

\item[$(\exptype{C}{\ol{T}} \lessdot a)$] If $\exptype{C}{\ol{X}} leq \theta$ and $\sigma$ is correct for $(a \doteq [\ol{T}/\ol{X}]N)$
then $\sigma$ is also correct for $(\exptype{C}{\ol{T}} \lessdot a))$.
When substitutin $a$ for $[\ol{T}/\ol{X}]N$ we get 
$(\exptype{C}{\ol{T}} \lessdot [\ol{T}/\ol{X}]N)$
, which is correct because $(\exptype{C}{\ol{X}} \leq \exptype{C}{\ol{Y}}$
(see \texttt{S-CLASS} rule).

\footnote{
Discussion:
Do we need to include the $\triangleleft \ol N$ bounds?
The S-CLASS rule does not mention them.

When ignoring those rules this could lead to an error
class T<X extends List> {
}
class S<X>{}

Constraint: S<String> <. a
Unify: T<String> =. a // ERROR!

This is not a problem because no error is gonna result from this.
TYPEExpr only hast to implement all of the typing rules of FJ
and unify has to solely respect the subtyping rules.

}
\end{description}

\item[Step 1]
\begin{description}
\item[erase-rules] remove correct constraints from the constraint set.
A unifier $\sigma$ that is correct for the constraint set $Eq$
is also correct for $Eq \cup \set{\theta \doteq \theta}$
and $Eq \cup \set{\theta \lessdot \theta'}$, when $\theta \leq \theta'$.
\item[swap-rule] does not change the unifier for the constraint set.
$\doteq$ is a symmetric operator and parameters can be swapped freely.
\item[match] The subtype relation is transitive, so if there is a correct solution for
$a \lessdot \exptype{C}{\ol{X}}, \exptype{C}{\ol{X}} \lessdot \exptype{D}{\ol{Y}}$
then this solution would also apply for $a \lessdot \exptype{C}{\ol{X}} \lessdot \exptype{D}{\ol{Y}}$
or $a \lessdot \exptype{D}{\ol{Y}}$.
\item[adopt] An unifier which is correct for $Eq \cup \set{a \lessdot \exptype{C}{\ol{X}}, a \lessdot b, b \lessdot \exptype{C}{\ol{X}}}$
is also correct for $Eq \cup \set{a \lessdot \exptype{C}{\ol{X}}, a \lessdot b}$.
\item[adapt] If there is a $\sigma$ which is a correct unifier for a set
$Eq \cup \set{ \exptype{C}{[\ol{A}/\ol{X}]\ol{Y}} \doteq \exptype{C}{\ol{B}}}$ then it is also
a correct unifier for the set $Eq \cup \set{ \exptype{D}{\ol{A}} \lessdot \exptype{C}{\ol{B}}}$,
if there is a subtype relation $\exptype{D}{\ol{X}} \leq^* \exptype{C}{\ol{Y}}$.
To make the set $Eq \cup \set{ [\ol{A}/\ol{X}]\exptype{C}{\ol{Y}} \doteq \exptype{C}{\ol{B}}}$ the unifier 
$\sigma$ must satisfy the condition $\sigma([\ol{A}/\ol{X}]\ol{Y}) = \sigma(\ol{B})$.
By substitution we get $Eq \cup \set{ \exptype{D}{\ol{A}} \lessdot \exptype{C}{[\ol{A}/\ol{X}]\ol{Y}}}$
which is correct under the \texttt{S-CLASS} rule.
\item[reduce] The \texttt{reduce1} and \texttt{reduce2} rules are obviously correct under the FJ typing rules.
\end{description}

\item[OrConstraints]
If $\sigma$ is a correct unifier for one of the constraint sets in $Eq_{set}$
then it is also a correct unifier for the input set $Cons_{in}$.
When building the cartesian product of the \textbf{OrConstraints} every possible
combination for $Cons_{in}$ is build.
No constraint is altered, deleted or modified during this step.
\end{description}
\hfill $\square$


%Completeness: If there is a Unifier (a solution), the algorithm will never alter the constraints in a way that this solution is removed
%\begin{theoremAndi}\label{theo:unifyCompleteness}
%  \textbf{(Completeness):} If there is a solution for the input constraints $Cons_{in}$, the \textbf{Unify} algorithm will not fail.
%  A solution is a non-empty set of unifiers $Uni = \set{\sigma_1, \ldots \sigma_n }$,
%  where each unifier is a injective function which maps every type variable in the input constraints $Cons_{in}$ to a
%  type in $S_\leq$.
%  \end{theoremAndi}
\begin{theoremAndi}\label{theo:unifyCompleteness}
  \textbf{(Completeness):} The \textbf{Unify} algorithm calculates all principal type solutions for the input set of constraints ($Cons_{in}$).
  A unifier $\sigma$ is a principal type solution for $Cons_{in}$ if it unifies $Cons_{in}$
  and for every other unifier $\omega$ there is a unifier $\lambda$ so that $\omega(x) = \lambda(\sigma(x))$.
\end{theoremAndi}
\textit{Proof:}
%The \textbf{Unify} calculates multiple solutions.

%Our proof goes as follows:
We look at every step of the algorithm, which alters the set of constraints $Eq$,
while assuming that there is at least one possible principal type solution $\sigma$ for the input.
We will show that the principal type is among them by proofing for every step of the algorithm that the principal type is never excluded.

%Assume there is a unifier and the Unify algorithm finds it.
%Then no rule makes this unifier impossible / removes this unifier.

\begin{description}
\item[Step 1:]
The first step applies the three rules from figure \ref{fig:unifyrules}.
\textbf{erase-rules:} The erase2 rule from figure \ref{fig:unifyrules} removes a
$\{\theta \leq \theta\}$ constraint from the constraint set.
The erase1 rule removes a $\{\theta \leq \theta\}$ constraint,
but only if the two types $\theta$ and $\theta'$ satisfy the constraint.
Both rules do not change the set of possible solutions for the given constraint set.

\textbf{swap-rule:} $\doteq$ is a symmetric operator and parameters can be swapped freely.
This operation does not change the meaning of the constraint set.

\textbf{match-rule:}
If there is a solution for $a \lessdot \exptype{C}{\ol{X}}, a \lessdot \exptype{D}{\ol{Y}}$,
this is also a solution for $a \lessdot \exptype{C}{\ol{X}}, \exptype{C}{\ol{X}} \lessdot \exptype{D}{\ol{Y}}$.
A correct unifier $\sigma$ has to find a type for $a$, which complies with $a \lessdot \exptype{C}{\ol{X}}$ and $a \lessdot \exptype{D}{\ol{Y}}$.
Due to the subtyping relation being transitive this means that $\sigma(a) \lessdot \exptype{C}{\ol{X}} \lessdot \exptype{D}{\ol{Y}}$.

\textbf{adopt-rule:} Subtyping in FJ is transitive,
which allows us to apply the adopt rule without excluding any possible unifier.

\textbf{adapt-rule:} Every solution which is correct for the constraints
$Eq \cup \set{ \exptype{C}{[\ol{A}/\ol{X}]\ol{Y}} \doteq \exptype{C}{\ol{B}}}$ is also
a correct solution for the set $Eq \cup \set{ \exptype{D}{\ol{A}} \lessdot \exptype{C}{\ol{B}}}$.
According to the \texttt{S-CLASS} rule there can only be a possible solution for 
$\exptype{C}{[\ol{A}/\ol{X}]\ol{Y}} \doteq \exptype{C}{\ol{B}}$
if $\ol{B} = [\ol{A}/\ol{X}]\ol{Y}$.
Therefore this transformation does not remove any possible solution from the constraint set.

\textbf{reduce-rule:}
%The constraint is not altered
For a constraint $\exptype{D}{\ol{A}} \lessdot \exptype{D}{\ol{A}}$ the FJ subtyping rule \texttt{S-REFL} ($\triangle \vdash T <: T$) is the only one which applies.
According to this rule the transformation to $\ol{A} \doteq \ol{B}$ is correct.
Only $D$ gets removed, which is not a type variable.
Therefore this step does not remove a possible solution.
This applies for both reduce rules \textbf{reduce1} and \textbf{reduce2}.

\item[Step 2:]
%The second step builds multiple constraint sets of all possible type combinations for the $\lessdot$-constraints.
The second step of the algorithm eliminates $\lessdot$-constraints
by replacing them with $\doteq$-constraints.
%The algorithm considers every possible type from $S_\leq$ which does not violate the eliminated $\lessdot$ -constraint itself.
%This step does not remove a solution from the constraint set.
For each $(a \lessdot \exptype{C}{\ol{X}})$ constraint the algorithm builds a set with every
possible subtype of $\exptype{C}{\ol{X}}$ set in for $a$.
So if there is a correct unifier $\sigma$ for the constraints before this conversion there will be at least one set of
constraints for which $\sigma$ is a correct unifier.
%TODO: what does soundness mean. if there is a possible type solution (with types in S) then unify will find it

\item[Step 3:]
In the third step the \textbf{substitution}-rule is applied.
If there is a constraint $a \doteq \theta$ then there is no other way to fulfill the constraint set
than replacing $a$ with $\theta$.
This does not remove a possible solution.

\item[Step 4:]
None of the constraints get modified.

\item[Step 5 a):]
The removed sets do not have a possible unifier, therefore no possible solution is
omitted in this step.

\textbf{Proof}:
In step 5.a all constraint sets that have a unifier are in solved form.
All other possibilities are eliminated in steps 1-4.
There are 8 different variations of constraints:\\
$(a \doteq a), (a \doteq C), (C \doteq a), (C \doteq C), (a \lessdot a), (a \lessdot C), (C \lessdot a), (C \lessdot C)$

After step 1 there are no $(C \doteq C)$, $(C \lessdot C)$ and $(C \doteq a)$ constraints anymore,
as long as the constraint set has a correct unifier.
Because a constraint set that has a correct unifier cannot contain constraints of the form $\theta_1 \doteq \theta_2$ with $\theta_1 \neq \theta_2$ and
$(\theta_1 \lessdot \theta_2)$ with $(\theta_1 \leq \theta_2) \notin S_\leq$.
By removing $(\theta \doteq \theta)$ and $(\theta \lessdot \theta')$ with $(\theta \leq \theta') \notin S_\leq$ constraints
no constraints of the form $(C \doteq C)$ and $(C \lessdot C)$
remain in a constraint set that has a correct unifier after step 1.

After step 2 there are no more $(a \lessdot C)$ constraints.

After step 3 there are no $(a \doteq C)$ anymore.

We only reach step 5 if the constraint set is not changed by the substitution (step 3).

%If at this point a set $Eq_i$ is not in solved form it has no correct unifier.

If the constraint set has a correct unifier only $(a \lessdot a)$, $(a \doteq a)$ and $(a \doteq C)$ constraints are left at this point.
The type variables in the $(a \lessdot a)$ and $(a \doteq a)$ constraints have to be independent type variables.
If a type variable $c$ is inside a $(c \doteq C)$ constraint it is not an independent type variable.
But this variable $c$ cannot be inside a $(a \doteq a)$ or $(a \lessdot a)$ constraint, because otherwise step 3 would have replaced it in there.


\item[Step 5 b):]
If the algorithm advances to this step we further only work on constraint sets in solved form.
This means there are only two kinds of constraints left.
($A \doteq \texttt{Typ}$), ($A \doteq B$) and ($A \lessdot B$) with $A$ and $B$ as type variables.

%We can set all TVs equal, because we allow only same TVs when having circles in a method call.
%This still will lead to the principal type.

The FGJ language does not allow subtype constraints for generic types.
A constraint like $(A \lessdot B)$ in a solution could be inserted as the typing shown in the example below.
But this is not allowed by the syntax of FGJ.
That is why we can treat this constraint as $(A \doteq B)$.

%TODO: This does not alter the outcome because the solution set is not modified anymore. All other TVs alread have a type like A =. Typ

\textit{Example:}
This would be a valid Java program but is not allowed in FGJ:
\begin{lstlisting}
class Example {
  <A extends Object, B extends A> A id(B a){
    return a;
  }
}
\end{lstlisting}

By replacing all ($A \lessdot B$) constraints with ($A \doteq B$) we do not remove a principal type solution.

\item[Step 6:]
In the last step all the constraint sets, which are in solved form, are converted to unifiers.

We see that only a constraint set which has no unifier does not reach solved form.
We showed that in every step of the \textbf{Unify} algorithm we never exclude a possible unifier.
Also we showed that after we reach step 5 only constraint sets with a correct unifier are in solved form.
By removing all constraint sets which are not in solved form the algorithm does not
remove a possible correct unifier.

If we assume that there is a possible principal type solution $\sigma$ for the input set $Cons_{in}$
and the \textbf{Unify} algorithm does not exclude any of the possible unifiers,
then the result \textbf{Unify} contains the principal type solution.
\hfill $\square$
\end{description}

\textbf{Termination Proof:}
Every step of the algorithm removes a $\lessdot$ constraint with the following exceptions:
A few rules of Step 1 do not change the amount of $\lessdot$ constraints.
Each of those rules can be applied only a finite amount of times on a finite set of constraints $Eq$.
So thos will stop to be applied automatically.
The only other exception is the \texttt{adopt} rule.
This rule can only be applied once for each combination of 

Does the adopt rule really terminate?
no -> if there would be a circle, but circles are deleted by equals
a <. b, b <. C<x1>, a <. C<x2>, b <. c, c <. C<x>, c <. a
yes -> only finite combination of a <. b and a <. C
