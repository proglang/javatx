%%
%% This is file `sample-acmsmall-submission.tex',
%% generated with the docstrip utility.
%%
%% The original source files were:
%%
%% samples.dtx  (with options: `acmsmall-submission')
%% 
%% IMPORTANT NOTICE:
%% 
%% For the copyright see the source file.
%% 
%% Any modified versions of this file must be renamed
%% with new filenames distinct from sample-acmsmall-submission.tex.
%% 
%% For distribution of the original source see the terms
%% for copying and modification in the file samples.dtx.
%% 
%% This generated file may be distributed as long as the
%% original source files, as listed above, are part of the
%% same distribution. (The sources need not necessarily be
%% in the same archive or directory.)
%%
%% The first command in your LaTeX source must be the \documentclass command.
\documentclass[acmsmall,screen,review]{acmart}

\usepackage{todonotes} % [disable]
\usepackage[utf8]{inputenc}
\usepackage{prolog}

%%
%% \BibTeX command to typeset BibTeX logo in the docs
\AtBeginDocument{%
  \providecommand\BibTeX{{%
    \normalfont B\kern-0.5em{\scshape i\kern-0.25em b}\kern-0.8em\TeX}}}

%% Rights management information.  This information is sent to you
%% when you complete the rights form.  These commands have SAMPLE
%% values in them; it is your responsibility as an author to replace
%% the commands and values with those provided to you when you
%% complete the rights form.
\setcopyright{acmcopyright}
\copyrightyear{2018}
\acmYear{2018}
\acmDOI{10.1145/1122445.1122456}


%%
%% These commands are for a JOURNAL article.
\acmJournal{JACM}
\acmVolume{37}
\acmNumber{4}
\acmArticle{111}
\acmMonth{8}

%%
%% Submission ID.
%% Use this when submitting an article to a sponsored event. You'll
%% receive a unique submission ID from the organizers
%% of the event, and this ID should be used as the parameter to this command.
%%\acmSubmissionID{123-A56-BU3}

%%
%% The majority of ACM publications use numbered citations and
%% references.  The command \citestyle{authoryear} switches to the
%% "author year" style.
%%
%% If you are preparing content for an event
%% sponsored by ACM SIGGRAPH, you must use the "author year" style of
%% citations and references.
%% Uncommenting
%% the next command will enable that style.
%%\citestyle{acmauthoryear}

%%
%% end of the preamble, start of the body of the document source.
\begin{document}

%%
%% The "title" command has an optional parameter,
%% allowing the author to define a "short title" to be used in page headers.
\title{Type Inference for Featherweight Generic Java}

%%
%% The "author" command and its associated commands are used to define
%% the authors and their affiliations.
%% Of note is the shared affiliation of the first two authors, and the
%% "authornote" and "authornotemark" commands
%% used to denote shared contribution to the research.
\author{Andreas Stadelmeier}
\affiliation{%
  \institution{DHBW Stuttgart, Campus Horb}
  \streetaddress{Tannenweg 4}
  \city{Tübingen}
  \country{Germany}}
\email{a.stadelmeier@hb.dhbw-stuttgart.de}


%%
%% By default, the full list of authors will be used in the page
%% headers. Often, this list is too long, and will overlap
%% other information printed in the page headers. This command allows
%% the author to define a more concise list
%% of authors' names for this purpose.
\renewcommand{\shortauthors}{Stadelmeier, Plümicke, Thiemann}

%%
%% The abstract is a short summary of the work to be presented in the
%% article.
\begin{abstract}
  Type Inference for Featherweight Generic Java
\end{abstract}

%%
%% The code below is generated by the tool at http://dl.acm.org/ccs.cfm.
%% Please copy and paste the code instead of the example below.
%%
\begin{CCSXML}
<ccs2012>
 <concept>
  <concept_id>10010520.10010553.10010562</concept_id>
  <concept_desc>Computer systems organization~Embedded systems</concept_desc>
  <concept_significance>500</concept_significance>
 </concept>
 <concept>
  <concept_id>10010520.10010575.10010755</concept_id>
  <concept_desc>Computer systems organization~Redundancy</concept_desc>
  <concept_significance>300</concept_significance>
 </concept>
 <concept>
  <concept_id>10010520.10010553.10010554</concept_id>
  <concept_desc>Computer systems organization~Robotics</concept_desc>
  <concept_significance>100</concept_significance>
 </concept>
 <concept>
  <concept_id>10003033.10003083.10003095</concept_id>
  <concept_desc>Networks~Network reliability</concept_desc>
  <concept_significance>100</concept_significance>
 </concept>
</ccs2012>
\end{CCSXML}

\ccsdesc[500]{Computer systems organization~Embedded systems}
\ccsdesc[300]{Computer systems organization~Redundancy}
\ccsdesc{Computer systems organization~Robotics}
\ccsdesc[100]{Networks~Network reliability}

%%
%% Keywords. The author(s) should pick words that accurately describe
%% the work being presented. Separate the keywords with commas.
\keywords{type inference, Java, compiler}


%%
%% This command processes the author and affiliation and title
%% information and builds the first part of the formatted document.
\maketitle

\section{Introduction}
\label{sec:introduction}

Java is one of the most important programming languages. In 2019, Java
was the second most popular language according to a study
based on GitHub
data.\footnote{\url{https://www.businessinsider.de/international/the-10-most-popular-programming-languages-according-to-github-2018-10/}} Estimates
for the number of Java programmers range between 7.6 and 9 million.\footnote{\url{https://www.zdnet.com/article/programming-languages-python-developers-now-outnumber-java-ones/},
\url{http://infomory.com/numbers/number-of-java-developers/}} Java
has been around since 1995 and progressed through 16 versions.

Swarms of programmers have taken their first steps in Java. Many more
have been introduced to object-oriented programming through Java, as
it is among the first mainstream languages supporting
object-orientation. Java is a class-based language with static single inheritance among
classes, hence it has nominal types with a specified subtyping
hierarchy. Besides classes there are interfaces to characterize common 
traits independent of the inheritance hierarchy. Since version J2SE~5.0,
the Java language supports F-bounded polymorphism in the form of generics.

Java is generally explicitly typed with some amendments introduced in
recent versions. That is, 
variables, fields, method parameters, and method returns must be
adorned with their type. Figure~\ref{fig:intro-example-generic-fj}
contains a simple example with generics.
% from Featherweight Java \cite{DBLP:journals/toplas/IgarashiPW01} 
\begin{figure}[tp]
%   \begin{subfigure}[t]{0.55\textwidth}
% \begin{lstlisting}
% class Pair {
%   Object fst;
%   Object snd;
%   Pair(Object fst, Object snd) {
%    this.fst=fst; 
%    this.snd=snd;
%   }
%   Pair setfst(Object fst) {
%     return new Pair(fst, this.snd);
%   }
% }
% \end{lstlisting}
%   \end{subfigure}
  \begin{subfigure}[t]{0.49\linewidth}
\begin{lstlisting}[style=fgj]
class Pair<X,Y> {
  X fst;
  Y snd;
  Pair<X,Y>(X fst, Y snd) {
    this.fst=fst;
    this.snd=snd;
  }
  Pair<X,Y> setfst(X fst) {
    return new Pair(fst, this.snd);
  }
  Pair<Y,X> swap() {
    return new Pair(this.snd, this.fst);
  }
}  
\end{lstlisting}
    \caption{Featherweight Generic Java (FGJ)}
    \label{fig:intro-example-generic-fj}
  \end{subfigure}
  ~
  \begin{subfigure}[t]{0.49\linewidth}
\begin{lstlisting}[style=tfgj]
class Pair<X,Y> {
  X fst;
 Y snd;
  Pair(fst, snd) {
    this.fst=fst; 
    this.snd=snd;
  }
  setfst(fst) {
    return new Pair(fst, this.snd);
  }
  swap() {
    return new Pair(this.snd, this.fst);
  }
}  
\end{lstlisting}
    \caption{FGJ with global type inference (\TFGJ)}
    \label{fig:intro-example-generic-jtx}
  \end{subfigure}
  \caption{Example code}
  \label{fig:intro-example-code}
\end{figure}
While the overhead of explicit types look reasonable in the example,
realistic programs often contain variable initializations like
the following:\footnote{Taken from
  \url{https://stackoverflow.com/questions/4120216/map-of-maps-how-to-keep-the-inner-maps-as-maps/4120268}.} 
\begin{lstlisting}[basicstyle=\ttfamily\fontsize{8}{9.6}\selectfont,style=fgj]
  HashMap<String, HashMap<String, Object>> outerMap =
    new HashMap<String, HashMap<String, Object>>();
\end{lstlisting}

Java's \emph{local variable type inference} (since version 10\footnote{\url{https://openjdk.java.net/jeps/286}}) deals
satisfactorily with examples like the initialization of
\lstinline{outerMap}. 
In many initialization scenarios for local variables, Java infers their type
if it is obvious from the context. In the
example, we can write
\begin{lstlisting}[basicstyle=\ttfamily\fontsize{8}{9.6}\selectfont,style=fgj]
var outerMap = new HashMap<String, HashMap<String, Object>>();
\end{lstlisting}
because the constructor of the map spells out the type in
full. More specifically,``obvious'' means that the right side of the initialization is
\begin{itemize}
\item a constant of known type (e.g., a string),
\item a constructor call, or
\item a method call (the return type is known from the method
  signature).
\end{itemize}
The \lstinline{var} declaration can also be used for an iteration
variable where the type can be obtained from the elements of the
container or from the initializer.
Alternatively, if the variable is used as the method's return value,
its type can be obtained from the current method's signature.

However, there are still many places where the programmer must provide types. In
particular, an explicit type must be given for
\begin{itemize}
\item a field of a class,
\item a local variable without initializer or initialized to \lstinline{NULL},
\item a method parameter, or
\item a method return type.
\end{itemize}

In this paper, we study \emph{global type inference} for Java. Our aim
is to write code that omits most type annotations, except for class
headers and field types. Returning to the \lstinline{Pair} example, it
is sufficient to write the code in Figure~\ref{fig:intro-example-generic-jtx}
and global type inference fills in the rest so that the result is
equivalent to Figure~\ref{fig:intro-example-generic-fj}. Our
motivation to study global type inference is threefold.
\begin{itemize}
\item Programmers are relieved from writing down obvious types. 
\item Programmers may write types that leak implementation details. The
  \lstinline{outerMap} example provides a good example of this
  problem. From a software engineering
  perspective, it would be better to use a more general abstract type like
\begin{lstlisting}[basicstyle=\ttfamily\fontsize{8}{9.6}\selectfont,style=fgj]
Map<String, Map<String, Object>> outerMap = ...
\end{lstlisting}
  Global type inference finds most general types.
\item Programmers may write types that are more specific than
  necessary instead of using generic types. Here, type
  inference helps programmers to find the most general type. Suppose
  we wanted to add a static  method \texttt{eqPair} for pairs of integers to the
  \lstinline/Pair/ class.
\begin{lstlisting}[basicstyle=\ttfamily\fontsize{8}{9.6}\selectfont,style=fgj]
boolean eqPair (Pair<Integer,Integer> p) {
  return p.fst.equals(p.snd);
}
\end{lstlisting}
  With global type inference it is sufficient to write the code on the
  left of Figure~\ref{fig:equal-pair} and obtain the FGJ code with the most general type on the right.
\end{itemize}
  \begin{figure}[t]
    \begin{minipage}[t]{0.49\linewidth}
\begin{lstlisting}[style=tfgj]
eqPair (p) {
  return p.fst.equals(p.snd);
}
\end{lstlisting}
    \end{minipage}
    \begin{minipage}[t]{0.49\linewidth}
\begin{lstlisting}[style=fgj]
<T> boolean eqPair (Pair<T,T> p){
  return p.fst.equals<T>(p.snd);
}
\end{lstlisting}
    \end{minipage}
    \caption{\lstinline{eqPair} in \TFGJ and FGJ}
    \label{fig:equal-pair}
  \end{figure}
% \item Sometimes, it can be hard to find a correct typing at all.
% \todo[inline]{example?}

To make our investigation palatable, we focus on global type inference for Featherweight
Generic Java \cite{DBLP:journals/toplas/IgarashiPW01} (FGJ), a
functional Java core language with full support for generics. Our type inference algorithm
applies to FGJ programs that specify the full class header and all field types,
but omit all method signatures. 
Given this input, our algorithm
infers a set of most general method signatures (parameter types and return types).
Inferred types are generic as much as possible and may contain
recursive upper bounds.

The inferred signatures have the following round-trip property
(relative completeness). If we
start with an FGJ program that does not make use of polymorphic
recursion (see Section~\ref{sec:polym-recurs}), strip all types from
method signatures, and run the algorithm on the 
resulting stripped program, then at least one of the inferred typings is more
general than the types in the original FGJ program.






\subsection*{Contributions}
\label{sec:contributions}


We specify syntax and type system of the language \FGJGT, which drops all method type
annotations from FGJ and the typing of which rules out polymorphic
recursion. This language is amenable to polymorphic type inference and
each \FGJGT program can be completed to an FGJ program. 

We define a constraint-based algorithm that performs global type
inference for \FGJGT. This algorithm is sound and relatively complete
for FGJ programs without polymorphic recursion. Our algorithm improves
on previous attempts at type inference for Java in the literature as detailed in
Section~\ref{sec:related-work}. 

We investigate the complexity of global type inference and show its NP-completeness.

We implemented a prototype of the type inference algorithm, which we
plan to submit for artifact evaluation.

\if0
\commentary{PT what else do you want to say? Do you want to point to
  your implementation for full Java? Can we point to Andi's prototype
  (check conditions in CFP)?
  Submit as an artifact?}

Our contributions in this paper are an algorithm for global type inference for
FGJ. Therefore we redraft the typing rules of FGJ such that the programs without
type annotations could be correct and in this case polymorphic recursion is
excluded. We prove soundness and completeness of the algorithm about the rules.

The type inference algorithm is reduced to a constraint solving type unification
algorithm. We improved our type unification algorithm such that constraints
of the form $a \lessdot ty$, where $a$ is type variable and $ty$ is is a
non-type variable type,
are not resolved rather converted to bounded type parameters \texttt{a extends
  ty}. This implicates an enormes reduction of solutions of the type
unification algorithm without restricting the generality of typings of
FGJ-programs.

We show that Global type inference for FGJ is NP complete.

Finally we have done an implementation of  global type inference for  a reduced
set of full Java.
\fi

%%% Local Variables:
%%% mode: latex
%%% TeX-master: "TIforGFJ"
%%% End:


\section{Motivation}
\label{sec:motivation}

Examples, examples, examples from simple to more advanced showing off
the (difficult) features of inference.

\section{Preliminaries}
\label{sec:preliminaries}

\subsection{Featherweight Java Typing Rules}
The input for our type inference algorithm is based on Generic Featherweight Java (GFJ).
GFJ is defined by syntax and typing rules.
We already changed the syntax to allow typeless GFJ programs as input for our algorithm.
Additionally we alter the typing rules slightly, which is presented in this chapter.

%Our type inference algorithm takes typeless GFJ classes as input.
%The generated output is correct GFJ, although we have to alter some rules.
%This chapter defines the typing rules for our version of GFJ,
%which our type inference algorithm is able to process.

Most of them stay the same as in the original GFJ language,
except from the following changes:
\begin{itemize}
\item We remove the \texttt{MT-CLASS}, \texttt{D-CAST}, \texttt{U-CAST}, \texttt{S-CAST} rules
\item The \texttt{GT-METHOD} rule is changed
\item Overriding of methods is removed for our typeless GFJ version. Therefore also the rule \texttt{MT-SUPER} is removed.
\item The \texttt{GT-INVK} rule is changed to support overloading
\end{itemize}

\fbox{
\begin{minipage}{\textwidth}
  \textbf{Subtyping:}\\
\begin{tabular}{l l}
%  $
%  \ddfrac{\texttt{class}\ \exptype{C}{\ol{X} \triangleleft \ol{N}} \triangleleft N \{ \ol{S}\ \ol{f};\ K \ \ol{M} \}
%  \quad \quad m \in \ol{M}}
%  {\mathit{mtype}(m, \exptype{C}{\ol{Z}}) = \mathit{mtype}(m, [\ol{T}/\ol{X}]N)}
%  $
%  & MT-SUPER \\
%& \\

$
\triangle \vdash T <: T
$
&   S-REFL \\

& \\
$\ddfrac{
    \triangle \vdash S <: T \quad \quad \triangle \vdash T <: U
}{
    \triangle \vdash S <: U
}$ & S-TRANS \\

& \\

$
\triangle \vdash X <: \triangle(X)
$ & S-VAR \\
& \\
$\ddfrac{
  \texttt{class}\ \exptype{C}{\ol{X} \triangleleft \ol{N}} \triangleleft N \set{ \ldots }
}{
  \triangle \vdash \exptype{C}{\ol{T}} <: [\ol{T}/\ol{X}]N
}$ & S-CLASS 
\end{tabular}
\end{minipage}
}

\fbox{
\begin{minipage}{\textwidth}
  \textbf{Well-formed types:}\\
\begin{tabular}{l l}
$\triangle \vdash \texttt{Object}\ \text{ok}
$ & WF-OBJECT\\

& \\
$\ddfrac{
    X \in \textit{dom}(\triangle)
}{
    \triangle \vdash X \ \text{ok}
}
$ & WF-VAR \\
& \\
$\ddfrac{\begin{array}{c}
\texttt{class}\ \exptype{C}{\ol{X} \triangleleft \ol{N}} \triangleleft N \{ \ldots \} \\
\triangle \vdash \ol{T} \ \text{ok} \quad \quad \triangle \vdash \ol{T} <: [\ol{T}/\ol{X}]\ol{N}
\end{array}
}{
\triangle \vdash \exptype{C}{\ol{T}} \ \text{ok}
}
$ & WF-CLASS
\end{tabular}
\end{minipage}
}


\fbox{
\begin{minipage}{\textwidth}
\textbf{Expression Typing:}\\
\begin{tabular}{l l}
$
\triangle ; \Gamma \vdash x : \Gamma(x)
$ & GT-VAR \\
& \\

$\ddfrac{\Gamma \vdash e_0:T_0 \quad \quad \mathit{fields}(\mathit{bound}_\triangle(T_0)) = \overline{T} \ \overline{f}}
{\Gamma \vdash e_0.\mathtt{f}_i : T_i}
$ & GT-FIELD \\
& \\
$ \ddfrac{\triangle \vdash N \ \texttt{ok} \quad \quad \textit{fields}(N) = \ol{T}\ \ol{f} \quad \quad
  \triangle; \Gamma \vdash \ol{e} : \ol{S} \quad \quad \triangle \vdash \ol{S} <: \ol{T}
}{
  \triangle; \Gamma \vdash \texttt{new N}(\ol{e}): N
}$ & GT-NEW \\

& \\

$\ddfrac{\begin{array}{c}
\mathit{mtype}(m, \mathit{bound}_\triangle (T_0)) = (\exptype{}{\ol{Y} \triangleleft \ol{P}} \ol{U} \to U)\\
\triangle; \Gamma \vdash e_0 : T_0 \quad \quad
\triangle \vdash \ol{V} \ \texttt{OK} \quad \quad
\triangle \vdash \ol{V} <: [\ol{V}/\ol{Y}]\ol{P} \\ %\quad \quad
\triangle; \Gamma \vdash \ol{e} : \ol{S} \quad \quad
\triangle \vdash \ol{S} <: [\ol{V}/\ol{Y}]\ol{U}
\end{array}}
{\triangle; \Gamma \vdash \mathtt{e_0.\exptype{m}{\ol{V}}(\overline{e}) : [\ol{V}/\ol{Y}]U }}
$ & GT-INVK
\end{tabular}
\end{minipage}
}



\fbox{
\begin{minipage}{\textwidth}
\begin{tabular}{l l}

  \textbf{Method Typing:} 
  & \\
  $\ddfrac{\begin{array}{c}
  \texttt{class}\ \exptype{C}{\ol{X} \triangleleft \ol{N}} \triangleleft N \{ \ldots\ \ol{M}\ \ldots\} \\
  \textit{mtype}(m, \exptype{C}{\ol{X}}) = \ol{T_m} \to T_m \textrm{ for } m \in \ol{M}\\
  \triangle \vdash \ol{X} <: \ol{N}  \quad \quad 
  \triangle \vdash \ol{T}, T \ \texttt{ok} \\
  \triangle ; \ol{x}:\ol{T_\mathit{meth}},\ this : \exptype{C}{\ol{X}} \vdash e_0 : S \quad \quad
  \triangle \vdash S <: T_\mathit{meth} \\
  \end{array}} {
  %{\exptype{}{\ol{Y} \triangleleft \ol{P}}\ T \ m(\ol{T}\ \ol{x}) \{
  %\texttt{return} \ e_0; \} \ \texttt{OK IN}\ \exptype{C}{\ol{X} \triangleleft
  %\ol{N}}}
   \exptype{}{\ol{Y}} T_\mathit{meth}\ \texttt{meth}(\ol{T_\mathit{meth}}\ \ol{\mathtt{x}}) \{\texttt{return}\ \mathtt{e}_0;\}
  \texttt{ OK in }\exptype{C}{\ol{X} \triangleleft \ol{N}} 
  }$ & GT-METHOD\\
  
  & \\

\textbf{Class Typing:} & \\
& \\
  %GT-CLASS: - This rule is modified by us
  $\ddfrac{
    \begin{array}{c}
      \ol{X} <: \ol{N} \vdash \ol{N}, N, \ol{T}\ \texttt{ok}
      \quad\quad fields(\mathtt{N}) = \ol{\mathtt{U}} \ \ol{\mathtt{g}}\\
      \exptype{}{\ol{Y}} T_\mathit{m}\ \texttt{m}(\ol{T_\mathit{m}}\ \ol{\mathtt{x}}) \{\texttt{return}\ \mathtt{e}_0;\}
  \texttt{ OK in }\exptype{C}{\ol{X} \triangleleft \ol{N}}  \textrm{ for all } m
      \in \ol{\mathtt{M}}\\
    K = C(\overline{D} \ \overline{g}, \overline{C} \ \overline{f}) \{ \texttt{super}(\overline{g}); \ \texttt{this}.\overline{f}=\overline{f}; \} \\
    %\quad \quad \overline{M} \ \texttt{OK IN C} \\
  \end{array}
    }
  {\texttt{class C extends D}\{ \overline{C} \ \overline{f}; \ K \ \overline{M} \} \ \texttt{OK}}
  $ & GT-CLASS
\end{tabular}
\end{minipage}
}
\commentarymargin{
\begin{tabular}{l l}

  \textbf{Method Typing:} & \\
  & \\
  $\ddfrac{\begin{array}{c}
  \texttt{class}\ \exptype{C}{\ol{X} \triangleleft \ol{N}} \triangleleft N \{ \ldots\ \ol{M}\ \ldots\} \\
  \texttt{meth}(\ol{x}) \{\texttt{return}\ e_0;\} \in \ol{M} \\
  \Gamma_C = \set{ (\textit{mtype}(m, \exptype{C}{\ol{X}}) = \ol{T_m} \to T_m) \ |\ m \in \ol{M}}\\
  (\textit{mtype}(\texttt{meth}, \exptype{C}{\ol{X}}) = \ol{T} \to T) \in \Gamma_C\\
  \triangle \vdash \ol{X} <: \ol{N}  \quad \quad 
  \triangle \vdash \ol{T}, T \ \texttt{ok} \\
  \triangle ; \Gamma_C,\ \ol{x}:\ol{T},\ this : \exptype{C}{\ol{X}} \vdash e_0 : S \quad \quad
  \triangle \vdash S <: T \\
  \end{array}} {
  %{\exptype{}{\ol{Y} \triangleleft \ol{P}}\ T \ m(\ol{T}\ \ol{x}) \{ \texttt{return} \ e_0; \} \ \texttt{OK IN}\ \exptype{C}{\ol{X} \triangleleft \ol{N}}}
  \Gamma_C \vdash \textit{mtype}(m, \exptype{C}{\ol{Z}}) = [\ol{Z} / \ol{X}](\exptype{}{\ol{Y}} \ol{T} \to T)
  }$ & GT-METHOD
\end{tabular}
}



\fbox{
\begin{minipage}{\textwidth}
\begin{tabular}{l l}

  \textbf{Method type lookup:} & \\
  & \\
  $\ddfrac{\begin{array}{c}
  \texttt{class}\ \exptype{C}{\ol{X} \triangleleft \ol{N}}\triangleleft
             \ol{N}\{ \overline{C} \ \overline{f}; \ K \ \overline{M} \} \ \mathtt{OK}\\
  \exptype{}{\ol{Y}} T_\mathit{m}\ \texttt{m}(\ol{T_\mathit{m}}\ \ol{\mathtt{x}}) \{\texttt{return}\ \mathtt{e}_0;\}
  \texttt{ OK in }\exptype{C}{\ol{X} \triangleleft \ol{N}}
  \end{array}} {
  %{\exptype{}{\ol{Y} \triangleleft \ol{P}}\ T \ m(\ol{T}\ \ol{x}) \{ \texttt{return} \ e_0; \} \ \texttt{OK IN}\ \exptype{C}{\ol{X} \triangleleft \ol{N}}}
  \textit{mtype}(m, \exptype{C}{\ol{Z}}) = [\ol{Z} / \ol{X}](\exptype{}{\ol{Y}} \ol{T} \to T)
  }$ & MT-CLASS
\end{tabular}
\end{minipage}
}


\commentarymargin{In the typing system of GFJ \textit{mtype} is a global function which gives the type for every method in every class.
It is defined by the \texttt{MT-CLASS} rule.
In our type system it is also a global function but it is defined by the \texttt{GT-METHOD} rule.
The difference to GFJ is that methods are typechecked one after another.
There has to be one method which initially does not call any other method in the input program,
except the ones in the same class.
%Also the resulting \textit{mtype} from the \texttt{GT-METHOD} rule is bound to the local method type context $\Gamma_C$.

The rule \texttt{GT-CLASS} only is valid, if every method in a class $C$ could satisfy the \texttt{GT-METHOD} rule with the same $\Gamma_C$.}


\medskip
In the typing system of GFJ \textit{mtype} is a global function which gives the type for every method in every class.
In our type system it is also a global function but it is differed between
methods declared in the actual class and methods from other classes.

The main difference between the type system of GFJ and our type system is that
in the \texttt{MT-CLASS} rule the correspondig class has to be proved as \texttt{OK}
by the \texttt{GT-CLASS} rule which means that for all methods of the class a type has to
be assumed and proved as correct be the \texttt{GT-METHOD} rule.
In this rule
for all methods in the actual class a type is assumed by the
declaration of the \texttt{mtype} function.
These assumptions have to be proved as correct. Then the assumed type is
\texttt{OK} in the correspondig class. 



%MT-CLASS: - This rule is not used by us
%\begin{align*}
%\ddfrac{\begin{array}{c}
%\texttt{class} \ \exptype{C}{\ol{X} \triangleleft \ol{N}} \ \{ \ol{S} \ \ol{f}; \ K \ol{M} \}\\
%\exptype{}{\ol{Y} \triangleleft \ol{P}} U \ m(\ol{U} \ \ol{x})\{  \texttt{return} \ e; \} \in \ol{M}
%\end{array}}
%{\mathit{mtype}(m, \exptype{C}{\ol{Z}}) = [\ol{T}/\ol{X}](\exptype{}{\ol{Y} \triangleleft \ol{P}} \ol{U} \to U)}
%\end{align*}

%GT-METHOD:
%\begin{align*}
%\ddfrac{\begin{array}{c}
%\triangle \vdash \ol{X} <: \ol{N}, \ol{Y} <: \ol{P} \quad \quad 
%\triangle \vdash \ol{T}, T, \ol{P} \ \texttt{ok} \\
%\triangle ; \ol{x}:\ol{T}, this : \exptype{C}{\ol{X}} \vdash e_0 : S \quad \quad
%\triangle \vdash S <: T \\
%\texttt{class}\ \exptype{C}{\ol{X} \triangleleft \ol{N}} \triangleleft N \{ \ldots \} \quad \quad
%\textit{override}(m, N, \exptype{}{\ol{Y} \triangleleft \ol{P}} \ol{T} \to T)
%\end{array}}
%{\exptype{}{\ol{Y} \triangleleft \ol{P}}\ T \ m(\ol{T}\ \ol{x}) \{ \texttt{return} \ e_0; \} \ \texttt{OK IN}\ \exptype{C}{\ol{X} \triangleleft \ol{N}}}
%\end{align*}


\subsection{Input assumptions}
% No overloaded methods
% No Or-Constraints
The input is a GFJ program lacking the type assignments for method parameters and method return types.


The Typeless Generic Featherweight Java (TGFJ) syntax is different in that from normal Generic Featherweight Java (GFJ) that it is possible
to omit the type annotations for methods.%, except the ones for casts and \texttt{new} calls.
We declare the syntax for TGFJ as follows:

\begin{align*}
  T ::=& X \, | \, N \\
  N ::=& \exptype{C}{\ol{T}}\\
  L ::=& \mathtt{class } \ \exptype{C}{\ol{X} \triangleleft \ol{N}} \ \triangleleft \ N \{ \overline{T} \ \overline{f}; \, K \, \overline{M} \} \\
  K ::=& C(\overline{T} \ \overline{f})\{\mathtt{super}(\overline{f}); \ \mathtt{this}.\overline{f}=\overline{f};\} \\
  %M ::=& \exptype{}{\ol{X} \triangleleft \ol{X}}\ T \ \mathtt{m}(\overline{T} \, \overline{x})\{ \mathtt{ return }\ e; \} \\
  M ::=& \mathtt{m}(\overline{x})\{ \mathtt{ return }\ e; \} \\
  e ::=& \mathtt{this} \, | \, x \, | \, e.f \, | \, e.\mathtt{m}(\overline{e}) \, | \, \mathtt{new }\ C(\overline{e})
  %M ::=& T \ \mathtt{m}(\overline{T} \, \overline{x})\{ \mathtt{ return }\ e; \} \\
  %e ::=& \mathtt{this} \, | \, x \, | \, e.f \, | \, e.\exptype{\mathtt{m}}{\ol{T}}(\overline{e}) \, | \, \mathtt{new }\ C(\overline{e}) \, | \, (C) e \\
\end{align*}

All type annotations in our TGFJ language can be omitted ($T = \epsilon$).
The only exception are fields which must be given a concrete type.

Another difference to the syntax of GFJ is that we added the special variable \texttt{this} to the syntax.
FJ treats \texttt{this} as a normal variable
but our algorithm treats it as a special variable which always has a predetermined type;
the type of the class it is used in.

We assume every method name is only used once in the input program.

Type inference for polymorphic recursion is undecidable.
Therefore we have to alter the GFJ typing rules to exclude polymorphic recursion in method calls:
\begin{enumerate}
    \item The \texttt{MT-CLASS} rule is removed.
    \item We change the \texttt{GT-METHOD} rule:
GT-METHOD:
\begin{align*}
\ddfrac{\begin{array}{c}
\triangle \vdash \ol{X} <: \ol{N}, \ol{Y} <: \ol{P}  \quad \quad 
\triangle \vdash \ol{T}, T \ \texttt{ok} \\
\triangle ; \ol{x}:\ol{T}, this : \exptype{C}{\ol{X}} \vdash e_0 : S \quad \quad
\triangle \vdash S <: T \\
\texttt{class}\ \exptype{C}{\ol{X} \triangleleft \ol{N}} \triangleleft N \{ \ldots \} \quad \quad
%\textit{override}(m, N, \exptype{}{\ol{Y} \triangleleft \ol{P}} \ol{T} \to T)
\ol{T} \to T  \in \textit{mtype}(m, \exptype{C}{\ol{X}}) \\
%\textit{override}(m, N,)
\end{array}}{
%{\exptype{}{\ol{Y} \triangleleft \ol{P}}\ T \ m(\ol{T}\ \ol{x}) \{ \texttt{return} \ e_0; \} \ \texttt{OK IN}\ \exptype{C}{\ol{X} \triangleleft \ol{N}}}
\textit{mtype}(m, \exptype{C}{\ol{Z}}) = [\ol{Z} / \ol{X}](\exptype{}{\ol{Y}} \ol{T} \to T)
}
\end{align*}
\end{enumerate}

\subsection{Principal Type and Intersection Types}
Generic Featherweight Java has no principal typing.
We can show this easily with an example.
We try to find the principal type for the method \texttt{method1}.
\begin{lstlisting}
class Global{
  method1(a){
    a.add(this);
    return a.get();
  }
}
class List<A> {
  add(A item){...}
  A get() ...
}
\end{lstlisting}
In \texttt{method1} neither the return type nor the type for the parameter \texttt{a} are specified.
The return type of the method depends on the type of \texttt{a}.
If we set in the type \texttt{List<Object>} here, then \texttt{method1} would return \texttt{Object}.
The type \texttt{List<Global>} would also be correct.
Then the return type of the method can also be the type \texttt{Global}.

The principal type would either be an intersection type or the method \texttt{method1} has to be overloaded.
GFJ neither supports intersection types nor overloading.
Therefore we cannot set in the principal type and have to stick with one of the possible solutions,
for example\\
\texttt{List<Global> method1(List<Global> a)}.

We therefore infer intersection types for methods.
The same method body can now have multiple return and parameter type combinations.

\begin{align*}
\textit{mtype}(\texttt{method}, \exptype{C}{\ol{X}}) = \set{ \exptype{List}{Object} \to \texttt{Object}
\ || \ \exptype{List}{String} \to \texttt{String}} 
\end{align*}


\section{Type inference algorithm}
In this chapter we present our type inference algorithm.
The algorithm is split into following parts:

\begin{enumerate}
\item Create assumptions and subtype relation
\item Constraint generation with \textbf{GFJTYPE}
\item Unification of those constraints
\item Set in principal type solution
\end{enumerate}

The Unify algorithm returns a set of possible type solutions.
This means that there are possibly multiple type solutions for each method.
The last step has to choose the principal type out of those possibilities.

\subsection{Process multiple classes}
The algorithm processes only one class at a time.
Only the first step creating the type assumptions is able to consider other classes as well.

Nevertheless we allow the input to consist out of multiple classes.
But in that case there are some additional requirements for the input.
%TODO: these requirements can also be in "Preliminaries"

We assume that the algorithm are given the input classes in the correct order $C_1, \ldots C_2$.
Hereby there must exist a correct typisation for the class $C_1$ when existing on its own.
When there are two classes which depend on one another this would be invalid input for our algorithm.
\begin{lstlisting}
class C1 extends Object {
  C1(){ super(); }
  m1(){ return new C2(); }
}
class C2 extends C1{
  C2(){ super(); }
}
\end{lstlisting}
Here the class \texttt{C2} extends \texttt{C1}, but the class \texttt{C1} cannot be processed first because it already uses \texttt{C2}.
Without the method \texttt{m1} the class \texttt{C1} would not reference any other class and stand on its own.
\begin{lstlisting}
  class C1 extends Object {
    C1(){ super(); }
  }
  class C2 extends C1{
    C2(){ super(); }
    m1(){ return new C2(); }
  }
  \end{lstlisting}
By moving \texttt{m1} to class \texttt{C2} we create a valid input for our algorithm.

When processing the second class the first step of the algorithm (create assumptions) also considers the first and already compiled class.
A class $C_n$ therefore can use and reference all the classes before it ($C_1, \ldots, C_{n-1}$).

\subsection{Generate Assumptions}
% Every empty Type T in the input is assigned a type variable.
% Assumptions saves every field, method and the class subtype relation

%Generate subtype relationships:

Generating assumptions consists of two parts.
At first we add type variables to the untyped class.
The second part generates the assumption set.
This is the same algorithm for the already typed classes as for the 
new untyped class, which is now equipped with type variables.

\begin{enumerate}
\item Every missing type in the input class gets assigned a fresh type variable.
For methods:
\begin{align*}
  \ddfrac{
  m(\ol{x}) \{ \ldots \} \quad \quad A \cup \ol{A} \ \text{are fresh type variables}
  }{
  A m(\ol{A}\ \ol{x}) \{ \ldots \}
  }
  \end{align*}
  For fields:
\begin{align*}
  \ddfrac{
  \texttt{class}\ \exptype{C}{\ol{X}} \{ \ol{f}; \quad \ldots \} \quad \quad \ol{F} \ \text{are fresh type variables}
  }{
    \texttt{class}\ \exptype{C}{\ol{X}} \{ \ol{F} \ \ol{f}; \quad \ldots \}
  }
\end{align*}
\item We define the two functions $\textit{ftype}_\textit{Ass}$ and $\textit{mtype}_\textit{Ass}$.
Both functions return a set of all types for a method \texttt{m} or a field \texttt{f}.
This is due to the fact that there can be multiple methods and fields with the same name.
\begin{align*}
  %TODO: fresh type variables for generic variables:
  \ddfrac{
    class\ \exptype{C}{\ol{X} \triangleleft \ol{N}}\ \{\ \ol{N}\ \ol{f};\ K\ \ol{M}\ \} \quad \quad
    \exptype{}{\ol{Y}}\ U\ \texttt{m}(\ol{U}\ \ol{x}) \{ \ldots \} \in \ol{M}
  }{
    \textit{mtype}_\textit{Ass}(\texttt{m}, \exptype{C}{\ol{X} \triangleleft \ol{N}}) =  \set{\exptype{}{\ol{Y}} (\ol{U} \to U )}
  }
\end{align*}
\begin{align*}
  \ddfrac{
    class\ \exptype{C}{\ol{X} \triangleleft \ol{N}}\ \{\ \ol{T}\ \ol{f};\ K\ \ol{M}\ \} \quad \quad
    T\ \texttt{f} \in \ol{f}
  }{
    \textit{ftype}_\textit{Ass}(\texttt{f}) = \exptype{C}{\ol{X} \triangleleft \ol{N}} \to T
  }
\end{align*}
\item If the input for the type inference algorithm consists out of multiple classes we compile them one by one.
Additionally we add the types of the already compiled classes to the assumption set.
Therefore it is possible to have intersection types already in the assumptions.
\end{enumerate}


\subsection{GFJTYPE}
The \textbf{GFJTYPE} algorithm produces two kinds of constraints.
\begin{description}
\item[Constraint] A constraint consists of two types or type variables and an operator.
The operator can either be a $\doteq$ (same type) or $\lessdot$ (subtype).
Example: $(a \lessdot \mathtt{Object})$, means that the type variable $a$ should be a subtype of \texttt{Object}.
\item[OrConstraint] An OrConstraint consists out of multiple constraint sets.
For example $\textbf{OrConstraint}(\{ \ \{ (a \lessdot b), (a \leq \mathtt{Object}) \} \ , \ \{ (a \lessdot b)\} \ \})$
is an Or-Constraint consisting of two constraint sets.
\end{description}

The algorithm \textbf{GFJTYPE} is given as follows:

\textbf{FJTYPE}:
$\texttt{TypeAssumptions} \times
\texttt{Class} \rightarrow \texttt{Constraints}\\
 \begin{array}{@{}l@{}l@{}l}
 \textbf{FJT}&\textbf{Y} & \textbf{PE}(Ass, \mathtt{class } \ C \ \mathtt{ extends } \ D \{ \overline{T} \ \overline{f}; \, K \, \overline{M} \}) =\\
& \multicolumn{2}{@{}l@{}}{ \{ \ \textbf{TYPEMethod}(\textit{Ass} \cup \{ \mathtt{this} : C \}, m_i) \quad | \quad m_i \in \overline{M} \ \} }\\ 
\end{array}$

The \textbf{FJTYPE} function gets called for every class in the input.
This function accumulates all the constraints generated from calling the
\textbf{TYPEMethod} function for each method declared in the given class.

$\textbf{TYPEMethod}:\texttt{TypeAssumptions} \times
\texttt{Method} \rightarrow \texttt{Constraints}\\
\begin{array}{@{}l@{}l@{}l}
\textbf{TY}& \textbf{PE} & \textbf{Method} (Ass, T_r \ \mathtt{m}(\overline{T} \, \overline{x})\{ \mathtt{ return }\ e; \}) =\\
& \textbf{let}
& Ass_m = Ass \cup \{ \overline{T} : \overline{x} \}\\
& & \ul{(e:rty, ConS)} = \textbf{TYPEExpr}(Ass_m, e)\\
& \mathbf{in}
& (ConS \cup (rty \lessdot T_r))\\
\end{array}
$

The \textbf{TYPEMethod} function for methods just calls the \textbf{TYPEExpr} function with the
return expression. It is significant to note that it adds the assumptions for the method parameters to the global assumptions before passing them to \textbf{TYPEExpr}.
%and the global assumptions plus the assumptions for the method parameters.

\smallskip

In the following we define the \textbf{TYPEExpr} function for every possible expression:

\smallskip

$\textbf{TYPEExpr}:\texttt{TypeAssumptions} \times
\texttt{Expression} \rightarrow \texttt{Type} \times \texttt{Constraints}\\
\begin{array}{@{}l@{}l}
\textbf{TY} \textbf{PE} & \textbf{Expr} (Ass, \mathtt{this}) = (t , \{\})\\
& \textbf{with } (\mathtt{this} : t) \in Ass 
\end{array}
$
\smallskip
$\begin{array}{@{}l@{}l}
\textbf{TY} \textbf{PE} & \textbf{Expr} (Ass,x) = (t , \{\})\\
& \textbf{with } (x : t) \in Ass 
\end{array}
$

\smallskip

$\begin{array}{@{}l@{}l@{}l}
\textbf{TY}& \textbf{PE} & \textbf{Expr} (Ass, e.f ) = \\
& \textbf{let} % \\
% &
& (rty, ConS) = \textbf{TYPEExpr}(Ass, e),\\
& & \textbf{fresh} = \text{a mapping from each variable in}\ \ol{X} \ \text{to a fresh type variable},\\
& & Ass_{f} = \textit{ftype}_{Ass}(f) = \exptype{C}{\ol{X} \triangleleft \ol{N}} \to T \\
& & Cons_{f} = \{\ rty \doteq \exptype{C}{\textbf{fresh}(\ol{X})}, a \doteq \textbf{fresh}(T)\},\\
%& & OrCons = \{ \{ rty \doteq cl, a \doteq t_f \} \ | \ cl.f : t_f \in Ass \},\\
& \mathbf{in}% \\
% &
& (a, ConS \cup Cons_{f})\\
& & \mathit{where\ } a \mathit{\ is\ a\ fresh\
  type\ variable}\\ 
\end{array}
$

\smallskip

$\begin{array}{@{}l@{}l@{}l}
\textbf{TY}& \textbf{PE} & \textbf{Expr} (Ass, e_r.\mathtt{m}(\overline{e}) ) = \\
& \textbf{let} % \\
% &
& (rty, ConS) = \textbf{TYPEExpr}(Ass, e_r),\\
& & \forall e_i \in \overline{e} : (pt_i, ConS_i) = \textbf{TYPEExpr}(Ass, e_i)  ,\\
& & \textbf{fresh} = \text{a mapping from each variable in}\ \ol{X} \ \text{to a fresh type variable},\\
& & Ass_{m} =  \set{\exptype{}{\ol{Y}} (\ol{T} \to T) \,|\, \exptype{}{\ol{Y}} (\ol{T} \to T) \in \textit{mtype}(m, \exptype{C}{\ol{X}})} \\
& & \begin{array}{@{}l@{}l}
        Cons_{m} = \{\ & rty \doteq \exptype{C}{\textbf{fresh}(\ol{X})}, a \doteq \textbf{fresh}(T), \bigcup_{T_i \in \overline{T}} (pt_i \lessdot \textbf{fresh}(T_i))\\
                    & |\, \exptype{}{\ol{Y}} (\ol{T} \to T) \in \textit{mtype}(m, \exptype{C}{\ol{X}}) \} %, \textbf{fresh}(\ol{X}) \lessdot \textbf{fresh}(\ol{N})\},\\
    \end{array}\\
& \mathbf{in}% \\
% &
& (a, ConS \cup \bigcup_i ConS_i \cup \textbf{OrConstraint}(Cons_{m}))\\
& & \text{where\ } a \text{\ is\ a\ fresh\
  type\ variable}\\ 
\end{array}
$

\smallskip

$\begin{array}{@{}l@{}l@{}l}
\textbf{TY}& \textbf{PE} & \textbf{Expr} (Ass, \mathtt{new }\ N(\overline{e}) ) = \\
& \textbf{let} % \\
& \forall e_i \in \overline{e} : (pt_i, ConS_i) = \textbf{TYPEExpr}(Ass, e_i)  ,\\
& & Cons = \{ \bigcup_{T_i \in \overline{T}} (pt_i \lessdot T_i) \ | \ \mathtt{constructor }\ \exptype{C}{\ol{X}}(\overline{T} \overline{x}) \in Ass \},\\
& \mathbf{in}% \\
% &
& (C, ConS \cup \bigcup_i ConS_i)\\
\end{array}
$

\subsubsection{Completeness}
Theorem: The Unify algorithm is complete
Theorem: \textbf{GFJTYPE} generates the principal type
Proof: The \textbf{Unify} algorithm is complete, so the principal type is included in the solution set.
We only have to choose the principal type out of those solutions.

All types that are possible under the GFJ typing rules, plus our additional assumptions,
also comply with the generated constraints.

We match every generated constraint with the respective type rule to show completeness of our \textbf{GFJTYPE} algorithm.
This shows that none of the generated constraints remove a type which otherwise would be possible under the GFJ typing rules.
The constraints are generated on expression statements.
We now compare the constraints for each expression with the appropriate type rule from GFJ:
\begin{description}
  \item [this]
  has always the type of the surrounding class and generates no constraints.
  \item [Local var]
  No constraints are generated.
  \item[Method invocation]
By direct comparison we show that each of the generated constraints do not apply more restrictions than the \texttt{GT-INVK} rule.
The \texttt{GT-INVK} rule states the condition $\textit{mtype}(m, \textit{bound}_\triangle(T_0)) = \exptype{}{\ol{Y}}\ \ol{U} \to U$.
In our version of typeless GFJ every method name is unique
and there is only one class with that particular method.
The constraint $rty \lessdot \exptype{C}{\textbf{fresh}(\ol{X})}$ assures that the type of the expression $e_0$ contains the method \texttt{m}.

\begin{tabular}{l|l}
  \textbf{GFJ Type rule} & \textbf{Constraints} \\
  $\triangle; \Gamma \vdash e_o : T_0$ & $(rty, ConS) = \textbf{TYPEExpr}(Ass, e_r)$\\ 
  $\quad \textit{mtype}(m, \textit{bound}_\triangle(T_0)) = \exptype{}{\ol{Y}}\ \ol{U} \to U$ & $rty \lessdot \exptype{C}{\textbf{fresh}(\ol{X})}$ \\
 %$\textit{mtype}(m, \textit{bound}_\triangle(T_0)) = \ol{U} \to U$ & $rty \doteq cl$\\
 $\triangle; \Gamma \vdash \ol{e} : \ol{S}$ & $\forall e_i \in \overline{e} : (pt_i, ConS_i) = \textbf{TYPEExpr}(Ass, e_i)$\\
 $\triangle \vdash \ol{S} <: \ol{U}$ & $ \bigcup_{T_i \in \overline{T}} (pt_i \lessdot \textbf{fresh}(T_i))$\\
 $\triangle; \Gamma \vdash \mathtt{e_0.m(\overline{e}) : U }$ & $a \doteq \textbf{fresh}(T)$ \\
\end{tabular}

\textit{Note}: The \textbf{TYPEExpr} function only generates constraints which apply to our assumption.
 \item[Field access]
Mostly the same as method invocation.
Fieldnames by default are unique in the GFJ language.

 \begin{tabular}{l|l}
   \textbf{GFJ Type rule} & \textbf{Constraints} \\
   $\Gamma \vdash e_0:T_0$ & $(rty, ConS) = \textbf{TYPEExpr}(Ass, e_r)$\\ 
   $\quad \mathit{fields}(\mathit{bound}_\triangle(T_0)) = \overline{T} \ \overline{f}$ & $rty \doteq \exptype{C}{\textbf{fresh}(\ol{X})}$ \\
  %$\textit{mtype}(m, \textit{bound}_\triangle(T_0)) = \ol{U} \to U$ & $rty \doteq cl$\\
  $\triangle; \Gamma \vdash \ol{e} : \ol{S}$ & $\forall e_i \in \overline{e} : (pt_i, ConS_i) = \textbf{TYPEExpr}(Ass, e_i)$\\
  $\triangle \vdash \ol{S} <: \ol{U}$ & $ \bigcup_{T_i \in \overline{T}} (pt_i \lessdot \textbf{fresh}(T_i))$\\
  $\triangle; \Gamma \vdash \mathtt{e_0.m(\overline{e}) : U }$ & $a \doteq \textbf{fresh}(T)$ \\
 \end{tabular}
 \item[Constructor]

\begin{tabular}{l|l}
  $\triangle; \Gamma \vdash \ol{e} : \ol{S}$ & $\forall e_i \in \overline{e} : (pt_i, ConS_i) = \textbf{TYPEExpr}(Ass, e_i)$\\
  $\triangle \vdash \ol{S} <: \ol{T}$ & $\bigcup_{T_i \in \overline{T}} (pt_i \lessdot T_i)$
\end{tabular}
  
\end{description}

\section{Unify}
\label{sec:unify}
This chapter describes the \textbf{GenericUnify} algorithm
which is used to find type solutions for the constraints generated by \textbf{FGJType}.

\begin{description}
\item[input] A set of type constraints $Cons_{in}$ and a set of subtype relationships $S_\leq$
\item[output] A set of type unifiers $Uni$
or fail $Uni = \emptyset$.
%The unifiers have the form of $Uni = \{ \sigma_1, \ldots , \sigma_n \}$.
\end{description}

The unify algorithm first has to build the cartesian product of all the \textbf{OrConstraint}s and the remaining normal constraints:
\begin{align*}
\Omega &= \text{All }\mathbf{OrConstraints} \text{ in } {Cons}_{in}\\
C &= {Cons}_{in} \setminus \Omega \\
Eq_{set} &= \Omega_1 \times \ldots \times \Omega_n \times C \quad \text{for all }\Omega_i \in \Omega
\end{align*}

%The algorithm starts by setting $Eq_{set} = \set{ Cons_{in} }$.
Afterwards the following steps are repeatedly executed on $Eq_{set}$ until the algorithm termiates:
%For every $Eq \in Eq_{set}$ the following steps are applied.
%The resulting unifiers $\sigma$ from each $Eq$ merged together form the result of the \textbf{Unify} algorithm.

%$\textbf{SubUnify} :: [Constraint] \to [Unifier]$
\begin{enumerate}
\item Repeated application of the rules depicted in figure \ref{fig:fgjreduce-rules} and \ref{fig:fgjerase-rules}.
The end configuration of $Eq$ is reached if for each element
no rule is applicable.

\item
(The function $\textbf{fresh}(i)$ returns an array of $i$ fresh type variables.)

\begin{align*}
Eq_1 =& \text{Subset of pairs where both type terms are type variables}\\
Eq_2 =& Eq / Eq_1 \\
Eq_{set}\\ 
    = 
     & \begin{array}[t]{l@{\,}ll}
      \times \, (\displaystyle{\bigotimes_{(a \lessdot \exptype{C}{\ol{X}}) \in Eq'_2}}
      \{\,(a \doteq \exptype{D}{\ol{A}}) \, \cup \, (\ol{X} \doteq \ol{Y'}) \ | \ \sarray{@{}l}{
        (\exptype{D}{\ol{Z}} \olsub \exptype{C}{\ol{Y}}) \in S_\leq, \\
        \ol{A} = \textbf{fresh}(\#(\ol{Z})), \\
        \ol{Y'} = [\ol{A}/\ol{Z}]\ol{Y}
        \,\})}\\ 
      \end{array}\\
   %& \times\, 
   %   (\displaystyle{\bigotimes_{(\exptype{C}{\overline{T}} \lessdot a) \in Eq'_2}}\!\!
   %   \set{(a \doteq [\ol{T}/\ol{X}]N ) \ | \ (\exptype{C}{\overline{X}} \leq N) \in
   %     S_\leq})\\
    & \times\, 
      (\displaystyle{\bigotimes_{(\exptype{C}{\overline{T}} \lessdot a) \in Eq'_2}}\!\!
      \set{(a \doteq [\ol{T}/\ol{X}]N ) \ | \ (\exptype{C}{\overline{X} \triangleleft \ol{N}} \leq N) \in
        S_\leq})\\
    & \times\, \set{[a \doteq \theta \ | \  (a \doteq \theta) \in Eq'_2]} \times Eq_1 \\
\end{align*}

\item \label{subst-step}  Application of the following \emph{subst} rule
    %\begin{enumerate}
    %\item Apply the following subst\_eq rule
    
      $$\begin{array}[c]{lll}
        (\mathrm{subst}) &
        \begin{array}[c]{l}
          Eq'' \cup \set{a \doteq \theta}\\
          \hline
          Eq''[a \mapsto \theta] \cup \set{a \doteq \theta}
        \end{array}
        & a \textrm{ occurs in } Eq'' \textrm{ but not in } \theta 
      \end{array}$$
      
      for each $a \doteq \theta$ in each element of $Eq' \in Eq'_{set}$.

\item 
    \begin{enumerate}
    \item Foreach $Eq \in Eq_{set}$ which has changed in the last step
      start again with the first step.
    \item Build the union $Eq_{set}$ of all results of (a) and all $Eq' \in
      Eq'_{set}$ which has not changed in the last step.
    \end{enumerate}
\item
\begin{enumerate}
\item Filter all constraint sets which are in solved form:\\
$Eq_{solved} = \set{ Eq \ | \ Eq \in Eq_{set}, Eq \ \text{is in solved form}}$
\item We apply the following rule to every constraint set in $Eq_{solved}$:
\begin{align*}
\ddfrac{
  Eq \cup \set{ a \lessdot b } %There are only Type variables left at this point
}{
  Eq \cup \set{ a \doteq b }
}
\end{align*}
\item $\emph{Uni} = \set{\sigma \ | \ Eq \in Eq_{solved},\ \sigma = \set{a \mapsto \theta \ | \ (a \doteq \texttt{T}) \in Eq} }$
%\item $\emph{Uni} = \sarray{l@{\ }l}{\set{\sigma \ | & Eq'' \in Eq''_{set},
%        Eq'' \textrm{ is in solved form,}\\ 
%        & \sigma = \set{a \mapsto \theta \ | \ (a \doteq \theta) \in Eq''}
%        \\ & \quad \cup \ \set{a \mapsto \texttt{A}, b \mapsto \texttt{B} \ | \ (a \lessdot b) \in Eq \ \text{and a is an isolated type variable}}
%        }}$
\end{enumerate}
\end{enumerate}

\begin{figure}
\begin{center}
    \leavevmode
    \fbox{
    \begin{tabular}[t]{ll}
      (adapt)
      & $
      \begin{array}[c]{ll}
      \begin{array}[c]{l}
         Eq \cup \, \set{\exptype{D}{\ol{A}} \lessdot
          \exptype{C}{\ol{B}}} \\ 
        \hline
        \vspace*{-0.4cm}\\
        Eq \cup \set{\exptype{C}{[ \ol{A} / \ol{X} ]\ol{Y}}
        \doteq \exptype{C}{\ol{B}}}
      %Eq \cup \set{\theta_1 \doteq \lambda'_1 \ldo \theta_n \doteq \lambda'_n}
      \end{array}
      & (\exptype{D}{\ol{X}} \olsub \exptype{C}{\ol{Y}}) \in S_\leq 
      \end{array}
      $
    \\\\
(reduce1) & $
\begin{array}[c]{l}
  Eq \cup \set{\exptype{D}{\ol{A}} \lessdot
    \exptype{D}{\ol{B}}}\\
  \hline
  Eq \cup \set{\ol{A} \doteq \ol{B}}
\end{array}
      $ \\\\
(reduce2) & $
\begin{array}[c]{l}
  Eq \cup \set{\exptype{D}{\ol{A}} \doteq
    \exptype{D}{\ol{B}}}\\
  \hline
  Eq \cup \set{\ol{A} \doteq \ol{B}}
\end{array}
      $ \\\\
    \end{tabular}}
  \end{center}
\caption{Reduce and adapt rules}\label{fig:fgjreduce-rules}
\end{figure}

\begin{figure}
\begin{align*}
&\begin{tabular}[t]{ll}
      (erase1)  & $ 
      \begin{array}[c]{ll}
        \begin{array}[c]{l}
          Eq \cup \set{C \lessdot D}\\
          \hline
          Eq
        \end{array}
        & C \leq^* D \in S_\leq
      \end{array}$\\
          \end{tabular}\\
&\begin{tabular}[t]{ll}
      (erase2)  & $ 
      \begin{array}[c]{ll}
        \begin{array}[c]{l}
          Eq \cup \set{C \doteq C}\\
          \hline
          Eq
        \end{array}
      \end{array}$\\
          \end{tabular}\\
    &      \begin{tabular}[t]{ll}
       (swap) & $
            \begin{array}[c]{ll}
              \begin{array}[c]{l}
                Eq \cup \set{C \doteq a}\\
                \hline
                Eq \cup \set{a \doteq C}
              \end{array}
            \end{array}$
          \end{tabular}
\end{align*}
\caption{Erase and swap rules}\label{fig:fgjerase-rules}
\end{figure}

\subsection{Unify proof}

%Soundness: We have to prove that each calculated result of the algorithm is a general unifier of the corresponding input
\begin{theoremAndi}
  \label{theo:unifySoundness}
  \textbf{(Soundness):}
  If the \textbf{Unify} algorithm finds a solution it does not contradict any of the input constraints:
  $\nexists (a \lessdot b) \in {Cons}_{in}$ where $\sigma(a) \nleq \sigma(b)$  
\end{theoremAndi}
%We show this by induction.
%No step alters the constraint set in a way that would make a wrong solution possible.
\textit{Proof:}
We show theorem \ref{theo:unifySoundness} by going backwards over every step of the algorithm.
We assume there exists a unifier $\sigma = \set {a_1 \mapsto \theta_1, \ldots , a_n \mapsto \theta_n}$ for the input constraints,
which is the result of the \textbf{Unify} algorithm.
This means for every constraint in the input set $(a \lessdot b) \in {Cons}_{in}$ and $(c \doteq d) \in {Cons}_{in}$
this unifier will substitute all variables in a way that all constraints are satisfied:
$\sigma(a) \leq \sigma(b)$, $\sigma(c) = \sigma(d)$\\

We now look at each step of the \textbf{Unify} algorithm
which transforms the input set of constraints $Eq$ to a set $Eq'$.
If we assume the unifier $\sigma$ is correct for the set $Eq'$,
then we can show that it will also be correct for the constraints $Eq$. 


%TODO:
%Assume we find a unifier. Then do every step backwards.
%For each step the constraint set before and after the step have the same unifier.
% -> This means no step adds an incorrect unifier / makes a incorrecto unifier possible
%At the end we must end at the correct unifier.

\begin{description}
\item[Step 5 c)]
The last step of the algorithm transforms a set of constraints $Eq$ of the the form

$Eq = \set{a_1 \doteq \theta_1, \ldots , a_n \doteq \theta_n}$

to the unifier $\sigma$.
Trivial.

\item[Step 5 b)]
A unifier which is correct for $a \doteq b$ is also correct for $a \lessdot b$.

\item[Step 5 a)]
Trivial, we do not alter the constraint set which lateron leads to the unifier.

\item[Step 4]
Trivial, the constraint sets are not altered here.

\item[Step 3]
An unifier $\sigma$ that is correct for a constraint set
$Eq[a \to \theta] \cup (a \doteq \theta)$ is also correct for
the set $Eq \cup (a \doteq \theta)$.
From the constraint $(a \doteq \theta)$ it follows that $\sigma(a) = \theta$.
This means that $\sigma(Eq) = \sigma(Eq[a \to \theta])$,
because every occurence of $a$ in $Eq$ will be raplaced by $\theta$ anyways when using the unifier $\sigma$.

\item[Step 2]
This step transforms constraints of the form $(\exptype{C}{\ol{X}} \lessdot a)$ and $(a \lessdot \exptype{C}{\ol{X}})$
into sets of constraints and builds the cartesian product with the remaining constraints.
We can show that if there is a resulting set of constraints which has $\sigma$ as its correct unifier
then $\sigma$ also has to be a correct unifier for the constraints before this transformation.
Proof:
%normal version:
%\begin{description}
%\item[$(a \lessdot C)$] If $\sigma$ is a correct unifier for a set containing $(a \doteq \theta)$
%and $\theta \leq C$, then $\sigma$ is also a correct unifier for the set containing $(a \lessdot C)$.
%\item[$(C \lessdot a)$] Same goes the other way. If $C leq \theta$ and $\sigma$ is correct for $(C \lessdot \theta)$
%then $\sigma$ is also correct for $(a \doteq \theta)$
%\end{description}

\begin{description}
\item[$(a \lessdot \exptype{C}{\ol{X}})$] If $\sigma$ is a correct unifier for a set containing $(a \doteq \exptype{D}{\ol{A}})$
and $(\ol{X} \doteq \ol{Y'})$
, then $\sigma$ is also a correct unifier for the set containing $(a \lessdot \exptype{C}{\ol{X}})$.
The reason is the \texttt{S-CLASS} subtype rule of GFJ. %TODO: Hier etwas ausführlicher?
%TODO:
%\item[$(\exptype{C}{\ol{T}} \lessdot a)$] Same goes the other way. If $\exptype{C}{\ol{X} leq \theta$ and $\sigma$ is correct for $(C \lessdot \theta)$
%then $\sigma$ is also correct for $(a \doteq \theta)$

%If $\sigma$ is a correct unifier for a set \\
%$Eq = Eq' \cup (a \doteq \exptype{D}{\ol{A}}) \, \cup \, (\ol{X} \doteq \ol{Y'})$ \\
%then it is also a correct unifier for a set $Eq = Eq' \cup (a \lessdot \exptype{C}{\ol{X}})$: \\
When substituting $(a \to \exptype{D}{\ol{A}})$ and $(\ol{X} \to \ol{Y'})$
and finally $ (\ol{Y'} \to [\ol{A}/\ol{Z}]\ol{Y})$ in the constraint $(a \lessdot \exptype{C}{\ol{X}})$
we get: $(\exptype{D}{\ol{A}} \lessdot [\ol{A}/\ol{Z}]\exptype{C}{\ol{Y}}$
which is correct under the \texttt{S-CLASS} rule (see figure \ref{fig:gfj-subtyping-rules}).

\item[$(\exptype{C}{\ol{T}} \lessdot a)$] If $\exptype{C}{\ol{X}} leq \theta$ and $\sigma$ is correct for $(a \doteq [\ol{T}/\ol{X}]N)$
then $\sigma$ is also correct for $(\exptype{C}{\ol{T}} \lessdot a))$.
When substitutin $a$ for $[\ol{T}/\ol{X}]N$ we get 
$(\exptype{C}{\ol{T}} \lessdot [\ol{T}/\ol{X}]N)$
, which is correct because $(\exptype{C}{\ol{X}} \leq \exptype{C}{\ol{Y}}$
(see \texttt{S-CLASS} rule).

\begin{figure}
$\ddfrac{
  \texttt{class}\ \exptype{C}{\ol{X} \triangleleft \ol{N}} \triangleleft N \set{ \ldots }
}{
  \triangle \vdash \exptype{C}{\ol{T}} <: [\ol{T}/\ol{X}]N
}$ 
\caption{Generic Featherweight Java subtyping rules}\label{fig:gfj-subtyping-rules}
\end{figure}

\footnote{
Discussion:
Do we need to include the $\triangleleft \ol N$ bounds?
The S-CLASS rule does not mention them.

When ignoring those rules this could lead to an error
class T<X extends List> {
}
class S<X>{}

Constraint: S<String> <. a
Unify: T<String> =. a // ERROR!

This is not a problem because no error is gonna result from this.
TYPEExpr only hast to implement all of the typing rules of FJ
and unify has to solely respect the subtyping rules.

}
\end{description}

\item[Step 1]
\begin{description}
\item[erase-rules] remove correct constraints from the constraint set.
A unifier $\sigma$ that is correct for the constraint set $Eq$
is also correct for $Eq \cup \set{\theta \doteq \theta}$
and $Eq \cup \set{\theta \lessdot \theta'}$, when $\theta \leq \theta'$.
\item[swap-rule] does not change the unifier for the constraint set.
$\doteq$ is a symmetric operator and parameters can be swapped freely.
\item[adapt] If there is a $\sigma$ which is a correct unifier for a set
$Eq \cup \set{ \exptype{C}{[\ol{A}/\ol{X}]\ol{Y}} \doteq \exptype{C}{\ol{B}}}$ then it is also
a correct unifier for the set $Eq \cup \set{ \exptype{D}{\ol{A}} \lessdot \exptype{C}{\ol{B}}}$,
if there is a subtype relation $\exptype{D}{\ol{X}} \leq^* \exptype{C}{\ol{Y}}$.
To make the set $Eq \cup \set{ [\ol{A}/\ol{X}]\exptype{C}{\ol{Y}} \doteq \exptype{C}{\ol{B}}}$ the unifier 
$\sigma$ must satisfy the condition $\sigma([\ol{A}/\ol{X}]\ol{Y}) = \sigma(\ol{B})$.
By substitution we get $Eq \cup \set{ \exptype{D}{\ol{A}} \lessdot \exptype{C}{[\ol{A}/\ol{X}]\ol{Y}}}$
which is correct under the \texttt{S-CLASS} rule.
\item[reduce] The \texttt{reduce1} and \texttt{reduce2} rules are obviously correct under the FJ typing rules.
\end{description}

\item[OrConstraints]
If $\sigma$ is a correct unifier for one of the constraint sets in $Eq_{set}$
then it is also a correct unifier for the input set $Cons_{in}$.
When building the cartesian product of the \textbf{OrConstraints} every possible
combination for $Cons_{in}$ is build.
No constraint is altered, deleted or modified during this step.
\end{description}
\hfill $\square$


%Completeness: If there is a Unifier (a solution), the algorithm will never alter the constraints in a way that this solution is removed
%\begin{theoremAndi}\label{theo:unifyCompleteness}
%  \textbf{(Completeness):} If there is a solution for the input constraints $Cons_{in}$, the \textbf{Unify} algorithm will not fail.
%  A solution is a non-empty set of unifiers $Uni = \set{\sigma_1, \ldots \sigma_n }$,
%  where each unifier is a injective function which maps every type variable in the input constraints $Cons_{in}$ to a
%  type in $S_\leq$.
%  \end{theoremAndi}
\begin{theoremAndi}\label{theo:unifyCompleteness}
  \textbf{(Completeness):} The \textbf{Unify} algorithm calculates the principal type solution for the input set of constraints ($Cons_{in}$).
  A unifier $\sigma$ is a principal type solution for $Cons_{in}$ if it unifies $Cons_{in}$
  and for every other unifier $\omega$ there is a unifier $\lambda$ so that $\omega(x) = \lambda(\sigma(x))$.
\end{theoremAndi}
\textit{Proof:}
%The \textbf{Unify} calculates multiple solutions.

%Our proof goes as follows:
We look at every step of the algorithm, which alters the set of constraints $Eq$,
while assuming that there is at least one possible principal type solution $\sigma$ for the input.
We will show that the principal type is among them by proofing for every step of the algorithm that the principal type is never excluded.

%Assume there is a unifier and the Unify algorithm finds it.
%Then no rule makes this unifier impossible / removes this unifier.

\begin{description}
\item[Step 1:]
The first step applies the three rules from figure \ref{fig:unifyrules}.
\textbf{erase-rules:} The erase2 rule from figure \ref{fig:unifyrules} removes a
$\{\theta \leq \theta\}$ constraint from the constraint set.
The erase1 rule removes a $\{\theta \leq \theta\}$ constraint,
but only if the two types $\theta$ and $\theta'$ satisfy the constraint.
Both rules do not change the set of possible solutions for the given constraint set.

\textbf{swap-rule:} $\doteq$ is a symmetric operator and parameters can be swapped freely.
This operation does not change the meaning of the constraint set.

\textbf{adapt-rule:} Every solution which is correct for the constraints
$Eq \cup \set{ \exptype{C}{[\ol{A}/\ol{X}]\ol{Y}} \doteq \exptype{C}{\ol{B}}}$ is also
a correct solution for the set $Eq \cup \set{ \exptype{D}{\ol{A}} \lessdot \exptype{C}{\ol{B}}}$.
According to the \texttt{S-CLASS} rule there can only be a possible solution for 
$\exptype{C}{[\ol{A}/\ol{X}]\ol{Y}} \doteq \exptype{C}{\ol{B}}$
if $\ol{B} = [\ol{A}/\ol{X}]\ol{Y}$.
Therefore this transformation does not remove any possible solution from the constraint set.

\textbf{reduce-rule:}
%The constraint is not altered
For a constraint $\exptype{D}{\ol{A}} \lessdot \exptype{D}{\ol{A}}$ the FJ subtyping rule \texttt{S-REFL} ($\triangle \vdash T <: T$) is the only one which applies.
According to this rule the transformation to $\ol{A} \doteq \ol{B}$ is correct.
Only $D$ gets removed, which is not a type variable.
Therefore this step does not remove a possible solution.
This applies for both reduce rules \textbf{reduce1} and \textbf{reduce2}.

\item[Step 2:]
%The second step builds multiple constraint sets of all possible type combinations for the $\lessdot$-constraints.
The second step of the algorithm eliminates $\lessdot$-constraints
by replacing them with $\doteq$-constraints.
%The algorithm considers every possible type from $S_\leq$ which does not violate the eliminated $\lessdot$ -constraint itself.
%This step does not remove a solution from the constraint set.
For each $(a \lessdot \exptype{C}{\ol{X}})$ constraint the algorithm builds a set with every
possible subtype of $\exptype{C}{\ol{X}}$ set in for $a$.
So if there is a correct unifier $\sigma$ for the constraints before this conversion there will be at least one set of
constraints for which $\sigma$ is a correct unifier.
%TODO: what does soundness mean. if there is a possible type solution (with types in S) then unify will find it

\item[Step 3:]
In the third step the \textbf{substitution}-rule is applied.
If there is a constraint $a \doteq \theta$ then there is no other way to fulfill the constraint set
than replacing $a$ with $\theta$.
This does not remove a possible solution.

\item[Step 4:]
None of the constraints get modified.

\item[Step 5 a):]
The removed sets do not have a possible unifier, therefore no possible solution is
omitted in this step.

\textbf{Proof}:
In step 5.a all constraint sets that have a unifier are in solved form.
All other possibilities are eliminated in steps 1-4.
There are 8 different variations of constraints:\\
$(a \doteq a), (a \doteq C), (C \doteq a), (C \doteq C), (a \lessdot a), (a \lessdot C), (C \lessdot a), (C \lessdot C)$

After step 1 there are no $(C \doteq C)$, $(C \lessdot C)$ and $(C \doteq a)$ constraints anymore,
as long as the constraint set has a correct unifier.
Because a constraint set that has a correct unifier cannot contain constraints of the form $\theta_1 \doteq \theta_2$ with $\theta_1 \neq \theta_2$ and
$(\theta_1 \lessdot \theta_2)$ with $(\theta_1 \leq \theta_2) \notin S_\leq$.
By removing $(\theta \doteq \theta)$ and $(\theta \lessdot \theta')$ with $(\theta \leq \theta') \notin S_\leq$ constraints
no constraints of the form $(C \doteq C)$ and $(C \lessdot C)$
remain in a constraint set that has a correct unifier after step 1.

After step 2 there are no more $(a \lessdot C)$ constraints.

After step 3 there are no $(a \doteq C)$ anymore.

We only reach step 5 if the constraint set is not changed by the substitution (step 3).

%If at this point a set $Eq_i$ is not in solved form it has no correct unifier.

If the constraint set has a correct unifier only $(a \lessdot a)$, $(a \doteq a)$ and $(a \doteq C)$ constraints are left at this point.
The type variables in the $(a \lessdot a)$ and $(a \doteq a)$ constraints have to be independent type variables.
If a type variable $c$ is inside a $(c \doteq C)$ constraint it is not an independent type variable.
But this variable $c$ cannot be inside a $(a \doteq a)$ or $(a \lessdot a)$ constraint, because otherwise step 3 would have replaced it in there.


\item[Step 5 b):]
If the algorithm advances to this step we further only work on constraint sets in solved form.
This means there are only two kinds of constraints left.
($A \doteq \texttt{Typ}$), ($A \doteq B$) and ($A \lessdot B$) with $A$ and $B$ as type variables.

%We can set all TVs equal, because we allow only same TVs when having circles in a method call.
%This still will lead to the principal type.

The GFJ language does not allow subtype constraints for generic types.
A constraint like $(A \lessdot B)$ in a solution could be inserted as the typing shown in the example below.
But this is not allowed by the syntax of GFJ.
That is why we can treat this constraint as $(A \doteq B)$.

%TODO: This does not alter the outcome because the solution set is not modified anymore. All other TVs alread have a type like A =. Typ

\textit{Example:}
This would be a valid Java program but is not allowed in GFJ:
\begin{lstlisting}
class Example {
  <A extends Object, B extends A> A id(B a){
    return a;
  }
}
\end{lstlisting}

By replacing all ($A \lessdot B$) constraints with ($A \doteq B$) we do not remove a principal type solution.

\item[Step 6:]
In the last step all the constraint sets, which are in solved form, are converted to unifiers.

We see that only a constraint set which has no unifier does not reach solved form.
We showed that in every step of the \textbf{Unify} algorithm we never exclude a possible unifier.
Also we showed that after we reach step 5 only constraint sets with a correct unifier are in solved form.
By removing all constraint sets which are not in solved form the algorithm does not
remove a possible correct unifier.

If we assume that there is a possible principal type solution $\sigma$ for the input set $Cons_{in}$
and the \textbf{Unify} algorithm does not exclude any of the possible unifiers,
then the result \textbf{Unify} contains the principal type solution.
\hfill $\square$
\end{description}

\section{Insert principal type}
After generating all possible unifiers in the \textbf{Unify} we can insert the principal types.

\begin{align*}
\ddfrac{
  A\ \texttt{m}(\ol{A}\ \ol{p}) \{ \ldots \} \in M \quad \quad \texttt{class}\ \exptype{C}{\ol{X}} \{ \ldots\ M,\ \ldots\}
}{
  \textit{mtype}(\texttt{m}, \exptype{C}{\ol{X}}) = \set{\sigma(\ol{A}) \to \sigma(A) \,|\, \sigma \in {Uni}}
}
\end{align*}

The \textbf{Unify} algorithm returns a set of unifiers ${Uni}$.
Each element of that set is a correct solution.
The unifiers $\sigma$ map type placeholders to types.
When generating the intersection types for the methods we have to make sure that the
type placeholders for the return type as well as for the parameter types get replaced by the same unifier $\sigma$.
It can happen that two unifiers $\sigma_1$ and $\sigma_2$ lead to the same method type ($\sigma_1(A) = \sigma_2(A), \sigma_1(\ol A) = \sigma_2(\ol A)$).
The set of all the distinct combinations then builds the intersection type for the method.

\textbf{Example:}
\begin{lstlisting}
class Global{
  method1(a){
    a.add(this);
    return a.get();
  }
}
class List<A> {
  add(A item){...}
  A get() ...
}
\end{lstlisting}

The method \texttt{method1} would get the type $\set{ \exptype{List}{Object} \to \texttt{Object}
\ || \ \exptype{List}{String} \to \texttt{String}}$.
If GFJ would support overloaded methods this could be written as:
\begin{lstlisting}
class Global{
  Object method1(List<Object> a){
    a.add(this);
    return a.get();
  }
  String method1(List<String> a){
    a.add(this);
    return a.get();
  }
}
class List<A> {
  add(A item){...}
  A get() ...
}
\end{lstlisting}

\section{Examples}

\textbf{Example 1}\\
The algorithm is able to infer the types of multiple classes under specific circumstances.
The individual classes must be given to him after one another.
This comes with the restriction, that the first class is correct on its own and does not use any other class.
The second class that gets compiled can use the first class and so on.

The following example shows how the algorithm infers and compiles multiple classes iteratively.
The class \texttt{Class1} is infered first.
It has only one method which is the identity function,
to which our algorithm allocates the type $\exptype{}{A}\ A \to A$.
The next class \texttt{Class2} is now able to use this generic method.
The blue colored types are inferred in the next iteration of our algorithm.

\begin{table}
\caption{Two classes as input. \texttt{Class1} is infered first (shown in {\color{red}red})}
\begin{tabular}{cc}
\begin{lstlisting}
class Class1 extends Object {
  Class1() { super(); }
  id(a){
    return a;
  }
}
class Class2 extends Class1 {
  Class2() { 
    super(); 
  }
  example(){
    return new Class1().id(this);
  }
}
\end{lstlisting}
&
\begin{lstlisting}
class Class1 extends Object {
  Class1() { super(); }
  (*@ \textcolor{red}{<A> A} @*) id((*@ \textcolor{red}{A} @*) a){
    return a;
  }
}
class Class2 extends Class1 {
  Class2() { 
    super(); 
  }
  (*@ \textcolor{blue}{Class1}@*) example(){
    return this.(*@\textcolor{blue}{<Class1>}@*)id(this);
  }
}
\end{lstlisting}
\end{tabular}
\end{table}

\textbf{Example 2}\\
When compiling a class like the following
we have to first split this class into two classes.
The \texttt{TwoMethods} class can be first split into the classes \texttt{Class1}
and \texttt{Class2} and after being processed by the type inference algorithm it can be assembled back together again.
This leads to a principal typing.
When using our type inference algorithm on the class \texttt{TwoMethods} alone
it would give the method \texttt{id} the type $\texttt{TwoMethods} \to \texttt{TwoMethods}$,
which is not the desired principal type.
\begin{lstlisting}
class TwoMethods extends Object {
  TwoMethods() { super(); }
  id(a){
    return a;
  }
  example(){
    return this.id(this);
  }
}
\end{lstlisting}

\textbf{Example 3}\\
%TODO: Ein Beispiel für die Unify-adapt Regel
GFJ allows subtype relations like the following:
\begin{lstlisting}
class Map<A,B> extends Object {
  Map<A,B>() { super(); }
}
class SpecialMap<A,B,C> extends Map<A,C> {
  SpecialMap<A,B,C>() { super(); }
}
\end{lstlisting}

If for example we have a method \texttt{method} like this:
\begin{lstlisting}
<X> void method(Map<X, String> map){
  ...
}
\end{lstlisting}
and call it:
\begin{lstlisting}
method(new SpecialMap<Object,Integer,String>());
\end{lstlisting}

Then the constraint $\exptype{SpecialMap}{Object,Integer,String} \lessdot \exptype{Map}{X,String}$
is generated by the \textbf{GFJTYPE} algorithm.
This constraint will be processed by the \texttt{adapt} rule of the \textbf{Unify} algorithm.
Remember that $(\exptype{SpecialMap}{A,B,C} \olsub \exptype{Map}{A,C}) \in S_\leq$.
\begin{align*}
  Eq& \cup \exptype{SpecialMap}{Object,Integer,String} \lessdot \exptype{Map}{X,String} \\
  \cline{1-2} 
  Eq& \cup \set{\exptype{Map}{[ Object / A ][ Integer / B ][ String / C ](A,C)}
  \doteq \exptype{Map}{X,Integer}} \\
  \cline{1-2} 
  Eq& \cup \set{\exptype{Map}{Object,String}
  \doteq \exptype{C}{X, Integer}}
%Eq \cup \set{\theta_1 \doteq \lambda'_1 \ldo \theta_n \doteq \lambda'_n}
\end{align*}

After the \texttt{adapt} rule got applied we can already see that a correct unificator for this constraint would be
$\sigma(X) = \texttt{Object}$.

\section{Assessment}
\label{sec:assessment}

\begin{itemize}
\item NP-hard 
\item cannot infer all possible generic methods
\item features that need to be addressed to make it practical (e.g.,
  what's necessary to move from FGJ to full Java: overloading,
  imperative, )
\end{itemize}

\section{Related Work}
\section{Related Work}
\label{sec:related-work}


\subsection{Formal models for Java}
\label{sec:formal-models-java}

There is a range of formal models for Java. Flatt et al
\cite{DBLP:conf/java/FlattKF99} define an elaborate model with
interfaces and classes and prove a type soundness result. They do not
address generics. Igarashi et al
\cite{DBLP:journals/toplas/IgarashiPW01} define Featherweight Java
and its generic sibling, Featherweight Generic Java. Their language is
a functional calculus reduced to the bare essentials, they develop the full metatheory, they
support generics, and study the type erasing transformation used by
the Java compiler. MJ \cite{UCAM-CL-TR-563} is a core calculus that
embraces imperative programming as it is targeted towards reasoning
about effects. It does not consider generics. Welterweight Java
\cite{DBLP:conf/tools/OstlundW10} and OOlong
\cite{DBLP:conf/sac/CastegrenW18} are different sketches for a core
language that includes concurrency, which none of the other core
languages considers. 

We chose to base our development on FGJ because it embraces a relevant
subset of Java without including too much complexity (e.g., no imperative
features, no interfaces, no concurrency). It seems that results for
FGJ are easily scalable to full Java. We leave the addition of these
feature to future work, as we see our results on FGJ as a first step
towards a formalized basis for global type inference for Java.

\subsection{Type inference}

Some object-oriented languages like Scala, C\#, and Java perform
\emph{local} type inference \cite{PT98,OZZ01}. Local type 
inference means that missing type annotations are recovered using only
information from adjacent nodes in the syntax tree without long distance
constraints. For instance, the type of a variable initialized with a
non-functional expression or the return type of a method can be
inferred. However, method argument types, in particular for recursive
methods, cannot be inferred by local type inference.

Milner's algorithm $\mathcal{W}$ \cite{DBLP:journals/jcss/Milner78} is
the gold standard for global type inference for languages with 
parametric polymorphism, which is used by ML-style languages. The fundamental idea
of the algorithm is to enforce type equality by many-sorted type
unification \cite{Rob65,MM82}. This approach is effective and results
in so-called principal types because many-sorted unification is
unitary, which means that there is at most one most general result.

Pl\"umicke \cite{Plue07_3} presents a first attempt to adopt Milner's
approach to Java. However, the presence of subtyping means that type
unification is no longer unitary, but still finitary. Thus, there is
no longer a single most general type, but any type is an instance of a
finite set of maximal types (for more details see Section
\ref{sec:unification}). Further work by the same author
\cite{plue15_2,plue17_2}, 
refines this approach by moving to a constraint-based algorithm and by
considering lambda expressions and Scale-like function types.
In Pl\"umicke's work there is no formal definition of the type system as a basis
of the type inference algorithm. One contribution of this paper is a
formal definition of the underlying type system. 

We rule out polymorphic recursion because its presence makes type
inference (but not type checking: see FGJ) undecidable. Henglein
\cite{DBLP:journals/toplas/Henglein93} as well as Kfoury et al
\cite{DBLP:journals/toplas/KfouryTU93} investigate type inference in
the presence of polymorphic recursion. They show that type inference
is reducible to semi-unification, which is undecidable
\cite{DBLP:journals/iandc/KfouryTU93}. However, the undecidability of
this problem apparently does not matter much in practice
\cite{DBLP:journals/tcs/EmmsL99}. 

Ancona, Damiani, Drossopoulou, and Zucca \cite{ADDZ05} consider polymorphic byte
code. Their approach is modular in the sense that it infers
polymorphic structural types. As {Java} does not support structural
types, their approach would have to be simulated with generated
interfaces. Pl\"umicke \cite{plue16_1} follows this
approach. Furthermore Ancona and coworkers do not consider generic classes. 



\subsection{Unification}
\label{sec:unification}

We reduce the type inference problem to constraint solving with
equality and subtype constraints.
The procedure presented in Section~\ref{sec:unify} is inspired by
polymorphic order-sorted unification which is used in logic 
programming languages with polymorphic order-sorted types
\cite{GS89,MH91,HiTo92,CB95}.

Smolka's thesis \cite{GS89} mentions type unification
as an open problem. He gives  an incomplete type inference algorithm
for the logical language \textsf{TEL}. The reason for incompleteness
is the admission of subtype relationships between polymorphic types of
different arities as in  $\texttt{List(a)} \sub
\texttt{myLi(a,b)}$. In consequence, the subtyping relation does
not fulfill the ascending chain condition.
For example, given  $\texttt{List(a)} \sub \texttt{myLi(a,b)}$, we obtain:
\begin{gather*}
  \texttt{List(a)} \sub \texttt{myLi(a,List(a))} \sub \texttt{myLi(a,myLi(a,List(a)))}  \sub \dots
\end{gather*}
However, this subtyping chain exploits covariant subtyping, which does
not apply to FGJ.
%(but it would apply in the presence of wildcards).

Smolka's algorithm also fails sometimes in the absence of infinite
chains, although there is a unifier. 
For example, given $\texttt{nat} \sub \texttt{int}$ and the set
of subtyping constraints $\set{\mathtt{nat} \lessdot \mathtt{a},
  \mathtt{int} \lessdot \mathtt{a}}$, it returns the substitution
$\set{\mathtt{a} \mapsto \mathtt{nat}}$ generated from the first
constraint encountered. This substitution is not a solution
because $\set{\mathtt{int} \lessdot \mathtt{nat}}$ fails.
However, $\set{\mathtt{a}\mapsto \mathtt{int}}$ is a unifier, which
can be obtained by processing the constraints in a different order: from $\set{\mathtt{int} \lessdot \mathtt{a}, \mathtt{nat} \lessdot
  \mathtt{a}}$ the algorithm calculates the unifier 
$\set{\mathtt{a}\mapsto \mathtt{int}}$.

Hill and Topor  \cite{HiTo92} propose a polymorphically typed logic
programming language with subtyping. They restrict subtyping to type
constructors of the same arity,  which guarantees that all subtyping
chains are finite.
In this approach a \emph{most general type unifier (mgtu)} is
defined as an upper bound of different principal type unifiers. In
general, two type terms need not have an upper bound in the subtype ordering,
which means that there is no mgtu in the sense of Hill and Topor.
For example, given  $\texttt{nat} \sub \texttt{int}$, $\texttt{neg} 
\sub \texttt{int}$, and the set of inequations $\set{\mathtt{nat} \lessdot
  \mathtt{a}$, $\mathtt{neg} \lessdot \mathtt{a}}$, the mgtu $\set{\mathtt{a} \mapsto \texttt{int}}$ is
determined. If the subtype ordering is extended by $\mathtt{int} \sub
\mathtt{index}$ and $\mathtt{int} \sub \mathtt{expr}$, then there are three
unifiers $\set{\mathtt{a} \mapsto \texttt{int}}$, $\set{\mathtt{a} \mapsto
  \mathtt{index}}$, and $\set{\mathtt{a} \mapsto 
\mathtt{expr}}$, but none of them is an mgtu \cite{HiTo92}.

The type system of \textsf{PROTOS-L} \cite{CB95} was
derived from \textsf{TEL} by disallowing any explicit subtype relationships
between polymorphic type constructors. 
Beierle \cite{CB95} gives a complete type unification algorithm, which can be extended to the
type system of Hill and Topor.
They also prove that the type unification problem is finitary.

Given the declarations  $\texttt{nat} \sub
\texttt{int}$, $\texttt{neg} \sub \texttt{int}$, $\mathtt{int} \sub
\mathtt{index}$, and $\mathtt{int} \sub \mathtt{expr}$, applying the
type unification algorithm of \textsf{PROTOS-L} to the set of
inequations $\set{\mathtt{nat} \lessdot
  \mathtt{a}$, $\mathtt{neg} \lessdot \mathtt{a}}$ yield three general
unifiers $\set{\mathtt{a} \mapsto \texttt{int}}$, $\set{\mathtt{a} \mapsto
  \mathtt{index}}$, and $\set{\mathtt{a} \mapsto \mathtt{expr}}$. 

Pl\"umicke \cite{plue09_1} realized that the type system of
\textsf{TEL} is related to subtyping in Java.
In contrast to \textsf{TEL}, where the ascending chain condition does
not hold,  Java with wildcards violates the descending chain
condition. For example, given $\exptypett{myLi}{b,a} \olsub
\exptypett{List}{a}$ we find:

\smallskip
{\centering
$\ldots \ \olsub\
\exptypett{myLi}{\exptypett{?\,$\extends$\,myLi}{\exptypett{\textrm{{\tt ?}}\,$\extends$\,List}{a},a},a}
\ \olsub\ \exptypett{myLi}{\exptypett{?\,$\extends$\,List}{a},a} \ \olsub\  \exptypett{List}{a}$\\}

\smallskip
Pl\"umicke \cite{plue09_1} solved the open problem of infinite chains
posed by Smolka \cite{GS89}.
He showed that in any infinite chain there is a finite number of elements such that
all other elements of the chain are instances of them. The resulting type
unification algorithm can be used for type inference of Java~5 with
wildcards \cite{Plue07_3}. As FGJ has no wildcards, we based our
algorithm on an earlier work \cite{Plue04_1}.
In contrast to that work, which only infers generic methods with
unbounded types, our algorithm  infers bounded generics.
To this end, we do not expand constraints
of the form $\tv{a} \lessdot \itype{N}$, where $\tv{a}$ is type variable and $\itype{N}$ is is a
non-variable type, but convert them to bounded type parameters of the form
\texttt{X extends N}. This change results in a significant reduction
of the number of solutions of the type
unification algorithm without restricting the generality of typings of
FGJ-programs. Unfortunately, constraints of the form $\itype{N} \lessdot \tv{a}$ have
to be expanded as FGJ (like Java) does not permit lower bounds for
generic parameters. If lower bounds were permitted  (as in Scala), the
number of solutions could be reduced even further.


%%% Local Variables:
%%% mode: latex
%%% TeX-master: "TIforGFJ"
%%% End:


\section{Conclusions}
\label{sec:conclusions}


\bibliographystyle{ACM-Reference-Format}
\bibliography{peter,martin}

\end{document}
\endinput
%%
%% End of file `TIforGFJ.tex'.
