% APLAS 21: 18 pages in the Springer LNCS format(LaTeX template),
% including bibliography and figures
% APLAS 2021 will use a lightweight double-blind reviewing process.
\documentclass[runningheads]{llncs}

\usepackage{todonotes} % [disable]
\usepackage[utf8]{inputenc}
\usepackage{hyperref}
\usepackage{amsmath}
\usepackage{amssymb}
\usepackage{subcaption}
\usepackage{prolog}
\usepackage{prftree}

\parindent=0mm
\begin{document}

%%
%% The "title" command has an optional parameter,
%% allowing the author to define a "short title" to be used in page headers.
\title{Type Inference for Featherweight Generic Java}


\author{Andreas Stadelmeier \and
Martin Plümicke\and
Peter Thiemann}
%
\authorrunning{F. Author et al.}
% First names are abbreviated in the running head.
% If there are more than two authors, 'et al.' is used.
%
\institute{DHBW Stuttgart, Campus Horb
\email{a.stadelmeier@hb.dhbw-stuttgart.de}}
\authorrunning{Stadelmeier, Plümicke, Thiemann}


%%
%% This command processes the author and affiliation and title
%% information and builds the first part of the formatted document.
\maketitle

%%
%% The abstract is a short summary of the work to be presented in the
%% article.
\begin{abstract}
  Type Inference for Featherweight Generic Java
  \keywords{type inference, Java, compiler}
\end{abstract}



\section{Introduction}
\label{sec:introduction}

Java is one of the most important programming languages. In 2019, Java
was the second most popular language according to a study
based on GitHub
data.\footnote{\url{https://www.businessinsider.de/international/the-10-most-popular-programming-languages-according-to-github-2018-10/}} Estimates
for the number of Java programmers range between 7.6 and 9 million.\footnote{\url{https://www.zdnet.com/article/programming-languages-python-developers-now-outnumber-java-ones/},
\url{http://infomory.com/numbers/number-of-java-developers/}} Java
has been around since 1995 and progressed through 16 versions.

Swarms of programmers have taken their first steps in Java. Many more
have been introduced to object-oriented programming through Java, as
it is among the first mainsteam languages supporting
object-orientation. Java is a class-based language with static single inheritance among
classes, hence it has nominal types with a specified subtyping
hierarchy. Besides classes there are interfaces to characterize common 
traits independent of the inheritance hierarchy. Since version J2SE~5.0,
the Java language supports F-bounded polymorphism in the form of generics.

Java is generally explicitly typed with some amendments introduced in
recent versions. That is, 
variables, fields, method parameters, and method returns must be
adorned with their type. An adaptation of a generic example from Featherweight Java
\cite{DBLP:journals/toplas/IgarashiPW01} 
is shown in Figure~\ref{fig:intro-example-generic-fj}.
\begin{figure}[tp]
%   \begin{subfigure}[t]{0.55\textwidth}
% \begin{lstlisting}
% class Pair {
%   Object fst;
%   Object snd;
%   Pair(Object fst, Object snd) {
%    this.fst=fst; 
%    this.snd=snd;
%   }
%   Pair setfst(Object fst) {
%     return new Pair(fst, this.snd);
%   }
% }
% \end{lstlisting}
%   \end{subfigure}
  \begin{subfigure}[t]{0.49\linewidth}
\begin{lstlisting}[style=fgj]
class Pair<X,Y> {
  X fst;
  Y snd;
  Pair<X,Y>(X fst, Y snd) {
    this.fst=fst;
    this.snd=snd;
  }
  Pair<X,Y> setfst(X fst) {
    return new Pair(fst, this.snd);
  }
  Pair<Y,X> swap() {
    return new Pair(this.snd, this.fst);
  }
}  
\end{lstlisting}
    \caption{Featherweight Generic Java (FGJ)}
    \label{fig:intro-example-generic-fj}
  \end{subfigure}
  ~
  \begin{subfigure}[t]{0.49\linewidth}
\begin{lstlisting}[style=tfgj]
class Pair<X,Y> {
  X fst;
 Y snd;
  Pair(fst, snd) {
    this.fst=fst; 
    this.snd=snd;
  }
  setfst(fst) {
    return new Pair(fst, this.snd);
  }
  swap() {
    return new Pair(this.snd, this.fst);
  }
}  
\end{lstlisting}
    \caption{FGJ with global type inference (\TFGJ)}
    \label{fig:intro-example-generic-jtx}
  \end{subfigure}
  \caption{Example code}
  \label{fig:intro-example-code}
\end{figure}
While the overhead of explicit types look reasonable in the example,
realistic programs often contain variable initializations like
the following:\footnote{Taken from
  \url{https://stackoverflow.com/questions/4120216/map-of-maps-how-to-keep-the-inner-maps-as-maps/4120268}.} 
\begin{lstlisting}[basicstyle=\ttfamily\fontsize{8}{9.6}\selectfont,style=fgj]
  HashMap<String, HashMap<String, Object>> outerMap =
    new HashMap<String, HashMap<String, Object>>();
\end{lstlisting}

Java's \emph{local variable type inference} (since version 10\footnote{\url{https://openjdk.java.net/jeps/286}}) deals
satisfactorily with examples like the initialization of
\lstinline{outerMap}. 
In many initialization scenarios for local variables, Java infers their type
if it is obvious from the context. In the
example, we can write
\begin{lstlisting}[basicstyle=\ttfamily\fontsize{8}{9.6}\selectfont,style=fgj]
var outerMap = new HashMap<String, HashMap<String, Object>>();
\end{lstlisting}
because the constructor of the map spells out the type in
full. More specifically,``obvious'' means that the right side of the initialization is
\begin{itemize}
\item a constant of known type (e.g., a string),
\item a constructor call, or
\item a method call (the return type is known from the method
  signature).
\end{itemize}
The \lstinline{var} declaration can also be used for an iteration
variable where the type can be obtained from the elements of the
container or from the initializer.
Alternatively, if the variable is used as the method's return value,
its type can be obtained from the current method's signature.

However, there are still many places where the programmer must provide types. In
particular, an explicit type must be given for
\begin{itemize}
\item a field of a class,
\item a local variable without initializer or initialized to \lstinline{NULL},
\item a method parameter, or
\item a method return type.
\end{itemize}

In this paper, we study \emph{global type inference} for Java. Our aim
is to write code that omits most type annotations, except for class
headers and field types. Returning to the \lstinline{Pair} example, it
is sufficient to write the code in Figure~\ref{fig:intro-example-generic-jtx}
and global type inference fills in the rest so that the result is
equivalent to Figure~\ref{fig:intro-example-generic-fj}. Our
motivation to study global type inference is threefold.
\begin{itemize}
\item Programmers are relieved from writing down obvious types. 
\item Programmers may write types that leak implementation details. The
  \lstinline{outerMap} example provides a good example of this
  problem. From a software engineering
  perspective, it would be better to use a more general abstract type like
\begin{lstlisting}[basicstyle=\ttfamily\fontsize{8}{9.6}\selectfont,style=fgj]
Map<String, Map<String, Object>> outerMap = ...
\end{lstlisting}
  Global type inference finds most general types.
\item Programmers may write types that are more specific than
  necessary instead of using generic types. Here, type
  inference helps programmers to find the most general type. Suppose
  we wanted to add a static  method \texttt{eqPair} for pairs of integers to the
  \lstinline/Pair/ class.
\begin{lstlisting}[basicstyle=\ttfamily\fontsize{8}{9.6}\selectfont,style=fgj]
boolean eqPair (Pair<Integer,Integer> p) {
  return p.fst.equals(p.snd);
}
\end{lstlisting}
  With global type inference it is sufficient to write the code on the
  left of Figure~\ref{fig:equal-pair} and obtain the FGJ code with the most general type on the right.
\end{itemize}
  \begin{figure}[t]
    \begin{minipage}[t]{0.49\linewidth}
\begin{lstlisting}[style=tfgj]
eqPair (p) {
  return p.fst.equals(p.snd);
}
\end{lstlisting}
    \end{minipage}
    \begin{minipage}[t]{0.49\linewidth}
\begin{lstlisting}[style=fgj]
<T> boolean eqPair (Pair<T,T> p){
  return p.fst.equals<T>(p.snd);
}
\end{lstlisting}
    \end{minipage}
    \caption{\lstinline{eqPair} in \TFGJ and FGJ}
    \label{fig:equal-pair}
  \end{figure}
% \item Sometimes, it can be hard to find a correct typing at all.
% \todo[inline]{example?}

In this paper, we study global type inference for Featherweight
Generic Java \cite{DBLP:journals/toplas/IgarashiPW01} (FGJ), which is backed
by an implementation for full Java. Our type inference algorithm
applies to FGJ programs that specify the full class header and all field types,
but omit all method signatures. 
Given this input, our algorithm
infers a set of most general method signatures (parameter types and return types).
Inferred types are generic as much as possible, but do not include
bounds.

The inferred signatures have the following round-trip property
(relative completeness). If we
start with an FGJ program that does not make use of polymorphic
recursion (see Section~\ref{sec:polym-recurs}), strip all types from
method signatures, and run the algorithm on the 
resulting stripped program, then at least one of the inferred typings is more
general than the types in the original FGJ program.






\subsection{Contributions}
\label{sec:contributions}

\begin{itemize}
\item Algorithm for global type inference for FGJ. This algorithm is
  sound and relatively complete for programs without polymorphic recursion.
\item Global type inference for FGJ is NP complete.
\item Implementation of global type inference for (full?) Java.
\end{itemize}


%%% Local Variables:
%%% mode: latex
%%% TeX-master: "TIforGFJ"
%%% End:


\section{Motivation}
\label{sec:motivation}

% Examples, examples, examples from simple to more advanced showing off
% the (difficult) features of inference.

In this section, we present a sequence of more and more challenging
examples for global type inference (GTI). To spice up our examples
somewhat, we assume some predefined utility classes with the following
interfaces.
\begin{lstlisting}
class Bool {
  Bool not(); 
}
class Int {
  Int negate ();
  Int add (Int that);
  Int mult (Int that);
}
class Double {
  Double negate ();
  Double add (Double that);
  Double multi (Double that);
}
\end{lstlisting}

\subsection{Multiplication}
\label{sec:multiplication}

Here is the TFGJ code  for multiplying the components of a pair.
\begin{lstlisting}
class MultPair {
  mult (p) { return p.fst.mult(p.snd); }
}
\end{lstlisting}
Assuming the parameter typing $\mv p:\mv P$, result type $\mv R$, and that
\texttt{mult} in the body refers to \texttt{Int.mult}, we
obtain the following constraints.
\begin{itemize}
\item From \texttt{p.fst}: $\mv P \le \mathtt{Pair}\Angle{X,Y}$ and
  $\texttt{p.fst} : X$
\item From \texttt{p.snd}: $\mv P \le \mathtt{Pair}\Angle{X,Y}$ and
  $\texttt{p.snd} : Y$
\item From \texttt{.mult (p.snd)}: $X \le \mathtt{Int}$, $Y \le
  \mathtt{Int}$, and $\mathtt{Int} \le \mv R$.
\end{itemize}
As the return type only occurs positively in the constraints, we can
set it to $\mathtt{Int}$.
% \begin{lstlisting}
% class MultPair {
%   Int mult (Pair<Int,Int> p) { return p.fst.mult(p.snd); }
% }
% \end{lstlisting}
\begin{lstlisting}
class MultPair {
  <X extends Int, Y extends Int>
  Int mult (Pair<X,Y> p) { return p.fst.mult(p.snd); }
}
\end{lstlisting}
We obtain a second version if we assume that \texttt{mult} refers to
\texttt{Double.mult}.
\begin{lstlisting}
class MultPair {
  <X extends Double, Y extends Double>
  Double mult (Pair<X,Y> p) { return p.fst.mult(p.snd); }
}
\end{lstlisting}

Finally, \texttt{mult} might be recursive, which results different constraints.
\begin{itemize}
\item From \texttt{.mult (p.snd)}: $X \le \mathtt{MultPair}$, $Y \le
  \mathtt{P}$, and $\mathtt{R} \le \mv R$.
\end{itemize}
Combining $\mv P \le \mathtt{Pair}\Angle{X,Y}$ and $Y \le \mathtt{P}$
yields $Y \le \mathtt{Pair}\Angle{X,Y}$, which is not solvable and
hence rejected. 

The corresponding example in full Java would be the dot product. In
full Java, we also have to deal with overloading of the multiplication
operator, which amounts to considering \texttt{Int.mult} and
\texttt{Double.mult}. Overloading is not allowed in FGJ, but method
names in different classes may overlap.


\subsection{Inheritance}
\label{sec:inheritance}

\begin{figure}[tp]
  \begin{subfigure}[t]{0.49\linewidth}
\begin{lstlisting}
class A1 {
  m(x) { return x.add(x); }
}
class B1 extends A1 {
  m(x) { return x; }
}
\end{lstlisting}
  \end{subfigure}
\begin{subfigure}[t]{0.49\linewidth}
\begin{lstlisting}
class A2 {
  m(x) { return x; }
}
class B2 extends A2 {
  m(x) { return x.add(x); }
}
\end{lstlisting}
  \end{subfigure}
  \caption{Method overriding}
  \label{fig:method-overriding}
\end{figure}

Let's start with the artificial example in the left
listing of Figure~\ref{fig:method-overriding} and ignore the \texttt{Double} class. Type
inference proceeds according 
to the inheritance hierarchy starting from the superclasses. Class \texttt{A1} has
a method \texttt{m} of inferred type \texttt{Int -> Int}. Class \texttt{B1} is a
subclass of \texttt{A1} which must override \texttt{m} as there is no
overloading in FGJ. However, the inferred
type of \texttt{B1.m()} is \texttt{<T> T -> T}, which is not a correct
method override for \texttt{A1.m()}.
Hence, we have to instantiate the type of \texttt{B1.m()} to
\texttt{Int -> Int}.

Conversely, for the right listing of Figure~\ref{fig:method-overriding}, GTI finds that
\texttt{A2.m} has the generic type \texttt{<T> T -> T} while \texttt{B2.m} has type
\texttt{Int -> Int}. Again, these types do not give rise to a correct method override and
GTI is forced to instantiate the type of \texttt{A2.m} to \texttt{Int -> Int}.

\todo[inline]{This example shows that GTI must first collect the constraints 
  for all methods \texttt{m{}} in a class hierarchy. Generalization can only happen after
  all constraints on \texttt{m}'s type have been considered.}

% Otherwise, we would get strange results. Consider extending the
% program with classes \texttt{A} and \texttt{B} by the following class.
% \begin{lstlisting}
% class C {
%   mbool(x) { return x.m(new Bool()); }
% }
% \end{lstlisting}
% Inferring the type of \texttt{C.mbool()} using \texttt{A.m()} would
% fail because \texttt{Int} and \texttt{Bool} are not
% unifiable. However, inferring the type of \texttt{C.mbool()} using
% \texttt{B.m()} (with the generic type) would succeed and yield the
% typing
% \begin{lstlisting}
% class C {
%   Bool mbool(B x) { return x.m(new Bool()); }
% }
% \end{lstlisting}


In full Java, type inference would have to offer two alternative
results: either two different
overloaded methods (one inherited and one local) in \texttt{B1}/\texttt{B2} or
impose the typing \texttt{Int B1.m(Int)} or \texttt{Int A2.m(Int)} to enforce correct overriding. 


\subsection{Inheritance and Generics}
\label{sec:inheritance-generics}


\begin{lstlisting}[float,caption={Function class}, label={lst:function-class}]
class Function<S,T> {
  T apply(S arg) { return this.apply (arg); }
}
\end{lstlisting}
Suppose we are given the generic FGJ class for modeling functions in Listing~\ref{lst:function-class}.
This code is constructed to serve as an ``abstract'' super class to derive more
interesting subclasses.
The class \texttt{Function<S,T>} must be presented in this explicit
way. Its type annotations \textbf{cannot} be inferred by GTI because
the use to the generic class parameters in the method type cannot be inferred from the
implementation.

If we applied GTI to the type-erased version of
Listing~\ref{lst:function-class}, the \texttt{apply} method would be considered a generic method: 
\begin{lstlisting}
  apply (arg) { ... }  --GTI-->    <A,B> B apply (A arg) { ... }
\end{lstlisting}
The typing of \texttt{apply} in Listing~\ref{lst:function-class} is an
instance of this result, so that completeness of GTI is preserved!

% While it is possible to  force GTI to infer the intended typing, it requires some awkward coding.
% \begin{lstlisting}
% class Function<S,T> {
%   S in;
%   T out;
%   dummy (arg) { return this.dummy(in); }
%   apply (arg) { return new Pair<>(this.out, this.dummy (arg)).fst; }
% }
% \end{lstlisting}

Now that we have the abstract class \texttt{Function<S,T>} at our
disposal, let us apply GTI to a class of boxed values with a
\texttt{map} function:
\begin{lstlisting}
class Box<S> {
  S val;
  map(f) {
    return new Box<>(f.apply(this.val));
} }
\end{lstlisting}
GTI finds the following constraints
\begin{itemize}
\item the return value must be of type \texttt{Box<T>}, for some type
  \texttt{T},
\item \texttt{T} is a supertype of the type returned by
  \texttt{f.apply},
\item \texttt{apply} is defined in class \texttt{Function<S1,T1>} with
  type \texttt{T1 apply(S1 arg)}, 
\item hence \texttt{T1 <: T} and \texttt{S <: S1} (because
  \texttt{this.val : S}),
\end{itemize}
and resolves them to the desired outcome where \texttt{T1=T} and
\texttt{S=S1} using the methods of
Simonet~\cite{DBLP:conf/aplas/Simonet03}. 
\begin{lstlisting}
class Box<S> {
  S val;
  <T> Box<T> map(Function<S,T> f) {
    return new Box<T>(f.apply<S,T>(this.val));
} }
\end{lstlisting}
But what happens if we add useful subclasses of \texttt{Function} as in
\begin{lstlisting}
class Not extends Function<Bool,Bool> {
  apply(b) { return b.not(); }
}
class Negate extends Function<Int,Int> {
  apply(x) { return x.negate(); }
}
\end{lstlisting}
\todo[inline]{Does this approach really make sense? If we have a
  subclass that overrides a method of the superclass, then the public
  interface should be the one of the superclass. Subsequently, the
  implementation of the method in the subclass should be checked
  against the type inferred for the superclass.}
If we rerun GTI with these classes, we now have additional
possibilities to invoke the \texttt{apply} method. With \texttt{Not}, we need to use the
generic type of \texttt{Function.apply()}, but instantiate it according to
\texttt{Function<Bool,Bool>}. Thus,
we obtain the constraints \texttt{Bool <: T} and \texttt{S <: Bool} for \texttt{T = Bool}
and \texttt{S = Bool}, which are both satisfiable. With
\texttt{Negate} we run into the same situation with the constraints
\texttt{Int <: Int} and \texttt{Int <: Int}.
% That means there is no way to
% \emph{directly} invoke \texttt{Not.apply} or \texttt{Negate.apply}
% from \texttt{Box.map}, but it may happen indirectly via inheritance.

Here is another subclass of \texttt{Function<S,T>} that we want
to consider.
\begin{lstlisting}
class Identity<S> extends Function<S,S> {
  S apply(S arg) { return arg; }
}
\end{lstlisting}
Here, we obtain the following type constraints
\begin{itemize}
\item \texttt{apply} is defined in class \texttt{Identity<S1>} with
  type \texttt{S1 apply (S1 arg)},
\item hence \texttt{S1 <: T} and \texttt{S <: S1}.
\end{itemize}
Resolving the constraints yields \texttt{S = T} thus the typing
\begin{lstlisting}
Box<S> map(Identity<S> f);
\end{lstlisting}
which is an instance of the previous typing.

\subsection{Multiple results}
\label{sec:multiple-results}
% Our global type inference algorithm is able to infer every type annotation.
% For the sake of simplicity we will only consider method types in this
% paper and assume that field types are specified.
\begin{figure}[tp]
  \begin{minipage}{0.49\linewidth}
\begin{lstlisting}
class List<A> {
  List<A> add(A item) {...}
  A get() { ... }
}
\end{lstlisting}
  \end{minipage}
  ~$\left|
  \begin{minipage}{0.49\linewidth}
\begin{lstlisting}
class Global{
  m(a){
    return a.add(this).get();
} }
\end{lstlisting}
  \end{minipage}\right.$
  \caption{Example for multiple inferred types}
  \label{fig:example-types-not-unique}
\end{figure}
Consider the classes \texttt{List<A>} and \texttt{Global} in Figure~\ref{fig:example-types-not-unique}.
Global type inference processes classes in order of
dependency. Class \texttt{Global} may depend on
class \texttt{List} because \texttt{List} defines methods \texttt{add} and
\texttt{get} which may be used in \texttt{Global}. The dependency is
not certain because, in general, there may be additional classes
providing methods \texttt{add} and \texttt{get}.

In the example, it is safe to assume that the types for the methods of class \texttt{List}
are already available, either because they are given (as in the code
fragment) or because they were inferred before considering class \texttt{Global}.

The method \texttt{m} in class \texttt{Global} first invokes
\texttt{add} on \texttt{a}, so the type of \texttt{a} as well as the
return type of \texttt{a.add(this)} must be
\texttt{List<T>}, for some \texttt{T}. As \texttt{this} has
type \texttt{Global}, it must be that \texttt{Global} is a
subtype of \texttt{T}. By the typing of \texttt{get()} we find that
the return type of method \texttt{m} is also \texttt{T}.

But now we are in a dilemma because FGJ only supports \emph{upper bounds} for
type variables,\footnote{Java has the same restriction. Lower bounds
  are only allowed for wildcards.} so that 
\texttt{Global <: T} is not a valid constraint in FGJ.
To stay compatible with this restriction, global type inference
expands the constraint by instantiating \texttt{T} with the (two) superclasses fulfilling
the constraint, \texttt{Global} and \texttt{Object}.  They give rise to two incomparable
types for 
\texttt{m}, \texttt{List<Global> -> Global} and
\texttt{List<Object> -> Object}. So there are two different FGJ
programs that are completions of the \texttt{Global} class.

GTI models these instances using the \emph{intersection type}
\texttt{List<Global> -> Global} \verb!&! \texttt{List<Object> -> Object}
for method \texttt{m} and the different FGJ-completions of class \texttt{Global} are
instances of the intersection type:
\begin{center}
  \begin{minipage}{0.49\linewidth}
\begin{lstlisting}
class Global {
  Global m(List<Global> a) {
    return a.add(this).get();
}
\end{lstlisting}
  \end{minipage}
  \begin{minipage}{0.49\linewidth}
\begin{lstlisting}
class Global {
  Object m(List<Object> a) {
    return a.add(this).get();
}
\end{lstlisting}
  \end{minipage}
\end{center}
% \begin{lstlisting}
% class Global {
%   <Global <: T> T m( List<T> a ) {
%     return a.add(this).get();
% }
% \end{lstlisting}
Additional classes in the program may further restrict the number of
viable types. Suppose we define a class \texttt{UseGlobal} as
follows:
\begin{lstlisting}
class UseGlobal {
  main() {
    return new Global().m(new List<Object>());
} }
\end{lstlisting}
Due to the dependency on \texttt{Global.m()}, type inference considers this class after class
\texttt{Global}. As it uses \texttt{m} at type
\texttt{List<Object> -> Object}, global type inference narrows the
type of \texttt{m} to just this alternative.

\subsection{Polymorphic recursion}
\label{sec:polym-recurs}
\begin{figure}[tp]
  \begin{minipage}{0.49\linewidth}
\begin{lstlisting}
class UsePair {
  <X,Y> Object prc(Pair<X,Y> p) {
    return this.prc<Y,X> (p.swap<X,Y>());
} }
\end{lstlisting}
  \end{minipage}
  ~$\left|
  \begin{minipage}{0.49\linewidth}
\begin{lstlisting}
class UsePair {
  prc(p) {
    return this.prc (p.swap());

} }
\end{lstlisting}
  \end{minipage}\right.$
  \caption{Example for polymorphic recursion}
  \label{fig:examples-poly-rec}
\end{figure}
A program uses \emph{polymorphic recursion} if there is a generic method that is invoked
recursively at a more specific type than its definition.
As a toy example for polymorphic recursion consider the class \texttt{UsePair} in FGJ with a
generic method \texttt{prc} that invokes itself
recursively on a swapped version of its argument pair
(Figure~\ref{fig:examples-poly-rec}, left).
This method makes use of polymorphic recursion because the type of the
recursive call is different from the declared type of the method. More
precisely, the declared argument type is \texttt{Pair<X,Y>} whereas
the argument of the recursive call has type
\texttt{Pair<Y,X>}---an instance of the declared type.

For this particular example, global type inference succeeds on the
corresponding stripped program shown in
Figure~\ref{fig:examples-poly-rec}, right, but it yields a more restrictive
typing of \texttt{<X> Object prc (Pair<X,X> p)} for the method. 
A minor variation of the program with a non-variable instantiation makes type inference fail entirely:
\begin{lstlisting}
class UsePair2 {
  <X,Y> Object prc(Pair<X,Y> p) {
    return this.prc<Y,Pair<X,Y>> (new Pair (p.snd, p));
  }
}
\end{lstlisting}



Polymorphic recursion is known to make type inference intractable
\cite{DBLP:journals/toplas/Henglein93,DBLP:journals/toplas/KfouryTU93}
because it can be reduced to an undecidable semi-unification problem
\cite{DBLP:journals/iandc/KfouryTU93}. However, \emph{type checking} with
polymorphic recursion is tractable and routinely used in languages like Haskell
and Java.

GTI does not infer method types with polymorphic recursion. Inference either fails or
returns a more restrictive type. Methods making use of this features need to supply
explicit typings. 


% Features:
% \begin{itemize}
% \item overloaded methods
% \item class header is complete in the form
%   $\mathtt{class}\ C\langle\overline X <: \overline N\rangle <: N \{
%   \overline T\ \overline f;\ K\ \overline M \}$
% \item constructor fully typed (as it is determined by the types of the
%   fields and the supertype)
% \item method signatures may be omitted, i.e., 
%   $M\ ::=\ m(\overline x) \{ \ \mathtt{return}\ e;\ \}$
% \item polymorphic recursive methods must be annotated.
% \end{itemize}

%%% Local Variables:
%%% mode: latex
%%% TeX-master: "TIforGFJ"
%%% End:


\section{Preliminaries}
\label{sec:preliminaries}

\subsection{Featherweight Java Typing Rules}
As \TFGJ omits some type annotations with respect to FGJ, we cannot use
FGJ's typing rules directly. Hence, we give new typing rules which are
close to FGJ's original rules, but adapted to cater for omitted types.

% The input for our type inference algorithm is based on Featherweight Generic Java (FGJ).
% FGJ is defined by syntax and typing rules.
% We already changed the syntax to allow typeless FGJ programs as input for our algorithm.
% Additionally we alter the typing rules slightly, which is presented in this chapter.

%Our type inference algorithm takes typeless FGJ classes as input.
%The generated output is correct FGJ, although we have to alter some rules.
%This chapter defines the typing rules for our version of FGJ,
%which our type inference algorithm is able to process.

Most rules stay the same as in FGJ, but we note the following changes:
\todo[inline]{PT: Why and how is overloading supported? \\
  GT-NEW should use \texttt{C} in place of \texttt{N}: see suggested GT-NEW'. \\
  The font for types and expressions changes between math and typewriter. \\
  $\triangle$ should be $\triangle$ (changed)
}
\begin{itemize}
\item We remove the \texttt{MT-CLASS} rule.
\item The \texttt{GT-METHOD} rule is changed.
\item Overriding of methods is removed for our typeless FGJ version. Therefore also the rule \texttt{MT-SUPER} is removed.
\item The \texttt{GT-INVK} rule is changed to support overloading.
\end{itemize}

\fbox{
\begin{minipage}{\textwidth}
  \begin{small}
  \textbf{Subtyping:}\\[1em]
  \begin{tabularx}{\textwidth}{X c X r}
%  $
%  \ddfrac{\texttt{class}\ \exptype{C}{\ol{X} \triangleleft \ol{N}} \triangleleft N \{ \ol{S}\ \ol{f};\ K \ \ol{M} \}
%  \quad \quad m \in \ol{M}}
%  {\mathit{mtype}(m, \exptype{C}{\ol{Z}}) = \mathit{mtype}(m, [\ol{T}/\ol{X}]N)}
%  $
%  & MT-SUPER \\
%& \\

& $\mathtt{
\triangle \vdash T <: T
}
$
&   & S-REFL \\

& \\
& $\mathtt{\ddfrac{
    \triangle \vdash S <: T \quad \quad \triangle \vdash T <: U
}{
    \triangle \vdash S <: U
}}$ & & S-TRANS \\

& \\

& $\mathtt{
\triangle \vdash X <: \triangle(X)
}$ & & S-VAR \\
& \\
& $\mathtt{\ddfrac{
  \texttt{class}\ \exptype{C}{\ol{X} \triangleleft \ol{N}}
  \triangleleft \mv N \set{ \ldots }
}{
  \triangle \vdash \exptype{C}{\ol{T}} <: [\ol{T}/\ol{X}]\mv N
}}$ & & S-CLASS 
\end{tabularx}
\end{small}
\end{minipage}
}

\fbox{
\begin{minipage}{\textwidth}
  \begin{small}
  \textbf{Well-formed types:}\\[1em]
\begin{tabularx}{\textwidth}{X c X r}
  & $\mathtt{
    \triangle \vdash \texttt{Object}\ \text{ok}
    }$ & & WF-OBJECT\\

& \\
& $\mathtt{\ddfrac{
    X \in \textit{dom}(\triangle)
}{
    \triangle \vdash X \ \text{ok}
}
}
$ & & WF-VAR \\
& \\
& $\mathtt{\ddfrac{\begin{array}{c}
\texttt{class}\ \exptype{C}{\ol{X} \triangleleft \ol{N}} \triangleleft N \{ \ldots \} \\
\triangle \vdash \ol{T} \ \text{ok} \quad \quad \triangle \vdash \ol{T} <: [\ol{T}/\ol{X}]\ol{N}
\end{array}
}{
\triangle \vdash \exptype{C}{\ol{T}} \ \text{ok}
}
}
$ & & WF-CLASS
\end{tabularx}
\end{small}
\end{minipage}
}


\fbox{
\begin{minipage}{\textwidth}
\begin{small}
\textbf{Expression Typing:}\\
\begin{tabularx}{\textwidth}{c X r}
%\begin{tabular}{l@{\quad}l}
  $\mathtt{
\triangle ; \Gamma \vdash x : \Gamma(x)
}$ & & GT-VAR \\
& \\

$\mathtt{\ddfrac{\Gamma \vdash e_0:T_0 \quad \quad \mathit{fields}(\mathit{bound}_\triangle(T_0)) = \overline{T} \ \overline{f}}
{\Gamma \vdash e_0.\mathtt{f}_i : T_i}
}
$ & & GT-FIELD \\
& \\
$\mathtt{ \ddfrac{\triangle \vdash {\mv N} \ \texttt{ok} \quad 
  \mv N = \exptype{C}{\ol{X}} \quad
  \textit{fields}(\mv N) = \ol{T}\ \ol{f} \quad 
  \triangle; \Gamma \vdash \ol{e} : \ol{S} \quad \triangle \vdash \ol{S} <: \ol{T}
}{
  \triangle; \Gamma \vdash \texttt{new C}(\ol{e}): \mv N
}
}$ & & GT-NEW \\

& \\

$\mathtt{\ddfrac{\begin{array}{c}
  \mathtt{\exptype{}{\ol{Y} \triangleleft \ol{P}} \ol{U} \to U \in \mathit{mtype}(m, \mathit{bound}_\triangle (T_0))} \\
  \mathtt{\triangle; \Gamma \vdash e_0 : T_0 } \quad \quad
  \mathtt{\triangle \vdash \ol{V} \ \texttt{OK} } \quad \quad
  \mathtt{\triangle \vdash \ol{V} <: [\ol{V}/\ol{Y}]\ol{P} } \\ %\quad \quad
  \mathtt{\triangle; \Gamma \vdash \ol{e} : \ol{S} } \quad \quad
  \mathtt{\triangle \vdash \ol{S} <: [\ol{V}/\ol{Y}]\ol{U}}
\end{array}}
{\triangle; \Gamma \vdash \mathtt{e_0.\mv{m}(\overline{e}) : [\ol{V}/\ol{Y}]U }}
}$ & & GT-INVK\\

& \\

$\ddfrac{\mathtt{\triangle; \Gamma \vdash e_0 : T_0 \quad \quad \triangle \vdash \textit{bound}_\triangle(T_0) <: N}}
{\mathtt{\triangle; \Gamma \vdash (N) e_0 : N}}
$ & & GT-UCAST \\

& \\

$ \ddfrac{\begin{array}{c}
  \mathtt{\triangle; \Gamma \vdash e_0 : T_0 \quad \quad \triangle \vdash N\ \texttt{ok} \quad \quad \triangle \vdash N <: \textit{bound}_\triangle(T_0) } \\
  \mathtt{N = \exptype{C}{\ol{T}} \quad \quad \textit{bound}_\triangle(T_0) = \exptype{D}{\ol{U}}  \quad \quad \textit{dcast}(C,D)}
\end{array}
}{\mathtt{\triangle; \Gamma \vdash (N) e_0 : N}}$ & & GT-DCAST \\

& \\

$\ddfrac{\begin{array}{c}
  \mathtt{\triangle; \Gamma \vdash e_0 : T_0 \quad \quad \triangle \vdash N\ \texttt{ok} \quad \quad N = \exptype{C}{\ol{T}}  } \\
  \mathtt{\textit{bound}_\triangle(T_0) = \exptype{D}{\ol{U}} \quad \quad C \ntrianglelefteq D \quad \quad D \ntrianglelefteq C \quad \quad \textit{stupid warning}}
\end{array}}
{\mathtt{\triangle; \Gamma \vdash (N) e_0 : N}}
$ & & GT-SCAST 
\end{tabularx}
\end{small}
\end{minipage}
}



\fbox{
\begin{minipage}{\textwidth}
\begin{small}
\textbf{Method Typing:}\\[1em]
\begin{tabularx}{\textwidth}{c X r}
  $\mathtt{\ddfrac{\begin{array}{c}
    \mathtt{\texttt{class}\ \exptype{C}{\ol{X} \triangleleft \ol{N}} \triangleleft N \{ \ldots\ \ol{M}\ \ldots\} }\\
    \mathtt{\textit{mtype}(m, \exptype{C}{\ol{X}}) = \ol{T_m} \to T_m \textrm{ for } m \in \ol{M}}\\
  \mathtt{\triangle \vdash \ol{X} <: \ol{N}  \quad \quad 
  \triangle \vdash \ol{T}, T \ \texttt{ok} } \\
  \mathtt{\triangle ; \ol{x}:\ol{T_\mathit{meth}},\ this : \exptype{C}{\ol{X}} \vdash e_0 : S \quad \quad
  \triangle \vdash S <: T_\mathit{meth} } \\
  \end{array}} {
  %{\exptype{}{\ol{Y} \triangleleft \ol{P}}\ T \ m(\ol{T}\ \ol{x}) \{
  %\texttt{return} \ e_0; \} \ \texttt{OK IN}\ \exptype{C}{\ol{X} \triangleleft
  %\ol{N}}}
   \exptype{}{\ol{Y}} T_\mathit{meth}\ \texttt{meth}(\ol{T_\mathit{meth}}\ \ol{\mathtt{x}}) \{\texttt{return}\ \mathtt{e}_0;\}
  \texttt{ OK in }\exptype{C}{\ol{X} \triangleleft \ol{N}} 
  }}$ & & GT-METHOD\\

\end{tabularx}\\[1em]
\textbf{Class Typing:}\\[1em]
  \begin{tabularx}{\textwidth}{c X r}
  %GT-CLASS: - This rule is modified by us
  $\ddfrac{
    \begin{array}{c}
      \mathtt{\ol{X} <: \ol{N} \vdash \ol{N}, N, \ol{T}\ \texttt{ok}
      \quad\quad fields(\mathtt{N}) = \ol{\mathtt{U}} \ \ol{\mathtt{g}}}\\
      \mathtt{\exptype{}{\ol{Y}} T_\mathit{m}\ \texttt{m}(\ol{T_\mathit{m}}\ \ol{\mathtt{x}}) \{\texttt{return}\ \mathtt{e}_0;\}
  \texttt{ OK in }\exptype{C}{\ol{X} \triangleleft \ol{N}}  \textrm{ for all } m
      \in \ol{\mathtt{M}}}\\
      \mathtt{K = C(\overline{D} \ \overline{g}, \overline{C} \ \overline{f}) \{ \texttt{super}(\overline{g}); \ \texttt{this}.\overline{f}=\overline{f}; \} }\\
    %\quad \quad \overline{M} \ \texttt{OK IN C} \\
  \end{array}
    }
  {\mathtt{\texttt{class C extends D}\{ \overline{C} \ \overline{f}; \ K \ \overline{M} \} \ \texttt{OK}}}
  $ & & GT-CLASS
\end{tabularx}
\end{small}
\end{minipage}
}


\fbox{
\begin{minipage}{\textwidth}
  \begin{small}
  \textbf{Method type lookup:} \\[1em]
\begin{tabularx}{\textwidth}{cXr}
  $\ddfrac{\begin{array}{c}
  \texttt{class}\ \exptype{C}{\ol{X} \triangleleft \ol{N}}\triangleleft
             N\{ \overline{C} \ \overline{f}; \ K \ \overline{M} \} \ \mathtt{OK}\\
  \exptype{}{\ol{Y}} T_\mathit{m}\ \texttt{m}(\ol{T_\mathit{m}}\ \ol{\mathtt{x}}) \{\texttt{return}\ \mathtt{e}_0;\}
  \texttt{ OK in }\exptype{C}{\ol{X} \triangleleft \ol{N}}
  \end{array}} {
  %{\exptype{}{\ol{Y} \triangleleft \ol{P}}\ T \ m(\ol{T}\ \ol{x}) \{ \texttt{return} \ e_0; \} \ \texttt{OK IN}\ \exptype{C}{\ol{X} \triangleleft \ol{N}}}
  \textit{mtype}(m, \exptype{C}{\ol{Z}}) = [\ol{Z} / \ol{X}](\exptype{}{\ol{Y}} \ol{T} \to T)
  }$ & & MT-CLASS \\
\end{tabularx}\\[1em]
\textbf{Valid method overriding:}\\[1em]
\begin{tabularx}{\textwidth}{c}
  $\ddfrac{
    \textit{mtype}(m, N) = \exptype{}{\ol{Y} \triangleleft \ol{Q}} \ol{U} \to U \ 
    \text{implies}\ \ol{P}, \ol{T} = [\ol{Y}/\ol{Z}](\ol{Q},\ol{U}) \ 
   \text{and}\ \ol{Y} <: \ol{P} \vdash T_0 <: [\ol{Y}/\ol{Z}]U_0 
  } {
  %{\exptype{}{\ol{Y} \triangleleft \ol{P}}\ T \ m(\ol{T}\ \ol{x}) \{ \texttt{return} \ e_0; \} \ \texttt{OK IN}\ \exptype{C}{\ol{X} \triangleleft \ol{N}}}
  \textit{override}(m, N, \exptype{}{\ol{Y} \triangleleft \ol{P}} \ol{T} \to T_0)
  }$ 
\end{tabularx}\\[1em]
  \textbf{Method body lookup:} \\[1em]
\begin{tabularx}{\textwidth}{cXr}
  $\ddfrac{\begin{array}{c}
  \texttt{class}\ \exptype{C}{\ol{X} \triangleleft \ol{N}}\triangleleft
             N\{ \overline{S} \ \overline{f}; \ K \ \overline{M} \} \\
  \mathtt{\exptype{}{\ol{Y} \triangleleft \ol{P}}\ T\ \texttt{m}(\ol{T}\ \ol{\mathtt{x}}) \{\texttt{return}\ \mathtt{e}_0;\}}
  \in \ol{M}
  \end{array}} {
  %{\exptype{}{\ol{Y} \triangleleft \ol{P}}\ T \ m(\ol{T}\ \ol{x}) \{ \texttt{return} \ e_0; \} \ \texttt{OK IN}\ \exptype{C}{\ol{X} \triangleleft \ol{N}}}
  \textit{mbody}(m, \exptype{C}{\ol{Z}}) = [\ol{Z} / \ol{X}](\exptype{}{\ol{Y}} \ol{T} \to T)
  }$ & & MB-CLASS \\
  & & \\
  $\ddfrac{
  \mathtt{\texttt{class}\ \exptype{C}{\ol{X} \triangleleft \ol{N}}\triangleleft
             \ol{N}\ \{ \overline{S} \ \overline{f}; \ K \ \overline{M} \} \quad \quad m \notin \ol{M}}
  } {
  %{\exptype{}{\ol{Y} \triangleleft \ol{P}}\ T \ m(\ol{T}\ \ol{x}) \{ \texttt{return} \ e_0; \} \ \texttt{OK IN}\ \exptype{C}{\ol{X} \triangleleft \ol{N}}}
  \mathtt{\textit{mbody}(m, \exptype{C}{\ol{T}}) = \textit{mbody}(m, [\ol{T}/\ol{X}]N)}
  }$ & & MB-SUPER \\
\end{tabularx}
\end{small}
\end{minipage}
}

\medskip
In FGJ, \textit{mtype} is a global function which returns the type for every method in every class.
In our system, \textit{mtype} is also a global function but it is differed between
methods declared in the actual class and methods from other classes.

The main difference between the type system of FGJ and our type system is that
in the \texttt{MT-CLASS} rule the correspondig class has to be proved as \texttt{OK}
by the \texttt{GT-CLASS} rule which means that for all methods of the class a type has to
be assumed and proved as correct by the \texttt{GT-METHOD} rule.
In this rule
for all methods in the actual class a type is assumed by the
declaration of the \texttt{mtype} function.
These assumptions have to be proved as correct. Then the assumed type is
\texttt{OK} in the correspondig class. Additionally, the containing type
variables are generalized (marked by $\exptype{}{\ol{Y}}$). The type variables
$\ol{\mathtt{Y}}$ are not generalized during the prove. The reason is the
undecidablity of polymorphic recursion (cp. Sec. \ref{sec:polym-recurs}).



%MT-CLASS: - This rule is not used by us
%\begin{align*}
%\ddfrac{\begin{array}{c}
%\texttt{class} \ \exptype{C}{\ol{X} \triangleleft \ol{N}} \ \{ \ol{S} \ \ol{f}; \ K \ol{M} \}\\
%\exptype{}{\ol{Y} \triangleleft \ol{P}} U \ m(\ol{U} \ \ol{x})\{  \texttt{return} \ e; \} \in \ol{M}
%\end{array}}
%{\mathit{mtype}(m, \exptype{C}{\ol{Z}}) = [\ol{T}/\ol{X}](\exptype{}{\ol{Y} \triangleleft \ol{P}} \ol{U} \to U)}
%\end{align*}

%GT-METHOD:
%\begin{align*}
%\ddfrac{\begin{array}{c}
%\triangle \vdash \ol{X} <: \ol{N}, \ol{Y} <: \ol{P} \quad \quad 
%\triangle \vdash \ol{T}, T, \ol{P} \ \texttt{ok} \\
%\triangle ; \ol{x}:\ol{T}, this : \exptype{C}{\ol{X}} \vdash e_0 : S \quad \quad
%\triangle \vdash S <: T \\
%\texttt{class}\ \exptype{C}{\ol{X} \triangleleft \ol{N}} \triangleleft N \{ \ldots \} \quad \quad
%\textit{override}(m, N, \exptype{}{\ol{Y} \triangleleft \ol{P}} \ol{T} \to T)
%\end{array}}
%{\exptype{}{\ol{Y} \triangleleft \ol{P}}\ T \ m(\ol{T}\ \ol{x}) \{ \texttt{return} \ e_0; \} \ \texttt{OK IN}\ \exptype{C}{\ol{X} \triangleleft \ol{N}}}
%\end{align*}

%%% Local Variables:
%%% mode: latex
%%% TeX-master: "TIforGFJ"
%%% End:

%we have to show completeness and Soundness for the new type rules
%we only have to show this for the changed rules (for preservation), because they use induction on the derivations
% what is the consequence on method overloading 
% the Proof of Theorem 3.4.1 works also with the mtype overload rule
\begin{theoremAndi}
Subject reduction: if $\triangle; \Gamma \vdash e : T$ and $e \to e'$,
then $\triangle; \Gamma \vdash e':T'$ for some $T'$ such that $\triangle \vdash T' <: T$.
\end{theoremAndi}



\subsection{Input assumptions}
% No overloaded methods
% No Or-Constraints
The input is a FGJ program lacking the type assignments for method parameters and method return types.


The Typeless Featherweight Generic Java (TFGJ) syntax is different in that from normal Featherweight Generic Java (FGJ) that it is possible
to omit the type annotations for methods.%, except the ones for casts and \texttt{new} calls.
We declare the syntax for TFGJ as follows:

\begin{align*}
  T ::=& X \, | \, N \\
  N ::=& \exptype{C}{\ol{T}}\\
  L ::=& \mathtt{class } \ \exptype{C}{\ol{X} \triangleleft \ol{N}} \ \triangleleft \ N \{ \overline{T} \ \overline{f}; \, K \, \overline{M} \} \\
  K ::=& C(\overline{T} \ \overline{f})\{\mathtt{super}(\overline{f}); \ \mathtt{this}.\overline{f}=\overline{f};\} \\
  %M ::=& \exptype{}{\ol{X} \triangleleft \ol{X}}\ T \ \mathtt{m}(\overline{T} \, \overline{x})\{ \mathtt{ return }\ e; \} \\
  M ::=& \mathtt{m}(\overline{x})\{ \mathtt{ return }\ e; \} \\
  e ::=& \mathtt{this} \, | \, x \, | \, e.f \, | \, e.\mathtt{m}(\overline{e}) \, | \, \mathtt{new }\ N(\overline{e})
  %M ::=& T \ \mathtt{m}(\overline{T} \, \overline{x})\{ \mathtt{ return }\ e; \} \\
  %e ::=& \mathtt{this} \, | \, x \, | \, e.f \, | \, e.\exptype{\mathtt{m}}{\ol{T}}(\overline{e}) \, | \, \mathtt{new }\ C(\overline{e}) \, | \, (C) e \\
\end{align*}

All type annotations in our TFGJ language can be omitted ($T = \epsilon$).
The only exception are fields which must be given a concrete type.

Another difference to the syntax of FGJ is that we added the special variable \texttt{this} to the syntax.
FJ treats \texttt{this} as a normal variable
but our algorithm treats it as a special variable which always has a predetermined type;
the type of the class it is used in.

Type inference for polymorphic recursion is undecidable.
Therefore we have to alter the FGJ typing rules to exclude polymorphic recursion in method calls:
\begin{enumerate}
    \item The \texttt{MT-CLASS} rule is removed.
    \item We change the \texttt{GT-METHOD} rule:
GT-METHOD:
\begin{align*}
\ddfrac{\begin{array}{c}
\triangle \vdash \ol{X} <: \ol{N}, \ol{Y} <: \ol{P}  \quad \quad 
\triangle \vdash \ol{T}, T \ \texttt{ok} \\
\triangle ; \ol{x}:\ol{T}, this : \exptype{C}{\ol{X}} \vdash e_0 : S \quad \quad
\triangle \vdash S <: T \\
\texttt{class}\ \exptype{C}{\ol{X} \triangleleft \ol{N}} \triangleleft N \{ \ldots \} \quad \quad
%\textit{override}(m, N, \exptype{}{\ol{Y} \triangleleft \ol{P}} \ol{T} \to T)
\ol{T} \to T  \in \textit{mtype}(m, \exptype{C}{\ol{X}}) \\
%\textit{override}(m, N,)
\end{array}}{
%{\exptype{}{\ol{Y} \triangleleft \ol{P}}\ T \ m(\ol{T}\ \ol{x}) \{ \texttt{return} \ e_0; \} \ \texttt{OK IN}\ \exptype{C}{\ol{X} \triangleleft \ol{N}}}
\textit{mtype}(m, \exptype{C}{\ol{Z}}) = [\ol{Z} / \ol{X}](\exptype{}{\ol{Y}} \ol{T} \to T)
}
\end{align*}
\end{enumerate}

\subsection{Principal Type}
Featherweight Generic Java has no unique principal typing.
We can show this easily with an example.
We try to find the principal type for the method \texttt{method1}.
\begin{lstlisting}
class Global{
  method1(a){
    a.add(this);
    return a.get();
  }
}
class List<A> {
  add(A item){...}
  A get() ...
}
\end{lstlisting}
In \texttt{method1} neither the return type nor the type for the parameter \texttt{a} are specified.
The return type of the method depends on the type of \texttt{a}.
If we set in the type \texttt{List<Object>} here, then \texttt{method1} would return \texttt{Object}.
The type \texttt{List<Global>} would also be correct.
Then the return type of the method can also be the type \texttt{Global}.

The principal type would either be an intersection type or the method \texttt{method1} has to be overloaded.
FGJ neither supports intersection types nor overloading.
Therefore we cannot set in the principal type and have to stick with one of the possible solutions,
for example\\
\texttt{List<Global> method1(List<Global> a)}.

It is possible for a class to have multiple principal type solutions.
This can lead to a type error when compiling multiple classes.
\begin{lstlisting}
  class Global{
    method1(a){
      a.add(this);
      return a.get();
    }
  }
  class Class2{
    Object test(){
      return new Global().method1(new List<Object>());
    }
  }
\end{lstlisting}
Our type inference algorithm is able to infer all of the principal type solutions, but only one of them can be set in.
If we set in \texttt{List<Global>} as the parameter type for the \texttt{method1},
then the class \texttt{Class2} would lead to a type error.
In this case the type inference algorithm has to try another type solution for \texttt{method1}
to render the program type correct.

%%% Local Variables:
%%% mode: latex
%%% TeX-master: "TIforGFJ"
%%% End:


\section{Type inference algorithm}
\label{sec:type-infer-algor}
In this chapter we present our type inference algorithm.
The algorithm is split into following parts, which are executed on a single class at a time:

\begin{enumerate}
\item Create assumptions and subtype relation
\item Constraint generation with \textbf{FJTYPE}
\item Unification of those constraints
\item Set in principal type solution
\end{enumerate}

The Unify algorithm returns a set of possible type solutions.
This means that there are possibly multiple type solutions for each method.
The last step has to choose the principal type out of those possibilities.

\subsection{Process multiple classes}
The algorithm processes only one class at a time.
Only the first step creating the type assumptions is able to consider other classes as well.

Nevertheless we allow the input to consist out of multiple classes.
But in that case there are some additional requirements for the input.
%TODO: these requirements can also be in "Preliminaries"

We assume that the algorithm are given the input classes in the correct order $C_1, \ldots C_2$.
Hereby there must exist a correct typisation for the class $C_1$ when existing on its own.
This is also regulated by our typing rules (see chapter \ref{chapter:type-rules}).

\textbf{Example:}
We give an example for a incorrect input program for our algorithm, where none of the given classes cannot be compiled on its own.
  \begin{figure}
    \centering
    \begin{minipage}{.5\textwidth}
      \centering
      \begin{lstlisting}
class C1 extends Object {
  C1(){ super(); }
  m1(){ return new C2().m2(); }
}
class C2 extends Object{
  C2(){ super(); }
  m2(){ return new C2().m1(); }
}
          \end{lstlisting}
            \caption{Invalid typeless GFJ program}
      \label{fig:invalidinput}
    \end{minipage}%
    \begin{minipage}{.5\textwidth}
      \centering

    \begin{lstlisting}
class C1 extends Object {
  C1(){ super(); }
}
class C2 extends Object{
  C2(){ super(); }
  m1(){ return new C2().m2(); }
  m2(){ return new C2().m1(); }
}
            \end{lstlisting}
          \caption{Correct typeless GFJ program}
      \label{fig:correctinput}
    \end{minipage}
\end{figure}

The problem in figure \ref{fig:invalidinput} is a circular method call graph.
Method \texttt{m1} calls method \texttt{m2} and other way round.
Our typing rules demand that one of the two classes has to only use his internal methods.
Figure \ref{fig:correctinput} shows a possible way to alter the incorrect input to make it comply with our typing rules.
The method \texttt{m1} was moved into the class \texttt{C2}.
Now both methods still call each other, but they are inherited by the same class.

Another problem we face when compiling multiple classes is the fact that there can be more than a single principal typing for a method.
When considering only one class at once it is not possible to set in the correc type right away.
This problem can be solved with backtracking.
Whenever the \textbf{Unify} algorithm gives more than one type solution, we pick the first one and continue.
If the algorithm fails at some point it has to backtrack to this point and try one of the other solutions.

\subsection{Generate Assumptions}
% Every empty Type T in the input is assigned a type variable.
% Assumptions saves every field, method and the class subtype relation

%Generate subtype relationships:

Generating assumptions consists of two parts.
At first we add type variables to the untyped class.
The second part generates the assumption set.
This is the same algorithm for the already typed classes as for the 
new untyped class, which is now equipped with type variables.

\begin{enumerate}
\item Every missing type in the input class gets assigned a fresh type variable.
For methods:
\begin{align*}
  \ddfrac{
  m(\ol{x}) \{ \ldots \} \quad \quad A \cup \ol{A} \ \text{are fresh type variables}
  }{
  A m(\ol{A}\ \ol{x}) \{ \ldots \}
  }
  \end{align*}
  For fields:
\begin{align*}
  \ddfrac{
  \texttt{class}\ \exptype{C}{\ol{X}} \{ \ol{f}; \quad \ldots \} \quad \quad \ol{F} \ \text{are fresh type variables}
  }{
    \texttt{class}\ \exptype{C}{\ol{X}} \{ \ol{F} \ \ol{f}; \quad \ldots \}
  }
\end{align*}
\item We define the two functions $\textit{ftype}_\textit{Ass}$ and $\textit{mtype}_\textit{Ass}$.
Both functions return a set of all types for a method \texttt{m} or a field \texttt{f}.
This is due to the fact that there can be multiple methods and fields with the same name.
\begin{align*}
  %TODO: fresh type variables for generic variables:
  \ddfrac{
    class\ \exptype{C}{\ol{X} \triangleleft \ol{N}}\ \{\ \ol{N}\ \ol{f};\ K\ \ol{M}\ \} \quad \quad
    \exptype{}{\ol{Y}}\ U\ \texttt{m}(\ol{U}\ \ol{x}) \{ \ldots \} \in \ol{M}
  }{
    \textit{mtype}_\textit{Ass}(\texttt{m}, \exptype{C}{\ol{X} \triangleleft \ol{N}}) =  \set{\exptype{}{\ol{Y}} (\ol{U} \to U )}
  }
\end{align*}
\begin{align*}
  \ddfrac{
    class\ \exptype{C}{\ol{X} \triangleleft \ol{N}}\ \{\ \ol{T}\ \ol{f};\ K\ \ol{M}\ \} \quad \quad
    T\ \texttt{f} \in \ol{f}
  }{
    \textit{ftype}_\textit{Ass}(\texttt{f}, \exptype{C}{\ol{X} \triangleleft \ol{N}}) = T
  }
\end{align*}
\item If the input for the type inference algorithm consists out of multiple classes we compile them one by one.
Additionally we add the types of the already compiled classes to the assumption set.
Therefore it is possible to have intersection types already in the assumptions.
\end{enumerate}


\subsection{FJTYPE}
The \textbf{FJTYPE} algorithm produces two kinds of constraints.
\begin{description}
\item[Constraint] A constraint consists of two types or type variables and an operator.
The operator can either be a $\doteq$ (same type) or $\lessdot$ (subtype).
Example: $(a \lessdot \mathtt{Object})$, means that the type variable $a$ should be a subtype of \texttt{Object}.
\item[OrConstraint] An OrConstraint consists out of multiple constraint sets.
For example $\textbf{OrConstraint}(\{ \ \{ (a \lessdot b), (a \leq \mathtt{Object}) \} \ , \ \{ (a \lessdot b)\} \ \})$
is an Or-Constraint consisting of two constraint sets.
\end{description}

Before the algorithm starts we equip every untyped method with type variables.
Every method parameter gets a unique type variable as a type aswell as every method gets a unique type variable as a return type.
After our algorithm found a correct typisation we replace the type variables with the inferred types and generate a GFJ program.

The algorithm \textbf{FJTYPE} is given as follows:

\textbf{FJTYPE}:
$
\texttt{Class} \rightarrow \texttt{Constraints}\\
 \begin{array}{@{}l@{}l@{}l}
 \textbf{FJT}&\textbf{Y} & \textbf{PE}(\mathtt{class } \ C \ \mathtt{ extends } \ D \{ \overline{T} \ \overline{f}; \, K \, \overline{M} \}) =\\
& \multicolumn{2}{@{}l@{}}{ \{ \ \textbf{TYPEMethod}(\{ \mathtt{this} : C \}, m_i) \quad | \quad m_i \in \overline{M} \ \} }\\ 
\end{array}$

The \textbf{FJTYPE} function gets called for every class in the input.
This function accumulates all the constraints generated from calling the
\textbf{TYPEMethod} function for each method declared in the given class.

$\textbf{TYPEMethod}:\texttt{TypeAssumptions} \times
\texttt{Method} \rightarrow \texttt{Constraints}\\
\begin{array}{@{}l@{}l@{}l}
\textbf{TY}& \textbf{PE} & \textbf{Method} (Ass, T_r \ \mathtt{m}(\overline{T} \, \overline{x})\{ \mathtt{ return }\ e; \}) =\\
& \textbf{let}
& Ass_m = Ass \cup \{ \overline{T} : \overline{x} \}\\
& & \ul{(e:rty, ConS)} = \textbf{TYPEExpr}(Ass_m, e)\\
& \mathbf{in}
& (ConS \cup (rty \lessdot T_r))\\
\end{array}
$

The \textbf{TYPEMethod} function for methods just calls the \textbf{TYPEExpr} function with the
return expression. It is significant to note that it adds the assumptions for the method parameters to the global assumptions before passing them to \textbf{TYPEExpr}.
%and the global assumptions plus the assumptions for the method parameters.

\smallskip

In the following we define the \textbf{TYPEExpr} function for every possible expression:

\smallskip

$\textbf{TYPEExpr}:\texttt{TypeAssumptions} \times
\texttt{Expression} \rightarrow \texttt{Type} \times \texttt{Constraints}\\
\begin{array}{@{}l@{}l}
\textbf{TY} \textbf{PE} & \textbf{Expr} (Ass, \mathtt{this}) = (t , \{\})\\
& \textbf{with } (\mathtt{this} : t) \in Ass 
\end{array}
$
\smallskip
$\begin{array}{@{}l@{}l}
\textbf{TY} \textbf{PE} & \textbf{Expr} (Ass,x) = (t , \{\})\\
& \textbf{with } (x : t) \in Ass 
\end{array}
$

\smallskip

$\begin{array}{@{}l@{}l@{}l}
\textbf{TY}& \textbf{PE} & \textbf{Expr} (Ass, e.f) = \\
& \textbf{let} % \\
% &
& (rty, ConS) = \textbf{TYPEExpr}(Ass, e),\\
& & \textbf{fresh} = \text{a mapping from each variable in}\ \ol{X} \ \text{to a fresh type variable},\\
& & Ass_{f} = \textit{ftype}_{Ass}(\textit{bound}_\triangle(f)) = \exptype{C}{\ol{X} \triangleleft \ol{N}} \to T \\
& & Cons_{f} = \{\ rty \doteq \exptype{C}{\textbf{fresh}(\ol{X})}, a \doteq \textbf{fresh}(T)\},\\
& & \begin{array}{@{}l@{}l}
  Cons_{f} = \{\ & rty \doteq \exptype{C}{\ol{X}}, a \doteq T \\
              & |\, \text{for every field}\ \exptype{}{\ol{Y}} T\ \texttt{m}(\ol{T}\  \ol{p}) \in \ol{M} \} \\%, \textbf{fresh}(\ol{X}) \lessdot \textbf{fresh}(\ol{N})\},\\
              & ,\, \text{in every class}\ \exptype{C}{\ol{X}} \{ \ldots \ol{M} \ldots \}\ \text{in the input} \} 
            \end{array}\\
%& & OrCons = \{ \{ rty \doteq cl, a \doteq t_f \} \ | \ cl.f : t_f \in Ass \},\\
& \mathbf{in}% \\
% &
& (a, ConS \cup Cons_{f})\\
& & \mathit{where\ } a \mathit{\ is\ a\ fresh\
  type\ variable}\\ 
\end{array}
$

\smallskip

$\begin{array}{@{}l@{}l@{}l}
\textbf{TY}& \textbf{PE} & \textbf{Expr} (Ass, e.\mathtt{m}(\overline{e}) ) = \\
& \textbf{let} % \\
% &
& (rty, ConS) = \textbf{TYPEExpr}(Ass, e),\\
& & \forall e_i \in \overline{e} : (pt_i, ConS_i) = \textbf{TYPEExpr}(e_i)  ,\\
& & \begin{array}{@{}l@{}l}
        Cons_{m} = \{\ & rty \doteq \exptype{C}{\ol{X}}, a \doteq T, \bigcup_{T_i \in \overline{T}} (pt_i \lessdot T_i)\\
                    & |\, \text{for every method}\ \exptype{}{\ol{Y}} T\ \texttt{m}(\ol{T}\  \ol{p}) \in \ol{M} \} \\%, \textbf{fresh}(\ol{X}) \lessdot \textbf{fresh}(\ol{N})\},\\
                    & ,\, \text{in every class}\ \exptype{C}{\ol{X}} \{ \ldots \ol{M} \ldots \}\ \text{in the input} \} 
                  \end{array}\\
& & OrCons = \textbf{OrConstraint}(Cons_{m})\\
& \mathbf{in}% \\
% &
& (a, [\textbf{fresh}(\ol{X})/\ol{X}][\textbf{fresh}(\ol{Y})/\ol{Y}](ConS \cup \bigcup_i ConS_i \cup OrCons))\\
& & \text{where\ } a \text{\ is\ a\ fresh\
  type\ variable}\\ 
\end{array}
$

\smallskip

The \texttt{new}-statement comes without the generic variables
(\texttt{new Classname(...)} instead of \texttt{new Classname<Class>(...)}).
The correct type will be inferred by our type inference algorithm.
We generate new type variables $\ol{X'}$ for the generic variables of the class \texttt{C},
so the \textbf{Unify} algorithm can later set in the correct types for these variables.
He has to comply to the bounds given by $\ol{N}$, which is why we add $\ol{X'} \lessdot \ol{N}$ to the constraints.
It is important to change every occurence of $\ol{X}$ with the fresh type variables $\ol{X'}$ in the generated constraints.
$\ol{X}$ can occur in the bound $\ol{N}$ aswell as in the types of the constructor parameters $\ol{T}$.

$\begin{array}{@{}l@{}l@{}l}
\textbf{TY}& \textbf{PE} & \textbf{Expr} (Ass, \mathtt{new }\ C(\overline{e}) ) = \\
& \textbf{let} % \\
& \forall e_i \in \overline{e} : (pt_i, Cons_i) = \textbf{TYPEExpr}(Ass, e_i)  ,\\
& & Cons = \{ \bigcup_{T_i \in \overline{T}} (pt_i \lessdot T_i) \ | \ \mathtt{class }\ \exptype{C}{\ol{X} \triangleleft \ol{N}}\{ \ldots, \texttt{C}(\overline{T} \overline{x}), \ldots \} \}\\
& \mathbf{in}% \\
% &
& (\exptype{C}{\ol{X'}}, [\textbf{fresh}(\ol{X})/\ol{X}](Cons \cup \bigcup_i Cons_i \cup \ol{X'} \lessdot \ol{N}))\\
\end{array}
$

We do not generate constraints for casts.

\subsubsection{Completeness of the type inference algorithm}
%Theorem: The Unify algorithm is complete
%Theorem: \textbf{FJTYPE} generates the principal type
\textbf{Proof:} The \textbf{Unify} algorithm is complete, so every correct type is included in the solution set.
We only have to choose the right type out of those solutions.
When compiling multiple classes the problem arises,
that only one of the type solutions calculated by \textbf{Unify} is correct
in respective to the other classes that will be compiled afterwards.

All types that are possible under the FGJ typing rules, plus our additional assumptions,
also comply with the generated constraints.

We match every generated constraint with the respective type rule to show completeness of our \textbf{FJTYPE} algorithm.
This shows that none of the generated constraints remove a type which otherwise would be possible under the FGJ typing rules.
The constraints are generated on expression statements.
We now compare the constraints for each expression with the appropriate type rule from FGJ:
\begin{description}
  \item [this]
  has always the type of the surrounding class and generates no constraints.
  \item [Local var]
  No constraints are generated.
  \item[Method invocation]
By direct comparison we show that each of the generated constraints do not apply more restrictions than the \texttt{GT-INVK} rule.
The \texttt{GT-INVK} rule states the condition $\textit{mtype}(m, \textit{bound}_\triangle(T_0)) = \exptype{}{\ol{Y}}\ \ol{U} \to U$.
%In our version of typeless FGJ every method name is unique
%and there is only one class with that particular method.
The constraint $rty \lessdot \exptype{C}{\textbf{fresh}(\ol{X})}$ assures that the type of the expression $e_0$ contains the method \texttt{m}.

\begin{tabular}{l|l}
  \textbf{FGJ Type rule} & \textbf{Constraints} \\
  $\triangle; \Gamma \vdash e_o : T_0$ & $(rty, ConS) = \textbf{TYPEExpr}(Ass, e_r)$\\ 
  $\quad \textit{mtype}(m, \textit{bound}_\triangle(T_0)) = \exptype{}{\ol{Y}}\ \ol{U} \to U$ & $rty \lessdot \exptype{C}{\textbf{fresh}(\ol{X})}$ \\
 %$\textit{mtype}(m, \textit{bound}_\triangle(T_0)) = \ol{U} \to U$ & $rty \doteq cl$\\
 $\triangle; \Gamma \vdash \ol{e} : \ol{S}$ & $\forall e_i \in \overline{e} : (pt_i, ConS_i) = \textbf{TYPEExpr}(Ass, e_i)$\\
 $\triangle \vdash \ol{S} <: \ol{U}$ & $ \bigcup_{T_i \in \overline{T}} (pt_i \lessdot \textbf{fresh}(T_i))$\\
 $\triangle; \Gamma \vdash \mathtt{e_0.m(\overline{e}) : U }$ & $a \doteq \textbf{fresh}(T)$ \\
\end{tabular}

\textit{Note}: The \textbf{TYPEExpr} function only generates constraints which apply to our assumption.
 \item[Field access]
Mostly the same as method invocation.
Fieldnames by default are unique in the FGJ language.

 \begin{tabular}{l|l}
   \textbf{FGJ Type rule} & \textbf{Constraints} \\
   $\Gamma \vdash e_0:T_0$ & $(rty, ConS) = \textbf{TYPEExpr}(Ass, e_r)$\\ 
   $\quad \mathit{fields}(\mathit{bound}_\triangle(T_0)) = \overline{T} \ \overline{f}$ & $rty \doteq \exptype{C}{\textbf{fresh}(\ol{X})}$ \\
  %$\textit{mtype}(m, \textit{bound}_\triangle(T_0)) = \ol{U} \to U$ & $rty \doteq cl$\\
  $\triangle; \Gamma \vdash \ol{e} : \ol{S}$ & $\forall e_i \in \overline{e} : (pt_i, ConS_i) = \textbf{TYPEExpr}(Ass, e_i)$\\
  $\triangle \vdash \ol{S} <: \ol{U}$ & $ \bigcup_{T_i \in \overline{T}} (pt_i \lessdot \textbf{fresh}(T_i))$\\
  $\triangle; \Gamma \vdash \mathtt{e_0.m(\overline{e}) : U }$ & $a \doteq \textbf{fresh}(T)$ \\
 \end{tabular}
 \item[Constructor]

\begin{tabular}{l|l}
  $\triangle; \Gamma \vdash \ol{e} : \ol{S}$ & $\forall e_i \in \overline{e} : (pt_i, ConS_i) = \textbf{TYPEExpr}(Ass, e_i)$\\
  $\triangle \vdash \ol{S} <: \ol{T}$ & $\bigcup_{T_i \in \overline{T}} (pt_i \lessdot T_i)$
\end{tabular}
  
\end{description}

\section{Unify}
\label{sec:unify}
This chapter describes the \textbf{GenericUnify} algorithm
which is used to find type solutions for the constraints generated by \textbf{FGJType}.

\begin{description}
\item[input] A set of type constraints $Cons_{in}$ %and a set of subtype relationships $S_\leq$
\item[output] A set of type unifiers $Uni$
or fail $Uni = \emptyset$.
%The unifiers have the form of $Uni = \{ \sigma_1, \ldots , \sigma_n \}$.
\end{description}

The unify algorithm first has to build the cartesian product of all the \textbf{OrConstraint}s and the remaining normal constraints:
\begin{align*}
\Omega &= \text{All }\mathbf{OrConstraints} \text{ in } {Cons}_{in}\\
C &= {Cons}_{in} \setminus \Omega \\
Eq_{set} &= \Omega_1 \times \ldots \times \Omega_n \times C \quad \text{for all }\Omega_i \in \Omega
\end{align*}

%The algorithm starts by setting $Eq_{set} = \set{ Cons_{in} }$.
Afterwards the following steps are repeatedly executed on $Eq_{set}$ until the algorithm termiates:
%For every $Eq \in Eq_{set}$ the following steps are applied.
%The resulting unifiers $\sigma$ from each $Eq$ merged together form the result of the \textbf{Unify} algorithm.

%$\textbf{SubUnify} :: [Constraint] \to [Unifier]$
\begin{enumerate}
\item Repeated application of the rules depicted in figure \ref{fig:fgjreduce-rules} and \ref{fig:fgjerase-rules}.
The end configuration of $Eq$ is reached if for each element
no rule is applicable.

\item
(The function $\textbf{fresh}(i)$ returns an array of $i$ fresh type variables.)

\begin{align*}
Eq_1 =& \text{Subset of pairs where both type terms are type variables}\\
Eq_2 =& Eq / Eq_1 \\
Eq_{set}\\ 
    = 
     %& \begin{array}[t]{l@{\,}ll}
     % \times \, (\displaystyle{\bigotimes_{(a \lessdot \exptype{C}{\ol{X}}) \in Eq'_2}}
     % \{\,(a \doteq \exptype{D}{\ol{A}}) \, \cup \, (\ol{X} \doteq \ol{Y'}) \ | \ \sarray{@{}l}{
     %   \exptype{D}{\ol{Z}} <: \exptype{C}{\ol{Y}}, \\
     %   \ol{A} = \textbf{fresh}(\#(\ol{Z})), \\
     %   \ol{Y'} = [\ol{A}/\ol{Z}]\ol{Y}
     %   \,\})}\\ 
     % \end{array}\\
   %& \times\, 
   %   (\displaystyle{\bigotimes_{(\exptype{C}{\overline{T}} \lessdot a) \in Eq'_2}}\!\!
   %   \set{(a \doteq [\ol{T}/\ol{X}]N ) \ | \ (\exptype{C}{\overline{X}} \leq N) \in
   %     S_\leq})\\
    & %\times\, 
    (\displaystyle{\bigotimes_{(\exptype{C}{\overline{T}} \lessdot a) \in Eq'_2}}\!\!
    \set{a \doteq [\ol{T}/\ol{X}]N ) \ | \ (\exptype{C}{\overline{X} \triangleleft \ol{N}} <: N})\\
    & \times\, 
      (\displaystyle{\bigotimes_{((X \triangleleft M)\lessdot a) \in Eq'_2}}\!\!
      \set{a \doteq (X \triangleleft M)} \cup
      \set{a \doteq N ) \ | \ M <: N } \cup \set{a \doteq (X \triangleleft M)})\\
      & \times\, \set{[a \doteq N \ | \  (a \doteq N) \in Eq'_2]} \\
      & \times\, \set{[a \lessdot N \ | \  (a \lessdot N) \in Eq'_2]} \times Eq_1 \\
\end{align*}

\item \label{subst-step}  Application of the following \emph{subst} rule
    %\begin{enumerate}
    %\item Apply the following subst\_eq rule
    
      $$\begin{array}[c]{lll}
        (\mathrm{subst}) &
        \begin{array}[c]{l}
          Eq'' \cup \set{a \doteq \theta}\\
          \hline
          Eq''[a \mapsto \theta] \cup \set{a \doteq \theta}
        \end{array}
        & a \textrm{ occurs in } Eq'' \textrm{ but not in } \theta 
      \end{array}$$
      
      for each $a \doteq \theta$ in each element of $Eq' \in Eq'_{set}$.

\item 
    \begin{enumerate}
    \item Foreach $Eq \in Eq_{set}$ which has changed in the last step
      start again with the first step.
    \item Build the union $Eq_{set}$ of all results of (a) and all $Eq' \in
      Eq'_{set}$ which has not changed in the last step.
    \end{enumerate}
\item
\begin{enumerate}
\item Filter all constraint sets which are in solved form:\\
$Eq_{solved} = \set{ Eq \ | \ Eq \in Eq_{set}, Eq \ \text{is in solved form}}$
\item We apply the following rule to every constraint set in $Eq_{solved}$:
\begin{align*}
\ddfrac{
  Eq \cup \set{ a \lessdot b } %There are only Type variables left at this point
}{
  Eq \cup \set{ a \doteq b }
}
\end{align*}
\item $\emph{Uni} = \set{\sigma \ | \ Eq \in Eq_{solved},\ \sigma = \set{a \mapsto \theta \ | \ (a \doteq \texttt{T}) \in Eq} }$
%\item $\emph{Uni} = \sarray{l@{\ }l}{\set{\sigma \ | & Eq'' \in Eq''_{set},
%        Eq'' \textrm{ is in solved form,}\\ 
%        & \sigma = \set{a \mapsto \theta \ | \ (a \doteq \theta) \in Eq''}
%        \\ & \quad \cup \ \set{a \mapsto \texttt{A}, b \mapsto \texttt{B} \ | \ (a \lessdot b) \in Eq \ \text{and a is an isolated type variable}}
%        }}$
\end{enumerate}
\end{enumerate}

\begin{definition}[Isolated type variable] \label{def:isolated-type-variable}
  \rm
An isolated type variable does only occur in constraints together with another isolated type variable.
\end{definition}

\begin{definition}[Solved form]\label{def:solved-form}
  \rm
  A set of constraints $Eq$ has reached solved form if it contains only the following kind of constraints:
  \begin{enumerate}
    \item $a \lessdot b$, with $a$ and $b$ both isolated type variables
    \item $a \lessdot \exptype{C}{\ol{X}}$
    \item $a \doteq \exptype{C}{\ol{X}}$
  \end{enumerate}j
\end{definition}

\begin{figure}
\begin{center}
    \leavevmode
    \fbox{
    \begin{tabular}[t]{ll}
      (match)
      & $
      \begin{array}[c]{ll}
      \begin{array}[c]{l}
         Eq \cup \, \set{a \lessdot
         \exptype{C}{\ol{X}},
         a \lessdot
          \exptype{D}{\ol{Y}}} \\ 
        \hline
        \vspace*{-0.4cm}\\
        Eq \cup \set{a \lessdot \exptype{C}{\ol{X}}
        , \exptype{C}{\ol{X}} \lessdot \exptype{D}{\ol{Y}}}
      %Eq \cup \set{\theta_1 \doteq \lambda'_1 \ldo \theta_n \doteq \lambda'_n}
      \end{array}
      & \exptype{C}{\ol{Z}} <: \exptype{D}{\ol{N}} 
      \end{array}
      $
    \\\\
    (adopt)
    & $
    \begin{array}[c]{ll}
    \begin{array}[c]{l}
       Eq \cup \, \set{a \lessdot
       \exptype{C}{\ol{X}},
       b \lessdot
        a} \\ 
      \hline
      \vspace*{-0.4cm}\\
      Eq \cup \set{
        a \lessdot
       \exptype{C}{\ol{X}},
       b \lessdot
        a
      , b \lessdot \exptype{C}{\ol{X}}}
    %Eq \cup \set{\theta_1 \doteq \lambda'_1 \ldo \theta_n \doteq \lambda'_n}
    \end{array}
    \end{array}
    $
  \\\\
      (adapt)
      & $
      \begin{array}[c]{ll}
      \begin{array}[c]{l}
         Eq \cup \, \set{\exptype{D}{\ol{A}} \lessdot
          \exptype{C}{\ol{B}}} \\ 
        \hline
        \vspace*{-0.4cm}\\
        Eq \cup \set{\exptype{C}{[ \ol{A} / \ol{X} ]\ol{Y}}
        \doteq \exptype{C}{\ol{B}}}
      %Eq \cup \set{\theta_1 \doteq \lambda'_1 \ldo \theta_n \doteq \lambda'_n}
      \end{array}
      & \exptype{D}{\ol{X}} <:\ \exptype{C}{\ol{Y}}
      \end{array}
      $
    \\\\
(reduce1) & $
\begin{array}[c]{l}
  Eq \cup \set{\exptype{D}{\ol{A}} \lessdot
    \exptype{D}{\ol{B}}}\\
  \hline
  Eq \cup \set{\ol{A} \doteq \ol{B}}
\end{array}
      $ \\\\
(reduce2) & $
\begin{array}[c]{l}
  Eq \cup \set{\exptype{D}{\ol{A}} \doteq
    \exptype{D}{\ol{B}}}\\
  \hline
  Eq \cup \set{\ol{A} \doteq \ol{B}}
\end{array}
      $ \\\\
(equals) & $
\begin{array}[c]{l}
  Eq \cup \set{a \lessdot
    b, b \lessdot a}\\
  \hline
  Eq \cup \set{a \doteq b}
\end{array}
      $ \\\\
    \end{tabular}}
  \end{center}
\caption{Reduce and adapt rules}\label{fig:fgjreduce-rules}
\end{figure}

\begin{figure}
  \begin{center}
      \leavevmode
      \fbox{
      \begin{tabular}[t]{ll}
        (matchG)
        & $
        \begin{array}[c]{ll}
        \begin{array}[c]{l}
           Eq \cup \, \set{a \lessdot
           \exptype{C}{\ol{X}},
           a \lessdot
            (X \triangleleft \exptype{D}{\ol{Y}})} \\ 
          \hline
          \vspace*{-0.4cm}\\
          Eq \cup \set{a \lessdot \exptype{C}{\ol{X}}
          , \exptype{C}{\ol{X}} \lessdot \exptype{D}{\ol{Y}}}
        %Eq \cup \set{\theta_1 \doteq \lambda'_1 \ldo \theta_n \doteq \lambda'_n}
        \end{array}
        & \exptype{C}{\ol{Z}} <: \exptype{D}{\ol{N}} 
        \end{array}
        $
      \\\\
      (adoptG)
      & (the same as adopt)
    \\\\
    (adaptG)
    & $
    \begin{array}[c]{ll}
    \begin{array}[c]{l}
       Eq \cup \, \set{(X \triangleleft \exptype{D}{\ol{A}}) \lessdot
        \exptype{C}{\ol{B}}} \\ 
      \hline
      \vspace*{-0.4cm}\\
      Eq \cup \set{\exptype{C}{[ \ol{A} / \ol{X} ]\ol{Y}}
      \doteq \exptype{C}{\ol{B}}}
    %Eq \cup \set{\theta_1 \doteq \lambda'_1 \ldo \theta_n \doteq \lambda'_n}
    \end{array}
    & \exptype{D}{\ol{X}} <:\ \exptype{C}{\ol{Y}}
    \end{array}
    $
  \\\\
  (adaptG2)
  & $
  \begin{array}[c]{ll}
  \begin{array}[c]{l}
     Eq \cup \, \set{(X \triangleleft \exptype{C}{\ol{A}}) \lessdot
      \exptype{D}{\ol{B}}} \\ 
    \hline
    \vspace*{-0.4cm}\\
    Eq \cup \set{\exptype{C}{\ol{A}}
    \doteq \exptype{C}{[ \ol{B} / \ol{X} ]\ol{Y}}}
  %Eq \cup \set{\theta_1 \doteq \lambda'_1 \ldo \theta_n \doteq \lambda'_n}
  \end{array}
  & \exptype{D}{\ol{X}} <:\ \exptype{C}{\ol{Y}}
  \end{array}
  $
\\\\
  (reduce1) & $
  \begin{array}[c]{l}
    Eq \cup \set{\exptype{D}{\ol{A}} \lessdot
      \exptype{D}{\ol{B}}}\\
    \hline
    Eq \cup \set{\ol{A} \doteq \ol{B}}
  \end{array}
        $ \\\\
  (reduce2) & $
  \begin{array}[c]{l}
    Eq \cup \set{\exptype{D}{\ol{A}} \doteq
      \exptype{D}{\ol{B}}}\\
    \hline
    Eq \cup \set{\ol{A} \doteq \ol{B}}
  \end{array}
        $ \\\\
      \end{tabular}}
    \end{center}
  \caption{Reduce and adapt rules with generic variables}\label{fig:fgjreduce-rules-generic}
  \end{figure}

\begin{figure}
\begin{align*}
&\begin{tabular}[t]{ll}
      (erase1)  & $ 
      \begin{array}[c]{ll}
        \begin{array}[c]{l}
          Eq \cup \set{C \lessdot D}\\
          \hline
          Eq
        \end{array}
        & C \leq^* D \in S_\leq
      \end{array}$\\
          \end{tabular}\\
&\begin{tabular}[t]{ll}
      (erase2)  & $ 
      \begin{array}[c]{ll}
        \begin{array}[c]{l}
          Eq \cup \set{C \doteq C}\\
          \hline
          Eq
        \end{array}
      \end{array}$\\
          \end{tabular}\\
    &      \begin{tabular}[t]{ll}
       (swap) & $
            \begin{array}[c]{ll}
              \begin{array}[c]{l}
                Eq \cup \set{C \doteq a}\\
                \hline
                Eq \cup \set{a \doteq C}
              \end{array}
            \end{array}$
          \end{tabular}
\end{align*}
\caption{Erase and swap rules}\label{fig:fgjerase-rules}
\end{figure}


\subsection{Unify proof}

%Soundness: We have to prove that each calculated result of the algorithm is a general unifier of the corresponding input
\begin{theoremAndi}
  \label{theo:unifySoundness}
  \textbf{(Soundness):}
  If the \textbf{Unify} algorithm finds a solution it does not contradict any of the input constraints:
  $\nexists (a \lessdot b) \in {Cons}_{in}$ where $\sigma(a) \nleq \sigma(b)$  
\end{theoremAndi}
%We show this by induction.
%No step alters the constraint set in a way that would make a wrong solution possible.
\textit{Proof:}
We show theorem \ref{theo:unifySoundness} by going backwards over every step of the algorithm.
We assume there exists a unifier $\sigma = \set {a_1 \mapsto \theta_1, \ldots , a_n \mapsto \theta_n}$ for the input constraints,
which is the result of the \textbf{Unify} algorithm.
This means for every constraint in the input set $(a \lessdot b) \in {Cons}_{in}$ and $(c \doteq d) \in {Cons}_{in}$
this unifier will substitute all variables in a way that all constraints are satisfied:
$\sigma(a) \leq \sigma(b)$, $\sigma(c) = \sigma(d)$\\

We now look at each step of the \textbf{Unify} algorithm
which transforms the input set of constraints $Eq$ to a set $Eq'$.
If we assume the unifier $\sigma$ is correct for the set $Eq'$,
then we can show that it will also be correct for the constraints $Eq$. 


%TODO:
%Assume we find a unifier. Then do every step backwards.
%For each step the constraint set before and after the step have the same unifier.
% -> This means no step adds an incorrect unifier / makes a incorrecto unifier possible
%At the end we must end at the correct unifier.

\begin{description}
\item[Step 5 c)]
The last step of the algorithm transforms a set of constraints $Eq$ of the the form

$Eq = \set{a_1 \doteq \theta_1, \ldots , a_n \doteq \theta_n}$

to the unifier $\sigma$.
Trivial.

\item[Step 5 b)]
A unifier which is correct for $a \doteq b$ is also correct for $a \lessdot b$.

\item[Step 5 a)]
Trivial, we do not alter the constraint set which lateron leads to the unifier.

\item[Step 4]
Trivial, the constraint sets are not altered here.

\item[Step 3]
An unifier $\sigma$ that is correct for a constraint set
$Eq[a \to \theta] \cup (a \doteq \theta)$ is also correct for
the set $Eq \cup (a \doteq \theta)$.
From the constraint $(a \doteq \theta)$ it follows that $\sigma(a) = \theta$.
This means that $\sigma(Eq) = \sigma(Eq[a \to \theta])$,
because every occurence of $a$ in $Eq$ will be raplaced by $\theta$ anyways when using the unifier $\sigma$.

\item[Step 2]
This step transforms constraints of the form $(\exptype{C}{\ol{X}} \lessdot a)$ and $(a \lessdot \exptype{C}{\ol{X}})$
into sets of constraints and builds the cartesian product with the remaining constraints.
We can show that if there is a resulting set of constraints which has $\sigma$ as its correct unifier
then $\sigma$ also has to be a correct unifier for the constraints before this transformation.
Proof:
%normal version:
%\begin{description}
%\item[$(a \lessdot C)$] If $\sigma$ is a correct unifier for a set containing $(a \doteq \theta)$
%and $\theta \leq C$, then $\sigma$ is also a correct unifier for the set containing $(a \lessdot C)$.
%\item[$(C \lessdot a)$] Same goes the other way. If $C leq \theta$ and $\sigma$ is correct for $(C \lessdot \theta)$
%then $\sigma$ is also correct for $(a \doteq \theta)$
%\end{description}

\begin{description}
\item[$(a \lessdot \exptype{C}{\ol{X}})$] If $\sigma$ is a correct unifier for a set containing $(a \doteq \exptype{D}{\ol{A}})$
and $(\ol{X} \doteq \ol{Y'})$
, then $\sigma$ is also a correct unifier for the set containing $(a \lessdot \exptype{C}{\ol{X}})$.
The reason is the \texttt{S-CLASS} subtype rule of FGJ. %TODO: Hier etwas ausführlicher?
%TODO:
%\item[$(\exptype{C}{\ol{T}} \lessdot a)$] Same goes the other way. If $\exptype{C}{\ol{X} leq \theta$ and $\sigma$ is correct for $(C \lessdot \theta)$
%then $\sigma$ is also correct for $(a \doteq \theta)$

%If $\sigma$ is a correct unifier for a set \\
%$Eq = Eq' \cup (a \doteq \exptype{D}{\ol{A}}) \, \cup \, (\ol{X} \doteq \ol{Y'})$ \\
%then it is also a correct unifier for a set $Eq = Eq' \cup (a \lessdot \exptype{C}{\ol{X}})$: \\
When substituting $(a \to \exptype{D}{\ol{A}})$ and $(\ol{X} \to \ol{Y'})$
and finally $ (\ol{Y'} \to [\ol{A}/\ol{Z}]\ol{Y})$ in the constraint $(a \lessdot \exptype{C}{\ol{X}})$
we get: $(\exptype{D}{\ol{A}} \lessdot [\ol{A}/\ol{Z}]\exptype{C}{\ol{Y}}$
which is correct under the \texttt{S-CLASS} rule (see chapter \ref{chapter:type-rules}).

\item[$(\exptype{C}{\ol{T}} \lessdot a)$] If $\exptype{C}{\ol{X}} leq \theta$ and $\sigma$ is correct for $(a \doteq [\ol{T}/\ol{X}]N)$
then $\sigma$ is also correct for $(\exptype{C}{\ol{T}} \lessdot a))$.
When substitutin $a$ for $[\ol{T}/\ol{X}]N$ we get 
$(\exptype{C}{\ol{T}} \lessdot [\ol{T}/\ol{X}]N)$
, which is correct because $(\exptype{C}{\ol{X}} \leq \exptype{C}{\ol{Y}}$
(see \texttt{S-CLASS} rule).

\footnote{
Discussion:
Do we need to include the $\triangleleft \ol N$ bounds?
The S-CLASS rule does not mention them.

When ignoring those rules this could lead to an error
class T<X extends List> {
}
class S<X>{}

Constraint: S<String> <. a
Unify: T<String> =. a // ERROR!

This is not a problem because no error is gonna result from this.
TYPEExpr only hast to implement all of the typing rules of FJ
and unify has to solely respect the subtyping rules.

}
\end{description}

\item[Step 1]
\begin{description}
\item[erase-rules] remove correct constraints from the constraint set.
A unifier $\sigma$ that is correct for the constraint set $Eq$
is also correct for $Eq \cup \set{\theta \doteq \theta}$
and $Eq \cup \set{\theta \lessdot \theta'}$, when $\theta \leq \theta'$.
\item[swap-rule] does not change the unifier for the constraint set.
$\doteq$ is a symmetric operator and parameters can be swapped freely.
\item[match] The subtype relation is transitive, so if there is a correct solution for
$a \lessdot \exptype{C}{\ol{X}}, \exptype{C}{\ol{X}} \lessdot \exptype{D}{\ol{Y}}$
then this solution would also apply for $a \lessdot \exptype{C}{\ol{X}} \lessdot \exptype{D}{\ol{Y}}$
or $a \lessdot \exptype{D}{\ol{Y}}$.
\item[adopt] An unifier which is correct for $Eq \cup \set{a \lessdot \exptype{C}{\ol{X}}, a \lessdot b, b \lessdot \exptype{C}{\ol{X}}}$
is also correct for $Eq \cup \set{a \lessdot \exptype{C}{\ol{X}}, a \lessdot b}$.
\item[adapt] If there is a $\sigma$ which is a correct unifier for a set
$Eq \cup \set{ \exptype{C}{[\ol{A}/\ol{X}]\ol{Y}} \doteq \exptype{C}{\ol{B}}}$ then it is also
a correct unifier for the set $Eq \cup \set{ \exptype{D}{\ol{A}} \lessdot \exptype{C}{\ol{B}}}$,
if there is a subtype relation $\exptype{D}{\ol{X}} \leq^* \exptype{C}{\ol{Y}}$.
To make the set $Eq \cup \set{ [\ol{A}/\ol{X}]\exptype{C}{\ol{Y}} \doteq \exptype{C}{\ol{B}}}$ the unifier 
$\sigma$ must satisfy the condition $\sigma([\ol{A}/\ol{X}]\ol{Y}) = \sigma(\ol{B})$.
By substitution we get $Eq \cup \set{ \exptype{D}{\ol{A}} \lessdot \exptype{C}{[\ol{A}/\ol{X}]\ol{Y}}}$
which is correct under the \texttt{S-CLASS} rule.
\item[reduce] The \texttt{reduce1} and \texttt{reduce2} rules are obviously correct under the FJ typing rules.
\end{description}

\item[OrConstraints]
If $\sigma$ is a correct unifier for one of the constraint sets in $Eq_{set}$
then it is also a correct unifier for the input set $Cons_{in}$.
When building the cartesian product of the \textbf{OrConstraints} every possible
combination for $Cons_{in}$ is build.
No constraint is altered, deleted or modified during this step.
\end{description}
\hfill $\square$


%Completeness: If there is a Unifier (a solution), the algorithm will never alter the constraints in a way that this solution is removed
%\begin{theoremAndi}\label{theo:unifyCompleteness}
%  \textbf{(Completeness):} If there is a solution for the input constraints $Cons_{in}$, the \textbf{Unify} algorithm will not fail.
%  A solution is a non-empty set of unifiers $Uni = \set{\sigma_1, \ldots \sigma_n }$,
%  where each unifier is a injective function which maps every type variable in the input constraints $Cons_{in}$ to a
%  type in $S_\leq$.
%  \end{theoremAndi}
\begin{theoremAndi}\label{theo:unifyCompleteness}
  \textbf{(Completeness):} The \textbf{Unify} algorithm calculates all principal type solutions for the input set of constraints ($Cons_{in}$).
  A unifier $\sigma$ is a principal type solution for $Cons_{in}$ if it unifies $Cons_{in}$
  and for every other unifier $\omega$ there is a unifier $\lambda$ so that $\omega(x) = \lambda(\sigma(x))$.
\end{theoremAndi}
\textit{Proof:}
%The \textbf{Unify} calculates multiple solutions.

%Our proof goes as follows:
We look at every step of the algorithm, which alters the set of constraints $Eq$,
while assuming that there is at least one possible principal type solution $\sigma$ for the input.
We will show that the principal type is among them by proofing for every step of the algorithm that the principal type is never excluded.

%Assume there is a unifier and the Unify algorithm finds it.
%Then no rule makes this unifier impossible / removes this unifier.

\begin{description}
\item[Step 1:]
The first step applies the three rules from figure \ref{fig:unifyrules}.
\textbf{erase-rules:} The erase2 rule from figure \ref{fig:unifyrules} removes a
$\{\theta \leq \theta\}$ constraint from the constraint set.
The erase1 rule removes a $\{\theta \leq \theta\}$ constraint,
but only if the two types $\theta$ and $\theta'$ satisfy the constraint.
Both rules do not change the set of possible solutions for the given constraint set.

\textbf{swap-rule:} $\doteq$ is a symmetric operator and parameters can be swapped freely.
This operation does not change the meaning of the constraint set.

\textbf{match-rule:}
If there is a solution for $a \lessdot \exptype{C}{\ol{X}}, a \lessdot \exptype{D}{\ol{Y}}$,
this is also a solution for $a \lessdot \exptype{C}{\ol{X}}, \exptype{C}{\ol{X}} \lessdot \exptype{D}{\ol{Y}}$.
A correct unifier $\sigma$ has to find a type for $a$, which complies with $a \lessdot \exptype{C}{\ol{X}}$ and $a \lessdot \exptype{D}{\ol{Y}}$.
Due to the subtyping relation being transitive this means that $\sigma(a) \lessdot \exptype{C}{\ol{X}} \lessdot \exptype{D}{\ol{Y}}$.

\textbf{adopt-rule:} Subtyping in FJ is transitive,
which allows us to apply the adopt rule without excluding any possible unifier.

\textbf{adapt-rule:} Every solution which is correct for the constraints
$Eq \cup \set{ \exptype{C}{[\ol{A}/\ol{X}]\ol{Y}} \doteq \exptype{C}{\ol{B}}}$ is also
a correct solution for the set $Eq \cup \set{ \exptype{D}{\ol{A}} \lessdot \exptype{C}{\ol{B}}}$.
According to the \texttt{S-CLASS} rule there can only be a possible solution for 
$\exptype{C}{[\ol{A}/\ol{X}]\ol{Y}} \doteq \exptype{C}{\ol{B}}$
if $\ol{B} = [\ol{A}/\ol{X}]\ol{Y}$.
Therefore this transformation does not remove any possible solution from the constraint set.

\textbf{reduce-rule:}
%The constraint is not altered
For a constraint $\exptype{D}{\ol{A}} \lessdot \exptype{D}{\ol{A}}$ the FJ subtyping rule \texttt{S-REFL} ($\triangle \vdash T <: T$) is the only one which applies.
According to this rule the transformation to $\ol{A} \doteq \ol{B}$ is correct.
Only $D$ gets removed, which is not a type variable.
Therefore this step does not remove a possible solution.
This applies for both reduce rules \textbf{reduce1} and \textbf{reduce2}.

\item[Step 2:]
%The second step builds multiple constraint sets of all possible type combinations for the $\lessdot$-constraints.
The second step of the algorithm eliminates $\lessdot$-constraints
by replacing them with $\doteq$-constraints.
%The algorithm considers every possible type from $S_\leq$ which does not violate the eliminated $\lessdot$ -constraint itself.
%This step does not remove a solution from the constraint set.
For each $(a \lessdot \exptype{C}{\ol{X}})$ constraint the algorithm builds a set with every
possible subtype of $\exptype{C}{\ol{X}}$ set in for $a$.
So if there is a correct unifier $\sigma$ for the constraints before this conversion there will be at least one set of
constraints for which $\sigma$ is a correct unifier.
%TODO: what does soundness mean. if there is a possible type solution (with types in S) then unify will find it

\item[Step 3:]
In the third step the \textbf{substitution}-rule is applied.
If there is a constraint $a \doteq \theta$ then there is no other way to fulfill the constraint set
than replacing $a$ with $\theta$.
This does not remove a possible solution.

\item[Step 4:]
None of the constraints get modified.

\item[Step 5 a):]
The removed sets do not have a possible unifier, therefore no possible solution is
omitted in this step.

\textbf{Proof}:
In step 5.a all constraint sets that have a unifier are in solved form.
All other possibilities are eliminated in steps 1-4.
There are 8 different variations of constraints:\\
$(a \doteq a), (a \doteq C), (C \doteq a), (C \doteq C), (a \lessdot a), (a \lessdot C), (C \lessdot a), (C \lessdot C)$

After step 1 there are no $(C \doteq C)$, $(C \lessdot C)$ and $(C \doteq a)$ constraints anymore,
as long as the constraint set has a correct unifier.
Because a constraint set that has a correct unifier cannot contain constraints of the form $\theta_1 \doteq \theta_2$ with $\theta_1 \neq \theta_2$ and
$(\theta_1 \lessdot \theta_2)$ with $(\theta_1 \leq \theta_2) \notin S_\leq$.
By removing $(\theta \doteq \theta)$ and $(\theta \lessdot \theta')$ with $(\theta \leq \theta') \notin S_\leq$ constraints
no constraints of the form $(C \doteq C)$ and $(C \lessdot C)$
remain in a constraint set that has a correct unifier after step 1.

After step 2 there are no more $(a \lessdot C)$ constraints.

After step 3 there are no $(a \doteq C)$ anymore.

We only reach step 5 if the constraint set is not changed by the substitution (step 3).

%If at this point a set $Eq_i$ is not in solved form it has no correct unifier.

If the constraint set has a correct unifier only $(a \lessdot a)$, $(a \doteq a)$ and $(a \doteq C)$ constraints are left at this point.
The type variables in the $(a \lessdot a)$ and $(a \doteq a)$ constraints have to be independent type variables.
If a type variable $c$ is inside a $(c \doteq C)$ constraint it is not an independent type variable.
But this variable $c$ cannot be inside a $(a \doteq a)$ or $(a \lessdot a)$ constraint, because otherwise step 3 would have replaced it in there.


\item[Step 5 b):]
If the algorithm advances to this step we further only work on constraint sets in solved form.
This means there are only two kinds of constraints left.
($A \doteq \texttt{Typ}$), ($A \doteq B$) and ($A \lessdot B$) with $A$ and $B$ as type variables.

%We can set all TVs equal, because we allow only same TVs when having circles in a method call.
%This still will lead to the principal type.

The FGJ language does not allow subtype constraints for generic types.
A constraint like $(A \lessdot B)$ in a solution could be inserted as the typing shown in the example below.
But this is not allowed by the syntax of FGJ.
That is why we can treat this constraint as $(A \doteq B)$.

%TODO: This does not alter the outcome because the solution set is not modified anymore. All other TVs alread have a type like A =. Typ

\textit{Example:}
This would be a valid Java program but is not allowed in FGJ:
\begin{lstlisting}
class Example {
  <A extends Object, B extends A> A id(B a){
    return a;
  }
}
\end{lstlisting}

By replacing all ($A \lessdot B$) constraints with ($A \doteq B$) we do not remove a principal type solution.

\item[Step 6:]
In the last step all the constraint sets, which are in solved form, are converted to unifiers.

We see that only a constraint set which has no unifier does not reach solved form.
We showed that in every step of the \textbf{Unify} algorithm we never exclude a possible unifier.
Also we showed that after we reach step 5 only constraint sets with a correct unifier are in solved form.
By removing all constraint sets which are not in solved form the algorithm does not
remove a possible correct unifier.

If we assume that there is a possible principal type solution $\sigma$ for the input set $Cons_{in}$
and the \textbf{Unify} algorithm does not exclude any of the possible unifiers,
then the result \textbf{Unify} contains the principal type solution.
\hfill $\square$
\end{description}

\textbf{Termination Proof:}
Every step of the algorithm removes a $\lessdot$ constraint with the following exceptions:
A few rules of Step 1 do not change the amount of $\lessdot$ constraints.
Each of those rules can be applied only a finite amount of times on a finite set of constraints $Eq$.
So thos will stop to be applied automatically.
The only other exception is the \texttt{adopt} rule.
This rule can only be applied once for each combination of 

Does the adopt rule really terminate?
no -> if there would be a circle, but circles are deleted by equals
a <. b, b <. C<x1>, a <. C<x2>, b <. c, c <. C<x>, c <. a
yes -> only finite combination of a <. b and a <. C


\section{Insert principal type}
After generating all possible unifiers in the \textbf{Unify} we can insert the principal types.

\begin{align*}
\ddfrac{
  A\ \texttt{m}(\ol{A}\ \ol{p}) \{ \ldots \} \in M \quad \quad \texttt{class}\ \exptype{C}{\ol{X}} \{ \ldots\ M,\ \ldots\}
}{
  \textit{mtype}(\texttt{m}, \exptype{C}{\ol{X}}) = \set{\sigma(\ol{A}) \to \sigma(A) \,|\, \sigma \in {Uni}}
}
\end{align*}

The \textbf{Unify} algorithm returns a set of unifiers ${Uni}$.
Each element of that set is a correct solution.
The unifiers $\sigma$ map type placeholders to types.
When generating the intersection types for the methods we have to make sure that the
type placeholders for the return type as well as for the parameter types get replaced by the same unifier $\sigma$.
It can happen that two unifiers $\sigma_1$ and $\sigma_2$ lead to the same method type ($\sigma_1(A) = \sigma_2(A), \sigma_1(\ol A) = \sigma_2(\ol A)$).
The set of all the distinct combinations then builds the intersection type for the method.

\textbf{Example:}
\begin{lstlisting}
class Global{
  method1(a){
    a.add(this);
    return a.get();
  }
}
class List<A> {
  add(A item){...}
  A get() ...
}
\end{lstlisting}

The method \texttt{method1} would get the type $\set{ \exptype{List}{Object} \to \texttt{Object}
\ || \ \exptype{List}{Global} \to \texttt{Global}}$.
If FGJ would support overloaded methods this could be written as:
\begin{lstlisting}
class Global{
  Object method1(List<Object> a){
    a.add(this);
    return a.get();
  }
  String method1(List<Global> a){
    a.add(this);
    return a.get();
  }
}
class List<A> {
  add(A item){...}
  A get() ...
}
\end{lstlisting}

\section{Complexity}
\section{Complexity}
\label{sec:complexity}

\begin{theorem}[NP-Hardness]
  \label{theo:np-hardness}
  The type inference algorithm for typeless Featherweight Java is NP-hard.
\end{theorem}

\textbf{Proof:} This section will show this by reducing the boolean satisfiability problem (SAT) to the \fjtypeinference{} algorithm.

\begin{figure}
\begin{lstlisting}
  class True extends Object{
  }
  class False extends Object{
  }

  class Nand1 extends Object{
    False nand(True a, True b){ return new False(); }
  }
  class Nand2 extends Object{
    True nand(False a, True b){ return new True(); }
  }
  class Nand3 extends Object{
    True nand(True a, False b){ return new True(); }
  }
  class Nand4 extends Object{
    True nand(False a, False b){ return new True(); }
  }

  class SATExample{
    True f;

    sat(v1, v2, v3, o1, o2){
      return o1.nand(v1, o2.nand(v2, v3));
    }

    forceSATtoTrue(v1, v2, v3, o1, o2){
      return new SATExample(this.sat(v1, v2, v3, o1, o2));
    }
  }
\end{lstlisting}

\caption{Representation for a SAT problem in FJ code}
\label{fig:fjSATcode}
\end{figure}

Any given boolean expression $B$ can be transformed to a typeless FJ program.
A type inference algorithm finding a possible typisation of this FJ program also solves the boolean expression $B$.
Figure \ref{fig:fjSATcode} shows an example of this.
The classes \texttt{True}, \texttt{False} and \texttt{Operations} always stay the same.
Here we assume that the boolean expression only consists out of $\neg \land$ (NAND) operators.
Now any boolean expression $B_\text{in} = v_1 \land \neg (v_2 \land v_3) \land \ldots$ can be expressed as a Java method.
The example in figure \ref{fig:fjSATcode} represents the problem $B_\text{in} = \neg(v_1 \land \neg (v_2 \land v_3))$.
Additionally we force the return type of the \texttt{sat} method to have the type \texttt{True}
by instancing the \texttt{SATExample} class, which requires the type \texttt{True}.
When using the \fjtypeinference{} algorithm on the generated FJ code it will
assign each parameter of the \texttt{sat} method with either the type \texttt{True} or \texttt{False}.
This represents a valid assignment for the expression $B_\text{in}$.
If \fjtypeinference{} fails to compute a solution the $B_\text{in}$ has no possible solution.
A correct solution for the \texttt{sat} method in figure \ref{fig:fjSATcode} would be:\\
\texttt{True sat(False v1, True v2, True v3, Nand4 o1, Nand1 o2)}

Any SAT problem can be transferred in polynomial time to a typeless FJ program.
Every literal $v$ in the SAT problem becomes a method parameter of the \texttt{sat} method, as well as every instance of a NAND operator used.

This reduction of SAT to our type inference algorithm proofs that its
complexity is at least NP-Hard.
\hfill $\square$

\begin{theorem}[NP-Completeness]
  \label{theo:np-completeness}
  The type inference algorithm for typeless Featherweight Java is NP-Complete.
\end{theorem}

\textbf{Proof:} We know the algorithm is NP-hard (see \ref{theo:np-completeness}).
To proof NP-Completeness we have to show that it is possible to verify a solution in polynomial time.
The verification of a type solution is the FJ typecheck.

It is easy to see that the expression typing rules can be checked in polynomial time as long as subtyping between two types is verifiable in polynomial time.

Subtyping is also solvable in polynomial time in \TFGJ{}.
Assume $\exptype{C}{\ol{X}} \leq \exptype{D}{\ol{Y}}$ with the number of generics $\ol{X}$ and $\ol{Y}$ less or equal $n$.
Also the number of classes in the subtyperelation is less or equal to $n$.
With $n$ classes the \texttt{S-TRANS} rule can be applied a maximum of $n$ times.
Each time the \texttt{S-CLASS} rules is applied which sets in the variables $\ol{X}$ into the supertype.
This operations also runs in polynomial time, so the subtyping relation is decidable in polynomial time.

This shows that the time complexity of the GFJ type check is at least
polynomial or better. 
\hfill $\square$


%%% Local Variables:
%%% mode: latex
%%% TeX-master: "TIforGFJ"
%%% End:


\section{Examples}

Our unify algorithm processes one or two constraints at a time.
Lets look at some examples of different compositions of constraints
and how our unify algorithm will transform them.

\begin{enumerate}
\item $Eq = \set{a \lessdot \exptype{C}{\ol{X}}, a \lessdot b}$ \\
If this is an end configuration, the constraint set is solved.
We can now put any type for $b$, which is a sub or supertype of $\exptype{C}{\ol{X}}$.
The only exception is, when $b \in \ol{X}$.
In this case $\sigma(b) = \tt{Object}$ is always a correct unifier.
So there is atleast one possible solution.

\item $Eq = \set{a \lessdot \exptype{C}{\ol{X}}, a \lessdot b, b \lessdot c}$
It's the same as the example before.
Replacing $b$ and $c$ with \texttt{Object} is also a correct solution. 

\item $Eq = \set{a \lessdot \exptype{C}{\ol{X}}, a \lessdot b, b \lessdot \exptype{D}{\ol{Y}}}$
Here the adopt rule transforms this to:
$\set{a \lessdot \exptype{C}{\ol{X}}, a \lessdot b, b \lessdot \exptype{D}{\ol{Y}}, a \lessdot \exptype{D}{\ol{Y}}}$
and here it depends if $\exptype{C}{\ol{A}} <: \exptype{D}{\ol{B}}$ or $\exptype{D}{\ol{A}} <: \exptype{C}{\ol{B}}$.
Either the $a \lessdot \exptype{C}{\ol{X}}$ or $b \lessdot \exptype{D}{\ol{Y}}$ constraint
gets removed by the match rule.
This leads to $\set{a \lessdot \exptype{C}{\ol{X}}, a \lessdot b, b \lessdot \exptype{D}{\ol{Y}}}$.
If $\texttt{C} = \texttt{D}$ and $\ol{X} = \ol{Y}$ we have the following situation:
$\set{a \lessdot \exptype{C}{\ol{X}}, a \lessdot b, b \lessdot \exptype{D}{\ol{Y}}}$.
Otherwise the match rule gets applied again.

\item $Eq = \set{a \lessdot \exptype{C}{\ol{X}}, a \lessdot b, b \lessdot \exptype{C}{\ol{X}}}$
TODO

\end{enumerate}

\textbf{match-rule:}\\
The match-rule takes two constraints of the form $a \lessdot \exptype{C}{\ol{X}}, a \lessdot \exptype{D}{\ol{Y}}$
and checks which one of the two types \texttt{C} and \texttt{D} is a subtype of the other.
We only leave the constraint, which is more restrictive.
If \texttt{C} and \texttt{D} do not have a subtype relation, the match-rule won't apply.
This will lead to the constraint set never reaching solved form.
We cannot remove the $a \lessdot \exptype{D}{\ol{Y}}$ constraint without matching the
$\ol{Y}$ and $\ol{X}$.
We do this by generating the constraint $\exptype{C}{\ol{X}} \lessdot \exptype{D}{\ol{Y}}$.
This constraint will then be processed by the adapt and the reduce rule.

Example:
$Eq = \set{a \lessdot \exptype{C}{b,\tt{String}}, a \lessdot \exptype{D}{b,b}}$
with $\exptype{C}{A,B} <: \exptype{D}{B,B}$.
\begin{displaymath}
\prftree[r]{
    reduce2
        }{
    \prftree[r]{
        adopt
            }{
    \prftree[r]{
        match
            }{
    \set{a \lessdot \exptype{C}{b,\tt{String}}, a \lessdot \exptype{D}{b,b}}
    }{
    a \lessdot \exptype{C}{b,\tt{String}}, \exptype{C}{b,\tt{String}} \lessdot \exptype{D}{b,b}
    }
    }{
         a \lessdot \exptype{C}{b,\tt{String}}, \exptype{D}{\tt{String},\tt{String}} \doteq \exptype{D}{b,b}
    }}
    {
       a \lessdot \exptype{C}{b,\tt{String}}, b \doteq \tt{String}, b \doteq \tt{String}
        }
\end{displaymath}

\textbf{adopt-rule:}\\
\begin{displaymath}
    \prftree[r]{
        ?
    }{
\prftree[r]{
    reduce2
        }{
    \prftree[r]{
        match
            }{
    \prftree[r]{
        adopt
            }{
    \set{a \lessdot \exptype{C}{\tt{String}}, b \lessdot a, b \lessdot \exptype{C}{c}}
    }{
        \set{a \lessdot \exptype{C}{\tt{String}}, b \lessdot a, b \lessdot \exptype{C}{\tt{String}}, b \lessdot \exptype{C}{c}}
    }
    }{
        \set{a \lessdot \exptype{C}{\tt{String}}, b \lessdot a, b \lessdot \exptype{C}{\tt{String}}, \exptype{C}{\tt{String}} \lessdot \exptype{C}{c}}
    }}
    {
        \set{a \lessdot \exptype{C}{\tt{String}}, b \lessdot a, b \lessdot \exptype{C}{\tt{String}}, \tt{String} \doteq c}
    }}{
    ?
        }
\end{displaymath}


\textbf{circle:}\\
\begin{displaymath}
    \prftree[r]{
        \ldots
    }{
\prftree[r]{
    match
        }{
    \prftree[r]{
        adopt
            }{
    \prftree[r]{
        adopt
            }{
\set{a \lessdot \exptype{C}{x}, b \lessdot a, a \lessdot b}
}{
    \set{a \lessdot \exptype{C}{x}, b \lessdot a, b \lessdot \exptype{C}{x}, a \lessdot b}
}
}{
    \set{a \lessdot \exptype{C}{x}, b \lessdot a, b \lessdot \exptype{C}{x}, a \lessdot b, a \lessdot \exptype{C}{x}}
}}
{
    \set{a \lessdot \exptype{C}{x}, b \lessdot a, b \lessdot \exptype{C}{x}, \exptype{C}{x} \lessdot \exptype{C}{x}, a \lessdot b, \exptype{C}{x} \lessdot \exptype{C}{x}}
}}{
 \ldots
        }
\end{displaymath}



\textbf{Example 1}\\
The algorithm is able to infer the types of multiple classes under specific circumstances.
The individual classes must be given to him after one another.
This comes with the restriction, that the first class is correct on its own and does not use any other class.
The second class that gets compiled can use the first class and so on.

The following example shows how the algorithm infers and compiles multiple classes iteratively.
The class \texttt{Class1} is infered first.
It has only one method which is the identity function,
to which our algorithm allocates the type $\exptype{}{A}\ A \to A$.
The next class \texttt{Class2} is now able to use this generic method.
The blue colored types are inferred in the next iteration of our algorithm.

\begin{table}
\caption{Two classes as input. \texttt{Class1} is infered first (shown in {\color{red}red})}
\begin{tabular}{cc}
\begin{lstlisting}
class Class1 extends Object {
  Class1() { super(); }
  id(a){
    return a;
  }
}
class Class2 extends Class1 {
  Class2() { 
    super(); 
  }
  example(){
    return new Class1().id(this);
  }
}
\end{lstlisting}
&
\begin{lstlisting}
class Class1 extends Object {
  Class1() { super(); }
  (*@ \textcolor{red}{<A> A} @*) id((*@ \textcolor{red}{A} @*) a){
    return a;
  }
}
class Class2 extends Class1 {
  Class2() { 
    super(); 
  }
  (*@ \textcolor{blue}{Class1}@*) example(){
    return this.(*@\textcolor{blue}{<Class1>}@*)id(this);
  }
}
\end{lstlisting}
\end{tabular}
\end{table}

\textbf{Example 2}\\
When compiling a class like the following
we have to first split this class into two classes.
The \texttt{TwoMethods} class can be first split into the classes \texttt{Class1}
and \texttt{Class2} and after being processed by the type inference algorithm it can be assembled back together again.
This leads to a principal typing.
When using our type inference algorithm on the class \texttt{TwoMethods} alone
it would give the method \texttt{id} the type $\texttt{TwoMethods} \to \texttt{TwoMethods}$,
which is not the desired principal type.
\begin{lstlisting}
class TwoMethods extends Object {
  TwoMethods() { super(); }
  id(a){
    return a;
  }
  example(){
    return this.id(this);
  }
}
\end{lstlisting}

\textbf{Example 3}\\
%TODO: Ein Beispiel für die Unify-adapt Regel
FGJ allows subtype relations like the following:
\begin{lstlisting}
class Map<A,B> extends Object {
  Map<A,B>() { super(); }
}
class SpecialMap<A,B,C> extends Map<A,C> {
  SpecialMap<A,B,C>() { super(); }
}
\end{lstlisting}

If for example we have a method \texttt{method} like this:
\begin{lstlisting}
<X> void method(Map<X, String> map){
  ...
}
\end{lstlisting}
and call it:
\begin{lstlisting}
method(new SpecialMap<Object,Integer,String>());
\end{lstlisting}

Then the constraint $\exptype{SpecialMap}{Object,Integer,String} \lessdot \exptype{Map}{X,String}$
is generated by the \textbf{FJTYPE} algorithm.
This constraint will be processed by the \texttt{adapt} rule of the \textbf{Unify} algorithm.
Remember that $(\exptype{SpecialMap}{A,B,C} \olsub \exptype{Map}{A,C}) \in S_\leq$.
\begin{align*}
  Eq& \cup \exptype{SpecialMap}{Object,Integer,String} \lessdot \exptype{Map}{X,String} \\
  \cline{1-2} 
  Eq& \cup \set{\exptype{Map}{[ Object / A ][ Integer / B ][ String / C ](A,C)}
  \doteq \exptype{Map}{X,Integer}} \\
  \cline{1-2} 
  Eq& \cup \set{\exptype{Map}{Object,String}
  \doteq \exptype{C}{X, Integer}}
%Eq \cup \set{\theta_1 \doteq \lambda'_1 \ldo \theta_n \doteq \lambda'_n}
\end{align*}

After the \texttt{adapt} rule got applied we can already see that a correct unificator for this constraint would be
$\sigma(X) = \texttt{Object}$.

\textbf{Example 4 (Multiple type solutions):}
\begin{lstlisting}
class List<A> extends Object {
  List<A> add(A p){...}
}
class C1 extends Object {
  m1(ls){
    return ls.add(this);
  }
}
class C2 extends Object {
  m2(){
    return new C1().m1(new List<C1>());
  }
}
\end{lstlisting}
When compiling the class \texttt{C1} there are two possible method typings for \texttt{m1}.
One of it would 
\begin{lstlisting}
class List<A> extends Object {
  List<A> add(A p){...}
}
class C1 extends Object {
  List<Object> m1((*@\color{red}List<Object>@*) ls){
    return ls.add(this);
  }
}
class C2 extends Object {
  m2(){
    return new C1().m1((*@\color{red}new List<C1>()@*));
  }
}
\end{lstlisting}
\texttt{List<Object>} would be a correct type for the parameter \texttt{ls}.
The call \texttt{ls.add(this)} still works because the type of \texttt{this} is a subtype of \texttt{Object}.
But then the method \texttt{m2} will be incorrect when compiling the class \texttt{C2}.
Our algorithm has to backtrack to the class \texttt{C1} and use the other possible typing.
The following would be the correct solution with \texttt{List<C1>} as the type for \texttt{ls}.
\begin{lstlisting}
class List<A> extends Object {
  List<A> add(A p){...}
}
class C1 extends Object {
  List<C1> m1((*@\color{green}List<C1>@*) ls){
    return ls.add(this);
  }
}
class C2 extends Object {
  List<C1> m2(){
    return new C1().m1((*@\color{green}new List<C1>()@*));
  }
}
\end{lstlisting}

\textbf{Example 5} (for global type inference)
\begin{lstlisting}

\end{lstlisting}

\section{Assessment}
\label{sec:assessment}

\begin{itemize}
\item NP-hard \todo[inline]{Show NP-completeness!}
\item cannot infer all possible generic methods
\item features that need to be addressed to make it practical (e.g.,
  what's necessary to move from FGJ to full Java: overloading,
  imperative, )
\end{itemize}

Java has some features which make global type inference hard:
\begin{itemize}
\item subtyping and overloading combined with a nominal type system;
  we show NP-hardness even without overloading
  \todo[inline]{correct?}
\item mutable local variables and mutable object state,
\item polymorphic recursion in method calls
\item Inside a Java method it is possible to call every other declared Java method
  \todo[inline]{meaning? The letrec-rule works on the fact that in a let statement:
  \lstinline{let x = e in e2} the expression \texttt{e} is not able to use \texttt{x}}
\item methods can have side effects.
\end{itemize}


For Featherweight Generic Java it is easier to decide because there is no state and therefore no wildcard types.
Also we exclude polymorphic recursion.

% Example for local type inference in Java:
% \begin{lstlisting}[language=java]
% // Java 8 code:
% class Local {
%   // type inference puts in "ArrayList<String>()"
%   List<String> field = new ArrayList<>();
% }
% \end{lstlisting}
% \todo[inline]{Is this a drawback for LVTI? Does GTI put
%   \lstinline{List<String>} or \lstinline{Map} in the
%   \lstinline{outerMap} example?}

\section{Related Work}

formal models for Java, why based on FGJ

\begin{itemize}
\item Welterweight Java
  \url{https://link.springer.com/chapter/10.1007/978-3-642-13953-6_6}
\item Igarashi, A., Pierce, B.C., Wadler, P.: Featherweight Java: a
  minimal core calculus for Java and GJ. ACM TOPLAS 23(3), 396–450
  (2001)
\item Flatt, M., Krishnamurthi, S., Felleisen, M.: A programmer’s
  reduction semantics for classes and mixins. In: Alves-Foss, J. (ed.)
  Formal Syntax and Semantics of Java. LNCS, vol. 1523,
  p. 241. Springer, Heidelberg (1999)
\item Bierman, G.M., Parkinson, M.J., Pitts, A.M.: MJ: An imperative
  core calculus for Java and Java with effects. Technical report,
  University of Cambridge (2003) 
\item 	Elias Castegren, Tobias Wrigstad:
OOlong: an extensible concurrent object calculus. SAC 2018: 1022-1029
\end{itemize}

Type inference

In sone object-oriented languages like Scala, C\# and also Java type inference
is included. But it is only local type inference \cite{PT98,OZZ01}. Local type
inference means that missing type annotations are recovered using only
information from adjacent nodes in the syntax tree without long distance
constraints such as unification variables. E.g. the types of variables which
a non-functional expression is assgned to or a return type of a method can bei
inferred. But the argument types of arbitrary especially recursive methods
cannot be inferred by local type inference.

The base of many global type inference algorithms is the algorithm $\mathcal{W}$
that was presented by Damas and Milner \cite{DM82}. The fundamental idea
of the algorithm is to determine types by many-sorted type term
unification \cite{Rob65, MM82} where the unification is called for any
application in the functional program. This is an efficient way to infer the types
as many-sorted unification is unitary which means that there is at most one
most general result.  In \cite{Plue07_3} the Milner's approach is adopted
directly to Java. This means that at any method application in the Java program
the finitary unfication is called which means that the result sets are multiplied in
many cases. In \cite{plue15_2} Pl\"umicke changes the approach. There 
type constraints are collected by parsing the abstract syntax tree. After that the
constraints are unified. The used Java-calculus in \cite{plue15_2} has
indeed lambda expressions but there are no methods. Functions are realized by
lambda expressions assingnd to class fields. This means that there is no
overloading, as fields in Java cannot be overloaded.



Martin: unification
The type unification problem which is addressed in Section \ref{sec:unify} is
well-known from polymorphic order-sorted unification which is used in logic
programming languages with polymorphically order-sorted types
\cite{GS89,MH91,HiTo92,CB95}.

In \cite{GS89} the type unification problem
is mentioned as an open problem. For the logical language \textsf{TEL} in
\cite{GS89} an incomplete type inference algorithm is given. The incompleteness
is caused by the fact, that subtype relationships
of polymorphic types which have different arities  (e.g. $\texttt{List(a)} \sub
\texttt{myLi(a,b)}$) are allowed. This leads to the property that there are
infinite chains in the type term ordering. In \textsf{TEL} for \texttt{List(a)}
$\,\leq\,$ \texttt{myLi(a,b)} holds: 

\smallskip
{\centering $\texttt{List(a)} \,\leq\, \texttt{myLi(a,List(a))} \,\leq\,
\texttt{myLi(a,myLi(a,List(a)))}  \leq \ldots$\\}

Caused by the invariance of Java--Types this problem arises only if wildcards
are allowed. This means in FGJ this problem is not present.

\medskip
Nevertheless, the type unification 
fails in some cases without infinite chains, although there is a unifier. Let
for example $\texttt{nat} 
\sub \texttt{int}$, and the set of inequations $\set{\mathtt{nat} \lessdot \mathtt{a}, \mathtt{int}
  \lessdot \mathtt{a}}$ be given, then $\set{\mathtt{a} \mapsto \mathtt{nat}}$
is determined, such that 
$\set{\mathtt{int} \lessdot \mathtt{nat}}$ fails, although $\set{\mathtt{a}\mapsto \mathtt{int}}$ is
a unifier. For $\set{\mathtt{int} \lessdot \mathtt{a}, \mathtt{nat} \lessdot
  \mathtt{a}}$ the algorithm determines the correct unifier
$\set{\mathtt{a}\mapsto \mathtt{int}}$.

In the typed logic programs of \cite{HiTo92} subtype
relationships of polymorphic types are allowed only between type
constructors of the same arity which means that there are no infinite chains.
In this approach a \emph{most general type unifier (mgtu)} is
defined as an upper bound of different principal type unifiers. In general
there are no upper bounds of two given type terms in the type term ordering,
which means that there is in general no mgtu in the sense of \cite{HiTo92}.
For example for  $\texttt{nat} \sub \texttt{int}$, $\texttt{neg} 
\sub \texttt{int}$, and the set of inequations $\set{\mathtt{nat} \lessdot
  \mathtt{a}$, $\mathtt{neg} \lessdot \mathtt{a}}$ the mgtu $\set{\mathtt{a} \mapsto \texttt{int}}$ is
determined. If the type term ordering is extended by $\mathtt{int} \sub
\mathtt{index}$ and $\mathtt{int} \sub \mathtt{expr}$, then there are three
unifiers $\set{\mathtt{a} \mapsto \texttt{int}}$, $\set{\mathtt{a} \mapsto
  \mathtt{index}}$, and $\set{\mathtt{a} \mapsto 
\mathtt{expr}}$, but none of them is a mgtu in the sense of \cite{HiTo92}.

The type system of \textsf{PROTOS-L} \cite{CB95} was
derived from \textsf{TEL} by disallowing any explicit subtype relationships
between polymorphic type constructors. 

In \cite{CB95}
a complete type unification algorithm is given, which can be extended to the
type system of \cite{HiTo92}. They solved the type unification problem for type
term orderings following the restrictions of \textsf{PROTOS-L} respectively the
restrictions of \cite{HiTo92}. Additionally, the result of this paper is, that
the type unification problem is not unitary, but finitary. This means in
general that there is more than one general type unifier. 

For the above example the algorithm determines, where $\texttt{nat} \sub
\texttt{int}$, $\texttt{neg} \sub \texttt{int}$, $\mathtt{int} \sub
\mathtt{index}$, and $\mathtt{int} \sub \mathtt{expr}$ and the set of
inequations $\set{\mathtt{nat} \lessdot
  \mathtt{a}$, $\mathtt{neg} \lessdot \mathtt{a}}$ is given, the three general
unifiers $\set{\mathtt{a} \mapsto \texttt{int}}$, $\set{\mathtt{a} \mapsto
  \mathtt{index}}$, and $\set{\mathtt{a} \mapsto \mathtt{expr}}$. 

In \cite{plue09_1} the open problem of infinite chains from \cite{GS89} was
solved. It is showed that in any infinite chain there is a finite number of elements such that
all other elements of the chain are instances of them which means that these
elements are more more genaral than the other elements of the chain. This type
unfication algorithm is used for the type inference algorithm for Java--5 with
wildcards in \cite{Plue07_3}. As FGJ allows no wildcards we used in this paper
a reduction  of the unification algorithm in \cite{plue09_1}.





\section{Conclusions}
\label{sec:conclusions}


\bibliographystyle{splncs04}
\bibliography{peter,martin}

\end{document}
\endinput
%%
%% End of file `TIforGFJ.tex'.

%%% Local Variables:
%%% mode: latex
%%% TeX-master: t
%%% End:
