%%
%% This is file `sample-acmsmall-submission.tex',
%% generated with the docstrip utility.
%%
%% The original source files were:
%%
%% samples.dtx  (with options: `acmsmall-submission')
%% 
%% IMPORTANT NOTICE:
%% 
%% For the copyright see the source file.
%% 
%% Any modified versions of this file must be renamed
%% with new filenames distinct from sample-acmsmall-submission.tex.
%% 
%% For distribution of the original source see the terms
%% for copying and modification in the file samples.dtx.
%% 
%% This generated file may be distributed as long as the
%% original source files, as listed above, are part of the
%% same distribution. (The sources need not necessarily be
%% in the same archive or directory.)
%%
%% The first command in your LaTeX source must be the \documentclass command.
\documentclass[acmsmall,screen,review]{acmart}

\usepackage{prolog}

%%
%% \BibTeX command to typeset BibTeX logo in the docs
\AtBeginDocument{%
  \providecommand\BibTeX{{%
    \normalfont B\kern-0.5em{\scshape i\kern-0.25em b}\kern-0.8em\TeX}}}

%% Rights management information.  This information is sent to you
%% when you complete the rights form.  These commands have SAMPLE
%% values in them; it is your responsibility as an author to replace
%% the commands and values with those provided to you when you
%% complete the rights form.
\setcopyright{acmcopyright}
\copyrightyear{2018}
\acmYear{2018}
\acmDOI{10.1145/1122445.1122456}


%%
%% These commands are for a JOURNAL article.
\acmJournal{JACM}
\acmVolume{37}
\acmNumber{4}
\acmArticle{111}
\acmMonth{8}

%%
%% Submission ID.
%% Use this when submitting an article to a sponsored event. You'll
%% receive a unique submission ID from the organizers
%% of the event, and this ID should be used as the parameter to this command.
%%\acmSubmissionID{123-A56-BU3}

%%
%% The majority of ACM publications use numbered citations and
%% references.  The command \citestyle{authoryear} switches to the
%% "author year" style.
%%
%% If you are preparing content for an event
%% sponsored by ACM SIGGRAPH, you must use the "author year" style of
%% citations and references.
%% Uncommenting
%% the next command will enable that style.
%%\citestyle{acmauthoryear}

%%
%% end of the preamble, start of the body of the document source.
\begin{document}

%%
%% The "title" command has an optional parameter,
%% allowing the author to define a "short title" to be used in page headers.
\title{Type inference for Generic Featherweight Java}

%%
%% The "author" command and its associated commands are used to define
%% the authors and their affiliations.
%% Of note is the shared affiliation of the first two authors, and the
%% "authornote" and "authornotemark" commands
%% used to denote shared contribution to the research.
\author{Andreas Stadelmeier}
\affiliation{%
  \institution{DHBW Stuttgart, Campus Horb}
  \streetaddress{Tannenweg 4}
  \city{Tübingen}
  \country{Germany}}
\email{a.stadelmeier@hb.dhbw-stuttgart.de}


%%
%% By default, the full list of authors will be used in the page
%% headers. Often, this list is too long, and will overlap
%% other information printed in the page headers. This command allows
%% the author to define a more concise list
%% of authors' names for this purpose.
\renewcommand{\shortauthors}{Stadelmeier, Plümicke, Thiemann}

%%
%% The abstract is a short summary of the work to be presented in the
%% article.
\begin{abstract}
  Type inference for Generic Featherweight Java
\end{abstract}

%%
%% The code below is generated by the tool at http://dl.acm.org/ccs.cfm.
%% Please copy and paste the code instead of the example below.
%%
\begin{CCSXML}
<ccs2012>
 <concept>
  <concept_id>10010520.10010553.10010562</concept_id>
  <concept_desc>Computer systems organization~Embedded systems</concept_desc>
  <concept_significance>500</concept_significance>
 </concept>
 <concept>
  <concept_id>10010520.10010575.10010755</concept_id>
  <concept_desc>Computer systems organization~Redundancy</concept_desc>
  <concept_significance>300</concept_significance>
 </concept>
 <concept>
  <concept_id>10010520.10010553.10010554</concept_id>
  <concept_desc>Computer systems organization~Robotics</concept_desc>
  <concept_significance>100</concept_significance>
 </concept>
 <concept>
  <concept_id>10003033.10003083.10003095</concept_id>
  <concept_desc>Networks~Network reliability</concept_desc>
  <concept_significance>100</concept_significance>
 </concept>
</ccs2012>
\end{CCSXML}

\ccsdesc[500]{Computer systems organization~Embedded systems}
\ccsdesc[300]{Computer systems organization~Redundancy}
\ccsdesc{Computer systems organization~Robotics}
\ccsdesc[100]{Networks~Network reliability}

%%
%% Keywords. The author(s) should pick words that accurately describe
%% the work being presented. Separate the keywords with commas.
\keywords{typeinference, java, compiler}


%%
%% This command processes the author and affiliation and title
%% information and builds the first part of the formatted document.
\maketitle

\section{Introduction}

\section{Preliminaries}

\subsection{Input assumptions}
% No overloaded methods
% No Or-Constraints
The input is a GFJ program lacking the type assignments for method parameters and method return types.


The Typeless Generic Featherweight Java (TGFJ) syntax is different in that from normal Generic Featherweight Java (GFJ) that it is possible
to omit any type annotation.%, except the ones for casts and \texttt{new} calls.
We declare the syntax for TGFJ as follows:

\begin{align*}
  T ::=& X \, | \, N \\
  N ::=& \exptype{C}{\ol{T}}\\
  L ::=& \mathtt{class } \ \exptype{C}{\ol{X} \triangleleft \ol{N}} \ \triangleleft \ N \{ \overline{T} \ \overline{f}; \, K \, \overline{M} \} \\
  K ::=& C(\overline{T} \ \overline{f})\{\mathtt{super}(\overline{f}); \ \mathtt{this}.\overline{f}=\overline{f};\} \\
  %M ::=& \exptype{}{\ol{X} \triangleleft \ol{X}}\ T \ \mathtt{m}(\overline{T} \, \overline{x})\{ \mathtt{ return }\ e; \} \\
  M ::=& \mathtt{m}(\overline{x})\{ \mathtt{ return }\ e; \} \\
  e ::=& \mathtt{this} \, | \, x \, | \, e.f \, | \, e.\mathtt{m}(\overline{e}) \, | \, \mathtt{new }\ C(\overline{e}) \, | \, (C) e
  %M ::=& T \ \mathtt{m}(\overline{T} \, \overline{x})\{ \mathtt{ return }\ e; \} \\
  %e ::=& \mathtt{this} \, | \, x \, | \, e.f \, | \, e.\exptype{\mathtt{m}}{\ol{T}}(\overline{e}) \, | \, \mathtt{new }\ C(\overline{e}) \, | \, (C) e \\
\end{align*}

All type annotations in our TGFJ language can be omitted ($T = \epsilon$).
The only exception are fields which must be given a concrete type.

Another difference to the syntax of FJ is that we added the special variable \texttt{this} to the syntax.
FJ treats \texttt{this} as a normal variable
but our algorithm treats it as a special variable which always has a predetermined type;
the type of the class it is used in.

We assume every method name is only used once in the input program.

Type inference for polymorphic recursion is undecidable.
Therefore we have to alter the GFJ typing rules to exclude polymorphic recursion in method calls:
\begin{enumerate}
    \item The \texttt{MT-CLASS} rule is removed.
    \item We change the \texttt{GT-METHOD} rule:
GT-METHOD:
\begin{align*}
\ddfrac{\begin{array}{c}
\triangle \vdash \ol{X} <: \ol{N}, \ol{Y} <: \ol{P}  \quad \quad 
\triangle \vdash \ol{T}, T \ \texttt{ok} \\
\triangle ; \ol{x}:\ol{T}, this : \exptype{C}{\ol{X}} \vdash e_0 : S \quad \quad
\triangle \vdash S <: T \\
\texttt{class}\ \exptype{C}{\ol{X} \triangleleft \ol{N}} \triangleleft N \{ \ldots \} \quad \quad
%\textit{override}(m, N, \exptype{}{\ol{Y} \triangleleft \ol{P}} \ol{T} \to T)
\textit{mtype}(m, \exptype{C}{\ol{X}}) := \ol{T} \to T \\
%\textit{override}(m, N,)
\end{array}}{
%{\exptype{}{\ol{Y} \triangleleft \ol{P}}\ T \ m(\ol{T}\ \ol{x}) \{ \texttt{return} \ e_0; \} \ \texttt{OK IN}\ \exptype{C}{\ol{X} \triangleleft \ol{N}}}
\textit{mtype}(m, \exptype{C}{\ol{Z}}) = [\ol{Z} / \ol{X}](\exptype{}{\ol{Y}} \ol{T} \to T)
}
\end{align*}
\end{enumerate}

\subsection{Principal Type}
%TODO: Subtype definition:

%In our input set we exclude method overloading.
%Also every method name has to be unique.
%These assumptions allows us to make the following definition of a principal type:

GFJ supports two kinds of types;
Type variables and nonvariable types.
Nonvariable types have the form $\exptype{C}{\ol{T}}$ and can contain multiple type variables.
The type $T_2 = [N/X]\exptype{C}{\ol{T}}$ is a generic instantiation,
where all occurences of the type variable $X$ are replaced by the nonvariable type $N$.

\textbf{Principal Type:}
A nonvariable type for a declaration is a principal nonvariable type,
if any other type-scheme for the declaration is a subtype of a generic instance
of the type-scheme.

Additionally we define the subtype relation of methods as follows:
\begin{align*}
\ddfrac{
\ol{T} \ \leq^* \ \ol{U} \quad \quad U \ \leq^* \ T
}{
\ol{T} \to T \ \leq^* \ \ol{U} \to U
}
\end{align*}

%\begin{theoremAndi}
%  \label{theo:uniquePrincipalType}
%  \textbf{There is one unique principal type for any set of types}
%\end{theoremAndi}

%what about: c<A,Int>, c<Int, A> which is principal type?

\begin{theoremAndi}
  \label{theo:uniquePrincipalType}
  \textbf{There is one unique principal type for every method}
\end{theoremAndi}
Proof:
%We excluded polymorphic recursion
%Every expression $e$ in GFJ has a definitive type.
%%There is only one possible type our type inference algorithm can apply to any expression $e$.
%We will show this for every kind of expression:
%\begin{description}
  %\item[NEW] \texttt{new} can only have one type according to the \texttt{GT-NEW} type rule.
  %\item[Field access] A field access $e.f$ can only have one type according to the \texttt{GT-FIELD} type rule.
  %This also applies to our typeless GFJ version, because every field name has to be unique.
  %\item[Method call] We assume method names are also unique and cannot be overloaded.
  %Due to this assumption and the changed \texttt{GT-METHOD} rule a method type of a method \texttt{m}
  %is: $\textit{mtype(m, N) = \ol{T} \to T}$.
  %%The type of the respective method call $e.m(\ldots)$ is defined by \texttt{GT-INVK} as $T$.
  %%The return type $T$ of the method \texttt{m} is unique, because the 
%\end{description}

We exclude polymorphic recursion, remove overloading and assume every method and field name to be unique.
Now every method $T\ \texttt{m}(\ol{T}\ \ol{x})\{\ \texttt{return}\ e;\ \}$ has one principal type.
This is when its return type $T$ is the same type as its return expression $e$.
For the parameter types $\ol{T}$ the principal type is chosen.

\textit{Example:}
A method \texttt{m} can have multiple possible types due to the fact that GFJ has subtypes.

\begin{lstlisting}
class C1 {
  C1 f;
}
class C2 extends C1{
  m(x){
    return x.f;
  }
}
\end{lstlisting}
The method \texttt{m} can either have the type \texttt{C1 m(C1 x)}
or \texttt{C1 m(C2 x)}.
In this case the principal type would be \texttt{C1 m(C1 x)}.
This type has the type of the return expression as its return type
and for every parameter type it has the principal type.
%The principal type is unique
%No overloading, therefore no types int -> int, byte -> byte

\section{Type inference algorithm}
In this chapter we present our type inference algorithm.
The algorithm is split into following parts:

\begin{enumerate}
\item Create assumptions and subtype relation
\item Constraint generation with \textbf{GFJTYPE}
\item Unification of those constraints
\item Set in principal type solution
\end{enumerate}

The Unify algorithm returns a set of possible type solutions.
This means that there are possibly multiple type solutions for each method.
The last step has to choose the principal type out of those possibilities.

\subsection{Generate Assumptions}
% Every empty Type T in the input is assigned a type variable.
% Assumptions saves every field, method and the class subtype relation

%Generate subtype relationships:

Generating assumptions consists of two parts.
At first we add type variables to the untyped class.
The second part generates the assumption set.
This is the same algorithm for the already typed classes as for the 
new untyped class, which is now equipped with type variables.

\begin{enumerate}
\item Every missing type in the input class gets assigned a fresh type variable.
For methods:
\begin{align*}
  \ddfrac{
  m(\ol{x}) \{ \ldots \} \quad \quad A \cup \ol{A} \ \text{are fresh type variables}
  }{
  A m(\ol{A}\ \ol{x}) \{ \ldots \}
  }
  \end{align*}
  For fields:
\begin{align*}
  \ddfrac{
  \texttt{class}\ \exptype{C}{\ol{X}} \{ \ol{f}; \quad \ldots \} \quad \quad \ol{F} \ \text{are fresh type variables}
  }{
    \texttt{class}\ \exptype{C}{\ol{X}} \{ \ol{F} \ \ol{f}; \quad \ldots \}
  }
\end{align*}
\item We define the two functions $\textit{ftype}_\textit{Ass}$ and $\textit{mtype}_\textit{Ass}$.
Both functions return a set of all types for a method \texttt{m} or a field \texttt{f}.
This is due to the fact that there can be multiple methods and fields with the same name.
\begin{align*}
  %TODO: fresh type variables for generic variables:
  \ddfrac{
    class\ \exptype{C}{\ol{X} \triangleleft \ol{N}}\ \{\ \ol{N}\ \ol{f};\ K\ \ol{M}\ \} \quad \quad
    \exptype{}{\ol{Y}}\ U\ \texttt{m}(\ol{U}\ \ol{x}) \{ \ldots \} \in \ol{M}
  }{
    \textit{mtype}_\textit{Ass}(\texttt{m}) = \exptype{C}{\ol{X} \triangleleft \ol{N}} \to \exptype{}{\ol{Y}} (\ol{U} \to U )
  }
\end{align*}
\begin{align*}
  \ddfrac{
    class\ \exptype{C}{\ol{X} \triangleleft \ol{N}}\ \{\ \ol{T}\ \ol{f};\ K\ \ol{M}\ \} \quad \quad
    T\ \texttt{f} \in \ol{f}
  }{
    \textit{ftype}_\textit{Ass}(\texttt{f}) = \exptype{C}{\ol{X} \triangleleft \ol{N}} \to T
  }
\end{align*}
\item We do not include casts in the syntax and therefore remove the \texttt{GT-UCAST} and \texttt{GT-DCAST} typing rules.
\end{enumerate}


\subsection{GFJTYPE}

The algorithm \textbf{GFJTYPE} is given as follows:

\textbf{FJTYPE}:
$\texttt{TypeAssumptions} \times
\texttt{Class} \rightarrow \texttt{Constraints}\\
 \begin{array}{@{}l@{}l@{}l}
 \textbf{FJT}&\textbf{Y} & \textbf{PE}(Ass, \mathtt{class } \ C \ \mathtt{ extends } \ D \{ \overline{T} \ \overline{f}; \, K \, \overline{M} \}) =\\
& \multicolumn{2}{@{}l@{}}{ \{ \ \textbf{TYPEMethod}(\textit{Ass} \cup \{ \mathtt{this} : C \}, m_i) \quad | \quad m_i \in \overline{M} \ \} }\\ 
\end{array}$

The \textbf{FJTYPE} function gets called for every class in the input.
This function accumulates all the constraints generated from calling the
\textbf{TYPEMethod} function for each method declared in the given class.

$\textbf{TYPEMethod}:\texttt{TypeAssumptions} \times
\texttt{Method} \rightarrow \texttt{Constraints}\\
\begin{array}{@{}l@{}l@{}l}
\textbf{TY}& \textbf{PE} & \textbf{Method} (Ass, T_r \ \mathtt{m}(\overline{T} \, \overline{x})\{ \mathtt{ return }\ e; \}) =\\
& \textbf{let}
& Ass_m = Ass \cup \{ \overline{T} : \overline{x} \}\\
& & \ul{(e:rty, ConS)} = \textbf{TYPEExpr}(Ass_m, e)\\
& \mathbf{in}
& (ConS \cup (rty \lessdot T_r))\\
\end{array}
$

The \textbf{TYPEMethod} function for methods just calls the \textbf{TYPEExpr} function with the
return expression. It is significant to note that it adds the assumptions for the method parameters to the global assumptions before passing them to \textbf{TYPEExpr}.
%and the global assumptions plus the assumptions for the method parameters.

\smallskip

In the following we define the \textbf{TYPEExpr} function for every possible expression:

\smallskip

$\textbf{TYPEExpr}:\texttt{TypeAssumptions} \times
\texttt{Expression} \rightarrow \texttt{Type} \times \texttt{Constraints}\\
\begin{array}{@{}l@{}l}
\textbf{TY} \textbf{PE} & \textbf{Expr} (Ass, \mathtt{this}) = (t , \{\})\\
& \textbf{with } (\mathtt{this} : t) \in Ass 
\end{array}
$
\smallskip
$\begin{array}{@{}l@{}l}
\textbf{TY} \textbf{PE} & \textbf{Expr} (Ass,x) = (t , \{\})\\
& \textbf{with } (x : t) \in Ass 
\end{array}
$

\smallskip

$\begin{array}{@{}l@{}l@{}l}
\textbf{TY}& \textbf{PE} & \textbf{Expr} (Ass, e.f ) = \\
& \textbf{let} % \\
% &
& (rty, ConS) = \textbf{TYPEExpr}(Ass, e),\\
& & \textbf{fresh} = \text{a mapping from each variable in}\ \ol{X} \ \text{to a fresh type variable},\\
& & Ass_{f} = \textit{ftype}_{Ass}(f) = \exptype{C}{\ol{X} \triangleleft \ol{N}} \to T \\
& & Cons_{f} = \{\ rty \doteq \exptype{C}{\textbf{fresh}(\ol{X})}, a \doteq \textbf{fresh}(T)\},\\
%& & OrCons = \{ \{ rty \doteq cl, a \doteq t_f \} \ | \ cl.f : t_f \in Ass \},\\
& \mathbf{in}% \\
% &
& (a, ConS \cup Cons_{f})\\
& & \mathit{where\ } a \mathit{\ is\ a\ fresh\
  type\ variable}\\ 
\end{array}
$

\smallskip

$\begin{array}{@{}l@{}l@{}l}
\textbf{TY}& \textbf{PE} & \textbf{Expr} (Ass, e_r.\mathtt{m}(\overline{e}) ) = \\
& \textbf{let} % \\
% &
& (rty, ConS) = \textbf{TYPEExpr}(Ass, e_r),\\
& & \forall e_i \in \overline{e} : (pt_i, ConS_i) = \textbf{TYPEExpr}(Ass, e_i)  ,\\
& & \textbf{fresh} = \text{a mapping from each variable in}\ \ol{X} \ \text{to a fresh type variable},\\
& & Ass_{m} = \textit{mtype}_{Ass}(m) = \exptype{C}{\ol{X} \triangleleft \ol{N}} \to \exptype{}{\ol{Y}} (\ol{T} \to T) \\
& & \begin{array}{@{}l@{}l}
        Cons_{m} = \{\ & rty \doteq \exptype{C}{\textbf{fresh}(\ol{X})}, a \doteq \textbf{fresh}(T),\\
                    & \bigcup_{T_i \in \overline{T}} (pt_i \lessdot \textbf{fresh}(T_i)) \} %, \textbf{fresh}(\ol{X}) \lessdot \textbf{fresh}(\ol{N})\},\\
    \end{array}\\
& \mathbf{in}% \\
% &
& (a, ConS \cup Cons_{m} \cup \bigcup_i ConS_i))\\
& & \text{where\ } a \text{\ is\ a\ fresh\
  type\ variable}\\ 
\end{array}
$

\smallskip

$\begin{array}{@{}l@{}l@{}l}
\textbf{TY}& \textbf{PE} & \textbf{Expr} (Ass, \mathtt{new }\ N(\overline{e}) ) = \\
& \textbf{let} % \\
& \forall e_i \in \overline{e} : (pt_i, ConS_i) = \textbf{TYPEExpr}(Ass, e_i)  ,\\
& & Cons = \{ \bigcup_{T_i \in \overline{T}} (pt_i \lessdot T_i) \ | \ \mathtt{constructor }\ \exptype{C}{\ol{X}}(\overline{T} \overline{x}) \in Ass \},\\
& \mathbf{in}% \\
% &
& (C, ConS \cup \bigcup_i ConS_i)\\
\end{array}
$

\subsubsection{Completeness}
Theorem: The Unify algorithm is complete
Theorem: \textbf{GFJTYPE} generates the principal type
Proof: The \textbf{Unify} algorithm is complete, so the principal type is included in the solution set.
We only have to choose the principal type out of those solutions.

All types that are possible under the GFJ typing rules, plus our additional assumptions,
also comply with the generated constraints.

We match every generated constraint with the respective type rule to show completeness of our \textbf{GFJTYPE} algorithm.
This shows that none of the generated constraints remove a type which otherwise would be possible under the GFJ typing rules.
The constraints are generated on expression statements.
We now compare the constraints for each expression with the appropriate type rule from GFJ:
\begin{description}
  \item [this]
  has always the type of the surrounding class and generates no constraints.
  \item [Local var]
  No constraints are generated.
  \item[Method invocation]
By direct comparison we show that each of the generated constraints do not apply more restrictions than the \texttt{GT-INVK} rule.
The \texttt{GT-INVK} rule states the condition $\textit{mtype}(m, \textit{bound}_\triangle(T_0)) = \exptype{}{\ol{Y}}\ \ol{U} \to U$.
In our version of typeless GFJ every method name is unique
and there is only one class with that particular method.
The constraint $rty \lessdot \exptype{C}{\textbf{fresh}(\ol{X})}$ assures that the type of the expression $e_0$ contains the method \texttt{m}.

\begin{tabular}{l|l}
  \textbf{GFJ Type rule} & \textbf{Constraints} \\
  $\triangle; \Gamma \vdash e_o : T_0$ & $(rty, ConS) = \textbf{TYPEExpr}(Ass, e_r)$\\ 
  $\quad \textit{mtype}(m, \textit{bound}_\triangle(T_0)) = \exptype{}{\ol{Y}}\ \ol{U} \to U$ & $rty \lessdot \exptype{C}{\textbf{fresh}(\ol{X})}$ \\
 %$\textit{mtype}(m, \textit{bound}_\triangle(T_0)) = \ol{U} \to U$ & $rty \doteq cl$\\
 $\triangle; \Gamma \vdash \ol{e} : \ol{S}$ & $\forall e_i \in \overline{e} : (pt_i, ConS_i) = \textbf{TYPEExpr}(Ass, e_i)$\\
 $\triangle \vdash \ol{S} <: \ol{U}$ & $ \bigcup_{T_i \in \overline{T}} (pt_i \lessdot \textbf{fresh}(T_i))$\\
 $\triangle; \Gamma \vdash \mathtt{e_0.m(\overline{e}) : U }$ & $a \doteq \textbf{fresh}(T)$ \\
\end{tabular}

\textit{Note}: The \textbf{TYPEExpr} function only generates constraints which apply to our assumption.
 \item[Field access]
Mostly the same as method invocation.
Fieldnames by default are unique in the GFJ language.

 \begin{tabular}{l|l}
   \textbf{GFJ Type rule} & \textbf{Constraints} \\
   $\Gamma \vdash e_0:T_0$ & $(rty, ConS) = \textbf{TYPEExpr}(Ass, e_r)$\\ 
   $\quad \mathit{fields}(\mathit{bound}_\triangle(T_0)) = \overline{T} \ \overline{f}$ & $rty \doteq \exptype{C}{\textbf{fresh}(\ol{X})}$ \\
  %$\textit{mtype}(m, \textit{bound}_\triangle(T_0)) = \ol{U} \to U$ & $rty \doteq cl$\\
  $\triangle; \Gamma \vdash \ol{e} : \ol{S}$ & $\forall e_i \in \overline{e} : (pt_i, ConS_i) = \textbf{TYPEExpr}(Ass, e_i)$\\
  $\triangle \vdash \ol{S} <: \ol{U}$ & $ \bigcup_{T_i \in \overline{T}} (pt_i \lessdot \textbf{fresh}(T_i))$\\
  $\triangle; \Gamma \vdash \mathtt{e_0.m(\overline{e}) : U }$ & $a \doteq \textbf{fresh}(T)$ \\
 \end{tabular}
 \item[Constructor]

\begin{tabular}{l|l}
  $\triangle; \Gamma \vdash \ol{e} : \ol{S}$ & $\forall e_i \in \overline{e} : (pt_i, ConS_i) = \textbf{TYPEExpr}(Ass, e_i)$\\
  $\triangle \vdash \ol{S} <: \ol{T}$ & $\bigcup_{T_i \in \overline{T}} (pt_i \lessdot T_i)$
\end{tabular}
  
\end{description}

\section{Unify}
This chapter describes the \textbf{GenericUnify} algorithm
which is used to find type solutions for the constraints generated by \textbf{FGJType}.

\begin{description}
\item[input] A set of type constraints $Cons_{in}$ and a set of subtype relationships $S_\leq$
\item[output] A set of type unifiers $Uni$
or fail $Uni = \emptyset$.
%The unifiers have the form of $Uni = \{ \sigma_1, \ldots , \sigma_n \}$.
\end{description}

The algorithm starts by setting $Eq_{set} = \set{ Cons_{in} }$.
Afterwards the following steps are repeatedly executed on $Eq_{set}$ until the algorithm termiates:
%For every $Eq \in Eq_{set}$ the following steps are applied.
%The resulting unifiers $\sigma$ from each $Eq$ merged together form the result of the \textbf{Unify} algorithm.

%$\textbf{SubUnify} :: [Constraint] \to [Unifier]$
\begin{enumerate}
\item Repeated application of the rules depicted in figure \ref{fig:fgjreduce-rules} and \ref{fig:fgjerase-rules}.
The end configuration of $Eq$ is reached if for each element
no rule is applicable.

\item
(The function $\textbf{fresh}(i)$ returns an array of $i$ fresh type variables.)

\begin{align*}
Eq_1 =& \text{Subset of pairs where both type terms are type variables}\\
Eq_2 =& Eq / Eq_1 \\
Eq_{set}\\ 
    = 
     & \begin{array}[t]{l@{\,}ll}
      \times \, (\displaystyle{\bigotimes_{(a \lessdot \exptype{C}{\ol{X}}) \in Eq'_2}}
      \{\,(a \doteq \exptype{D}{\ol{A}}) \, \cup \, (\ol{X} \doteq \ol{Y'}) \ | \ \sarray{@{}l}{
        (\exptype{D}{\ol{Z}} \olsub \exptype{C}{\ol{Y}}) \in S_\leq, \\
        \ol{A} = \textbf{fresh}(\#(\ol{Z})), \\
        \ol{Y'} = [\ol{A}/\ol{Z}]\ol{Y}
        \,\})}\\ 
      \end{array}\\
   %& \times\, 
   %   (\displaystyle{\bigotimes_{(\exptype{C}{\overline{T}} \lessdot a) \in Eq'_2}}\!\!
   %   \set{(a \doteq [\ol{T}/\ol{X}]N ) \ | \ (\exptype{C}{\overline{X}} \leq N) \in
   %     S_\leq})\\
    & \times\, 
      (\displaystyle{\bigotimes_{(\exptype{C}{\overline{T}} \lessdot a) \in Eq'_2}}\!\!
      \set{(a \doteq [\ol{T}/\ol{X}]N ) \ | \ (\exptype{C}{\overline{X} \triangleleft \ol{N}} \leq N) \in
        S_\leq})\\
    & \times\, \set{[a \doteq \theta \ | \  (a \doteq \theta) \in Eq'_2]} \times Eq_1 \\
\end{align*}

\item \label{subst-step}  Application of the following \emph{subst} rule
    %\begin{enumerate}
    %\item Apply the following subst\_eq rule
    
      $$\begin{array}[c]{lll}
        (\mathrm{subst}) &
        \begin{array}[c]{l}
          Eq'' \cup \set{a \doteq \theta}\\
          \hline
          Eq''[a \mapsto \theta] \cup \set{a \doteq \theta}
        \end{array}
        & a \textrm{ occurs in } Eq'' \textrm{ but not in } \theta 
      \end{array}$$
      
      for each $a \doteq \theta$ in each element of $Eq' \in Eq'_{set}$.

\item 
    \begin{enumerate}
    \item Foreach $Eq \in Eq_{set}$ which has changed in the last step
      start again with the first step.
    \item Build the union $Eq_{set}$ of all results of (a) and all $Eq' \in
      Eq'_{set}$ which has not changed in the last step.
    \end{enumerate}
\item
\begin{enumerate}
\item Filter all constraint sets which are in solved form:\\
$Eq_{solved} = \set{ Eq \ | \ Eq \in Eq_{set}, Eq \ \text{is in solved form}}$
\item We apply the following rule to every constraint set in $Eq_{solved}$:
\begin{align*}
\ddfrac{
  Eq \cup \set{ a \lessdot b } %There are only Type variables left at this point
}{
  Eq \cup \set{ a \doteq b }
}
\end{align*}
\item $\emph{Uni} = \set{\sigma \ | \ Eq \in Eq_{solved},\ \sigma = \set{a \mapsto \theta \ | \ (a \doteq \texttt{T}) \in Eq} }$
%\item $\emph{Uni} = \sarray{l@{\ }l}{\set{\sigma \ | & Eq'' \in Eq''_{set},
%        Eq'' \textrm{ is in solved form,}\\ 
%        & \sigma = \set{a \mapsto \theta \ | \ (a \doteq \theta) \in Eq''}
%        \\ & \quad \cup \ \set{a \mapsto \texttt{A}, b \mapsto \texttt{B} \ | \ (a \lessdot b) \in Eq \ \text{and a is an isolated type variable}}
%        }}$
\end{enumerate}
\end{enumerate}

\begin{figure}
\begin{center}
    \leavevmode
    \fbox{
    \begin{tabular}[t]{ll}
      (adapt)
      & $
      \begin{array}[c]{ll}
      \begin{array}[c]{l}
         Eq \cup \, \set{\exptype{D}{\ol{A}} \lessdot
          \exptype{C}{\ol{B}}} \\ 
        \hline
        \vspace*{-0.4cm}\\
        Eq \cup \set{\exptype{C}{[ \ol{A} / \ol{X} ]\ol{Y}}
        \doteq \exptype{C}{\ol{B}}}
      %Eq \cup \set{\theta_1 \doteq \lambda'_1 \ldo \theta_n \doteq \lambda'_n}
      \end{array}
      & (\exptype{D}{\ol{X}} \olsub \exptype{C}{\ol{Y}}) \in S_\leq 
      \end{array}
      $
    \\\\
(reduce1) & $
\begin{array}[c]{l}
  Eq \cup \set{\exptype{D}{\ol{A}} \lessdot
    \exptype{D}{\ol{B}}}\\
  \hline
  Eq \cup \set{\ol{A} \doteq \ol{B}}
\end{array}
      $ \\\\
(reduce2) & $
\begin{array}[c]{l}
  Eq \cup \set{\exptype{D}{\ol{A}} \doteq
    \exptype{D}{\ol{B}}}\\
  \hline
  Eq \cup \set{\ol{A} \doteq \ol{B}}
\end{array}
      $ \\\\
    \end{tabular}}
  \end{center}
\caption{Reduce and adapt rules}\label{fig:fgjreduce-rules}
\end{figure}

\begin{figure}
\begin{align*}
&\begin{tabular}[t]{ll}
      (erase1)  & $ 
      \begin{array}[c]{ll}
        \begin{array}[c]{l}
          Eq \cup \set{C \lessdot D}\\
          \hline
          Eq
        \end{array}
        & C \leq^* D \in S_\leq
      \end{array}$\\
          \end{tabular}\\
&\begin{tabular}[t]{ll}
      (erase2)  & $ 
      \begin{array}[c]{ll}
        \begin{array}[c]{l}
          Eq \cup \set{C \doteq C}\\
          \hline
          Eq
        \end{array}
      \end{array}$\\
          \end{tabular}\\
    &      \begin{tabular}[t]{ll}
       (swap) & $
            \begin{array}[c]{ll}
              \begin{array}[c]{l}
                Eq \cup \set{C \doteq a}\\
                \hline
                Eq \cup \set{a \doteq C}
              \end{array}
            \end{array}$
          \end{tabular}
\end{align*}
\caption{Erase and swap rules}\label{fig:fgjerase-rules}
\end{figure}

\subsection{Unify proof}

%Soundness: We have to prove that each calculated result of the algorithm is a general unifier of the corresponding input
\begin{theoremAndi}
  \label{theo:unifySoundness}
  \textbf{(Soundness):}
  If the \textbf{Unify} algorithm finds a solution it does not contradict any of the input constraints:
  $\nexists (a \lessdot b) \in {Cons}_{in}$ where $\sigma(a) \nleq \sigma(b)$  
\end{theoremAndi}
%We show this by induction.
%No step alters the constraint set in a way that would make a wrong solution possible.
\textit{Proof:}
We show theorem \ref{theo:unifySoundness} by going backwards over every step of the algorithm.
We assume there exists a unifier $\sigma = \set {a_1 \mapsto \theta_1, \ldots , a_n \mapsto \theta_n}$ for the input constraints,
which is the result of the \textbf{Unify} algorithm.
This means for every constraint in the input set $(a \lessdot b) \in {Cons}_{in}$ and $(c \doteq d) \in {Cons}_{in}$
this unifier will substitute all variables in a way that all constraints are satisfied:
$\sigma(a) \leq \sigma(b)$, $\sigma(c) = \sigma(d)$\\

We now look at each step of the \textbf{Unify} algorithm
which transforms the input set of constraints $Eq$ to a set $Eq'$.
If we assume the unifier $\sigma$ is correct for the set $Eq'$,
then we can show that it will also be correct for the constraints $Eq$. 


%TODO:
%Assume we find a unifier. Then do every step backwards.
%For each step the constraint set before and after the step have the same unifier.
% -> This means no step adds an incorrect unifier / makes a incorrecto unifier possible
%At the end we must end at the correct unifier.

\begin{description}
\item[Step 5 c)]
The last step of the algorithm transforms a set of constraints $Eq$ of the the form

$Eq = \set{a_1 \doteq \theta_1, \ldots , a_n \doteq \theta_n}$

to the unifier $\sigma$.
Trivial.

\item[Step 5 b)]
A unifier which is correct for $a \doteq b$ is also correct for $a \lessdot b$.

\item[Step 5 a)]
Trivial, we do not alter the constraint set which lateron leads to the unifier.

\item[Step 4]
Trivial, the constraint sets are not altered here.

\item[Step 3]
An unifier $\sigma$ that is correct for a constraint set
$Eq[a \to \theta] \cup (a \doteq \theta)$ is also correct for
the set $Eq \cup (a \doteq \theta)$.
From the constraint $(a \doteq \theta)$ it follows that $\sigma(a) = \theta$.
This means that $\sigma(Eq) = \sigma(Eq[a \to \theta])$,
because every occurence of $a$ in $Eq$ will be raplaced by $\theta$ anyways when using the unifier $\sigma$.

\item[Step 2]
This step transforms constraints of the form $(\exptype{C}{\ol{X}} \lessdot a)$ and $(a \lessdot \exptype{C}{\ol{X}})$
into sets of constraints and builds the cartesian product with the remaining constraints.
We can show that if there is a resulting set of constraints which has $\sigma$ as its correct unifier
then $\sigma$ also has to be a correct unifier for the constraints before this transformation.
Proof:
%normal version:
%\begin{description}
%\item[$(a \lessdot C)$] If $\sigma$ is a correct unifier for a set containing $(a \doteq \theta)$
%and $\theta \leq C$, then $\sigma$ is also a correct unifier for the set containing $(a \lessdot C)$.
%\item[$(C \lessdot a)$] Same goes the other way. If $C leq \theta$ and $\sigma$ is correct for $(C \lessdot \theta)$
%then $\sigma$ is also correct for $(a \doteq \theta)$
%\end{description}

\begin{description}
\item[$(a \lessdot \exptype{C}{\ol{X}})$] If $\sigma$ is a correct unifier for a set containing $(a \doteq \exptype{D}{\ol{A}})$
and $(\ol{X} \doteq \ol{Y'})$
, then $\sigma$ is also a correct unifier for the set containing $(a \lessdot \exptype{C}{\ol{X}})$.
The reason is the \texttt{S-CLASS} subtype rule of GFJ. %TODO: Hier etwas ausführlicher?
%TODO:
%\item[$(\exptype{C}{\ol{T}} \lessdot a)$] Same goes the other way. If $\exptype{C}{\ol{X} leq \theta$ and $\sigma$ is correct for $(C \lessdot \theta)$
%then $\sigma$ is also correct for $(a \doteq \theta)$

%If $\sigma$ is a correct unifier for a set \\
%$Eq = Eq' \cup (a \doteq \exptype{D}{\ol{A}}) \, \cup \, (\ol{X} \doteq \ol{Y'})$ \\
%then it is also a correct unifier for a set $Eq = Eq' \cup (a \lessdot \exptype{C}{\ol{X}})$: \\
When substituting $(a \to \exptype{D}{\ol{A}})$ and $(\ol{X} \to \ol{Y'})$
and finally $ (\ol{Y'} \to [\ol{A}/\ol{Z}]\ol{Y})$ in the constraint $(a \lessdot \exptype{C}{\ol{X}})$
we get: $(\exptype{D}{\ol{A}} \lessdot [\ol{A}/\ol{Z}]\exptype{C}{\ol{Y}}$
which is correct under the \texttt{S-CLASS} rule (see figure \ref{fig:gfj-subtyping-rules}).

\item[$(\exptype{C}{\ol{T}} \lessdot a)$] If $\exptype{C}{\ol{X}} leq \theta$ and $\sigma$ is correct for $(a \doteq [\ol{T}/\ol{X}]N)$
then $\sigma$ is also correct for $(\exptype{C}{\ol{T}} \lessdot a))$.
When substitutin $a$ for $[\ol{T}/\ol{X}]N$ we get 
$(\exptype{C}{\ol{T}} \lessdot [\ol{T}/\ol{X}]N)$
, which is correct because $(\exptype{C}{\ol{X}} \leq \exptype{C}{\ol{Y}}$
(see \texttt{S-CLASS} rule).

\begin{figure}
$\ddfrac{
  \texttt{class}\ \exptype{C}{\ol{X} \triangleleft \ol{N}} \triangleleft N \set{ \ldots }
}{
  \triangle \vdash \exptype{C}{\ol{T}} <: [\ol{T}/\ol{X}]N
}$ 
\caption{Generic Featherweight Java subtyping rules}\label{fig:gfj-subtyping-rules}
\end{figure}

\footnote{
Discussion:
Do we need to include the $\triangleleft \ol N$ bounds?
The S-CLASS rule does not mention them.

When ignoring those rules this could lead to an error
class T<X extends List> {
}
class S<X>{}

Constraint: S<String> <. a
Unify: T<String> =. a // ERROR!

This is not a problem because no error is gonna result from this.
TYPEExpr only hast to implement all of the typing rules of FJ
and unify has to solely respect the subtyping rules.

}
\end{description}

\item[Step 1]
\begin{description}
\item[erase-rules] remove correct constraints from the constraint set.
A unifier $\sigma$ that is correct for the constraint set $Eq$
is also correct for $Eq \cup \set{\theta \doteq \theta}$
and $Eq \cup \set{\theta \lessdot \theta'}$, when $\theta \leq \theta'$.
\item[swap-rule] does not change the unifier for the constraint set.
$\doteq$ is a symmetric operator and parameters can be swapped freely.
\item[adapt] If there is a $\sigma$ which is a correct unifier for a set
$Eq \cup \set{ \exptype{C}{[\ol{A}/\ol{X}]\ol{Y}} \doteq \exptype{C}{\ol{B}}}$ then it is also
a correct unifier for the set $Eq \cup \set{ \exptype{D}{\ol{A}} \lessdot \exptype{C}{\ol{B}}}$,
if there is a subtype relation $\exptype{D}{\ol{X}} \leq^* \exptype{C}{\ol{Y}}$.
To make the set $Eq \cup \set{ [\ol{A}/\ol{X}]\exptype{C}{\ol{Y}} \doteq \exptype{C}{\ol{B}}}$ the unifier 
$\sigma$ must satisfy the condition $\sigma([\ol{A}/\ol{X}]\ol{Y}) = \sigma(\ol{B})$.
By substitution we get $Eq \cup \set{ \exptype{D}{\ol{A}} \lessdot \exptype{C}{[\ol{A}/\ol{X}]\ol{Y}}}$
which is correct under the \texttt{S-CLASS} rule.
\item[reduce] The \texttt{reduce1} and \texttt{reduce2} rules are obviously correct under the FJ typing rules.
\end{description}

\item[OrConstraints]
If $\sigma$ is a correct unifier for one of the constraint sets in $Eq_{set}$
then it is also a correct unifier for the input set $Cons_{in}$.
When building the cartesian product of the \textbf{OrConstraints} every possible
combination for $Cons_{in}$ is build.
No constraint is altered, deleted or modified during this step.
\end{description}
\hfill $\square$


%Completeness: If there is a Unifier (a solution), the algorithm will never alter the constraints in a way that this solution is removed
%\begin{theoremAndi}\label{theo:unifyCompleteness}
%  \textbf{(Completeness):} If there is a solution for the input constraints $Cons_{in}$, the \textbf{Unify} algorithm will not fail.
%  A solution is a non-empty set of unifiers $Uni = \set{\sigma_1, \ldots \sigma_n }$,
%  where each unifier is a injective function which maps every type variable in the input constraints $Cons_{in}$ to a
%  type in $S_\leq$.
%  \end{theoremAndi}
\begin{theoremAndi}\label{theo:unifyCompleteness}
  \textbf{(Completeness):} The \textbf{Unify} algorithm calculates the principal type solution for the input set of constraints ($Cons_{in}$).
  A unifier $\sigma$ is a principal type solution for $Cons_{in}$ if it unifies $Cons_{in}$
  and for every other unifier $\omega$ there is a unifier $\lambda$ so that $\omega(x) = \lambda(\sigma(x))$.
\end{theoremAndi}
\textit{Proof:}
%The \textbf{Unify} calculates multiple solutions.

%Our proof goes as follows:
We look at every step of the algorithm, which alters the set of constraints $Eq$,
while assuming that there is at least one possible principal type solution $\sigma$ for the input.
We will show that the principal type is among them by proofing for every step of the algorithm that the principal type is never excluded.

%Assume there is a unifier and the Unify algorithm finds it.
%Then no rule makes this unifier impossible / removes this unifier.

\begin{description}
\item[Step 1:]
The first step applies the three rules from figure \ref{fig:unifyrules}.
\textbf{erase-rules:} The erase2 rule from figure \ref{fig:unifyrules} removes a
$\{\theta \leq \theta\}$ constraint from the constraint set.
The erase1 rule removes a $\{\theta \leq \theta\}$ constraint,
but only if the two types $\theta$ and $\theta'$ satisfy the constraint.
Both rules do not change the set of possible solutions for the given constraint set.

\textbf{swap-rule:} $\doteq$ is a symmetric operator and parameters can be swapped freely.
This operation does not change the meaning of the constraint set.

\textbf{adapt-rule:} Every solution which is correct for the constraints
$Eq \cup \set{ \exptype{C}{[\ol{A}/\ol{X}]\ol{Y}} \doteq \exptype{C}{\ol{B}}}$ is also
a correct solution for the set $Eq \cup \set{ \exptype{D}{\ol{A}} \lessdot \exptype{C}{\ol{B}}}$.
According to the \texttt{S-CLASS} rule there can only be a possible solution for 
$\exptype{C}{[\ol{A}/\ol{X}]\ol{Y}} \doteq \exptype{C}{\ol{B}}$
if $\ol{B} = [\ol{A}/\ol{X}]\ol{Y}$.
Therefore this transformation does not remove any possible solution from the constraint set.

\textbf{reduce-rule:}
%The constraint is not altered
For a constraint $\exptype{D}{\ol{A}} \lessdot \exptype{D}{\ol{A}}$ the FJ subtyping rule \texttt{S-REFL} ($\triangle \vdash T <: T$) is the only one which applies.
According to this rule the transformation to $\ol{A} \doteq \ol{B}$ is correct.
Only $D$ gets removed, which is not a type variable.
Therefore this step does not remove a possible solution.
This applies for both reduce rules \textbf{reduce1} and \textbf{reduce2}.

\item[Step 2:]
%The second step builds multiple constraint sets of all possible type combinations for the $\lessdot$-constraints.
The second step of the algorithm eliminates $\lessdot$-constraints
by replacing them with $\doteq$-constraints.
%The algorithm considers every possible type from $S_\leq$ which does not violate the eliminated $\lessdot$ -constraint itself.
%This step does not remove a solution from the constraint set.
For each $(a \lessdot \exptype{C}{\ol{X}})$ constraint the algorithm builds a set with every
possible subtype of $\exptype{C}{\ol{X}}$ set in for $a$.
So if there is a correct unifier $\sigma$ for the constraints before this conversion there will be at least one set of
constraints for which $\sigma$ is a correct unifier.
%TODO: what does soundness mean. if there is a possible type solution (with types in S) then unify will find it

\item[Step 3:]
In the third step the \textbf{substitution}-rule is applied.
If there is a constraint $a \doteq \theta$ then there is no other way to fulfill the constraint set
than replacing $a$ with $\theta$.
This does not remove a possible solution.

\item[Step 4:]
None of the constraints get modified.

\item[Step 5 a):]
The removed sets do not have a possible unifier, therefore no possible solution is
omitted in this step.

\textbf{Proof}:
In step 5.a all constraint sets that have a unifier are in solved form.
All other possibilities are eliminated in steps 1-4.
There are 8 different variations of constraints:\\
$(a \doteq a), (a \doteq C), (C \doteq a), (C \doteq C), (a \lessdot a), (a \lessdot C), (C \lessdot a), (C \lessdot C)$

After step 1 there are no $(C \doteq C)$, $(C \lessdot C)$ and $(C \doteq a)$ constraints anymore,
as long as the constraint set has a correct unifier.
Because a constraint set that has a correct unifier cannot contain constraints of the form $\theta_1 \doteq \theta_2$ with $\theta_1 \neq \theta_2$ and
$(\theta_1 \lessdot \theta_2)$ with $(\theta_1 \leq \theta_2) \notin S_\leq$.
By removing $(\theta \doteq \theta)$ and $(\theta \lessdot \theta')$ with $(\theta \leq \theta') \notin S_\leq$ constraints
no constraints of the form $(C \doteq C)$ and $(C \lessdot C)$
remain in a constraint set that has a correct unifier after step 1.

After step 2 there are no more $(a \lessdot C)$ constraints.

After step 3 there are no $(a \doteq C)$ anymore.

We only reach step 5 if the constraint set is not changed by the substitution (step 3).

%If at this point a set $Eq_i$ is not in solved form it has no correct unifier.

If the constraint set has a correct unifier only $(a \lessdot a)$, $(a \doteq a)$ and $(a \doteq C)$ constraints are left at this point.
The type variables in the $(a \lessdot a)$ and $(a \doteq a)$ constraints have to be independent type variables.
If a type variable $c$ is inside a $(c \doteq C)$ constraint it is not an independent type variable.
But this variable $c$ cannot be inside a $(a \doteq a)$ or $(a \lessdot a)$ constraint, because otherwise step 3 would have replaced it in there.


\item[Step 5 b):]
If the algorithm advances to this step we further only work on constraint sets in solved form.
This means there are only two kinds of constraints left.
($A \doteq \texttt{Typ}$), ($A \doteq B$) and ($A \lessdot B$) with $A$ and $B$ as type variables.

%We can set all TVs equal, because we allow only same TVs when having circles in a method call.
%This still will lead to the principal type.

The GFJ language does not allow subtype constraints for generic types.
A constraint like $(A \lessdot B)$ in a solution could be inserted as the typing shown in the example below.
But this is not allowed by the syntax of GFJ.
That is why we can treat this constraint as $(A \doteq B)$.

%TODO: This does not alter the outcome because the solution set is not modified anymore. All other TVs alread have a type like A =. Typ

\textit{Example:}
This would be a valid Java program but is not allowed in GFJ:
\begin{lstlisting}
class Example {
  <A extends Object, B extends A> A id(B a){
    return a;
  }
}
\end{lstlisting}

By replacing all ($A \lessdot B$) constraints with ($A \doteq B$) we do not remove a principal type solution.

\item[Step 6:]
In the last step all the constraint sets, which are in solved form, are converted to unifiers.

We see that only a constraint set which has no unifier does not reach solved form.
We showed that in every step of the \textbf{Unify} algorithm we never exclude a possible unifier.
Also we showed that after we reach step 5 only constraint sets with a correct unifier are in solved form.
By removing all constraint sets which are not in solved form the algorithm does not
remove a possible correct unifier.

If we assume that there is a possible principal type solution $\sigma$ for the input set $Cons_{in}$
and the \textbf{Unify} algorithm does not exclude any of the possible unifiers,
then the result \textbf{Unify} contains the principal type solution.
\hfill $\square$
\end{description}

\end{document}
\endinput
%%
%% End of file `TIforGFJ.tex'.
