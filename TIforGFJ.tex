\documentclass[runningheads]{llncs}

\usepackage{todonotes} % [disable]
\usepackage[utf8]{inputenc}
\usepackage{hyperref}
\usepackage{amsmath}
\usepackage{amssymb}
\usepackage{subcaption}
\usepackage{prftree}
\usepackage{xspace}
\usepackage{color,ulem}
\usepackage{listings}
\lstset{language=Java,
  showspaces=false,
  showtabs=false,
  breaklines=true,
  showstringspaces=false,
  breakatwhitespace=true,
  basicstyle=\ttfamily\fontsize{8}{9.6}\selectfont, %\footnotesize
  escapeinside={(*@}{@*)},
  captionpos=b,
}
\lstdefinestyle{fgj}{backgroundcolor=\color{white}}
\lstdefinestyle{tfgj}{backgroundcolor=\color{lightgray}}


\newtheorem{theoremAndi}{Theorem}
\newtheorem{definitionAndi}{Definition}

\newcommand\mv[1]{{\tt #1}}
\newcommand{\ol}[1]{\overline{\tt #1}}
\newcommand{\exptype}[2]{\mathtt{#1 \texttt{<} #2 \texttt{>} }}
\newcommand\ddfrac[2]{\frac{\displaystyle #1}{\displaystyle #2}}

\newcommand{\sarray}[2]{\begin{array}[t]{#1} #2 \end{array}}

\newcommand{\olsub}{\textrm{$\, \leq^\ast \,$}\ }

\newcommand{\sub}{\mbox{$<$}}

\newcommand{\set}[1]{\{ #1 \} }

\definecolor{red}{rgb}{1,0,0}
\newcommand{\red}[1]{\textcolor{red}{#1}}

\newcommand{\commentarymargin}[1]{\red{\({}^*\)}\marginpar[\tiny \red{\({}^*\)#1}]{\tiny \red{\({}^*\)#1}}}
\newcommand{\erased}[1]{\commentarymargin{\sout{#1}}}

\newcommand\Erase[1]{|#1|}
\newcommand\Angle[1]{\langle#1\rangle}

\newcommand\TFGJ{FGJ-GT\xspace}
\newcommand\FGJType{\textbf{FGJType} }

\newcommand\TVX{\mv X}
\newcommand\TVY{\mv Y}
\newcommand\TVZ{\mv Z}
\newcommand\TVW{\mv W}

\newcommand\CL[1]{\textit{Cl}$_{#1}$}
\newcommand\subconstr{\lessdot}
\newcommand\eqconstr{\doteq}
\newcommand\subeq{\mathbin{\texttt{<:}}}
\newcommand\extends{\triangleleft}

\newcommand\rulename[1]{\textup{\textrm{(#1)}}}

%%% Commands for FGJTYPE algorithm
\newcommand{\tv}[1]{\mathit{ #1 }}
\newcommand{\fjtype}{\textbf{FJTYPE}}
\newcommand{\orCons}{C_{||}}
\newcommand{\typeMethod}{\textbf{TYPEMethod}}

%%% Local Variables:
%%% mode: latex
%%% TeX-master: "TIforGFJ"
%%% End:


\parindent=0mm
\begin{document}

%%
%% The "title" command has an optional parameter,
%% allowing the author to define a "short title" to be used in page headers.
\title{Global Type Inference for Featherweight Generic Java}


\author{Andreas Stadelmeier \and
Martin Plümicke\and
Peter Thiemann}
%
\authorrunning{A. Stadelmeier et al.}
% First names are abbreviated in the running head.
% If there are more than two authors, 'et al.' is used.
%
\institute{DHBW Stuttgart, Campus Horb
\email{a.stadelmeier@hb.dhbw-stuttgart.de}}


%%
%% This command processes the author and affiliation and title
%% information and builds the first part of the formatted document.
\maketitle

%%
%% The abstract is a short summary of the work to be presented in the
%% article.
\begin{abstract}
  Type Inference for Featherweight Generic Java
  \keywords{type inference, Java, compiler}
\end{abstract}



\section{Introduction}
\label{sec:introduction}

Java is one of the most important programming languages. In 2019, Java
was the second most popular language according to a study
based on GitHub
data.\footnote{\url{https://www.businessinsider.de/international/the-10-most-popular-programming-languages-according-to-github-2018-10/}} Estimates
for the number of Java programmers range between 7.6 and 9 million.\footnote{\url{https://www.zdnet.com/article/programming-languages-python-developers-now-outnumber-java-ones/},
\url{http://infomory.com/numbers/number-of-java-developers/}} Java
has been around since 1995 and progressed through 16 versions.

Swarms of programmers have taken their first steps in Java. Many more
have been introduced to object-oriented programming through Java, as
it is among the first mainstream languages supporting
object-orientation. Java is a class-based language with static single inheritance among
classes, hence it has nominal types with a specified subtyping
hierarchy. Besides classes there are interfaces to characterize common 
traits independent of the inheritance hierarchy. Since version J2SE~5.0,
the Java language supports F-bounded polymorphism in the form of generics.

Java is generally explicitly typed with some amendments introduced in
recent versions. That is, 
variables, fields, method parameters, and method returns must be
adorned with their type. Figure~\ref{fig:intro-example-generic-fj}
contains a simple example with generics.
% from Featherweight Java \cite{DBLP:journals/toplas/IgarashiPW01} 
\begin{figure}[tp]
%   \begin{subfigure}[t]{0.55\textwidth}
% \begin{lstlisting}
% class Pair {
%   Object fst;
%   Object snd;
%   Pair(Object fst, Object snd) {
%    this.fst=fst; 
%    this.snd=snd;
%   }
%   Pair setfst(Object fst) {
%     return new Pair(fst, this.snd);
%   }
% }
% \end{lstlisting}
%   \end{subfigure}
  \begin{subfigure}[t]{0.49\linewidth}
\begin{lstlisting}[style=fgj]
class Pair<X,Y> {
  X fst;
  Y snd;
  Pair<X,Y>(X fst, Y snd) {
    this.fst=fst;
    this.snd=snd;
  }
  Pair<X,Y> setfst(X fst) {
    return new Pair(fst, this.snd);
  }
  Pair<Y,X> swap() {
    return new Pair(this.snd, this.fst);
  }
}  
\end{lstlisting}
    \caption{Featherweight Generic Java (FGJ)}
    \label{fig:intro-example-generic-fj}
  \end{subfigure}
  ~
  \begin{subfigure}[t]{0.49\linewidth}
\begin{lstlisting}[style=tfgj]
class Pair<X,Y> {
  X fst;
 Y snd;
  Pair(fst, snd) {
    this.fst=fst; 
    this.snd=snd;
  }
  setfst(fst) {
    return new Pair(fst, this.snd);
  }
  swap() {
    return new Pair(this.snd, this.fst);
  }
}  
\end{lstlisting}
    \caption{FGJ with global type inference (\TFGJ)}
    \label{fig:intro-example-generic-jtx}
  \end{subfigure}
  \caption{Example code}
  \label{fig:intro-example-code}
\end{figure}
While the overhead of explicit types look reasonable in the example,
realistic programs often contain variable initializations like
the following:\footnote{Taken from
  \url{https://stackoverflow.com/questions/4120216/map-of-maps-how-to-keep-the-inner-maps-as-maps/4120268}.} 
\begin{lstlisting}[basicstyle=\ttfamily\fontsize{8}{9.6}\selectfont,style=fgj]
  HashMap<String, HashMap<String, Object>> outerMap =
    new HashMap<String, HashMap<String, Object>>();
\end{lstlisting}

Java's \emph{local variable type inference} (since version 10\footnote{\url{https://openjdk.java.net/jeps/286}}) deals
satisfactorily with examples like the initialization of
\lstinline{outerMap}. 
In many initialization scenarios for local variables, Java infers their type
if it is obvious from the context. In the
example, we can write
\begin{lstlisting}[basicstyle=\ttfamily\fontsize{8}{9.6}\selectfont,style=fgj]
var outerMap = new HashMap<String, HashMap<String, Object>>();
\end{lstlisting}
because the constructor of the map spells out the type in
full. More specifically,``obvious'' means that the right side of the initialization is
\begin{itemize}
\item a constant of known type (e.g., a string),
\item a constructor call, or
\item a method call (the return type is known from the method
  signature).
\end{itemize}
The \lstinline{var} declaration can also be used for an iteration
variable where the type can be obtained from the elements of the
container or from the initializer.
Alternatively, if the variable is used as the method's return value,
its type can be obtained from the current method's signature.

However, there are still many places where the programmer must provide types. In
particular, an explicit type must be given for
\begin{itemize}
\item a field of a class,
\item a local variable without initializer or initialized to \lstinline{NULL},
\item a method parameter, or
\item a method return type.
\end{itemize}

In this paper, we study \emph{global type inference} for Java. Our aim
is to write code that omits most type annotations, except for class
headers and field types. Returning to the \lstinline{Pair} example, it
is sufficient to write the code in Figure~\ref{fig:intro-example-generic-jtx}
and global type inference fills in the rest so that the result is
equivalent to Figure~\ref{fig:intro-example-generic-fj}. Our
motivation to study global type inference is threefold.
\begin{itemize}
\item Programmers are relieved from writing down obvious types. 
\item Programmers may write types that leak implementation details. The
  \lstinline{outerMap} example provides a good example of this
  problem. From a software engineering
  perspective, it would be better to use a more general abstract type like
\begin{lstlisting}[basicstyle=\ttfamily\fontsize{8}{9.6}\selectfont,style=fgj]
Map<String, Map<String, Object>> outerMap = ...
\end{lstlisting}
  Global type inference finds most general types.
\item Programmers may write types that are more specific than
  necessary instead of using generic types. Here, type
  inference helps programmers to find the most general type. Suppose
  we wanted to add a static  method \texttt{eqPair} for pairs of integers to the
  \lstinline/Pair/ class.
\begin{lstlisting}[basicstyle=\ttfamily\fontsize{8}{9.6}\selectfont,style=fgj]
boolean eqPair (Pair<Integer,Integer> p) {
  return p.fst.equals(p.snd);
}
\end{lstlisting}
  With global type inference it is sufficient to write the code on the
  left of Figure~\ref{fig:equal-pair} and obtain the FGJ code with the most general type on the right.
\end{itemize}
  \begin{figure}[t]
    \begin{minipage}[t]{0.49\linewidth}
\begin{lstlisting}[style=tfgj]
eqPair (p) {
  return p.fst.equals(p.snd);
}
\end{lstlisting}
    \end{minipage}
    \begin{minipage}[t]{0.49\linewidth}
\begin{lstlisting}[style=fgj]
<T> boolean eqPair (Pair<T,T> p){
  return p.fst.equals<T>(p.snd);
}
\end{lstlisting}
    \end{minipage}
    \caption{\lstinline{eqPair} in \TFGJ and FGJ}
    \label{fig:equal-pair}
  \end{figure}
% \item Sometimes, it can be hard to find a correct typing at all.
% \todo[inline]{example?}

To make our investigation palatable, we focus on global type inference for Featherweight
Generic Java \cite{DBLP:journals/toplas/IgarashiPW01} (FGJ), a
functional Java core language with full support for generics. Our type inference algorithm
applies to FGJ programs that specify the full class header and all field types,
but omit all method signatures. 
Given this input, our algorithm
infers a set of most general method signatures (parameter types and return types).
Inferred types are generic as much as possible and may contain
recursive upper bounds.

The inferred signatures have the following round-trip property
(relative completeness). If we
start with an FGJ program that does not make use of polymorphic
recursion (see Section~\ref{sec:polym-recurs}), strip all types from
method signatures, and run the algorithm on the 
resulting stripped program, then at least one of the inferred typings is more
general than the types in the original FGJ program.






\subsection*{Contributions}
\label{sec:contributions}


We specify syntax and type system of the language \FGJGT, which drops all method type
annotations from FGJ and the typing of which rules out polymorphic
recursion. This language is amenable to polymorphic type inference and
each \FGJGT program can be completed to an FGJ program. 

We define a constraint-based algorithm that performs global type
inference for \FGJGT. This algorithm is sound and relatively complete
for FGJ programs without polymorphic recursion. Our algorithm improves
on previous attempts at type inference for Java in the literature as detailed in
Section~\ref{sec:related-work}. 

We investigate the complexity of global type inference and show its NP-completeness.

We implemented a prototype of the type inference algorithm, which we
plan to submit for artifact evaluation.

\if0
\commentary{PT what else do you want to say? Do you want to point to
  your implementation for full Java? Can we point to Andi's prototype
  (check conditions in CFP)?
  Submit as an artifact?}

Our contributions in this paper are an algorithm for global type inference for
FGJ. Therefore we redraft the typing rules of FGJ such that the programs without
type annotations could be correct and in this case polymorphic recursion is
excluded. We prove soundness and completeness of the algorithm about the rules.

The type inference algorithm is reduced to a constraint solving type unification
algorithm. We improved our type unification algorithm such that constraints
of the form $a \lessdot ty$, where $a$ is type variable and $ty$ is is a
non-type variable type,
are not resolved rather converted to bounded type parameters \texttt{a extends
  ty}. This implicates an enormes reduction of solutions of the type
unification algorithm without restricting the generality of typings of
FGJ-programs.

We show that Global type inference for FGJ is NP complete.

Finally we have done an implementation of  global type inference for  a reduced
set of full Java.
\fi

%%% Local Variables:
%%% mode: latex
%%% TeX-master: "TIforGFJ"
%%% End:


\section{Motivation}
\label{sec:motivation}

% Examples, examples, examples from simple to more advanced showing off
% the (difficult) features of inference.

This section presents a sequence of more and more challenging
examples for global type inference (GTI). To spice up our examples
somewhat, we assume some predefined utility classes with the following
interfaces.
\begin{lstlisting}[style=fgj]
class Bool {
  Bool not(); 
}
class Int {
  Int negate ();
  Int add (Int that);
  Int mult (Int that);
}
class Double {
  Double negate ();
  Double add (Double that);
  Double mult (Double that);
}
\end{lstlisting}

We generally use upper case single-letter identifiers like $\TVX,
\TVY, \dots$ for type variables.
Given a FGJ-GT class \CL 0, 
we call any FGJ class \CL i that can be transformed to \CL 0 by
erasing type annotations a \emph{completion of  \CL 0}.

\subsection{Multiplication}
\label{sec:multiplication}

Here is the \TFGJ code  for multiplying the components of a
pair.\footnote{We indicate \TFGJ code fragments by using a
  {gray background}.}
\begin{lstlisting}[style=tfgj]
class MultPair {
  mult (p) { return p.fst.mult(p.snd); }
}
\end{lstlisting}
Assuming the parameter typing $\mv p:\mv P$, result type $\mv R$, and that
\texttt{mult} in the body refers to \texttt{Int.mult}, we
obtain the following constraints.
\begin{itemize}
\item From \texttt{p.fst}: $\mv P \subconstr \mathtt{Pair}\Angle{\TVX,\TVY}$ and
  $\texttt{p.fst} : \TVX$.
\item From \texttt{p.snd}: $\mv P \subconstr \mathtt{Pair}\Angle{\TVZ,\TVW}$ and
  $\texttt{p.snd} : \TVW$.
\item The two constraints on $\mv P$ imply that $\TVX \eqconstr \TVZ$ and
  $\TVY \eqconstr \TVW$.
\item From \texttt{.mult (p.snd)}: $\TVX \subconstr \mathtt{Int}$, $\TVY \subconstr
  \mathtt{Int}$, and $\mathtt{Int} \subconstr \mv R$.
\end{itemize}
The return type $\mv R$ only occurs positively in the constraints, so we can
set $\mv R = \mathtt{Int}$.
The argument type $\mv P$ only occurs negatively in the constraints,
so $\mv P = \mathtt{Pair} \Angle{\TVX,\TVY}$.
This reasoning gives rise to the following completion.
% \begin{lstlisting}[style=fgj]
% class MultPair {
%   Int mult (Pair<Int,Int> p) { return p.fst.mult(p.snd); }
% }
% \end{lstlisting}
\begin{lstlisting}[style=fgj]
class MultPair {
  <X extends Int, Y extends Int>
  Int mult (Pair<X,Y> p) { return p.fst.mult(p.snd); }
}
\end{lstlisting}
We obtain a second completion if we assume that \texttt{mult} refers to
\texttt{Double.mult}.
\begin{lstlisting}[style=fgj]
class MultPair {
  <X extends Double, Y extends Double>
  Double mult (Pair<X,Y> p) { return p.fst.mult(p.snd); }
}
\end{lstlisting}

Finally, the definition of \texttt{mult} might be recursive, which
generates different constraints for the method invocation of \texttt{mult}.
\begin{itemize}
\item From \texttt{.mult (p.snd)}: $\TVX \subconstr \mathtt{MultPair}$, $\TVY \subconstr
  \mathtt{P}$, and $\mathtt{R} \subconstr \mv R$.
\end{itemize}
Transitivity of subtyping applied to $\TVY \subconstr \mathtt{P}$ and $\mv P \subconstr \mathtt{Pair}\Angle{\TVX,\TVY}$
yields the constraint $\TVY \subconstr \mathtt{Pair}\Angle{\TVX,\TVY}$, which triggers the
occurs-check in unification and is hence rejected. 

% The corresponding example in full Java would be the dot product. In
% full Java, we also have to deal with overloading of the multiplication
% operator, which amounts to considering \texttt{Int.mult} and
% \texttt{Double.mult}. Overloading is not allowed in FGJ, but method
% names in different classes may overlap.

The two solutions can be combined to
\begin{lstlisting}[style=fgj]
class MultPair {
  <X extends T1, Y extends T2>
  T0 mult (Pair<X,Y> p) { return p.fst.mult(p.snd); }
}
\end{lstlisting}
where
$ (T_0, T_1, T_2)  \in \{ (\mv{Int}, \mv{Int}, \mv{Int}), (\mv{Double}, \mv{Double}, \mv{Double}) \}$.

\subsection{Inheritance}
\label{sec:inheritance}

\begin{figure}[tp]
  \begin{subfigure}[t]{0.49\linewidth}
\begin{lstlisting}[style=tfgj]
class A1 {
  m(x) { return x.add(x); }
}
class B1 extends A1 {
  m(x) { return x; }
}
\end{lstlisting}
  \end{subfigure}
\begin{subfigure}[t]{0.49\linewidth}
\begin{lstlisting}[style=tfgj]
class A2 {
  m(x) { return x; }
}
class B2 extends A2 {
  m(x) { return x.add(x); }
}
\end{lstlisting}
  \end{subfigure}
  \caption{Method overriding}
  \label{fig:method-overriding}
\end{figure}

Let's start with the artificial example in the left
listing of Figure~\ref{fig:method-overriding} and ignore the \texttt{Double} class. Type
inference proceeds according 
to the inheritance hierarchy starting from the superclasses. In class
\texttt{A1}, the inferred method type is \texttt{Int A1.m (Int)}. Class \texttt{B1} is a
subclass of \texttt{A1} which must override \texttt{m} as there is no
overloading in FGJ. However, the inferred method 
type is \texttt{<T> T B1.m(T)}, which is not a correct
method override for \texttt{A1.m()}.
Hence, GTI must instantiate the method type in the subclass \texttt{B1} to
\texttt{Int B1.m(Int)}.

Conversely, for the right listing of
Figure~\ref{fig:method-overriding}, GTI infers the types
\texttt{<T> T A2.m (T)} and \texttt{Int B2.m (Int)}. Again, these
types do not give rise to a correct method override and 
GTI is now forced to instantiate the type in the superclass to
\texttt{Int A2.m (Int)}.

\todo[inline]{This example shows that GTI must first collect the constraints 
  for all methods \texttt{m{}} in a class hierarchy. Generalization can only happen after
  all constraints on \texttt{m}'s type have been considered.}

% Otherwise, we would get strange results. Consider extending the
% program with classes \texttt{A} and \texttt{B} by the following class.
% \begin{lstlisting}[style=tfgj]
% class C {
%   mbool(x) { return x.m(new Bool()); }
% }
% \end{lstlisting}
% Inferring the type of \texttt{C.mbool()} using \texttt{A.m()} would
% fail because \texttt{Int} and \texttt{Bool} are not
% unifiable. However, inferring the type of \texttt{C.mbool()} using
% \texttt{B.m()} (with the generic type) would succeed and yield the
% typing
% \begin{lstlisting}[style=fgj]
% class C {
%   Bool mbool(B x) { return x.m(new Bool()); }
% }
% \end{lstlisting}


In full Java, type inference would have to offer two alternative
results: either two different
overloaded methods (one inherited and one local) in \texttt{B1}/\texttt{B2} or
impose the typing \texttt{Int B1.m(Int)} or \texttt{Int A2.m(Int)} to enforce correct overriding. 


\subsection{Inheritance and Generics}
\label{sec:inheritance-generics}


\begin{lstlisting}[float,caption={Function class}, label={lst:function-class},style=fgj]
class Function<S,T> {
  T apply(S arg) { return this.apply (arg); }
}
\end{lstlisting}
Suppose we are given a generic class for modeling functions in  FGJ (Listing~\ref{lst:function-class}).
This code is constructed to serve as an ``abstract'' super class to derive more
interesting subclasses.
The class \texttt{Function<S,T>} must be presented in this explicit
way. Its type annotations \textbf{cannot} be inferred by GTI because
the use of the generic class parameters in the method type cannot be inferred from the
implementation.

If we applied GTI to the type-erased version of
Listing~\ref{lst:function-class}, the \texttt{apply} method would be considered a generic method: 
\begin{center}
  \begin{minipage}{0.3\linewidth}
\begin{lstlisting}[style=tfgj]
apply (arg) { ... }
\end{lstlisting}
  \end{minipage}
  \hfill\texttt{ --GTI--> }\hfill
  \begin{minipage}{0.45\linewidth}
\begin{lstlisting}[style=fgj]
<A,B> B apply (A arg) { ... }
\end{lstlisting}
  \end{minipage}
\end{center}
The typing of \texttt{apply} in Listing~\ref{lst:function-class} is an
instance of this result, so that completeness of GTI is preserved!

% While it is possible to  force GTI to infer the intended typing, it requires some awkward coding.
% \begin{lstlisting}[style=tfgj]
% class Function<S,T> {
%   S in;
%   T out;
%   dummy (arg) { return this.dummy(in); }
%   apply (arg) { return new Pair<>(this.out, this.dummy (arg)).fst; }
% }
% \end{lstlisting}

Now that we have the abstract class \texttt{Function<S,T>} at our
disposal, let us apply GTI to a class of boxed values with a
\texttt{map} function:
\begin{lstlisting}[style=tfgj]
class Box<S> {
  S val;
  map(f) {
    return new Box<>(f.apply(this.val));
} }
\end{lstlisting}
GTI finds the following constraints
\begin{itemize}
\item the return value must be of type \texttt{Box<T>}, for some type
  \texttt{T},
\item \texttt{T} is a supertype of the type returned by
  \texttt{f.apply},
\item \texttt{apply} is defined in class \texttt{Function<S1,T1>} with
  type \texttt{T1 apply(S1 arg)}, 
\item hence \texttt{T1\,<:\,T} and \texttt{S\,<:\,S1} (because
  \texttt{this.val\,:\,S}),
\end{itemize}
and resolves them to the desired outcome where \texttt{T1=T} and
\texttt{S=S1} using the methods of
Simonet~\cite{DBLP:conf/aplas/Simonet03}. 
\begin{lstlisting}[style=fgj]
class Box<S> {
  S val;
  <T> Box<T> map(Function<S,T> f) {
    return new Box<T>(f.apply<S,T>(this.val));
} }
\end{lstlisting}
But what happens if we add subclasses of \texttt{Function}?
For example:
\begin{lstlisting}[style=tfgj]
class Not extends Function<Bool,Bool> {
  apply(b) { return b.not(); }
}
class Negate extends Function<Int,Int> {
  apply(x) { return x.negate(); }
}
\end{lstlisting}
\todo[inline]{Does this approach really make sense? If we have a
  subclass that overrides a method of the superclass, then the public
  interface should be the one of the superclass. Subsequently, the
  implementation of the method in the subclass should be checked
  against the type inferred for the superclass.}
If we rerun GTI with these classes, we now have additional
possibilities to invoke the \texttt{apply} method. With \texttt{Not}, we need to use the
generic type of \texttt{Function.apply()}, but instantiate it according to
\texttt{Function<Bool,Bool>}. Thus,
we obtain the constraints \texttt{Bool $\subconstr$ T} and \texttt{S $\subconstr$ Bool} for \texttt{T = Bool}
and \texttt{S = Bool}, which are both satisfiable. With
\texttt{Negate} we run into the same situation with the constraints
\texttt{Int $\subconstr$ Int} and \texttt{Int $\subconstr$ Int}.
% That means there is no way to
% \emph{directly} invoke \texttt{Not.apply} or \texttt{Negate.apply}
% from \texttt{Box.map}, but it may happen indirectly via inheritance.

Here is another subclass of \texttt{Function<S,T>} that we want
to consider.
\begin{lstlisting}[style=fgj]
class Identity<S> extends Function<S,S> {
  S apply(S arg) { return arg; }
}
\end{lstlisting}
Here, we obtain the following type constraints
\begin{itemize}
\item \texttt{apply} is defined in class \texttt{Identity<S1>} with
  type \texttt{S1 apply (S1 arg)},
\item hence \texttt{S1 $\subconstr$ T} and \texttt{S $\subconstr$ S1}.
\end{itemize}
Resolving the constraints yields \texttt{S = T} thus the typing
\begin{lstlisting}[style=fgj]
Box<S> map(Identity<S> f);
\end{lstlisting}
which is an instance of the previous typing.

\subsection{Multiple typings}
\label{sec:multiple-results}
% Our global type inference algorithm is able to infer every type annotation.
% For the sake of simplicity we will only consider method types in this
% paper and assume that field types are specified.
\begin{figure}[tp]
  \begin{minipage}{0.49\linewidth}
\begin{lstlisting}[style=fgj]
class List<A> {
  List<A> add(A item) {...}
  A get() { ... }
}
\end{lstlisting}
  \end{minipage}
  ~$\left|
  \begin{minipage}{0.49\linewidth}
\begin{lstlisting}[style=tfgj]
class Global{
  m(a){
    return a.add(this).get();
} }
\end{lstlisting}
  \end{minipage}\right.$
  \caption{Example for multiple inferred types}
  \label{fig:example-types-not-unique}
\end{figure}
Global type inference processes classes in order of
dependency.
To see why, consider the classes \texttt{List<A>} and \texttt{Global}
in Figure~\ref{fig:example-types-not-unique}. 
Class \texttt{Global} may depend on
class \texttt{List} because \texttt{Global} uses methods \texttt{add} and
\texttt{get} and \texttt{List} defines methods with the same names.
The dependency is only approximate because, in general, there may be additional classes
providing methods \texttt{add} and \texttt{get}.

In the example, it is safe to assume that the types for the methods of class \texttt{List}
are already available, either because they are given (as in the code
fragment) or because they were inferred before considering class \texttt{Global}.

The method \texttt{m} in class \texttt{Global} first invokes
\texttt{add} on \texttt{a}, so the type of \texttt{a} as well as the
return type of \texttt{a.add(this)} must be
\texttt{List<T>}, for some \texttt{T}. As \texttt{this} has
type \texttt{Global}, it must be that \texttt{Global} is a
subtype of \texttt{T}, which gives rise to the constraint
\texttt{Global $\subconstr$ T}. By the typing of \texttt{get()} we
find that the return type of method \texttt{m} is also \texttt{T}.

But now we are in a dilemma because FGJ only supports \emph{upper bounds} for
type variables,\footnote{Java has the same restriction. Lower bounds
  are only allowed for wildcards.} so that 
\texttt{Global $\subconstr$ T} is not a valid constraint in FGJ.
To stay compatible with this restriction, global type inference
expands the constraint by instantiating \texttt{T} with the (two) superclasses fulfilling
the constraint, \texttt{Global} and \texttt{Object}.  They give rise to two incomparable
types for 
\texttt{m}, \texttt{List<Global> -> Global} and
\texttt{List<Object> -> Object}. So there are two different FGJ
programs that are completions of the \texttt{Global} class.

GTI models these instances by inferring an \emph{intersection type}
\texttt{List<Global> -> Global \& List<Object> -> Object}
for method \texttt{m} and the different FGJ-completions of class \texttt{Global} are
instances of the intersection type:\footnote{The cognoscenti will be
  reminded of overloading. As FGJ does not support overloading,
  we rely on resolution by subsequent uses of the method. Moreover,
  this intersection type cannot be realized by overloading in a Java
  source program because it is resolved according to the raw classes
  of the arguments, in this case \lstinline{List}. It can be realized
  in bytecode which supports overloading on the return type, too.% 
}
\begin{center}
  \begin{minipage}{0.49\linewidth}
\begin{lstlisting}[style=fgj]
class Global {
  Global m(List<Global> a) {
    return a.add(this).get();
}
\end{lstlisting}
  \end{minipage}
  \begin{minipage}{0.49\linewidth}
\begin{lstlisting}[style=fgj]
class Global {
  Object m(List<Object> a) {
    return a.add(this).get();
}
\end{lstlisting}
  \end{minipage}
\end{center}
In this sense, the inferred intersection type represents a principal
typing for the class.
% \begin{lstlisting}[style=fgj]
% class Global {
%   <Global <: T> T m( List<T> a ) {
%     return a.add(this).get();
% }
% \end{lstlisting}
Additional classes in the program may further restrict the number of
viable types. Suppose we define a class \texttt{UseGlobal} as
follows:
\begin{lstlisting}[style=tfgj]
class UseGlobal {
  main() {
    return new Global().m((List<Object>) new List());
} }
\end{lstlisting}
Due to the dependency on \texttt{Global.m()}, type inference considers this class after class
\texttt{Global}. As it uses \texttt{m} at type
\texttt{List<Object> -> Object}, global type inference narrows the
type of \texttt{m} to just this alternative.

% \todo[inline]{If there are conflicting uses of a method with an
%   intersection type \texttt{T1\&T2}, one might consider splitting the
%   class into one providing the method with type \texttt{T1} and
%   another providing it with \texttt{T2}.}

\subsection{Polymorphic recursion}
\label{sec:polym-recurs}
\begin{figure}[tp]
  \begin{minipage}{0.49\linewidth}
\begin{lstlisting}[style=fgj]
class UsePair {
  <X,Y> Object prc(Pair<X,Y> p) {
    return this.prc<Y,X> (p.swap<X,Y>());
} }
\end{lstlisting}
  \end{minipage}
  ~$\left|
  \begin{minipage}{0.49\linewidth}
\begin{lstlisting}[style=tfgj]
class UsePair {
  prc(p) {
    return this.prc (p.swap());

} }
\end{lstlisting}
  \end{minipage}\right.$
  \caption{Example for polymorphic recursion}
  \label{fig:examples-poly-rec}
\end{figure}
A program uses \emph{polymorphic recursion} if there is a generic method that is invoked
recursively at a more specific type than its definition.
As a toy example for polymorphic recursion consider the FGJ class \texttt{UsePair} with a
generic method \texttt{prc} that invokes itself
recursively on a swapped version of its argument pair
(Figure~\ref{fig:examples-poly-rec}, left).
This method makes use of polymorphic recursion because the type of the
recursive call is different from the declared type of the method. More
precisely, the declared argument type is \texttt{Pair<X,Y>} whereas
the argument of the recursive call has type
\texttt{Pair<Y,X>}---an instance of the declared type.

For this particular example, global type inference succeeds on the
corresponding stripped program shown in
Figure~\ref{fig:examples-poly-rec}, right, but it yields a more restrictive
typing of \texttt{<X> Object prc (Pair<X,X> p)} for the method. 
A minor variation of the FGJ program with a non-variable instantiation makes type inference fail entirely:
\begin{lstlisting}[style=fgj]
class UsePair2 {
  <X,Y> Object prc(Pair<X,Y> p) {
    return this.prc<Y,Pair<X,Y>> (new Pair (p.snd, p));
  }
}
\end{lstlisting}





Polymorphic recursion is known to make type inference intractable
\cite{DBLP:journals/toplas/Henglein93,DBLP:journals/toplas/KfouryTU93}
because it can be reduced to an undecidable semi-unification problem
\cite{DBLP:journals/iandc/KfouryTU93}. However, \emph{type checking} with
polymorphic recursion is tractable and routinely used in languages like Haskell
and Java.

GTI does not infer method types with polymorphic recursion. Inference either fails or
returns a more restrictive type. Classes making use of polymorphic recursion need to supply
explicit typings for methods in question. 



% Features:
% \begin{itemize}
% \item overloaded methods
% \item class header is complete in the form
%   $\mathtt{class}\ C\langle\overline X <: \overline N\rangle <: N \{
%   \overline T\ \overline f;\ K\ \overline M \}$
% \item constructor fully typed (as it is determined by the types of the
%   fields and the supertype)
% \item method signatures may be omitted, i.e., 
%   $M\ ::=\ m(\overline x) \{ \ \mathtt{return}\ e;\ \}$
% \item polymorphic recursive methods must be annotated.
% \end{itemize}

%%% Local Variables:
%%% mode: latex
%%% TeX-master: "TIforGFJ"
%%% End:


\section{Featherweight Generic Java with Global Type Inference}
\label{sec:preliminaries}

This section defines the syntax and type system of a modified version
of the language Featherweight Generic Java
(FGJ)~\cite{DBLP:journals/toplas/IgarashiPW01}, which we call \TFGJ
(with Global Type Inference). The main omissions with respect to FGJ are method types specifications
and polymorphic recursion. We finish the section by formally
connecting FGJ and \TFGJ and by establishing some properties about
polymorphic recursion in FGJ.

\subsection{Syntax}\label{chapter:syntax}
% \commentarymargin{Input assumptions}
% No overloaded methods
% No Or-Constraints
\begin{figure}[tp]
\begin{align*}
  \mv T &::= \mv X \mid \mv N \\
  \mv N &::= \exptype{C}{\ol{T}}\\
  \mv L &::= \mathtt{class} \ \exptype{C}{\ol{X} \triangleleft \ol{N}} \triangleleft \ \mv N\ \{ \ol{T} \ \ol{f}; \,\mv K \, \ol{M} \} \\
  \mv K &::= \mv C(\ol{f})\ \{\mathtt{super}(\ol{f}); \ \mathtt{this}.\ol{f}=\ol{f};\} \\
  \mv M &::= \mathtt{m}(\ol{x})\ \{ \mathtt{ return}\ \mv e; \} \\
  \mv e &::= \mv x \mid \mv e.\mv f \mid
             \mv e.\mathtt{m}(\ol{e}) \mid \mathtt{new}\ \mathtt{C}(\ol{e})
             \mid (\mv N)\ \mv e
\end{align*}
  \caption{Syntax of \TFGJ}
  \label{fig:syntax-tfgj}
\end{figure}
Figure~\ref{fig:syntax-tfgj} defines the syntax of \TFGJ.
Compared to FGJ,
type annotations for method parameters and method return types are omitted.
Object creation via \texttt{new} as well as method calls come do not
require instantiation of their generic parameters.
We keep the class constraints ${\ol{X} \triangleleft \ol{N}}$ as well as the types
for fields $\ol{T} \ \ol{f}$ as we consider them as part of the
specification of a class.

% \textbf{Prechecks:}
We make the following assumptions for the input program:
\begin{itemize}
\item All types $\mv N$ and $\mv T$ are well formed according to the
  rules of FGJ, which carry over to \TFGJ (see Fig.~\ref{fig:well-formedness-and-subtyping}).
  % This can be checked easily, because no types annotations are omitted in the class header.
  % The generic variables of each class as well as their bounds are given in the input.
  % We assume that the each written type in the input in bounds of
  % classes, super class types, field types and casts is already
  % checked for well-formedness.
\item The methods of a class call each other mutually recursively.
\item The classes in the input are topologically sorted so that later
  classes only call methods in classes that come earlier in the
  sorting order.
  % to comply with the  rules
  % \rulename{GT-METHOD} and \rulename{GT-CLASS} (see
  % Section~\ref{chapter:type-rules}) of \TFGJ. 
\end{itemize}
Our requirements on the method calls do not impose serious
restrictions as any class, say \mv{C}, can be transformed to meet them as
follows. A preliminary dependency analysis determines an 
approximate call graph. We cluster the methods of \mv{C} according to
the $n$ strongly
connected components of the call graph. Then we split the class into a
class hierarchy 
$\mv{C}_1 \extends \dots \extends \mv{C}_n$ such that each class $\mv{C}_i$ contains
exactly the methods of one strongly connected component and assign a
method cluster to $\mv{C}_i$ if all calls to methods of  $\mv{C}$ now
target methods assigned to $\mv{C}_j$, for some $j\ge i$. The class
$\mv{C}_1$ replaces $\mv{C}$ everywhere in the program: in subtype
bounds, in \texttt{new} expressions, and in casts. More precisely, if
\mv{C} is defined by $\mathtt{class\ \exptype{C}{\ol X \extends \ol N}
\extends N \dots} $, then the class headers for the $\mv{C}_i$ are
defined as follows:
\begin{itemize}
\item  $\mathtt{class\ \exptype{C_i}{\ol X \extends \ol N}
\extends \exptype{C_{i+1}}{\ol X} \dots} $, for $1\le i < n$ and
\item  $\mathtt{class\ \exptype{C_n}{\ol X \extends \ol N}
\extends N \dots} $.
\end{itemize}
It follows from this discussion that the resulting classes have to be
processed backwards starting with $\mv{C}_n, \mv{C}_{n-1}, \dots, \mv{C}_1$.
Figure \ref{fig:example-decluster} showcases this process with a short example.

\begin{figure}[tp]
    \begin{subfigure}[t]{0.49\linewidth}
\begin{lstlisting}[style=tfgj]
class C extends Object {
  m1(a){
    return a;
  }
  m2(b){
    return this.id(a);
  }
}
\end{lstlisting}
      \caption{The methods \texttt{m1} and \texttt{m2} can be separated}
    \end{subfigure}
    ~
    \begin{subfigure}[t]{0.49\linewidth}
\begin{lstlisting}[style=tfgj]
class C1 extends C2 {
  m2(b){
    return this.id(a);
  }
}
class C2 extends Object {
  m1(a){
    return a;
  }
}
\end{lstlisting}
      \caption{After the transformation}
    \end{subfigure}
    \caption{Example for splitting a class into its strongly connected components}
    \label{fig:example-decluster}
  \end{figure}

\subsection{Typing}
\label{chapter:type-rules}
The input for our type inference algorithm is based on Generic Featherweight Java (GFJ).
GFJ is defined by syntax and typing rules.
We already changed the syntax to allow typeless GFJ programs as input for our algorithm.
Additionally we alter the typing rules slightly, which is presented in this chapter.

%Our type inference algorithm takes typeless GFJ classes as input.
%The generated output is correct GFJ, although we have to alter some rules.
%This chapter defines the typing rules for our version of GFJ,
%which our type inference algorithm is able to process.

Most of them stay the same as in the original GFJ language,
except from the following changes:
\begin{itemize}
\item We remove the \texttt{MT-CLASS}, \texttt{D-CAST}, \texttt{U-CAST}, \texttt{S-CAST} rules
\item The \texttt{GT-METHOD} rule is changed
\item Overriding of methods is removed for our typeless GFJ version. Therefore also the rule \texttt{MT-SUPER} is removed.
\item The \texttt{GT-INVK} rule is changed to support overloading
\end{itemize}

\fbox{
\begin{minipage}{\textwidth}
  \textbf{Subtyping:}\\
\begin{tabular}{l l}
%  $
%  \ddfrac{\texttt{class}\ \exptype{C}{\ol{X} \triangleleft \ol{N}} \triangleleft N \{ \ol{S}\ \ol{f};\ K \ \ol{M} \}
%  \quad \quad m \in \ol{M}}
%  {\mathit{mtype}(m, \exptype{C}{\ol{Z}}) = \mathit{mtype}(m, [\ol{T}/\ol{X}]N)}
%  $
%  & MT-SUPER \\
%& \\

$
\triangle \vdash T <: T
$
&   S-REFL \\

& \\
$\ddfrac{
    \triangle \vdash S <: T \quad \quad \triangle \vdash T <: U
}{
    \triangle \vdash S <: U
}$ & S-TRANS \\

& \\

$
\triangle \vdash X <: \triangle(X)
$ & S-VAR \\
& \\
$\ddfrac{
  \texttt{class}\ \exptype{C}{\ol{X} \triangleleft \ol{N}} \triangleleft N \set{ \ldots }
}{
  \triangle \vdash \exptype{C}{\ol{T}} <: [\ol{T}/\ol{X}]N
}$ & S-CLASS 
\end{tabular}
\end{minipage}
}

\fbox{
\begin{minipage}{\textwidth}
  \textbf{Well-formed types:}\\
\begin{tabular}{l l}
$\triangle \vdash \texttt{Object}\ \text{ok}
$ & WF-OBJECT\\

& \\
$\ddfrac{
    X \in \textit{dom}(\triangle)
}{
    \triangle \vdash X \ \text{ok}
}
$ & WF-VAR \\
& \\
$\ddfrac{\begin{array}{c}
\texttt{class}\ \exptype{C}{\ol{X} \triangleleft \ol{N}} \triangleleft N \{ \ldots \} \\
\triangle \vdash \ol{T} \ \text{ok} \quad \quad \triangle \vdash \ol{T} <: [\ol{T}/\ol{X}]\ol{N}
\end{array}
}{
\triangle \vdash \exptype{C}{\ol{T}} \ \text{ok}
}
$ & WF-CLASS
\end{tabular}
\end{minipage}
}


\fbox{
\begin{minipage}{\textwidth}
\textbf{Expression Typing:}\\
\begin{tabular}{l l}
$
\triangle ; \Gamma \vdash x : \Gamma(x)
$ & GT-VAR \\
& \\

$\ddfrac{\Gamma \vdash e_0:T_0 \quad \quad \mathit{fields}(\mathit{bound}_\triangle(T_0)) = \overline{T} \ \overline{f}}
{\Gamma \vdash e_0.\mathtt{f}_i : T_i}
$ & GT-FIELD \\
& \\
$ \ddfrac{\triangle \vdash N \ \texttt{ok} \quad \quad \textit{fields}(N) = \ol{T}\ \ol{f} \quad \quad
  \triangle; \Gamma \vdash \ol{e} : \ol{S} \quad \quad \triangle \vdash \ol{S} <: \ol{T}
}{
  \triangle; \Gamma \vdash \texttt{new N}(\ol{e}): N
}$ & GT-NEW \\

& \\

$\ddfrac{\begin{array}{c}
\mathit{mtype}(m, \mathit{bound}_\triangle (T_0)) = (\exptype{}{\ol{Y} \triangleleft \ol{P}} \ol{U} \to U)\\
\triangle; \Gamma \vdash e_0 : T_0 \quad \quad
\triangle \vdash \ol{V} \ \texttt{OK} \quad \quad
\triangle \vdash \ol{V} <: [\ol{V}/\ol{Y}]\ol{P} \\ %\quad \quad
\triangle; \Gamma \vdash \ol{e} : \ol{S} \quad \quad
\triangle \vdash \ol{S} <: [\ol{V}/\ol{Y}]\ol{U}
\end{array}}
{\triangle; \Gamma \vdash \mathtt{e_0.\exptype{m}{\ol{V}}(\overline{e}) : [\ol{V}/\ol{Y}]U }}
$ & GT-INVK
\end{tabular}
\end{minipage}
}



\fbox{
\begin{minipage}{\textwidth}
\begin{tabular}{l l}

  \textbf{Method Typing:} 
  & \\
  $\ddfrac{\begin{array}{c}
  \texttt{class}\ \exptype{C}{\ol{X} \triangleleft \ol{N}} \triangleleft N \{ \ldots\ \ol{M}\ \ldots\} \\
  \textit{mtype}(m, \exptype{C}{\ol{X}}) = \ol{T_m} \to T_m \textrm{ for } m \in \ol{M}\\
  \triangle \vdash \ol{X} <: \ol{N}  \quad \quad 
  \triangle \vdash \ol{T}, T \ \texttt{ok} \\
  \triangle ; \ol{x}:\ol{T_\mathit{meth}},\ this : \exptype{C}{\ol{X}} \vdash e_0 : S \quad \quad
  \triangle \vdash S <: T_\mathit{meth} \\
  \end{array}} {
  %{\exptype{}{\ol{Y} \triangleleft \ol{P}}\ T \ m(\ol{T}\ \ol{x}) \{
  %\texttt{return} \ e_0; \} \ \texttt{OK IN}\ \exptype{C}{\ol{X} \triangleleft
  %\ol{N}}}
   \exptype{}{\ol{Y}} T_\mathit{meth}\ \texttt{meth}(\ol{T_\mathit{meth}}\ \ol{\mathtt{x}}) \{\texttt{return}\ \mathtt{e}_0;\}
  \texttt{ OK in }\exptype{C}{\ol{X} \triangleleft \ol{N}} 
  }$ & GT-METHOD\\
  
  & \\

\textbf{Class Typing:} & \\
& \\
  %GT-CLASS: - This rule is modified by us
  $\ddfrac{
    \begin{array}{c}
      \ol{X} <: \ol{N} \vdash \ol{N}, N, \ol{T}\ \texttt{ok}
      \quad\quad fields(\mathtt{N}) = \ol{\mathtt{U}} \ \ol{\mathtt{g}}\\
      \exptype{}{\ol{Y}} T_\mathit{m}\ \texttt{m}(\ol{T_\mathit{m}}\ \ol{\mathtt{x}}) \{\texttt{return}\ \mathtt{e}_0;\}
  \texttt{ OK in }\exptype{C}{\ol{X} \triangleleft \ol{N}}  \textrm{ for all } m
      \in \ol{\mathtt{M}}\\
    K = C(\overline{D} \ \overline{g}, \overline{C} \ \overline{f}) \{ \texttt{super}(\overline{g}); \ \texttt{this}.\overline{f}=\overline{f}; \} \\
    %\quad \quad \overline{M} \ \texttt{OK IN C} \\
  \end{array}
    }
  {\texttt{class C extends D}\{ \overline{C} \ \overline{f}; \ K \ \overline{M} \} \ \texttt{OK}}
  $ & GT-CLASS
\end{tabular}
\end{minipage}
}
\commentarymargin{
\begin{tabular}{l l}

  \textbf{Method Typing:} & \\
  & \\
  $\ddfrac{\begin{array}{c}
  \texttt{class}\ \exptype{C}{\ol{X} \triangleleft \ol{N}} \triangleleft N \{ \ldots\ \ol{M}\ \ldots\} \\
  \texttt{meth}(\ol{x}) \{\texttt{return}\ e_0;\} \in \ol{M} \\
  \Gamma_C = \set{ (\textit{mtype}(m, \exptype{C}{\ol{X}}) = \ol{T_m} \to T_m) \ |\ m \in \ol{M}}\\
  (\textit{mtype}(\texttt{meth}, \exptype{C}{\ol{X}}) = \ol{T} \to T) \in \Gamma_C\\
  \triangle \vdash \ol{X} <: \ol{N}  \quad \quad 
  \triangle \vdash \ol{T}, T \ \texttt{ok} \\
  \triangle ; \Gamma_C,\ \ol{x}:\ol{T},\ this : \exptype{C}{\ol{X}} \vdash e_0 : S \quad \quad
  \triangle \vdash S <: T \\
  \end{array}} {
  %{\exptype{}{\ol{Y} \triangleleft \ol{P}}\ T \ m(\ol{T}\ \ol{x}) \{ \texttt{return} \ e_0; \} \ \texttt{OK IN}\ \exptype{C}{\ol{X} \triangleleft \ol{N}}}
  \Gamma_C \vdash \textit{mtype}(m, \exptype{C}{\ol{Z}}) = [\ol{Z} / \ol{X}](\exptype{}{\ol{Y}} \ol{T} \to T)
  }$ & GT-METHOD
\end{tabular}
}



\fbox{
\begin{minipage}{\textwidth}
\begin{tabular}{l l}

  \textbf{Method type lookup:} & \\
  & \\
  $\ddfrac{\begin{array}{c}
  \texttt{class}\ \exptype{C}{\ol{X} \triangleleft \ol{N}}\triangleleft
             \ol{N}\{ \overline{C} \ \overline{f}; \ K \ \overline{M} \} \ \mathtt{OK}\\
  \exptype{}{\ol{Y}} T_\mathit{m}\ \texttt{m}(\ol{T_\mathit{m}}\ \ol{\mathtt{x}}) \{\texttt{return}\ \mathtt{e}_0;\}
  \texttt{ OK in }\exptype{C}{\ol{X} \triangleleft \ol{N}}
  \end{array}} {
  %{\exptype{}{\ol{Y} \triangleleft \ol{P}}\ T \ m(\ol{T}\ \ol{x}) \{ \texttt{return} \ e_0; \} \ \texttt{OK IN}\ \exptype{C}{\ol{X} \triangleleft \ol{N}}}
  \textit{mtype}(m, \exptype{C}{\ol{Z}}) = [\ol{Z} / \ol{X}](\exptype{}{\ol{Y}} \ol{T} \to T)
  }$ & MT-CLASS
\end{tabular}
\end{minipage}
}


\commentarymargin{In the typing system of GFJ \textit{mtype} is a global function which gives the type for every method in every class.
It is defined by the \texttt{MT-CLASS} rule.
In our type system it is also a global function but it is defined by the \texttt{GT-METHOD} rule.
The difference to GFJ is that methods are typechecked one after another.
There has to be one method which initially does not call any other method in the input program,
except the ones in the same class.
%Also the resulting \textit{mtype} from the \texttt{GT-METHOD} rule is bound to the local method type context $\Gamma_C$.

The rule \texttt{GT-CLASS} only is valid, if every method in a class $C$ could satisfy the \texttt{GT-METHOD} rule with the same $\Gamma_C$.}


\medskip
In the typing system of GFJ \textit{mtype} is a global function which gives the type for every method in every class.
In our type system it is also a global function but it is differed between
methods declared in the actual class and methods from other classes.

The main difference between the type system of GFJ and our type system is that
in the \texttt{MT-CLASS} rule the correspondig class has to be proved as \texttt{OK}
by the \texttt{GT-CLASS} rule which means that for all methods of the class a type has to
be assumed and proved as correct be the \texttt{GT-METHOD} rule.
In this rule
for all methods in the actual class a type is assumed by the
declaration of the \texttt{mtype} function.
These assumptions have to be proved as correct. Then the assumed type is
\texttt{OK} in the correspondig class. 



%MT-CLASS: - This rule is not used by us
%\begin{align*}
%\ddfrac{\begin{array}{c}
%\texttt{class} \ \exptype{C}{\ol{X} \triangleleft \ol{N}} \ \{ \ol{S} \ \ol{f}; \ K \ol{M} \}\\
%\exptype{}{\ol{Y} \triangleleft \ol{P}} U \ m(\ol{U} \ \ol{x})\{  \texttt{return} \ e; \} \in \ol{M}
%\end{array}}
%{\mathit{mtype}(m, \exptype{C}{\ol{Z}}) = [\ol{T}/\ol{X}](\exptype{}{\ol{Y} \triangleleft \ol{P}} \ol{U} \to U)}
%\end{align*}

%GT-METHOD:
%\begin{align*}
%\ddfrac{\begin{array}{c}
%\triangle \vdash \ol{X} <: \ol{N}, \ol{Y} <: \ol{P} \quad \quad 
%\triangle \vdash \ol{T}, T, \ol{P} \ \texttt{ok} \\
%\triangle ; \ol{x}:\ol{T}, this : \exptype{C}{\ol{X}} \vdash e_0 : S \quad \quad
%\triangle \vdash S <: T \\
%\texttt{class}\ \exptype{C}{\ol{X} \triangleleft \ol{N}} \triangleleft N \{ \ldots \} \quad \quad
%\textit{override}(m, N, \exptype{}{\ol{Y} \triangleleft \ol{P}} \ol{T} \to T)
%\end{array}}
%{\exptype{}{\ol{Y} \triangleleft \ol{P}}\ T \ m(\ol{T}\ \ol{x}) \{ \texttt{return} \ e_0; \} \ \texttt{OK IN}\ \exptype{C}{\ol{X} \triangleleft \ol{N}}}
%\end{align*}


\subsection{Soundness of Typing}
\label{sec:soundness-typing}
\begin{figure}[tp]
    \begin{align*}
      \Erase{\mv x} &= \mv x \\
      \Erase{\mv e.\mv f} &= \Erase{\mv e}.\mv f \\
      \Erase{\exptype{e}{\ol T}.\mathtt{m}(\ol{e})} &= \Erase{\mv e}. \mathtt{m} (\Erase{\ol e}) \\
      \Erase{\mathtt{new}\ \exptype{C}{\ol T}(\ol{e})} & = \mathtt{new}\ \mv{C}(\Erase{\ol{e}}) \\
      \Erase{(\mv N)\ \mv e} & = (\mv N)\ \Erase{\mv e} \\
      \Erase{\exptype{}{\ol{X} \triangleleft \ol{N}}\ \mv{T}\ \mathtt{m}(\ol T\ \ol{x})\ \{ \mathtt{
      return}\ \mv e; \}} & = \mathtt{m}(\ol{x})\ \{ \mathtt{ return}\ \Erase{\mv e}; \} \\
      \Erase{\mv C(\ol{U}\ \ol{g}, \ol{T}\ \ol{f})\ \{\mathtt{super}(\ol{g}); \ \mathtt{this}.\ol{f}=\ol{f};\}} & = \mv C(\ol{g}, \ol{f})\ \{\mathtt{super}(\ol{g}); \ \mathtt{this}.\ol{f}=\ol{f};\} \\
      \Erase{\mathtt{class} \ \exptype{C}{\ol{X} \triangleleft \ol{N}} \triangleleft \ \mv N\ \{ \ol{T} \ \ol{f}; \,\mv K \, \ol{M} \}} & = 
                                                                                                                                          \mathtt{class} \ \exptype{C}{\ol{X} \triangleleft \ol{N}} \triangleleft \ \mv N\ \{ \ol{T} \ \ol{f}; \,\Erase{\mv K} \, \Erase{\ol{M}} \}
    \end{align*}
    \caption{Erasure functions}
    \label{fig:erasure}
  \end{figure}

We show that every typing derived by the \TFGJ rules gives rise to a
completion, that is, a well-typed FGJ program with the same structure.
\begin{definition}[Erasure]\label{def:erasure}
  Let $\mv{e}'$, $\mv{M}'$, $\mv{K}'$, $\mv{L}'$ be expression, method definition, constructor definition, class definition for FGJ. Define erasure functions 
  $\Erase{\mv{e}'}$, $\Erase{\mv{M}'}$, $\Erase{\mv{K}'}$,
  $\Erase{\mv{L}'}$ that map to the corresponding syntactic categories
  of \TFGJ as shown in Figure~\ref{fig:erasure}.
  \end{definition}
\begin{definition}[Completion]\label{def:completion}
  An FGJ expression $\mv{e}'$ is a \emph{completion} of a \TFGJ expression $\mv{e}$ if $\mv{e} = \Erase{\mv{e}'}$. Completions for method definitions, constructor definitions, and class definitions
  are defined analogously.
\end{definition}
\begin{theorem}
  Suppose that $\mathtt{\vdash \ol L : \Pi}$ such that $|\mathtt{\Pi (\exptype{C}{\ol{X} \triangleleft \ol{N}}.m)}| = 1$, for all $\mv{C.m}$ defined in $\ol L$. Then there is a completion $\ol{L}'$ of $\ol L$ such that
  $\ol{L}'\ \mathtt{OK}$ is derivable in FGJ.
\end{theorem}
\begin{proof}
  The proof is by induction on the length of $\ol L$.

  Consider the class typing $\mathtt{ \Pi \vdash \texttt{class}\ \exptype{C}{\ol{X} 
        \triangleleft \ol{N}} \triangleleft N\ \{ \ol{T} \
      \ol{f}; \ K \ \ol{M} \} \ \texttt{OK} : \Pi''}$ for an element of $\ol
    L$.

    We assume that all classes before $\mv{L}$ are completed according to the incoming $\mathtt{\Pi}$:
    If $\mathtt{\Pi (\exptype{D}{\ol{X} \triangleleft \ol{N}}.n)} = \mathtt{\exptype{}{\ol{Y} \triangleleft  \ol{P}}
      \ol{T} \to T}$, then $\mathtt{\exptype{}{\ol{Y} \triangleleft  \ol{P}}\ T\ \mv{n}(\ol{T}\
      \ol{x}) \dots}$ is in the completion of $\mv{D}$.

    Clearly, we can construct a completion for the class, if we can do so for each method. So we
    have to construct $\ol{M}'$ such that $\mathtt{\ol{M}'\ \mathtt{OK\ IN}\ \exptype{C}{\ol{X} 
        \triangleleft \ol{N}}}$. 

    Inversion of \rulename{GT-CLASS} yields
    \begin{gather}
      \label{eq:3}
      \mathtt{\Pi' = \Pi \cup \set{\exptype{C}{\ol{X} \triangleleft \ol{N}}.m \mapsto \exptype{}{} \ol{T_m} \to T_m \mid m \in \ol{M}} } \\
      \mathtt{\Pi'' = \Pi \cup \set{\exptype{C}{\ol{X} \triangleleft \ol{N}}.m \mapsto
          \exptype{}{\ol{Y} \triangleleft  \ol{P}} \ol{T_m} \to T_m \mid m \in \ol{M}} } \\
      \label{eq:5}
      \mathtt{\Pi', \Delta \vdash \ol{M} \ \texttt{OK IN}\
        \exptype{C}{\ol{X} \triangleleft \ol{N}} \extends N  \texttt{
          with } \exptype{}{\ol{Y} \triangleleft  \ol{P}}} \\
      \label{eq:7}
      \mathtt{\Delta = \ol{X} \subeq  \ol{N}, \ol{Y} \subeq  \ol{P}}
    \end{gather}
    Given some $\mv{M} = \texttt{m}(\ol{\mathtt{x}}) \{\texttt{return}\ \mathtt{e}_0;\} \in \ol{M}$,
    we show that
    \begin{gather}
      \label{eq:4}
      \exptype{}{\ol{Y} \triangleleft  \ol{P}}\ \mathtt{T_m}\ \texttt{m}(\ol{T_m}\ \ol{{x}})
      \{\texttt{return}\ \mathtt{e}'_0;\} \texttt{ OK IN }\exptype{C}{\ol{X} \triangleleft \ol{N}}
    \end{gather}
    is derivable for such completion $\mathtt{e_0'}$ of $\mathtt{e_0}$.

    By inversion of \eqref{eq:5} for $\mv{M}$, we obtain
    \begin{gather}
      \label{eq:6}
      \mathtt{\textit{override}(m, N, \exptype{}{\ol{Y} \triangleleft \ol{P}}\ol{T_m} \to T_m,
        \Pi) } \\
      \label{eq:9}
      \mathtt{\Pi; \Delta ; \ol{x}:\ol{T_m},\ this : \exptype{C}{\ol{X}} \vdash e_0 : S} \\
      \label{eq:8}
      \mathtt{\Delta \vdash S \subeq  T_m }
    \end{gather}
    As $\mathtt{\Delta}$ in~\eqref{eq:7} is defined as in \rulename{GT-METHOD'}, the well-formedness
    judgments are all given, the subtyping judgment \eqref{eq:8} is given as well as the override
    \eqref{eq:9}, the rule \rulename{GT-METHOD'} applies if we can establish
    \begin{gather}
      \label{eq:10}
      \mathtt{\Delta ; \ol{x}:\ol{T_m},\ this : \exptype{C}{\ol{X}} \vdash e_0' : S}
    \end{gather}
    for a completion of $\mathtt{e_0}$.

    To see that, we need to consider the rules \rulename{GT-NEW},
    \rulename{GT-CAST}, and \rulename{GT-INVK}. The 
    \rulename{GT-NEW} rule poses the existence of some $\ol{U}$ such that $\mv{N} =
    \exptype{C}{\ol{U}}$ for checking $\mv{e} = \mv{new}\ \mv{C} (\ol{e}) : \mv{N}$. In the completion, we
    define $\mv{e'} = \mv{new}\ \mv{N} (\ol{e}') : \mv{N}$ to apply rule \rulename{GT-NEW'} to the completions
    of the arguments.

    The rule \rulename{GT-CAST} splits into three rules
    \rulename{GT-UCAST'}, \rulename{GT-DCAST'}, and
    \rulename{GT-SCAST'}. These rules are disjoint, so that at most one of
    them applies to each occurrence of a cast. Here we assume a more
    liberal version of \rulename{GT-DCAST'} that admits downcasts that
    are not stable under type erasure semantics.

    For the rule \rulename{GT-INVK}, we first consider calls to methods not defined in the current
    class. By our assumption on previously checked classes $\mv{D}$ and their methods $\mv{n}$,
    $\mathit{mtype} (\mv{n}, \mv{D}, \mv\Pi) \allowbreak = \{\mathit{mtype}' (\mv{n}, \mv{D}')\}$ where the right
    side lookup happens in the completion following the definitions for FGJ (i.e., $\mv{D'}$ is the
    completion for $\mv D$). The \rulename{GT-INVK} rule poses the existence of some $\ol{V}$ that
    satisfies the same conditions as in \rulename{GT-INVK'}. Hence, we define the completion of
    $\mathtt{e_0.\mv{n}(\ol{e}) : [\ol{V}/\ol{Y}]U }$ as
    $\mathtt{e_0'.\exptype{n}{\ol{V}}(\ol{e}') : [\ol{V}/\ol{Y}]U }$.

    Next we consider calls to methods $\mv{n}$ defined in the current class, say, $\mv{C}$. For those methods,
    $\mathit{mtype} (\mv{n}, \mv{C}, \mv\Pi) = \exptype{}{} \ol{U} \to \mv U$, a non-generic
    type. By the definition of $\mathtt{\Pi''}$, we know that the type of this method will be
    published in the completion as $\exptype{}{\ol{Y} \triangleleft  \ol{P}} \ol{U} \to \mv
    U$. Hence, $\mathit{mtype}' (\mv{n}, \mv{C}') = \exptype{}{\ol{Y} \triangleleft  \ol{P}} \ol{U}
    \to \mv U$. As methods in $\mv{C}$ are mutually recursive, the rule must pose that $\ol{V} = \ol{Y}$ (cf.\
    Proposition~\ref{prop:polymorphi-recursion}). This setting fulfills all assumptions:
    \begin{gather}
      \label{eq:11}
      \mathtt{\Delta \vdash \ol{Y} \ \texttt{ok} } \\
      \mathtt{\Delta \vdash \ol{Y} \subeq  [\ol{Y}/\ol{Y}]\ol{P} }       
    \end{gather}
    We set the completion of
    $\mathtt{e_0.\mv{n}(\ol{e}) : [\ol{Y}/\ol{Y}]U }$ to
    $\mathtt{e_0'.\exptype{n}{\ol{Y}}(\ol{e}') : [\ol{Y}/\ol{Y}]U }$, which is derivable in FGJ.

    The remaining expression typing rules are shared between FGJ and \TFGJ, so they do not
    affect completions.
\end{proof}

\subsection{Polymorphic Recursion, Formally}
\label{sec:polym-recurs-form}

Consider an FGJ class $\mv{C}$ with $n$ mutually recursive methods $\mv{m}_i :
\forall\ol{X}_i. \ol{A}_i \to \ol{A}_i$, for $1\le i\le n$. Define the \emph{instantiation
  multigraph $IG(\mv{C})$} as a directed multigraph with vertices $\{1,
\dots, n\}$.
Edges between $i$ and $j$ in this graph are labeled with a
substitution from $\ol{X}_j$ to types in $\mv{m}_i$, which may contain
type variables from $\ol{X}_i$.
In particular, if $\mv{m}_i$ invokes $\mv{m}_j$ where the generic type variables in
the type of $\mv{m}_j$ are instantiated with substitution
$\ol{U}/\ol{X}_j$ (see rule GT-INVK), then 
$
i \stackrel{\ol{U}/\ol{X}_j}{\longrightarrow} j
$
is an edge of $IG (\mv{C})$.

Define the \emph{closure of the instantiation multigraph $IG^*(\mv{C})$} as the multigraph
obtained from $IG(\mv{C})$ by applying the following rule, which
composes the instantiating substitutions, exhaustively:
\begin{gather}\label{eq:1}
  i \stackrel{\ol{U}/\ol{X}_j}{\longrightarrow} j
  \quad\wedge\quad
  j \stackrel{\ol{V}/\ol{X}_k}{\longrightarrow} k
  \qquad\Rightarrow\qquad
  i \stackrel{[\ol{U}/\ol{X}_j]\ol{V}/\ol{X}_k}{\longrightarrow} k
\end{gather}

\begin{definition}\label{def:method-in-poly-rec}
  Method $\mv{m}_i$ is \emph{involved in polymorphic recursion}
  if there is an edge
  \begin{gather}\label{eq:2}
    i \stackrel{\ol{W}/\ol{X}_i}{\longrightarrow} i \quad \in IG^*
    (\mv{C}) \qquad \text{such that} \qquad \ol{W} \ne \ol{X}_i
  \end{gather}
\end{definition}
For the toy example in Figure~\ref{fig:examples-poly-rec}, we obtain
the multigraph $IG^* (\mv{UsePair})$ which indicates that \mv{prc} is
involved in polymorphic recursion:
\begin{gather*}
  \begin{array}{l@{\quad}|@{\quad}l}
    IG(\mv{UsePair})& IG^* (\mv{UsePair}) \\\hline
    \mv{prc} \stackrel{\mv{Y,X}/\mv{X,Y}}{\longrightarrow} \mv{prc} &
    \mv{prc} \stackrel{\mv{Y,X}/\mv{X,Y}}{\longrightarrow} \mv{prc} \qquad
    \mv{prc} \stackrel{\mv{X,Y}/\mv{X,Y}}{\longrightarrow} \mv{prc}
  \end{array}
\end{gather*}
The call to \mv{swap} does not appear in the graph because
\mv{swap} is defined in a different class.

For \mv{UsePair2}, we obtain a multigraph $IG^* (\mv{UsePair2})$ with
infinitely many edges which is also clear indication for polymorphic recursion:
\begin{gather*}
  \begin{array}{l@{\quad}|@{\quad}l}
    IG(\mv{UsePair2})& IG^* (\mv{UsePair2}) \\\hline
    \mv{prc} \stackrel{\mv{Y,Pair<X,Y>}/\mv{XY}}{\longrightarrow} \mv{prc}
    &
    \mv{prc} \stackrel{\mv{Y,Pair<X,Y>}/\mv{XY}}{\longrightarrow}
      \mv{prc} \\
    &
      \mv{prc}
      \stackrel{\mv{Pair<X,Y>,Pair<Y,Pair<X,Y>>}/\mv{XY}}{\longrightarrow}
      \mv{prc}
      \\
                     & \dots
  \end{array}
\end{gather*}


Clearly, $IG (\mv{C})$ is finite and can be constructed effectively by
collecting the instantiating substitutions from all method call
sites.
Repeated application of the propagation rule~\eqref{eq:1} either
results in saturation where no edge of the resulting multigraph satisfies~\eqref{eq:2} or it
detects an instantiating edge as in condition~\eqref{eq:2}. 

The following condition is necessary for the absence of
polymorphic recursion.

\begin{proposition}\label{prop:polymorphi-recursion}
  Suppose an FGJ class \mv{C} has $n$ methods, which are mutually
  recursive.
  If \mv{C} does not exhibit
  polymorphic recursion, then
  \begin{itemize}
  \item all methods quantify over the same number of generic
    variables;
  \item if a method has generic variables $\ol{X}$, then each call to
    a method of \mv{C} instantiates with a permutation of the
    $\ol{X}$;
  \item $IG^* (\mv{C})$ is finite.
  \end{itemize}
\end{proposition}
\begin{proof}
  Suppose for a contradiction that there are two distinct methods $\mv{m}_i$ and
  $\mv{m}_j$ with generic variables $\ol{X}_i$ and $\ol{X}_j$, respectively,
  where $|\ol{X}_i| < |\ol{X}_j|$. By mutual recursion, $\mv{m}_i$ invokes
  $\mv{m}_j$ directly or indirectly and vice versa. Hence, $IG^* (\mv{C})$
  contains edges from $i$ to $j$ and back:
  \begin{gather*}
    i \stackrel{\ol{U}/\ol{X}_j}{\longrightarrow} j
    \qquad
    j \stackrel{\ol{V}/\ol{X}_i}{\longrightarrow} i
  \end{gather*}
  As $IG^*(\mv{C})$ is closed under composition, it must also contain
  the edge
  \begin{gather*}
    j \stackrel{[\ol{V}/\ol{X}_i]\ol{U}/\ol{X}_j}{\longrightarrow} j
    \text{.}
  \end{gather*}
  By assumption $\mv{C}$ does not use polymorphic recursion, so it
  must be that $[\ol{V}/\ol{X}_i]\ol{U}/\ol{X}_j =
  \ol{X}_j/\ol{X}_j$. To fulfill this condition, all components of
  $\ol{U}$ must be variables $\in \ol{X}_i$. As  $|\ol{X}_i| <
  |\ol{X}_j| = |\ol{U}|$, there must be some variable $\mv{X} \in \ol{X}_i$ that
  occurs more than once in $\ol{U}$, say, at positions $j_1$ and $j_2$. 
  But that means the variables at positions $j_1$ and $j_2$ in
  $\ol{X}_j$ are mapped to the same component of $\ol{V}$. This is a
  contradiction because this substitution cannot be the identity
  substitution $\ol{X}_j/\ol{X}_j$.

  Hence, all methods have the same number of generic variables and all
  instantiations must use variables.

  Suppose now that there is a direct call from $\mv{m}_i$ to
  $\mv{m}_j$ where the instantiation $\ol{U}/\ol{X}_j$ is not a permutation. Hence,
  there is a variable that appears more than once in $\ol{U}$, which
  leads to a contradiction using similar reasoning as before.

  Hence, all instantiations must be permutations over a finite set of
  variables, so that $IG^*(\mv{C})$ is finite. 
\end{proof}

Moreover, if a class has only mutually recursive methods without
polymorphic recursion, we can assume that each method uses the same
generic variables, say $\ol{X}$, and each instantiation for
class-internal method calls is the identity $\ol X/\ol X$.

Using the same generic variables is achieved by $\alpha$ conversion.
By Proposition~\ref{prop:polymorphi-recursion}, we already know that
each instantiation is a permutation. Each self-recursive call must use
an identity instantiation already, otherwise it would constitute an
instance of polymorphic recursion. Suppose that method \mv{m} calls
method \mv{n} instantiated with a non-identity permutation, say
$\pi$ so that parameter $\mv{X_i}$ of \mv{n} gets instantiated with
$\mv{X_{\pi(i)}}$ of \mv{m}. In this case, we reorder the generic
parameters of \mv{n}  according to the inverse permutation
$\pi^{-1}$ and propagate this permutation to all call sites of
\mv{n}. For the call in \mv{m}, we obtain the identity permutation
$\pi \cdot \pi^{-1}$, for self-recursive calls inside \mv{n}, the
instantiation remains the identity (for the same reason), for a
call site in another method which instantiates \mv{n} with permutation
$\sigma$, we change that permutation to $\sigma \cdot \pi^{-1}$, which
is again a permutation.
This way, we can eliminate all non-identity instantiations from calls
inside \mv{m}.

We move our attention to \mv{n}. Each self-recursive call and each call to \mv{m} uses the
identity instantiation, the latter by construction. So we only need to
consider calls to $\mv{n'}\notin\{\mv{n}, \mv{m}\}$ with an
instantiation which is not the identity permutation. We can also
assume that $\mv{n'}$ is not called from $\mv{m}$: otherwise, \mv{n'}
would have the generic variables in the same order as \mv{m} and hence
as \mv{n}. But that means we can fix all calls to $\mv{n'}$ by
applying the inverse permutations as for \mv{n} \emph{without disturbing the already
  established identity instantiations}.

Each such step eliminates all non-identity instantiations for at least
one method without disturbing previous identity instantiations. Hence,
the procedure terminates after finitely many steps with a class with
all instantiations being identity permutations.

\if0 

\subsection{Principal Type}
\todo[inline]{PJT: is this subsumed by
  Section~\ref{sec:multiple-results}? Do we have principal types when
  we admit intersection types?}

\todo[inline]{MP: I think so. But we have to define an partial ording
  $\triangleright$ of princpality:

% \newpage
%   \begin{definition}[Partial Ordering of Princpality]
%      For a method $$\mathtt{m}(\ol{x})\ \{ \mathtt{ return}\ \mv e; \}$$ 
%      and a type $\mathtt{T}$ with $\environmentvdash x :  \mathtt{T}$.

%      $\mathtt{\vdash \ol L : \Pi_n}$

%      $\exptype{C}{\ol{X}}.{\tt m} \mapsto
%       \exptype{}{\ol{Y} \triangleleft  \ol{P}} \ol{T_m} \to {\tt T_m}$
%   \end{definition}

  $ty_1 \& \ldots \& ty_n \triangleright ty'_i$, if $ty'_i$ is a supertype of any
  $ty_i$ and\\ $ty'_i \triangleright ty'_i[\ol{U}/\ol{X}]$

  For this we have to define a rule for the derivation of an intersection type.
  Then there is minimal number of funtions types for any method.
}
\todo[inline]{PT: so should we add overloading to FGJ? Which style of
  overloading? (i.e., should it be compatible with Java's restrictions
  that overloading must be resolvable on raw types?)}


Featherweight Generic Java has no unique principal typing.
We can show this easily with an example.
We try to find the principal type for the method \texttt{method1}.
\begin{lstlisting}
class Global{
  method1(a){
    a.add(this);
    return a.get();
  }
}
class List<A> {
  add(A item){...}
  A get() ...
}
\end{lstlisting}
In \texttt{method1} neither the return type nor the type for the parameter \texttt{a} are specified.
The return type of the method depends on the type of \texttt{a}.
If we set in the type \texttt{List<Object>} here, then \texttt{method1} would return \texttt{Object}.
The type \texttt{List<Global>} would also be correct.
Then the return type of the method can also be the type \texttt{Global}.

The principal type would either be an intersection type or the method \texttt{method1} has to be overloaded.
FGJ neither supports intersection types nor overloading.
Therefore we cannot set in the principal type and have to stick with one of the possible solutions,
for example\\
\texttt{List<Global> method1(List<Global> a)}.

It is possible for a class to have multiple principal type solutions.
This can lead to a type error when compiling multiple classes.
\begin{lstlisting}
  class Global{
    method1(a){
      a.add(this);
      return a.get();
    }
  }
  class Class2{
    Object test(){
      return new Global().method1(new List<Object>());
    }
  }
\end{lstlisting}
Our type inference algorithm is able to infer all of the principal type solutions, but only one of them can be set in.
If we set in \texttt{List<Global>} as the parameter type for the \texttt{method1},
then the class \texttt{Class2} would lead to a type error.
In this case the type inference algorithm has to try another type solution for \texttt{method1}
to render the program type correct.

\fi

%%% Local Variables:
%%% mode: latex
%%% TeX-master: "TIforGFJ"
%%% End:


\section{Type inference algorithm}
\label{sec:type-infer-algor}
In this chapter we present our type inference algorithm.
The algorithm is split into following parts, which are executed on a single class at a time:

\begin{enumerate}
\item Create assumptions and subtype relation
\item Constraint generation with \textbf{FJTYPE}
\item Unification of those constraints
\item Set in principal type solution
\end{enumerate}

The Unify algorithm returns a set of possible type solutions.
This means that there are possibly multiple type solutions for each method.
The last step has to choose the principal type out of those possibilities.

\subsection{Process multiple classes}
The algorithm processes only one class at a time.
Only the first step creating the type assumptions is able to consider other classes as well.

Nevertheless we allow the input to consist out of multiple classes.
But in that case there are some additional requirements for the input.
%TODO: these requirements can also be in "Preliminaries"

We assume that the algorithm are given the input classes in the correct order $C_1, \ldots C_2$.
Hereby there must exist a correct typisation for the class $C_1$ when existing on its own.
This is also regulated by our typing rules (see chapter \ref{chapter:type-rules}).

\textbf{Example:}
We give an example for a incorrect input program for our algorithm, where none of the given classes cannot be compiled on its own.
  \begin{figure}
    \centering
    \begin{minipage}{.5\textwidth}
      \centering
      \begin{lstlisting}
class C1 extends Object {
  C1(){ super(); }
  m1(){ return new C2().m2(); }
}
class C2 extends Object{
  C2(){ super(); }
  m2(){ return new C2().m1(); }
}
          \end{lstlisting}
            \caption{Invalid typeless GFJ program}
      \label{fig:invalidinput}
    \end{minipage}%
    \begin{minipage}{.5\textwidth}
      \centering

    \begin{lstlisting}
class C1 extends Object {
  C1(){ super(); }
}
class C2 extends Object{
  C2(){ super(); }
  m1(){ return new C2().m2(); }
  m2(){ return new C2().m1(); }
}
            \end{lstlisting}
          \caption{Correct typeless GFJ program}
      \label{fig:correctinput}
    \end{minipage}
\end{figure}

The problem in figure \ref{fig:invalidinput} is a circular method call graph.
Method \texttt{m1} calls method \texttt{m2} and other way round.
Our typing rules demand that one of the two classes has to only use his internal methods.
Figure \ref{fig:correctinput} shows a possible way to alter the incorrect input to make it comply with our typing rules.
The method \texttt{m1} was moved into the class \texttt{C2}.
Now both methods still call each other, but they are inherited by the same class.

Another problem we face when compiling multiple classes is the fact that there can be more than a single principal typing for a method.
When considering only one class at once it is not possible to set in the correc type right away.
This problem can be solved with backtracking.
Whenever the \textbf{Unify} algorithm gives more than one type solution, we pick the first one and continue.
If the algorithm fails at some point it has to backtrack to this point and try one of the other solutions.

\subsection{Generate Assumptions}
% Every empty Type T in the input is assigned a type variable.
% Assumptions saves every field, method and the class subtype relation

%Generate subtype relationships:

Generating assumptions consists of two parts.
At first we add type variables to the untyped class.
The second part generates the assumption set.
This is the same algorithm for the already typed classes as for the 
new untyped class, which is now equipped with type variables.

\begin{enumerate}
\item Every missing type in the input class gets assigned a fresh type variable.
For methods:
\begin{align*}
  \ddfrac{
  m(\ol{x}) \{ \ldots \} \quad \quad A \cup \ol{A} \ \text{are fresh type variables}
  }{
  A m(\ol{A}\ \ol{x}) \{ \ldots \}
  }
  \end{align*}
  For fields:
\begin{align*}
  \ddfrac{
  \texttt{class}\ \exptype{C}{\ol{X}} \{ \ol{f}; \quad \ldots \} \quad \quad \ol{F} \ \text{are fresh type variables}
  }{
    \texttt{class}\ \exptype{C}{\ol{X}} \{ \ol{F} \ \ol{f}; \quad \ldots \}
  }
\end{align*}
\item We define the two functions $\textit{ftype}_\textit{Ass}$ and $\textit{mtype}_\textit{Ass}$.
Both functions return a set of all types for a method \texttt{m} or a field \texttt{f}.
This is due to the fact that there can be multiple methods and fields with the same name.
\begin{align*}
  %TODO: fresh type variables for generic variables:
  \ddfrac{
    class\ \exptype{C}{\ol{X} \triangleleft \ol{N}}\ \{\ \ol{N}\ \ol{f};\ K\ \ol{M}\ \} \quad \quad
    \exptype{}{\ol{Y}}\ U\ \texttt{m}(\ol{U}\ \ol{x}) \{ \ldots \} \in \ol{M}
  }{
    \textit{mtype}_\textit{Ass}(\texttt{m}, \exptype{C}{\ol{X} \triangleleft \ol{N}}) =  \set{\exptype{}{\ol{Y}} (\ol{U} \to U )}
  }
\end{align*}
\begin{align*}
  \ddfrac{
    class\ \exptype{C}{\ol{X} \triangleleft \ol{N}}\ \{\ \ol{T}\ \ol{f};\ K\ \ol{M}\ \} \quad \quad
    T\ \texttt{f} \in \ol{f}
  }{
    \textit{ftype}_\textit{Ass}(\texttt{f}, \exptype{C}{\ol{X} \triangleleft \ol{N}}) = T
  }
\end{align*}
\item If the input for the type inference algorithm consists out of multiple classes we compile them one by one.
Additionally we add the types of the already compiled classes to the assumption set.
Therefore it is possible to have intersection types already in the assumptions.
\end{enumerate}

\subsection{Multiple classes as input}
We process each class individually.

%Besprechung Peter:
%Ich betrachte immer Klassen zusammen
%Wo ich annehme, dass die sich rekursiv aufrufen
%
%Es gibt einen Präprozessor, der die aufrufgraphen ausrechnet
%und die Klassen in Pakete packt

\subsection{FJTYPE}
The \textbf{FJTYPE} algorithm produces two kinds of constraints.
\begin{description}
\item[Constraint] A constraint consists of two types or type variables and an operator.
The operator can either be a $\doteq$ (same type) or $\lessdot$ (subtype).
Example: $(a \lessdot \mathtt{Object})$, means that the type variable $a$ should be a subtype of \texttt{Object}.
\item[OrConstraint] An OrConstraint consists out of multiple constraint sets.
For example $\textbf{OrConstraint}(\{ \ \{ (a \lessdot b), (a \leq \mathtt{Object}) \} \ , \ \{ (a \lessdot b)\} \ \})$
is an Or-Constraint consisting of two constraint sets.
\end{description}

Before the algorithm starts we equip every untyped method with type variables.
Every method parameter gets a unique type variable as a type aswell as every method gets a unique type variable as a return type.
After our algorithm found a correct typisation we replace the type variables with the inferred types and generate a GFJ program.

The algorithm \textbf{FJTYPE} is given as follows:

\textbf{FJTYPE}:
$
\texttt{Class} \rightarrow \texttt{Constraints}\\
 \begin{array}{@{}l@{}l@{}l}
 \textbf{FJT}&\textbf{Y} & \textbf{PE}(\mathtt{class } \ \exptype{C}{\ol{X} \triangleleft \ol{N}} \ \mathtt{ extends } \ D \{ \overline{T} \ \overline{f}; \, K \, \overline{M} \}) =\\
& \multicolumn{2}{@{}l@{}}{ \{ \ \textbf{TYPEMethod}(\{ \mathtt{this} : \exptype{C}{\ol{X}} \}, m_i) \quad | \quad m_i \in \overline{M} \ \} }\\ 
\end{array}$

The \textbf{FJTYPE} function gets called for every class in the input.
This function accumulates all the constraints generated from calling the
\textbf{TYPEMethod} function for each method declared in the given class.

$\textbf{TYPEMethod}:\texttt{TypeAssumptions} \times
\texttt{Method} \rightarrow \texttt{Constraints}\\
\begin{array}{@{}l@{}l@{}l}
\textbf{TY}& \textbf{PE} & \textbf{Method} (Ass, T_r \ \mathtt{m}(\overline{T} \, \overline{x})\{ \mathtt{ return }\ e; \}) =\\
& \textbf{let}
& Ass_m = Ass \cup \{ \overline{T} : \overline{x} \}\\
& & \ul{(e:rty, ConS)} = \textbf{TYPEExpr}(Ass_m, e)\\
& \mathbf{in}
& (ConS \cup (rty \lessdot T_r))\\
\end{array}
$

The \textbf{TYPEMethod} function for methods just calls the \textbf{TYPEExpr} function with the
return expression. It is significant to note that it adds the assumptions for the method parameters to the global assumptions before passing them to \textbf{TYPEExpr}.
%and the global assumptions plus the assumptions for the method parameters.

\smallskip

In the following we define the \textbf{TYPEExpr} function for every possible expression:

\smallskip

$\textbf{TYPEExpr}:\texttt{TypeAssumptions} \times
\texttt{Expression} \rightarrow \texttt{Type} \times \texttt{Constraints}\\
\begin{array}{@{}l@{}l}
\textbf{TY} \textbf{PE} & \textbf{Expr} (Ass, \mathtt{this}) = (t , \{\})\\
& \textbf{with } (\mathtt{this} : t) \in Ass 
\end{array}
$
\smallskip
$\begin{array}{@{}l@{}l}
\textbf{TY} \textbf{PE} & \textbf{Expr} (Ass,x) = (t , \{\})\\
& \textbf{with } (x : t) \in Ass 
\end{array}
$

\smallskip

For our \textbf{TYPE} algorithm a field variable can be seen as a method call without parameters.
Therefore generating constraints for a field access is the same as for a method call,
with the difference that we search for fields with the same name instead of methods.\\
$\begin{array}{@{}l@{}l@{}l}
\textbf{TY}& \textbf{PE} & \textbf{Expr} (Ass, e.f) = \\
& \textbf{let} % \\
% &
& (rty, ConS) = \textbf{TYPEExpr}(Ass, e),\\
& & \begin{array}{@{}l@{}l}
        Cons_{m} = \{\ & rty \doteq \exptype{C}{\ol{X}}, a \doteq T\\
                    & |\, \text{for every field}\ T\ \texttt{f} \in \ol{f} \} \\%, \textbf{fresh}(\ol{X}) \lessdot \textbf{fresh}(\ol{N})\},\\
                    & ,\, \text{in every class}\ \exptype{C}{\ol{X}} \{ \ldots \ol{f} \ldots \}\ \text{in the input} \} 
                  \end{array}\\
& & OrCons = \textbf{OrConstraint}(Cons_{m})\\
& \mathbf{in}% \\
% &
& (a, [\textbf{fresh}(\ol{X})/\ol{X}][\textbf{fresh}(\ol{Y})/\ol{Y}]ConS \cup Cons_{f})\\
& & \mathit{where\ } a \mathit{\ is\ a\ fresh\
  type\ variable}\\ 
\end{array}
$

\smallskip

$\begin{array}{@{}l@{}l@{}l}
\textbf{TY}& \textbf{PE} & \textbf{Expr} (Ass, e.\mathtt{m}(\overline{e}) ) = \\
& \textbf{let} % \\
% &
& (rty, ConS) = \textbf{TYPEExpr}(Ass, e),\\
& & \forall e_i \in \overline{e} : (pt_i, ConS_i) = \textbf{TYPEExpr}(e_i)  ,\\
& & \begin{array}{@{}l@{}l}
        Cons_{m} = \{\ & rty \doteq \exptype{C}{\ol{X}}, a \doteq T, \bigcup_{T_i \in \overline{T}} (pt_i \lessdot T_i)\\
                    & |\, \text{for every method}\ \exptype{}{\ol{Y}} T\ \texttt{m}(\ol{T}\  \ol{p}) \in \ol{M} \} \\%, \textbf{fresh}(\ol{X}) \lessdot \textbf{fresh}(\ol{N})\},\\
                    & ,\, \text{in every class}\ \exptype{C}{\ol{X}} \{ \ldots \ol{M} \ldots \}\ \text{in the input} \} 
                  \end{array}\\
& & OrCons = \textbf{OrConstraint}(Cons_{m})\\
& \mathbf{in}% \\
% &
& (a, [\textbf{fresh}(\ol{X})/\ol{X}][\textbf{fresh}(\ol{Y})/\ol{Y}](ConS \cup \bigcup_i ConS_i \cup OrCons))\\
& & \text{where\ } a \text{\ is\ a\ fresh\
  type\ variable}\\ 
\end{array}
$

\smallskip

The \texttt{new}-statement comes without the generic variables
(\texttt{new Classname(...)} instead of \texttt{new Classname<Class>(...)}).
The correct type will be inferred by our type inference algorithm.
We generate new type variables $\ol{X'}$ for the generic variables of the class \texttt{C},
so the \textbf{Unify} algorithm can later set in the correct types for these variables.
He has to comply to the bounds given by $\ol{N}$, which is why we add $\ol{X'} \lessdot \ol{N}$ to the constraints.
It is important to change every occurence of $\ol{X}$ with the fresh type variables $\ol{X'}$ in the generated constraints.
$\ol{X}$ can occur in the bound $\ol{N}$ aswell as in the types of the constructor parameters $\ol{T}$.

$\begin{array}{@{}l@{}l@{}l}
\textbf{TY}& \textbf{PE} & \textbf{Expr} (Ass, \mathtt{new }\ C(\overline{e}) ) = \\
& \textbf{let} % \\
& \forall e_i \in \overline{e} : (pt_i, Cons_i) = \textbf{TYPEExpr}(Ass, e_i)  ,\\
& & Cons = \{ \bigcup_{T_i \in \overline{T}} (pt_i \lessdot T_i) \ | \ \mathtt{class }\ \exptype{C}{\ol{X} \triangleleft \ol{N}}\{ \ldots, \texttt{C}(\overline{T} \overline{x}), \ldots \} \}\\
& \mathbf{in}% \\
% &
& (\textbf{fresh}(\ol{X})/\ol{X}]\exptype{C}{\ol{X}}, [\textbf{fresh}(\ol{X})/\ol{X}](Cons \cup \bigcup_i Cons_i \cup \ol{X'} \lessdot \ol{N}))\\
\end{array}
$

We do not generate constraints for casts.

\subsubsection{Completeness of the type inference algorithm}
%Theorem: The Unify algorithm is complete
%Theorem: \textbf{FJTYPE} generates the principal type
\textbf{Proof:} The \textbf{Unify} algorithm is complete, so every correct type is included in the solution set.
We only have to choose the right type out of those solutions.
When compiling multiple classes the problem arises,
that only one of the type solutions calculated by \textbf{Unify} is correct
in respective to the other classes that will be compiled afterwards.

All types that are possible under the FGJ typing rules, plus our additional assumptions,
also comply with the generated constraints.

We match every generated constraint with the respective type rule to show completeness of our \textbf{FJTYPE} algorithm.
This shows that none of the generated constraints remove a type which otherwise would be possible under the FGJ typing rules.
The constraints are generated on expression statements.
We now compare the constraints for each expression with the appropriate type rule from FGJ:
\begin{description}
  \item [this]
  has always the type of the surrounding class and generates no constraints.
  \item [Local var]
  No constraints are generated.
  \item[Method invocation]
By direct comparison we show that each of the generated constraints do not apply more restrictions than the \texttt{GT-INVK} rule.
The \texttt{GT-INVK} rule states the condition $\textit{mtype}(m, \textit{bound}_\triangle(T_0)) = \exptype{}{\ol{Y}}\ \ol{U} \to U$.
%In our version of typeless FGJ every method name is unique
%and there is only one class with that particular method.
The constraint $rty \lessdot \exptype{C}{\textbf{fresh}(\ol{X})}$ assures that the type of the expression $e_0$ contains the method \texttt{m}.

\begin{tabular}{l|l}
  \textbf{FGJ Type rule} & \textbf{Constraints} \\
  $\triangle; \Gamma \vdash e_o : T_0$ & $(rty, ConS) = \textbf{TYPEExpr}(Ass, e_r)$\\ 
  $\quad \textit{mtype}(m, \textit{bound}_\triangle(T_0)) = \exptype{}{\ol{Y}}\ \ol{U} \to U$ & $rty \lessdot \exptype{C}{\textbf{fresh}(\ol{X})}$ \\
 %$\textit{mtype}(m, \textit{bound}_\triangle(T_0)) = \ol{U} \to U$ & $rty \doteq cl$\\
 $\triangle; \Gamma \vdash \ol{e} : \ol{S}$ & $\forall e_i \in \overline{e} : (pt_i, ConS_i) = \textbf{TYPEExpr}(Ass, e_i)$\\
 $\triangle \vdash \ol{S} <: \ol{U}$ & $ \bigcup_{T_i \in \overline{T}} (pt_i \lessdot \textbf{fresh}(T_i))$\\
 $\triangle; \Gamma \vdash \mathtt{e_0.m(\overline{e}) : U }$ & $a \doteq \textbf{fresh}(T)$ \\
\end{tabular}

\textit{Note}: The \textbf{TYPEExpr} function only generates constraints which apply to our assumption.
 \item[Field access]
Mostly the same as method invocation.
Fieldnames by default are unique in the FGJ language.

 \begin{tabular}{l|l}
   \textbf{FGJ Type rule} & \textbf{Constraints} \\
   $\Gamma \vdash e_0:T_0$ & $(rty, ConS) = \textbf{TYPEExpr}(Ass, e_r)$\\ 
   $\quad \mathit{fields}(\mathit{bound}_\triangle(T_0)) = \overline{T} \ \overline{f}$ & $rty \doteq \exptype{C}{\textbf{fresh}(\ol{X})}$ \\
  %$\textit{mtype}(m, \textit{bound}_\triangle(T_0)) = \ol{U} \to U$ & $rty \doteq cl$\\
  $\triangle; \Gamma \vdash \ol{e} : \ol{S}$ & $\forall e_i \in \overline{e} : (pt_i, ConS_i) = \textbf{TYPEExpr}(Ass, e_i)$\\
  $\triangle \vdash \ol{S} <: \ol{U}$ & $ \bigcup_{T_i \in \overline{T}} (pt_i \lessdot \textbf{fresh}(T_i))$\\
  $\triangle; \Gamma \vdash \mathtt{e_0.m(\overline{e}) : U }$ & $a \doteq \textbf{fresh}(T)$ \\
 \end{tabular}
 \item[Constructor]

\begin{tabular}{l|l}
  $\triangle; \Gamma \vdash \ol{e} : \ol{S}$ & $\forall e_i \in \overline{e} : (pt_i, ConS_i) = \textbf{TYPEExpr}(Ass, e_i)$\\
  $\triangle \vdash \ol{S} <: \ol{T}$ & $\bigcup_{T_i \in \overline{T}} (pt_i \lessdot T_i)$
\end{tabular}
  
\end{description}

\section{Unify}
\input{Unify}


\subsection{Unify proof}

\begin{theoremAndi}
  \label{theo:unifySoundness}
  \textbf{(Soundness):}
  If the \unify algorithm finds a solution it does not contradict any of the input constraints:
  $\nexists (a \lessdot b) \in {Cons}_{in}$ where $\sigma(a) \nless : \sigma(b)$
\end{theoremAndi}

\textit{Proof:}
We show theorem \ref{theo:unifySoundness} by going backwards over every step of the algorithm.
We assume there exists a unifier $\sigma = \set {a_1 \mapsto \theta_1, \ldots , a_n \mapsto \theta_n}$ for the input constraints,
which is the result of the \unify algorithm.
This means for every constraint in the input set $(a \lessdot b) \in {Cons}_{in}$ and $(c \doteq d) \in {Cons}_{in}$
this unifier will substitute all variables in a way that all constraints are satisfied:
$\sigma(a) \leq \sigma(b)$, $\sigma(c) = \sigma(d)$\\

We now look at each step of the \unify algorithm
which transforms the input set of constraints $Eq$ to a set $Eq'$.
If we assume the unifier $\sigma$ is correct for the set $Eq'$,
then we can show that it will also be correct for the constraints $Eq$. 

\begin{description}
\item[Step 5 c)]
The last step of the algorithm transforms a set of constraints $Eq$ in solved form
No changes to the constraint set are applied here.

\item[Step 5 b)]
A unifier which is correct for $a \doteq b$ is also correct for $a \lessdot b$.
The transformation $Eq' = [a/b]Eq$ does not change this.
The type variable $b$ appears only in constraints of the form $a \lessdot b$,
with both sides being type variables.
There cannot be two constraints $a \lessdot b, b \lessdot a$ in $Eq$.
This would have been removed by the \texttt{equals} rule.
%In the $Eq$ set there can only be one constraint with the type variables $a \lessdot b$.
So the only two other possible combinations for constraints containing $b$ would be $b \lessdot c$ and $c \lessdot b$.
\begin{itemize}
\item If $\set{a \lessdot b, b \lessdot c} \in Eq$ then $\set{a \doteq b, a \lessdot c} \in Eq'$.
%After using the unifier $a \to b$ on $Eq'$: $\set{a \doteq b, b \lessdot c} \in Eq'$.
\item If $\set{a \lessdot b, c \lessdot b} \in Eq$ then $\set{a \doteq b, c \lessdot b} \in Eq'$.
%After using the unifier $a \to b$ on $Eq'$: $\set{a \doteq b, b \lessdot c} \in Eq'$.
\end{itemize}
In both cases a correct unifier for $Eq'$ would also be correct for $Eq$.

\item[Step 5 a)]
We do not alter the constraint set which lateron lead to the unifier.

\item[Step 4]
The constraint sets are not altered here.

\item[Step 3]
An unifier $\sigma$ that is correct for a constraint set
$Eq[a \to \theta] \cup (a \doteq \theta)$ is also correct for
the set $Eq \cup (a \doteq \theta)$.
From the constraint $(a \doteq \theta)$ it follows that $\sigma(a) = \theta$.
This means that $\sigma(Eq) = \sigma(Eq[a \to \theta])$,
because every occurence of $a$ in $Eq$ will be replaced by $\theta$ anyways when using the unifier $\sigma$.

\item[Step 2]
This step transforms constraints of the form $(\exptype{C}{\ol{X}} \lessdot a)$ and $(a \lessdot \exptype{C}{\ol{X}})$
into sets of constraints and builds the cartesian product with the remaining constraints.
We can show that if there is a resulting set of constraints which has $\sigma$ as its correct unifier
then $\sigma$ also has to be a correct unifier for the constraints before this transformation.

We look at each transformation done in step 2:
\begin{description}
\item[$\set{\exptype{C}{\ol{T}} \lessdot a} \in Eq \to \set{a \doteq [\ol{T}/\ol{X}]N} \in Eq'$:]
If $\exptype{C}{\ol{X}} <: N$ and $\sigma$ is correct for $(a \doteq [\ol{T}/\ol{X}]N)$
then $\sigma$ is also correct for $(\exptype{C}{\ol{T}} \lessdot a))$.
When substituting $a$ for $[\ol{T}/\ol{X}]N$ we get 
$(\exptype{C}{\ol{T}} \lessdot [\ol{T}/\ol{X}]N)$
, which is correct because $\exptype{C}{\ol{X}} <: \exptype{C}{\ol{Y}}$
(see \texttt{S-CLASS} rule).
\item[$\set{a \lessdot \exptype{C}{\overline{T}},\ a \lessdot^* b} \in Eq \to \set{T \lessdot b, a \lessdot \exptype{C}{\overline{T}},\ a \lessdot^* b} \in Eq'$]
Trivial
\item[$\set{a \lessdot \exptype{C}{\overline{T}},\ a \lessdot^* b} \in Eq \to \set{a \doteq [\ol{T}/\ol{X}]N, a \lessdot \exptype{C}{\overline{T}},\ a \lessdot^* b} \in Eq'$]
This is the same as in the first transformation.
Here we can also show correctness via the \texttt{S-CLASS} rule.

\end{description}

\item[Step 1]
\begin{description}
\item[erase-rules] remove correct constraints from the constraint set.
A unifier $\sigma$ that is correct for the constraint set $Eq$
is also correct for $Eq \cup \set{\theta \doteq \theta}$
and $Eq \cup \set{\theta \lessdot \theta'}$, when $\theta \leq \theta'$.
\item[swap-rule] does not change the unifier for the constraint set.
$\doteq$ is a symmetric operator and parameters can be swapped freely.
\item[match] The subtype relation is transitive, so if there is a correct solution for
$a \lessdot \exptype{C}{\ol{X}}, \exptype{C}{\ol{X}} \lessdot \exptype{D}{\ol{Y}}$
then this solution would also apply for $a \lessdot \exptype{C}{\ol{X}} \lessdot \exptype{D}{\ol{Y}}$
or $a \lessdot \exptype{D}{\ol{Y}}$.
\item[adopt] An unifier which is correct for $Eq \cup \set{a \lessdot \exptype{C}{\ol{X}}, b \lessdot^* a, b \lessdot \exptype{D}{\ol{Y}}, b \lessdot \exptype{C}{\ol{X}}}$
is also correct for $Eq \cup \set{a \lessdot \exptype{C}{\ol{X}}, b \lessdot^* a, b \lessdot \exptype{D}{\ol{Y}}}$.
\item[adapt] If there is a $\sigma$ which is a correct unifier for a set
$Eq \cup \set{ \exptype{C}{[\ol{A}/\ol{X}]\ol{Y}} \doteq \exptype{C}{\ol{B}}}$ then it is also
a correct unifier for the set $Eq \cup \set{ \exptype{D}{\ol{A}} \lessdot \exptype{C}{\ol{B}}}$,
if there is a subtype relation $\exptype{D}{\ol{X}} \leq^* \exptype{C}{\ol{Y}}$.
To make the set $Eq \cup \set{ [\ol{A}/\ol{X}]\exptype{C}{\ol{Y}} \doteq \exptype{C}{\ol{B}}}$ the unifier 
$\sigma$ must satisfy the condition $\sigma([\ol{A}/\ol{X}]\ol{Y}) = \sigma(\ol{B})$.
By substitution we get $Eq \cup \set{ \exptype{D}{\ol{A}} \lessdot \exptype{C}{[\ol{A}/\ol{X}]\ol{Y}}}$
which is correct under the \texttt{S-CLASS} rule.
\item[reduce] The \texttt{reduce1} and \texttt{reduce2} rules are obviously correct under the FJ typing rules.
\end{description}

\item[OrConstraints]
If $\sigma$ is a correct unifier for one of the constraint sets in $Eq_{set}$
then it is also a correct unifier for the input set $Cons_{in}$.
When building the cartesian product of the \textbf{OrConstraints} every possible
combination for $Cons_{in}$ is build.
No constraint is altered, deleted or modified during this step.
\end{description}
\hfill $\square$


\begin{theoremAndi}\label{theo:unifyCompleteness}
  \textbf{(Completeness):} The \unify algorithm calculates a general unifier for the input set of constraints ($Cons_{in}$).
  A unifier $\sigma$ is a general unifier for $Cons_{in}$ if it unifies $Cons_{in}$
  and for every other unifier $\omega$ there is a substitution $\lambda$ so that $\omega(x) = \lambda(\sigma(x))$.
\end{theoremAndi}
\textit{Proof:}
%The \unify calculates multiple solutions.

%Our proof goes as follows:
We look at every step of the algorithm, which alters the set of constraints $Eq$,
while assuming that there is at least one possible principal type solution $\sigma$ for the input.
We will show that the principal type is among them by proofing for every step of the algorithm that the principal type is never excluded.

%Assume there is a unifier and the Unify algorithm finds it.
%Then no rule makes this unifier impossible / removes this unifier.

\begin{description}
\item[Step 1:]
The first step applies the three rules from figure \ref{fig:fgjerase-rules}.
\textbf{erase-rules:} The erase2 rule from figure \ref{fig:fgjerase-rules} removes a
$\{C \doteq D\}$ constraint from the constraint set.
The erase1 rule removes a $\{C \doteq C\}$ constraint,
but only if the two types $C$ and $D$ satisfy the constraint.
Both rules do not change the set of possible solutions for the given constraint
set.

\textbf{swap-rule:} $\doteq$ is a symmetric operator and parameters can be swapped freely.
This operation does not change the meaning of the constraint set.

\textbf{match-rule:}
If there is a solution for $a \lessdot \exptype{C}{\ol{X}}, a \lessdot \exptype{D}{\ol{Y}}$,
this is also a solution for $a \lessdot \exptype{C}{\ol{X}}, \exptype{C}{\ol{X}} \lessdot \exptype{D}{\ol{Y}}$.
A correct unifier $\sigma$ has to find a type for $a$, which complies with $a \lessdot \exptype{C}{\ol{X}}$ and $a \lessdot \exptype{D}{\ol{Y}}$.
Due to the subtyping relation being transitive this means that $\sigma(a) \lessdot \exptype{C}{\ol{X}} \lessdot \exptype{D}{\ol{Y}}$.

\textbf{adopt-rule:} Subtyping in FJ is transitive,
which allows us to apply the adopt rule without excluding any possible unifier.

\textbf{adapt-rule:} Every solution which is correct for the constraints
$Eq \cup \set{ \exptype{C}{[\ol{A}/\ol{X}]\ol{Y}} \doteq \exptype{C}{\ol{B}}}$ is also
a correct solution for the set $Eq \cup \set{ \exptype{D}{\ol{A}} \lessdot \exptype{C}{\ol{B}}}$.
According to the \texttt{S-CLASS} rule there can only be a possible solution for 
$\exptype{C}{[\ol{A}/\ol{X}]\ol{Y}} \doteq \exptype{C}{\ol{B}}$
if $\ol{B} = [\ol{A}/\ol{X}]\ol{Y}$.
Therefore this transformation does not remove any possible solution from the constraint set.

\textbf{reduce-rule:}
%The constraint is not altered
For a constraint $\exptype{D}{\ol{A}} \lessdot \exptype{D}{\ol{A}}$ the FJ subtyping rule \texttt{S-REFL} ($\triangle \vdash T <: T$) is the only one which applies.
According to this rule the transformation to $\ol{A} \doteq \ol{B}$ is correct.
Only $D$ gets removed, which is not a type variable.
Therefore this step does not remove a possible solution.
This applies for both reduce rules \textbf{reduce1} and \textbf{reduce2}.

\textbf{equals-rule:}
This rule removes a circle in the constraints.
This does not remove a general unifier.

\item[Step 2:]
%The second step builds multiple constraint sets of all possible type combinations for the $\lessdot$-constraints.
The second step of the algorithm eliminates $\lessdot$-constraints
by replacing them with $\doteq$-constraints.
For each $(\exptype{C}{\ol{X}} \lessdot a)$ constraint the algorithm builds a set with every
possible supertype of $\exptype{C}{\ol{X}}$.
So if there is a correct unifier $\sigma$ for the constraints before this conversion there will be at least one set of
constraints for which $\sigma$ is a correct unifier.

\item[Step 3:]
In the third step the \textbf{substitution}-rule is applied.
If there is a constraint $a \doteq N$ then there is no other way to fulfill the constraint set
than replacing $a$ with $N$.
This does not remove a possible solution.

\item[Step 4:]
None of the constraints get modified.

\item[Step 5 a):]
The removed sets do not have a possible unifier, therefore no possible solution is
omitted in this step.

\textbf{Proof}:
In step 5.a all constraint sets that have a unifier are in solved form.
All other possibilities are eliminated in steps 1-4.
There are 8 different variations of constraints:\\
$(a \doteq a), (a \doteq C), (C \doteq a), (C \doteq C), (a \lessdot a), (a \lessdot C), (C \lessdot a), (C \lessdot C)$

After step 1 there are no $(C \doteq C)$, $(C \lessdot C)$ and $(C \doteq a)$ constraints anymore,
as long as the constraint set has a correct unifier.
Because a constraint set that has a correct unifier cannot contain constraints of the form $N_1 \doteq N_2$ with $N_1 \neq N_2$ and
$(N_1 \lessdot N_2)$ with $(N_1 \nleq: N_2)$.
By removing $(N \doteq N)$ and $(C \lessdot D)$ with $(C <: D)$ constraints
no constraints of the form $(C \doteq C)$ and $(C \lessdot D)$
remain in a constraint set that has a correct unifier after step 1.

After step 2 there are no more $(C \lessdot a)$ constraints.

After step 3 there are no $(a \doteq C)$ anymore.

We only reach step 5 if the constraint set is not changed by the substitution (step 3).

%If at this point a set $Eq_i$ is not in solved form it has no correct unifier.

If the constraint set has a correct unifier only $(a \lessdot a)$, $(a \doteq a)$, $a \lessdot C$ and $(a \doteq C)$ constraints are left at this point.
The type variables in the $(a \lessdot a)$ and $(a \doteq a)$ constraints have to be independent type variables.
If a type variable $c$ is inside a $(c \doteq C)$ constraint it is not an independent type variable.
But this variable $c$ cannot be inside a $(a \doteq a)$ or $(a \lessdot a)$ constraint, because otherwise step 3 would have replaced it in there.

So this step only excludes constraint sets which do not have a correct unifier.

\item[Step 5 b):]
If the algorithm advances to this step we further only work on constraint sets in solved form.
This means there are only two kinds of constraints left.
($a \doteq \texttt{T}$), ($a \lessdot \texttt{T}$), ($a \doteq b$) and ($a \lessdot b$) with $a$ and $b$ as type variables.

%We can set all TVs equal, because we allow only same TVs when having circles in a method call.
%This still will lead to the principal type.

The FGJ language does not allow subtype constraints for generic types.
A constraint like $(a \lessdot b)$ in a solution could be inserted as the typing shown in the example below.
But this is not allowed by the syntax of FGJ.
That is why we can treat this constraint as $(a \doteq b)$.

%TODO: This does not alter the outcome because the solution set is not modified anymore. All other TVs alread have a type like A =. Typ

\textit{Example:}
This would be a valid Java program but is not allowed in FGJ:
\begin{lstlisting}
class Example {
  <A extends Object, B extends A> A id(B a){
    return a;
  }
}
\end{lstlisting}

By replacing all ($a \lessdot b$) constraints with ($a \doteq b$) we do not remove a principal type solution.

\item[Step 6:]
In the last step all the constraint sets, which are in solved form, are converted to unifiers.

We see that only a constraint set which has no unifier does not reach solved form.
We showed that in every step of the \unify algorithm we never exclude a possible unifier.
Also we showed that after we reach step 5 only constraint sets with a correct unifier are in solved form.
By removing all constraint sets which are not in solved form the algorithm does not
remove a possible correct unifier.

If we assume that there is a possible principal type solution $\sigma$ for the input set $Cons_{in}$
and the \unify algorithm does not exclude any of the possible unifiers,
then the result \unify contains the principal type solution.
\hfill $\square$
\end{description}

\begin{theoremAndi}\label{theo:unifyTermination}
  \textbf{(Termination):} The \unify algorithm terminates on every finite input set.
\end{theoremAndi}
The \unify algorithm gets called with a set of input constraints.
After resolving the \textbf{OrConstraints} we end up with multiple $Eq$ sets.
Afterwards the algorithm iterates over each of those sets (see Chapter \ref{sec:unify}).
We will show that \unify terminates on each of those sets by showing,
that each step of the algorithm removes at least one type variable
until the finishing state is reached.
The finishing state for a constraint set is reached when step 3 is not able to substitute a type variable.
This is checked by step 4 of the algorithm.
Then the $Eq$ set is either in solved form or determined to be unsolvable.

\textit{Proof:}
The \unify algorithm reduces the amount of type variables with every iteration.
No step adds a new type variable to the constraint set.
Additionally we have to show that the first step of the algorithm also terminates on every finite input set.

\begin{description}
\item[Step 1] 
Step 1 of the algorithm always terminates. \textit{Proof:}
Every rule either removes a $\lessdot$ constraint or reduces a $\exptype{C}{\ol{X}}$ to $\ol{X}$ inside a constraint.
None of the rules add a new $\lessdot$ constraint or a $\exptype{C}{\ol{X}}$ type to the constraint set.
Step 1 has to come to a stop once there are no more $\lessdot$ constraints or $\exptype{C}{\ol{X}}$ types to reduce.

The rule \textbf{match} seems to generate a new $\exptype{C}{\ol{X}}$ constraint,
but the $\exptype{C}{\ol{X}} \lessdot \exptype{D}{\ol{Y}}$ constraint added by \texttt{match}
will be changed immidiatly into a $\doteq$ constraint by the \texttt{adapt} rule.
Afterwards the \texttt{reduce1} rule will remove this freshly added $\exptype{C}{\ol{X}}$ type.
So effectively a $\doteq$ constraint is removed by this rule in combination with \texttt{adapt} and \texttt{reduce1}.

The \textbf{adopt} rule seems to generate a new $\lessdot$ constraint.
But the \texttt{adopt} rule triggers two other rules. The \texttt{match} and the \texttt{adapt} rule.
\begin{enumerate}
  \item We start with the \texttt{adopt} rule: \\
   $
  \begin{array}[c]{ll}
      \begin{array}[c]{l}
          Eq \cup \, \set{a \lessdot
          \exptype{C}{\ol{X}},
          b \lessdot^* a, b \lessdot \exptype{D}{\ol{Y}}} \\ 
          \hline
          \vspace*{-0.4cm}\\
          Eq \cup \set{
          a \lessdot
          \exptype{C}{\ol{X}},
          b \lessdot^*
          a
          , b \lessdot \exptype{C}{\ol{X}}
          , b \lessdot \exptype{D}{\ol{Y}}
          }
      %Eq \cup \set{\theta_1 \doteq \lambda'_1 \ldo \theta_n \doteq \lambda'_n}
      \end{array}
      \end{array}
      $
  \item We can now apply the \texttt{match} rule to the two resulting $(b \lessdot \ldots)$-constraints.
  If this is not possible due to type \texttt{C} not being a subtype of \texttt{D} or vice versa,
  then the $Eq$ set has no possible solution and \unify would terminate as fail $Uni = \emptyset$: \\
  $
  \begin{array}[c]{ll}
  \begin{array}[c]{l}
      Eq \cup \, \set{b \lessdot
      \exptype{C}{\ol{X}},
      b \lessdot
      \exptype{D}{\ol{Y}}} \\ 
      \hline
      \vspace*{-0.4cm}\\
      Eq \cup \set{b \lessdot \exptype{C}{\ol{X}}
      , \exptype{C}{\ol{X}} \lessdot \exptype{D}{\ol{Y}}}
  %Eq \cup \set{\theta_1 \doteq \lambda'_1 \ldo \theta_n \doteq \lambda'_n}
  \end{array}
  & \exptype{C}{\ol{Z}} <: \exptype{D}{\ol{N}} 
  \end{array}
      $\\
  \item The constraint added by the \texttt{match} rule fits the \texttt{adapt} rule, which we apply in the next step: $
  \begin{array}[c]{ll}
  \begin{array}[c]{l}
     Eq \cup \, \set{\exptype{C}{\ol{X}} \lessdot
      \exptype{D}{\ol{Y}}} \\ 
    \hline
    \vspace*{-0.4cm}\\
    Eq \cup \set{\exptype{D}{[ \ol{X} / \ol{Z} ]\ol{N}}
    \doteq \exptype{D}{\ol{Y}}}
  %Eq \cup \set{\theta_1 \doteq \lambda'_1 \ldo \theta_n \doteq \lambda'_n}
  \end{array}
  & \exptype{C}{\ol{Z}} <:\ \exptype{D}{\ol{N}}
  \end{array}
  $
  \end{enumerate}
In the end we have the conversion:\\
\begin{align*}\ddfrac{
  Eq \cup \, \set{a \lessdot
  \exptype{C}{\ol{X}},
  b \lessdot^* a, b \lessdot \exptype{D}{\ol{Y}}}
}{
  Eq \cup \set{a \lessdot
  \exptype{C}{\ol{X}},
  b \lessdot^* a, b \lessdot \exptype{D}{\ol{Y}}, \exptype{D}{[ \ol{X} / \ol{Z} ]\ol{N}}
  \doteq \exptype{D}{\ol{Y}}}
}\end{align*}

We can see now, that only a $\doteq$ constraint is added.
The \texttt{adopt} alone adds a $\lessdot$ constraint,
but due to the fact that it is always used together with \texttt{match} and \texttt{adapt} it effectively just adds a $\doteq$ constraint.

\item[Step 2] This step does not add new type variables to the constraint set.
\item[Step 3] The third step of the \unify algorithm removes at least one type variable
from the constraint set or otherwise does not alter $Eq$ at all.
If $Eq$ is not altered the algorithm terminates in the next step.
The type variable is not completely removed but stays inside $Eq$ only in one $a \doteq N$ constraint.
All other occurences are replaced by $N$.
The \texttt{subst} step can therefore only be executed once per type variable.
\end{description}

We see that with each iteration over the steps 1-3 at least one type variable is removed from the constraint set.
Due to the fact that there is never added a fresh type variable during the \unify algorithm,
the algorithm will terminate for any given finite set of constraints. \hfill $\square$

\section{Elimination Matrix}
\begin{table}[]
  \begin{tabular}{lllll}
  Constraints    & Step 2 & Reduce rule & Adapt rule & swap rule \\
  $a \lessdot N$ &        &             &            &           \\
  $a \doteq N$   &        &             &            &           \\
  $a \doteq b$   &        &             &            &           \\
  $a \lessdot b$ &        &             &            &           \\
  $N \lessdot T$ &        &             & X          &           \\
  $N \doteq T$   &        & X           &            &           \\
  $N \lessdot a$ & X      &             &            &           \\
  $N \doteq a$   &        &             &            & X        
  \end{tabular}
  \end{table}

%\section{Insert principal type}
\section{Insert intersection type}
The outcome of the \textbf{Unify} algorithm is a set of type solutions $Uni_{set}$.
The following function sets in all type solutions by generating overloaded methods:

\begin{description}
\item[Input] is a class with methods of the form%$\texttt{class}\ \exptype{C}{\ol{X} \triangleleft \ol{N}} \triangleleft N \{ \ol{M}\}$
$\mathtt{A\ m(\ol{A}\ \ol{p})\ldots}$, where $\mathtt{A}$ and $\mathtt{\ol{A}}$ are type variables.
And the set of possible type unifiers $Uni_{set}$.
\item[Output] is a type FJ class, where each method has the form
$\mathtt{\exptype{}{\ol{X} \triangleleft \ol{N}}\ T\ m(\ol{T}\ \ol{p})\ldots}$
\end{description}

For every $Uni \in Uni_{set}$ and every method \texttt{m} in the input class we apply the following steps:

\begin{enumerate}
\item In the first step we apply the \texttt{flatten} rule to remove all $a \lessdot b$ constraints. by setting them to $a \doteq b$ and substituting $[a/b]Uni$.
\begin{align*}
  \ddfrac{
    Uni \cup \set{ a \lessdot b}
  }{
    [b/a]Uni \cup \set{ a \doteq b }
  }
  \quad \texttt{flatten}
\end{align*}
\item 
Afterwards we determine which constraints belong to each method.
At this point, the method is still untyped and has type variables as return and parameter types.
After the type-insert step every type variable should either be replaced by a generic type variable or a regular type.

First we look for constraints of the form $a \lessdot N$, where $a \in T$ or $a \in \ol{T}$.
These constraints will be converted to the generic type variables of the method.
The whole set of generic type variable declarations $G_{tvs}$ consists out of:
\begin{itemize}
\item All the type variables, which are used by the method type, but do not have a bound given by the result set $Uni$,
will be assigned \texttt{Object} as bound:\\
$\set{A \triangleleft \texttt{Object} \ |\ (a \lessdot N) \notin Uni',\ a \in T \cup \ol{T}}$
\item Additionally all type variables of the method, which are given a direct bound by a constraint in $Uni$: \\
$\set{A \triangleleft \exptype{C}{\ol{X}} \ |\ (a \lessdot \exptype{C}{\ol{x}}) \in Uni',\ a \in T \cup \ol{T}}$
\item At last we have to add a bound of every type variable added in the last two steps:\\
$\set{A \triangleleft \texttt{Object} \ |\ a \in G_{tvs},\ (a \lessdot N) \notin Uni'}$ and \\
$\set{A \triangleleft \exptype{C}{\ol{X}}  \ |\ a \in G_{tvs},\ (a \lessdot \exptype{C}{\ol{x}} ) \in Uni'}$
\end{itemize}
\item With the resulting $G_{tvs}$ we can finally create the typed method:

$\mathtt{\exptype{}{G_{tvs}}\ \sigma(T)\ m(\sigma(\ol{T})\ \ol{p}))\{ \ldots \}}$

%$\sigma$ is a function which replaces TODO
\end{enumerate}

\subsection{Example}
%TODO


\subsection{Proof}
%Proof that we insert the principle type:
Only constraints of the form $a \lessdot \texttt{N}$ can be inserted as generic variables infront of a method.
After the \textbf{Unify} algorithm the following constraints remain.
\begin{itemize}
  \item $a \lessdot b$, this is removed by replacing every $b$ with $a$ in the result set
  \item $a \lessdot N$, this is used as generic type variables
  \item $a \doteq N$
  \item $a \doteq b$
\end{itemize}

At first remove the $a \lessdot b$ constraints with the \texttt{flatten} rule.
\begin{align*}
  \ddfrac{
    Eq \cup \set{ a \lessdot b}
  }{
    [b/a]Eq \cup \set{ a \doteq b }
  }
  \quad \text{flatten}
\end{align*}


The \textbf{Unify} algorithm returns a set of unifiers ${Uni}$.
Each element of that set is a correct solution.
The unifiers $\sigma$ map type placeholders to types.
When generating the intersection types for the methods we have to make sure that the
type placeholders for the return type as well as for the parameter types get replaced by the same unifier $\sigma$.
It can happen that two unifiers $\sigma_1$ and $\sigma_2$ lead to the same method type ($\sigma_1(A) = \sigma_2(A), \sigma_1(\ol A) = \sigma_2(\ol A)$).
The set of all the distinct combinations then builds the intersection type for the method.

\textbf{Example:}
\begin{lstlisting}
class Global{
  method1(a){
    a.add(this);
    return a.get();
  }
}
class List<A> {
  add(A item){...}
  A get() ...
}
\end{lstlisting}

The method \texttt{method1} would get the type $\set{ \exptype{List}{Object} \to \texttt{Object}
\ || \ \exptype{List}{Global} \to \texttt{Global}}$.
If FGJ would support overloaded methods this could be written as:
\begin{lstlisting}
class Global{
  Object method1(List<Object> a){
    a.add(this);
    return a.get();
  }
  String method1(List<Global> a){
    a.add(this);
    return a.get();
  }
}
class List<A> {
  add(A item){...}
  A get() ...
}
\end{lstlisting}

\section{Complexity}
\section{Complexity}
\label{sec:complexity}

\begin{theorem}[NP-Hardness]
  \label{theo:np-hardness}
  The type inference algorithm for typeless Featherweight Java is NP-hard.
\end{theorem}

\textbf{Proof:} This section will show this by reducing the boolean satisfiability problem (SAT) to the \fjtypeinference{} algorithm.

\begin{figure}
\begin{lstlisting}
  class True extends Object{
  }
  class False extends Object{
  }

  class Nand1 extends Object{
    False nand(True a, True b){ return new False(); }
  }
  class Nand2 extends Object{
    True nand(False a, True b){ return new True(); }
  }
  class Nand3 extends Object{
    True nand(True a, False b){ return new True(); }
  }
  class Nand4 extends Object{
    True nand(False a, False b){ return new True(); }
  }

  class SATExample extends Object{
    True f;

    sat(v1, v2, v3, o1, o2){
      return o1.nand(v1, o2.nand(v2, v3));
    }

    forceSATtoTrue(v1, v2, v3, o1, o2){
      return new SATExample(this.sat(v1, v2, v3, o1, o2));
    }
  }
\end{lstlisting}

\caption{Representation for a SAT problem in FJ code}
\label{fig:fjSATcode}
\end{figure}

Any given boolean expression $B$ can be transformed to a typeless FJ program.
A type inference algorithm finding a possible typisation of this FJ program also solves the boolean expression $B$.
Figure \ref{fig:fjSATcode} shows an example of this.
The classes \texttt{True}, \texttt{False} and \texttt{Operations} always stay the same.
Here we assume that the boolean expression only consists out of $\neg \land$ (NAND) operators.
Now any boolean expression $B_\text{in} = v_1 \land \neg (v_2 \land v_3) \land \ldots$ can be expressed as a Java method.
The example in figure \ref{fig:fjSATcode} represents the problem $B_\text{in} = \neg(v_1 \land \neg (v_2 \land v_3))$.
Additionally we force the return type of the \texttt{sat} method to have the type \texttt{True}
by instancing the \texttt{SATExample} class, which requires the type \texttt{True}.
When using the \fjtypeinference{} algorithm on the generated FJ code it will
assign each parameter of the \texttt{sat} method with either the type \texttt{True} or \texttt{False}.
This represents a valid assignment for the expression $B_\text{in}$.
If \fjtypeinference{} fails to compute a solution the $B_\text{in}$ has no possible solution.
A correct solution for the \texttt{sat} method in figure \ref{fig:fjSATcode} would be:\\
\texttt{True sat(False v1, True v2, True v3, Nand4 o1, Nand1 o2)}

Any SAT problem can be transferred in polynomial time to a typeless FJ program.
Every literal $v$ in the SAT problem becomes a method parameter of the \texttt{sat} method, as well as every instance of a NAND operator used.

This reduction of SAT to our type inference algorithm proofs that its
complexity is at least NP-Hard.
\hfill $\square$

\begin{theorem}[NP-Completeness]
  \label{theo:np-completeness}
  The type inference algorithm for typeless Featherweight Java is NP-Complete.
\end{theorem}

\textbf{Proof:} We know the algorithm is NP-hard (see \ref{theo:np-completeness}).
To proof NP-Completeness we have to show that it is possible to verify a solution in polynomial time.
The verification of a type solution is the FJ typecheck.

It is easy to see that the expression typing rules can be checked in polynomial time as long as subtyping between two types is verifiable in polynomial time.

Subtyping is also solvable in polynomial time in \TFGJ{}.
Assume $\exptype{C}{\ol{X}} \leq \exptype{D}{\ol{Y}}$ with the number of generics $\ol{X}$ and $\ol{Y}$ less or equal $n$.
Also the number of classes in the subtyperelation is less or equal to $n$.
With $n$ classes the \texttt{S-TRANS} rule can be applied a maximum of $n$ times.
Each time the \texttt{S-CLASS} rules is applied which sets in the variables $\ol{X}$ into the supertype.
This operations also runs in polynomial time, so the subtyping relation is decidable in polynomial time.

This shows that the time complexity of the GFJ type check is at least
polynomial or better. 
\hfill $\square$


%%% Local Variables:
%%% mode: latex
%%% TeX-master: "TIforGFJ"
%%% End:


\section{Examples}

Our unify algorithm processes one or two constraints at a time.
Lets look at some examples of different compositions of constraints
and how our unify algorithm will transform them.

\begin{enumerate}
\item $Eq = \set{a \lessdot \exptype{C}{\ol{X}}, a \lessdot b}$ \\
If this is an end configuration, the constraint set is solved.
We can now put any type for $b$, which is a sub or supertype of $\exptype{C}{\ol{X}}$.
The only exception is, when $b \in \ol{X}$.
In this case $\sigma(b) = \tt{Object}$ is always a correct unifier.
So there is atleast one possible solution.

\item $Eq = \set{a \lessdot \exptype{C}{\ol{X}}, a \lessdot b, b \lessdot c}$
It's the same as the example before.
Replacing $b$ and $c$ with \texttt{Object} is also a correct solution. 

\item $Eq = \set{a \lessdot \exptype{C}{\ol{X}}, a \lessdot b, b \lessdot \exptype{D}{\ol{Y}}}$
Here the adopt rule transforms this to:
$\set{a \lessdot \exptype{C}{\ol{X}}, a \lessdot b, b \lessdot \exptype{D}{\ol{Y}}, a \lessdot \exptype{D}{\ol{Y}}}$
and here it depends if $\exptype{C}{\ol{A}} <: \exptype{D}{\ol{B}}$ or $\exptype{D}{\ol{A}} <: \exptype{C}{\ol{B}}$.
Either the $a \lessdot \exptype{C}{\ol{X}}$ or $b \lessdot \exptype{D}{\ol{Y}}$ constraint
gets removed by the match rule.
This leads to $\set{a \lessdot \exptype{C}{\ol{X}}, a \lessdot b, b \lessdot \exptype{D}{\ol{Y}}}$.
If $\texttt{C} = \texttt{D}$ and $\ol{X} = \ol{Y}$ we have the following situation:
$\set{a \lessdot \exptype{C}{\ol{X}}, a \lessdot b, b \lessdot \exptype{D}{\ol{Y}}}$.
Otherwise the match rule gets applied again.

\item $Eq = \set{a \lessdot \exptype{C}{\ol{X}}, a \lessdot b, b \lessdot \exptype{C}{\ol{X}}}$
TODO

\end{enumerate}

\textbf{match-rule:}\\
The match-rule takes two constraints of the form $a \lessdot \exptype{C}{\ol{X}}, a \lessdot \exptype{D}{\ol{Y}}$
and checks which one of the two types \texttt{C} and \texttt{D} is a subtype of the other.
We only leave the constraint, which is more restrictive.
If \texttt{C} and \texttt{D} do not have a subtype relation, the match-rule won't apply.
This will lead to the constraint set never reaching solved form.
We cannot remove the $a \lessdot \exptype{D}{\ol{Y}}$ constraint without matching the
$\ol{Y}$ and $\ol{X}$.
We do this by generating the constraint $\exptype{C}{\ol{X}} \lessdot \exptype{D}{\ol{Y}}$.
This constraint will then be processed by the adapt and the reduce rule.

Example:
$Eq = \set{a \lessdot \exptype{C}{b,\tt{String}}, a \lessdot \exptype{D}{b,b}}$
with $\exptype{C}{A,B} <: \exptype{D}{B,B}$.
\begin{displaymath}
\prftree[r]{
    reduce2
        }{
    \prftree[r]{
        adapt
            }{
    \prftree[r]{
        match
            }{
    \set{a \lessdot \exptype{C}{b,\tt{String}}, a \lessdot \exptype{D}{b,b}}
    }{
    a \lessdot \exptype{C}{b,\tt{String}}, \exptype{C}{b,\tt{String}} \lessdot \exptype{D}{b,b}
    }
    }{
         a \lessdot \exptype{C}{b,\tt{String}}, \exptype{D}{\tt{String},\tt{String}} \doteq \exptype{D}{b,b}
    }}
    {
       a \lessdot \exptype{C}{b,\tt{String}}, b \doteq \tt{String}, b \doteq \tt{String}
        }
\end{displaymath}

\textbf{adopt-rule:}\\
\begin{displaymath}
    \prftree[r]{
        ?
    }{
\prftree[r]{
    reduce2
        }{
    \prftree[r]{
        match
            }{
    \prftree[r]{
        adopt
            }{
    \set{a \lessdot \exptype{C}{\tt{String}}, b \lessdot a, b \lessdot \exptype{C}{c}}
    }{
        \set{a \lessdot \exptype{C}{\tt{String}}, b \lessdot a, b \lessdot \exptype{C}{\tt{String}}, b \lessdot \exptype{C}{c}}
    }
    }{
        \set{a \lessdot \exptype{C}{\tt{String}}, b \lessdot a, b \lessdot \exptype{C}{\tt{String}}, \exptype{C}{\tt{String}} \lessdot \exptype{C}{c}}
    }}
    {
        \set{a \lessdot \exptype{C}{\tt{String}}, b \lessdot a, b \lessdot \exptype{C}{\tt{String}}, \tt{String} \doteq c}
    }}{
    ? \text{(solved form reached?)}
        }
\end{displaymath}


\textbf{circle:}\\
\begin{displaymath}
    \prftree[r]{
        \ldots
    }{
\prftree[r]{
    match
        }{
    \prftree[r]{
        adopt
            }{
    \prftree[r]{
        adopt
            }{
\set{a \lessdot \exptype{C}{x}, b \lessdot a, a \lessdot b}
}{
    \set{a \lessdot \exptype{C}{x}, b \lessdot a, b \lessdot \exptype{C}{x}, a \lessdot b}
}
}{
    \set{a \lessdot \exptype{C}{x}, b \lessdot a, b \lessdot \exptype{C}{x}, a \lessdot b, a \lessdot \exptype{C}{x}}
}}
{
    \set{a \lessdot \exptype{C}{x}, b \lessdot a, b \lessdot \exptype{C}{x}, \exptype{C}{x} \lessdot \exptype{C}{x}, a \lessdot b, \exptype{C}{x} \lessdot \exptype{C}{x}}
}}{
 \ldots
        }
\end{displaymath}

\textbf{longer circle:}\\
\begin{displaymath}
    \prftree[r]{
        adopt, adopt
}{
\set{a \lessdot \exptype{C}{x}, b \lessdot a, c \lessdot b, c \lessdot a}
}{
\set{a \lessdot \exptype{C}{x}, b \lessdot a,
c \lessdot b, c \lessdot a, b \lessdot \exptype{C}{x}, c \lessdot \exptype{C}{x}}
}
\end{displaymath}

\textbf{longer circle 2:}\\
\begin{displaymath}
    \prftree[r]{
        match, reduce2
    }{
\prftree[r]{
    adopt
        }{
    \prftree[r]{
        match, reduce2
            }{
    \prftree[r]{
        adopt, adopt
}{
\set{a \lessdot \exptype{C}{x}, b \lessdot a, c \lessdot b, c \lessdot a, b \lessdot \exptype{C}{String}}
}{
\set{a \lessdot \exptype{C}{x}, b \lessdot a,
c \lessdot b, c \lessdot a, b \lessdot \exptype{C}{x}, b \lessdot \exptype{C}{\tt{String}}, c \lessdot \exptype{C}{\tt{String}}}
}
}{
    \set{a \lessdot \exptype{C}{x}, b \lessdot a,
    c \lessdot b, c \lessdot a, b \lessdot \exptype{C}{x}, c \lessdot \exptype{C}{\tt{String}}, x \doteq \tt{String}}    
}}
{
    \set{a \lessdot \exptype{C}{x}, b \lessdot a,
    c \lessdot b, c \lessdot a, b \lessdot \exptype{C}{x}, c \lessdot \exptype{C}{\tt{String}}, c \lessdot \exptype{C}{x}, x \doteq \tt{String}}   
}}{
    \set{a \lessdot \exptype{C}{x}, b \lessdot a,
    c \lessdot b, c \lessdot a, b \lessdot \exptype{C}{x}, c \lessdot \exptype{C}{x}, x \doteq \tt{String}}   
        }
\end{displaymath}



\textbf{Example 1}\\
The algorithm is able to infer the types of multiple classes under specific circumstances.
The individual classes must be given to him after one another.
This comes with the restriction, that the first class is correct on its own and does not use any other class.
The second class that gets compiled can use the first class and so on.

The following example shows how the algorithm infers and compiles multiple classes iteratively.
The class \texttt{Class1} is infered first.
It has only one method which is the identity function,
to which our algorithm allocates the type $\exptype{}{A}\ A \to A$.
The next class \texttt{Class2} is now able to use this generic method.
The blue colored types are inferred in the next iteration of our algorithm.

\begin{table}
\caption{Two classes as input. \texttt{Class1} is infered first (shown in {\color{red}red})}
\begin{tabular}{cc}
\begin{lstlisting}
class Class1 extends Object {
  Class1() { super(); }
  id(a){
    return a;
  }
}
class Class2 extends Class1 {
  Class2() { 
    super(); 
  }
  example(){
    return new Class1().id(this);
  }
}
\end{lstlisting}
&
\begin{lstlisting}
class Class1 extends Object {
  Class1() { super(); }
  (*@ \textcolor{red}{<A> A} @*) id((*@ \textcolor{red}{A} @*) a){
    return a;
  }
}
class Class2 extends Class1 {
  Class2() { 
    super(); 
  }
  (*@ \textcolor{blue}{Class1}@*) example(){
    return this.(*@\textcolor{blue}{<Class1>}@*)id(this);
  }
}
\end{lstlisting}
\end{tabular}
\end{table}

\textbf{Example 2}\\
When compiling a class like the following
we have to first split this class into two classes.
The \texttt{TwoMethods} class can be first split into the classes \texttt{Class1}
and \texttt{Class2} and after being processed by the type inference algorithm it can be assembled back together again.
This leads to a principal typing.
When using our type inference algorithm on the class \texttt{TwoMethods} alone
it would give the method \texttt{id} the type $\texttt{TwoMethods} \to \texttt{TwoMethods}$,
which is not the desired principal type.
\begin{lstlisting}
class TwoMethods extends Object {
  TwoMethods() { super(); }
  id(a){
    return a;
  }
  example(){
    return this.id(this);
  }
}
\end{lstlisting}

\textbf{Example 3}\\
%TODO: Ein Beispiel für die Unify-adapt Regel
FGJ allows subtype relations like the following:
\begin{lstlisting}
class Map<A,B> extends Object {
  Map<A,B>() { super(); }
}
class SpecialMap<A,B,C> extends Map<A,C> {
  SpecialMap<A,B,C>() { super(); }
}
\end{lstlisting}

If for example we have a method \texttt{method} like this:
\begin{lstlisting}
<X> void method(Map<X, String> map){
  ...
}
\end{lstlisting}
and call it:
\begin{lstlisting}
method(new SpecialMap<Object,Integer,String>());
\end{lstlisting}

Then the constraint $\exptype{SpecialMap}{Object,Integer,String} \lessdot \exptype{Map}{X,String}$
is generated by the \textbf{FJTYPE} algorithm.
This constraint will be processed by the \texttt{adapt} rule of the \textbf{Unify} algorithm.
Remember that $(\exptype{SpecialMap}{A,B,C} \olsub \exptype{Map}{A,C}) \in S_\leq$.
\begin{align*}
  Eq& \cup \exptype{SpecialMap}{Object,Integer,String} \lessdot \exptype{Map}{X,String} \\
  \cline{1-2} 
  Eq& \cup \set{\exptype{Map}{[ Object / A ][ Integer / B ][ String / C ](A,C)}
  \doteq \exptype{Map}{X,Integer}} \\
  \cline{1-2} 
  Eq& \cup \set{\exptype{Map}{Object,String}
  \doteq \exptype{C}{X, Integer}}
%Eq \cup \set{\theta_1 \doteq \lambda'_1 \ldo \theta_n \doteq \lambda'_n}
\end{align*}

After the \texttt{adapt} rule got applied we can already see that a correct unificator for this constraint would be
$\sigma(X) = \texttt{Object}$.

\textbf{Example 4 (Multiple type solutions):}
\begin{lstlisting}
class List<A> extends Object {
  List<A> add(A p){...}
}
class C1 extends Object {
  m1(ls){
    return ls.add(this);
  }
}
class C2 extends Object {
  m2(){
    return new C1().m1(new List<C1>());
  }
}
\end{lstlisting}
When compiling the class \texttt{C1} there are two possible method typings for \texttt{m1}.
One of it would 
\begin{lstlisting}
class List<A> extends Object {
  List<A> add(A p){...}
}
class C1 extends Object {
  List<Object> m1((*@\color{red}List<Object>@*) ls){
    return ls.add(this);
  }
}
class C2 extends Object {
  m2(){
    return new C1().m1((*@\color{red}new List<C1>()@*));
  }
}
\end{lstlisting}
\texttt{List<Object>} would be a correct type for the parameter \texttt{ls}.
The call \texttt{ls.add(this)} still works because the type of \texttt{this} is a subtype of \texttt{Object}.
But then the method \texttt{m2} will be incorrect when compiling the class \texttt{C2}.
Our algorithm has to backtrack to the class \texttt{C1} and use the other possible typing.
The following would be the correct solution with \texttt{List<C1>} as the type for \texttt{ls}.
\begin{lstlisting}
class List<A> extends Object {
  List<A> add(A p){...}
}
class C1 extends Object {
  List<C1> m1((*@\color{green}List<C1>@*) ls){
    return ls.add(this);
  }
}
class C2 extends Object {
  List<C1> m2(){
    return new C1().m1((*@\color{green}new List<C1>()@*));
  }
}
\end{lstlisting}

\textbf{Example 5} (for global type inference)
\begin{lstlisting}

\end{lstlisting}

\section{Assessment}
\label{sec:assessment}

\begin{itemize}
\item NP-hard \todo[inline]{Show NP-completeness!}
\item cannot infer all possible generic methods
\item features that need to be addressed to make it practical (e.g.,
  what's necessary to move from FGJ to full Java: overloading,
  imperative, )
\end{itemize}

Java has some features which make global type inference hard:
\begin{itemize}
\item subtyping and overloading combined with a nominal type system;
  we show NP-hardness even without overloading
  \todo[inline]{correct?}
\item mutable local variables and mutable object state,
\item polymorphic recursion in method calls
\item Inside a Java method it is possible to call every other declared Java method
  \todo[inline]{meaning? The letrec-rule works on the fact that in a let statement:
  \lstinline{let x = e in e2} the expression \texttt{e} is not able to use \texttt{x}}
\item methods can have side effects.
\end{itemize}


For Featherweight Generic Java it is easier to decide because there is no state and therefore no wildcard types.
Also we exclude polymorphic recursion.

% Example for local type inference in Java:
% \begin{lstlisting}[language=java]
% // Java 8 code:
% class Local {
%   // type inference puts in "ArrayList<String>()"
%   List<String> field = new ArrayList<>();
% }
% \end{lstlisting}
% \todo[inline]{Is this a drawback for LVTI? Does GTI put
%   \lstinline{List<String>} or \lstinline{Map} in the
%   \lstinline{outerMap} example?}

\section{Limits of the algorithm}
Java does not allow generic variables with lower bounds.
The expression \texttt{<A super String>} is not a correct definition in FJ.

Take the following input program as an example:
\begin{lstlisting}{java}
class Example{
    method(list){
        return list.add(this).get();
    }
}
\end{lstlisting}
Lets assume this yields the constraints:
$p \lessdot \exptype{List}{b}, r \lessdot \exptype{List}{b}, \texttt{Example} \lessdot b$
Now we cannot insert a generic type variable with super bounds like this:
\texttt{B super Example}

We have to still calculate every possible supertype of \texttt{Example} which will lead to an array of possible solutions.


\section{Related Work}
\section{Related Work}
\label{sec:related-work}


\subsection{Formal models for Java}
\label{sec:formal-models-java}

There is a range of formal models for Java. Flatt et al
\cite{DBLP:conf/java/FlattKF99} define an elaborate model with
interfaces and classes and prove a type soundness result. They do not
address generics. Igarashi et al
\cite{DBLP:journals/toplas/IgarashiPW01} define Featherweight Java
and its generic sibling, Featherweight Generic Java. Their language is
a functional calculus reduced to the bare essentials, they develop the full metatheory, they
support generics, and study the type erasing transformation used by
the Java compiler. MJ \cite{UCAM-CL-TR-563} is a core calculus that
embraces imperative programming as it is targeted towards reasoning
about effects. It does not consider generics. Welterweight Java
\cite{DBLP:conf/tools/OstlundW10} and OOlong
\cite{DBLP:conf/sac/CastegrenW18} are different sketches for a core
language that includes concurrency, which none of the other core
languages considers. 

We chose to base our development on FGJ because it embraces a relevant
subset of Java without including too much complexity (e.g., no imperative
features, no interfaces, no concurrency). It seems that results for
FGJ are easily scalable to full Java. We leave the addition of these
feature to future work, as we see our results on FGJ as a first step
towards a formalized basis for global type inference for Java.

\subsection{Type inference}

Some object-oriented languages like Scala, C\#, and Java perform
\emph{local} type inference \cite{PT98,OZZ01}. Local type 
inference means that missing type annotations are recovered using only
information from adjacent nodes in the syntax tree without long distance
constraints. For instance, the type of a variable initialized with a
non-functional expression or the return type of a method can be
inferred. However, method argument types, in particular for recursive
methods, cannot be inferred by local type inference.

Milner's algorithm $\mathcal{W}$ \cite{DBLP:journals/jcss/Milner78} is
the gold standard for global type inference for languages with 
parametric polymorphism, which is used by ML-style languages. The fundamental idea
of the algorithm is to enforce type equality by many-sorted type
unification \cite{Rob65,MM82}. This approach is effective and results
in so-called principal types because many-sorted unification is
unitary, which means that there is at most one most general result.

Pl\"umicke \cite{Plue07_3} presents a first attempt to adopt Milner's
approach to Java. However, the presence of subtyping means that type
unification is no longer unitary, but still finitary. Thus, there is
no longer a single most general type, but any type is an instance of a
finite set of maximal types (for more details see Section
\ref{sec:unification}). Further work by the same author
\cite{plue15_2,plue17_2}, 
refines this approach by moving to a constraint-based algorithm and by
considering lambda expressions and Scale-like function types.
In Pl\"umicke's work there is no formal definition of the type system as a basis
of the type inference algorithm. One contribution of this paper is a
formal definition of the underlying type system. 

We rule out polymorphic recursion because its presence makes type
inference (but not type checking: see FGJ) undecidable. Henglein
\cite{DBLP:journals/toplas/Henglein93} as well as Kfoury et al
\cite{DBLP:journals/toplas/KfouryTU93} investigate type inference in
the presence of polymorphic recursion. They show that type inference
is reducible to semi-unification, which is undecidable
\cite{DBLP:journals/iandc/KfouryTU93}. However, the undecidability of
this problem apparently does not matter much in practice
\cite{DBLP:journals/tcs/EmmsL99}. 

Ancona, Damiani, Drossopoulou, and Zucca \cite{ADDZ05} consider polymorphic byte
code. Their approach is modular in the sense that it infers
polymorphic structural types. As {Java} does not support structural
types, their approach would have to be simulated with generated
interfaces. Pl\"umicke \cite{plue16_1} follows this
approach. Furthermore Ancona and coworkers do not consider generic classes. 



\subsection{Unification}
\label{sec:unification}

We reduce the type inference problem to constraint solving with
equality and subtype constraints.
The procedure presented in Section~\ref{sec:unify} is inspired by
polymorphic order-sorted unification which is used in logic 
programming languages with polymorphic order-sorted types
\cite{GS89,MH91,HiTo92,CB95}.

Smolka's thesis \cite{GS89} mentions type unification
as an open problem. He gives  an incomplete type inference algorithm
for the logical language \textsf{TEL}. The reason for incompleteness
is the admission of subtype relationships between polymorphic types of
different arities as in  $\texttt{List(a)} \sub
\texttt{myLi(a,b)}$. In consequence, the subtyping relation does
not fulfill the ascending chain condition.
For example, given  $\texttt{List(a)} \sub \texttt{myLi(a,b)}$, we obtain:
\begin{gather*}
  \texttt{List(a)} \sub \texttt{myLi(a,List(a))} \sub \texttt{myLi(a,myLi(a,List(a)))}  \sub \dots
\end{gather*}
However, this subtyping chain exploits covariant subtyping, which does
not apply to FGJ.
%(but it would apply in the presence of wildcards).

Smolka's algorithm also fails sometimes in the absence of infinite
chains, although there is a unifier. 
For example, given $\texttt{nat} \sub \texttt{int}$ and the set
of subtyping constraints $\set{\mathtt{nat} \lessdot \mathtt{a},
  \mathtt{int} \lessdot \mathtt{a}}$, it returns the substitution
$\set{\mathtt{a} \mapsto \mathtt{nat}}$ generated from the first
constraint encountered. This substitution is not a solution
because $\set{\mathtt{int} \lessdot \mathtt{nat}}$ fails.
However, $\set{\mathtt{a}\mapsto \mathtt{int}}$ is a unifier, which
can be obtained by processing the constraints in a different order: from $\set{\mathtt{int} \lessdot \mathtt{a}, \mathtt{nat} \lessdot
  \mathtt{a}}$ the algorithm calculates the unifier 
$\set{\mathtt{a}\mapsto \mathtt{int}}$.

Hill and Topor  \cite{HiTo92} propose a polymorphically typed logic
programming language with subtyping. They restrict subtyping to type
constructors of the same arity,  which guarantees that all subtyping
chains are finite.
In this approach a \emph{most general type unifier (mgtu)} is
defined as an upper bound of different principal type unifiers. In
general, two type terms need not have an upper bound in the subtype ordering,
which means that there is no mgtu in the sense of Hill and Topor.
For example, given  $\texttt{nat} \sub \texttt{int}$, $\texttt{neg} 
\sub \texttt{int}$, and the set of inequations $\set{\mathtt{nat} \lessdot
  \mathtt{a}$, $\mathtt{neg} \lessdot \mathtt{a}}$, the mgtu $\set{\mathtt{a} \mapsto \texttt{int}}$ is
determined. If the subtype ordering is extended by $\mathtt{int} \sub
\mathtt{index}$ and $\mathtt{int} \sub \mathtt{expr}$, then there are three
unifiers $\set{\mathtt{a} \mapsto \texttt{int}}$, $\set{\mathtt{a} \mapsto
  \mathtt{index}}$, and $\set{\mathtt{a} \mapsto 
\mathtt{expr}}$, but none of them is an mgtu \cite{HiTo92}.

The type system of \textsf{PROTOS-L} \cite{CB95} was
derived from \textsf{TEL} by disallowing any explicit subtype relationships
between polymorphic type constructors. 
Beierle \cite{CB95} gives a complete type unification algorithm, which can be extended to the
type system of Hill and Topor.
They also prove that the type unification problem is finitary.

Given the declarations  $\texttt{nat} \sub
\texttt{int}$, $\texttt{neg} \sub \texttt{int}$, $\mathtt{int} \sub
\mathtt{index}$, and $\mathtt{int} \sub \mathtt{expr}$, applying the
type unification algorithm of \textsf{PROTOS-L} to the set of
inequations $\set{\mathtt{nat} \lessdot
  \mathtt{a}$, $\mathtt{neg} \lessdot \mathtt{a}}$ yield three general
unifiers $\set{\mathtt{a} \mapsto \texttt{int}}$, $\set{\mathtt{a} \mapsto
  \mathtt{index}}$, and $\set{\mathtt{a} \mapsto \mathtt{expr}}$. 

Pl\"umicke \cite{plue09_1} realized that the type system of
\textsf{TEL} is related to subtyping in Java.
In contrast to \textsf{TEL}, where the ascending chain condition does
not hold,  Java with wildcards violates the descending chain
condition. For example, given $\exptypett{myLi}{b,a} \olsub
\exptypett{List}{a}$ we find:

\smallskip
{\centering
$\ldots \ \olsub\
\exptypett{myLi}{\exptypett{?\,$\extends$\,myLi}{\exptypett{\textrm{{\tt ?}}\,$\extends$\,List}{a},a},a}
\ \olsub\ \exptypett{myLi}{\exptypett{?\,$\extends$\,List}{a},a} \ \olsub\  \exptypett{List}{a}$\\}

\smallskip
Pl\"umicke \cite{plue09_1} solved the open problem of infinite chains
posed by Smolka \cite{GS89}.
He showed that in any infinite chain there is a finite number of elements such that
all other elements of the chain are instances of them. The resulting type
unification algorithm can be used for type inference of Java~5 with
wildcards \cite{Plue07_3}. As FGJ has no wildcards, we based our
algorithm on an earlier work \cite{Plue04_1}.
In contrast to that work, which only infers generic methods with
unbounded types, our algorithm  infers bounded generics.
To this end, we do not expand constraints
of the form $\tv{a} \lessdot \itype{N}$, where $\tv{a}$ is type variable and $\itype{N}$ is is a
non-variable type, but convert them to bounded type parameters of the form
\texttt{X extends N}. This change results in a significant reduction
of the number of solutions of the type
unification algorithm without restricting the generality of typings of
FGJ-programs. Unfortunately, constraints of the form $\itype{N} \lessdot \tv{a}$ have
to be expanded as FGJ (like Java) does not permit lower bounds for
generic parameters. If lower bounds were permitted  (as in Scala), the
number of solutions could be reduced even further.


%%% Local Variables:
%%% mode: latex
%%% TeX-master: "TIforGFJ"
%%% End:


\section{Conclusions}
\label{sec:conclusions}


\bibliographystyle{splncs04}
\bibliography{peter,martin}

\end{document}
\endinput
%%
%% End of file `TIforGFJ.tex'.

%%% Local Variables:
%%% mode: latex
%%% TeX-master: t
%%% End:
