\usepackage{xspace}
\usepackage{color,ulem}
\usepackage{listings}
\lstset{language=Java,
  showspaces=false,
  showtabs=false,
  breaklines=true,
  showstringspaces=false,
  breakatwhitespace=true,
  basicstyle=\ttfamily\fontsize{8}{9.6}\selectfont, %\footnotesize
  escapeinside={(*@}{@*)},
  captionpos=b,
}
\lstdefinestyle{fgj}{backgroundcolor=\color{lime!20}}
\lstdefinestyle{tfgj}{backgroundcolor=\color{lightgray!20}}


\newcommand\mv[1]{{\tt #1}}
\newcommand{\ol}[1]{\overline{\tt #1}}
\newcommand{\exptype}[2]{\mathtt{#1 \texttt{<} #2 \texttt{>} }}
\newcommand\ddfrac[2]{\frac{\displaystyle #1}{\displaystyle #2}}

\newcommand{\sarray}[2]{\begin{array}[t]{#1} #2 \end{array}}

\newcommand{\olsub}{\textrm{$\, \leq^\ast \,$}\ }

\newcommand{\sub}{\mbox{$<$}}

\newcommand{\set}[1]{\{ #1 \} }

\definecolor{red}{rgb}{1,0,0}
\newcommand{\red}[1]{\textcolor{red}{#1}}

\newcommand{\commentarystar}[1]{\red{\({}^*\)}\marginpar[\tiny
  \red{\({}^*\)#1}]{\tiny \red{\({}^*\)#1}}}
\newcommand{\commentary}[1]{\marginpar[\tiny
  \red{#1}]{\tiny \red{#1}}}
\newcommand{\commentarymark}[1]{\color{red} #1\ensuremath{^*}\color{black}}
\newcommand{\commentarymargintext}[2]{\color{red} #1$^*$
  \color{black}\marginpar[\tiny \red{\({}^*\)#2}]{\tiny \red{\({}^*\)#2}}}
\newcommand{\commentaryintext}[2]{\color{red} #1\textrm{$^*${\tiny #2}}\color{black}}
\newcommand{\commentarymath}[2]{\color{red} #1^*\color{black}\)\marginpar[\tiny \red{\({}^*\)#2}]{\tiny \red{\({}^*\)#2}}\(}
\newcommand{\replaced}[2]{\erased{#1}\color{red}#2\color{black}}
\newcommand{\erased}[1]{\commentary{\sout{#1}}}

\newcommand\Erase[1]{|#1|}
\newcommand\Angle[1]{\langle#1\rangle}

\newcommand\TFGJ{FGJ-GT\xspace}
\newcommand\FGJType{\textbf{FGJType} }

\newcommand\TVX{\mv X}
\newcommand\TVY{\mv Y}
\newcommand\TVZ{\mv Z}
\newcommand\TVW{\mv W}

\newcommand\CL[1]{\textit{Cl}$_{#1}$}
\newcommand\subconstr{\lessdot}
\newcommand\eqconstr{\doteq}
\newcommand\subeq{\mathbin{\texttt{<:}}}
\newcommand\extends{\triangleleft}

\newcommand\rulename[1]{\textup{\textrm{(#1)}}}

%%% Commands for FGJTYPE algorithm
\newcommand{\tv}[1]{\mathit{ #1 }}
\newcommand{\fjtype}{\textbf{FJTYPE}}
\newcommand{\unify}{\textbf{Unify}}
\newcommand{\typeMethod}{\textbf{TYPEMethod}}
\newcommand{\typeExpr}{\textbf{TYPEExpr}}
\newcommand{\constraint}{\ensuremath{\mathit{c}}}%{\ensuremath{\mathtt{C}}}
\newcommand{\consSet}{C}%{\ensuremath{\overline{\mathtt{C}}}}
\newcommand{\orCons}{\ensuremath{oc}}%{\ensuremath{\textbf{C}_{||}}}
\newcommand{\simpleCons}{\ensuremath{sc}}
\newcommand{\overridesFunc}{\textit{overrides}}
\newcommand{\typeAssumptionsSymbol}{\ensuremath{\Theta}}
\newcommand{\typeAssumptions}{\ensuremath{(\mv{\Pi} ; \overline{\fieldAssumption}; \overline{\localVarAssumption})}}%{\ensuremath{(\overline{\methodAssumption} ; \overline{\fieldAssumption}; \overline{\localVarAssumption})}}
\newcommand{\constraints}{\ensuremath{\mathit{\overline{c}}}}
\newcommand{\itype}[1]{\ensuremath{\mathit{#1}}}
\newcommand{\type}[1]{\texttt{#1}}
\newcommand{\gType}[1]{\texttt{#1}}
\newcommand{\mtypeEnvironment}{\ensuremath{\Pi}}
\newcommand{\methodAssumption}{\ensuremath{\mathtt{\lambda}}}
\newcommand{\fieldAssumption}{\ensuremath{\mathtt{\varphi}}}
\newcommand{\localVarAssumption}{\ensuremath{\mathtt{\eta}}}


%%% Local Variables:
%%% mode: latex
%%% TeX-master: "TIforGFJ"
%%% End:
