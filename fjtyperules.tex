As \TFGJ omits some type annotations, we are forced to adapt some of 
FGJ's typing rules. The new rules cater for omitted type annotations
and also disable polymorphic recursion.

An environment $\mv\Gamma$ is a finite mapping from variables to type,
written $\ol x:\ol T$; a type environment $\mv\Delta$ is a finite mapping
from type variables to nonvariable types, written $\ol X\subeq\ol
N$, which takes each type variables to its bound. As in FGJ, we do not
impose an ordering on environment entries to enable F-bounded
polymorphism.

There is a new method environment $\mv\Pi$ which maps pairs of a class header
$\exptype{C}{\ol X}$ and a method name $\mv m$ to a set of method
types of the form $\exptype{}{\ol{Y} \triangleleft  \ol{P}} \ol{T} \to \mv{T}$. 
It supports the \textit{mtype} function that relates a nonvariable type
$\mv N$ and a method name $\mv m$ to a method type.

The judgments for subtyping $\mathtt{\Delta \vdash S \subeq T}$ and
well-formedness of types $\mathtt{\Delta \vdash T\ \mathtt{ok}}$
(Figure~\ref{fig:well-formedness-and-subtyping}) stay the same as in
FGJ.

The rules for expression typing $\mathtt{\Pi; \Delta; \Gamma \vdash e
  : T}$ change subtly (see
Figure~\ref{fig:expression-typing}). Firstly, the judgment includes
the method 
environment $\mv\Pi$, which remains unchanged in expression
typing. This environment is only used in the revised rule
\rulename{GT-INVK}, where it is passed as an additional parameter to
\textit{mtype}. The revised definition of \textit{mtype} (in
Figure~\ref{fig:auxiliary-functions}) locates the class that contains
the method definition by traversing the subtype hierarchy and looks up the
method type in environment $\mv\Pi$, which contains the method types
that were already inferred. Our definition of \textit{mtype} supports
overloading as $\mv\Pi$ may relate several types to the same method
definition. The instantiation of the method's type parameters is
inferred in \TFGJ.

The other typing rule that has changed is \rulename{GT-NEW} where the
instantiation of the class's type parameters is now inferred: the rule
simply assumes a suitable instantiation by some $\ol U$.


% The input for our type inference algorithm is based on Featherweight Generic Java (FGJ).
% FGJ is defined by syntax and typing rules.
% We already changed the syntax to allow typeless FGJ programs as input for our algorithm.
% Additionally we alter the typing rules slightly, which is presented in this chapter.

%Our type inference algorithm takes typeless FGJ classes as input.
%The generated output is correct FGJ, although we have to alter some rules.
%This chapter defines the typing rules for our version of FGJ,
%which our type inference algorithm is able to process.

Moving on, the typing rule for a method $\mv m$, \rulename{GT-METHOD}, changes
significantly. The typing of $\mv m$ is already provided by the new
method environment $\mv\Pi$.  The type environment $\mv\Delta$ is
also provided as an input. Moreover, to rule out polymorphic
recursion, the assumptions about the local methods of class $\mv{C}$
are monomorphic at this stage. So the rule type checks the body for
all overloadings of method $\mv m$.

All this information is provided and generated by the rule for class
typing, \rulename{GT-CLASS}. A class typing for \mv{C} receives an incoming
method type environment $\mv\Pi$ and generates and extended one
$\mv{\Pi''}$ which additionally contains the method types inferred for
\mv{C}.

In $\mv{\Pi'}$, we first generate some monomorphic types for all
methods. We use these types to check the methods. Afterwards, we
return generalized versions of these same types in $\mv{\Pi''}$. All
method types using the same generic type variables $\ol Y$ with the
same constraints $\ol P$. This is sufficient in the absence of
polymorphic recursion as we showed in
Proposition~\ref{prop:polymorphi-recursion}.

The new program typing rule \rulename{GT-PROGRAM} starts with an empty
method environment and applies class typing to each class in the
sequence provided. Each processed class adds its method typings to the method
environment which is threaded through to constitute the program type. 

\todo[inline]{PJT 20220128: is this still an active remark?}
\begin{itemize}
\item We omit the rules for valid downcasts.
\todo[inline]{Andi: Is this correct? We need another constraint to be correct:
$N \lessdot_{downcast} a$ for a cast of the form: \texttt{(N)a}.
This is a different constraint then the normal subtype relationship (see FJ paper)
}
%The downcast problem:

%This is wrong:
%m(List<SortedList> p, Object o){
%  p.add((SortedList) o);
%}

%This is correct:
%m(List<SortedList> p, List<Object> o){
%  p.add((SortedList) o);
%}
%Currently we do not generate constraints for casts
\end{itemize}


\begin{figure}[tp]
  \fbox{
    \begin{minipage}{\textwidth}
      \begin{small}
        \textbf{Subtyping:}\\[1em]
        \begin{tabularx}{\textwidth}{X c X r}
          & $\mathtt{ \Delta \vdash T \subeq  T } $
          &   & \rulename{S-REFL} \\

          & \\
          &
          $\mathtt{\ddfrac{ \Delta \vdash S \subeq  T \quad \quad
              \Delta \vdash T \subeq  U }{ \Delta \vdash S \subeq  U
            }}$ & & \rulename{S-TRANS} \\

          & \\

          & $\mathtt{ \Delta \vdash X \subeq  \Delta(X)
          }$ & & \rulename{S-VAR} \\
          & \\
          &
          $\mathtt{\ddfrac{ \texttt{class}\ \exptype{C}{\ol{X}
                \triangleleft \ol{N}} \triangleleft \mv N \set{ \ldots
              } }{ \Delta \vdash \exptype{C}{\ol{T}} \subeq 
              [\ol{T}/\ol{X}]\mv N
            }}$ & & \rulename{S-CLASS} \\
          & & & \\
          \hline
          \multicolumn{1}{l}{\textbf{Well-formed types:}}\\
          & & & \\
          & $\mathtt{ \Delta \vdash \texttt{Object}\ \texttt{ok}
          }$ & & \rulename{WF-OBJECT}\\
          & \\
          &
          $\mathtt{\ddfrac{ X \in \textit{dom}(\Delta) }{ \Delta
              \vdash X \ \texttt{ok} } }
          $ & & \rulename{WF-VAR} \\
          & \\
          & $\mathtt{\ddfrac{\begin{array}{c}
                               \texttt{class}\ \exptype{C}{\ol{X}
                               \triangleleft \ol{N}} \triangleleft \mv N \{ \ldots \} \\
                               \mathtt{\Delta} \vdash \ol{T} \ \texttt{ok}
                               \quad \quad \mathtt{\Delta} \vdash \ol{T} \subeq 
                               [\ol{T}/\ol{X}]\ol{N}
                             \end{array}
                           }{ \Delta \vdash \exptype{C}{\ol{T}} \
                             \texttt{ok} } } $ & & \rulename{WF-CLASS}
                       \end{tabularx}
                     \end{small}
                   \end{minipage}
                 }
                 \caption{Well-formedness and subtyping}
                 \label{fig:well-formedness-and-subtyping}
               \end{figure}


\newcommand{\environmentvdash}{\Pi;\Delta;\Gamma \vdash}

\begin{figure}[tp]
\fbox{
\begin{minipage}{\textwidth}
\begin{small}
\textbf{Expression typing:}\\
\begin{tabularx}{\textwidth}{c X r}
%\begin{tabular}{l@{\quad}l}
  $\mathtt{
\environmentvdash x : \Gamma(x)
}$ & & \rulename{GT-VAR} \\
& \\

$\mathtt{\ddfrac{
    \Pi; \Delta; \Gamma \vdash e_0:T_0 \qquad
    \mathit{fields}(\mathit{bound}_\Delta(T_0)) = \overline{T} \ \overline{f}}
  {\Pi; \Delta; \Gamma \vdash e_0.\mathtt{f}_i : T_i}
}
$ & & \rulename{GT-FIELD} \\
& \\

$\mathtt{\ddfrac{\begin{array}{c}
  \mathtt{\environmentvdash e_0 : T_0 } \quad \quad 
  \mathtt{\exptype{}{\ol{Y} \triangleleft \ol{P}} \ol{U} \to U \in
                   \mathit{mtype}(m, \mathit{bound}_\Delta (T_0), \Pi)} \\
  \mathtt{\Delta \vdash \ol{V} \ \texttt{ok} } \quad \quad
  \mathtt{\Delta \vdash \ol{V} \subeq  [\ol{V}/\ol{Y}]\ol{P} } \qquad
  \mathtt{\environmentvdash \ol{e} : \ol{S} } \quad \quad
  \mathtt{\Delta \vdash \ol{S} \subeq  [\ol{V}/\ol{Y}]\ol{U}}
\end{array}}
{\environmentvdash \mathtt{e_0.\mv{m}(\overline{e}) : [\ol{V}/\ol{Y}]U }}
}$ & & \rulename{GT-INVK} \\
& \\
$\mathtt{ \ddfrac{\Delta \vdash {\mv N} \ \texttt{ok} \quad 
  \mv N = \exptype{C}{\ol{U}} \quad
  \textit{fields}(\mv N) = \ol{T}\ \ol{f} \quad 
  \environmentvdash \ol{e} : \ol{S} \quad \Delta \vdash \ol{S} \subeq  \ol{T}
}{
  \environmentvdash \texttt{new C}(\ol{e}): \mv N
}
}$ & & \rulename{GT-NEW} \\


% & \\

% $\mathtt{\ddfrac{\begin{array}{c}
%   \mathtt{\Pi(\exptype{C}{\ol{X}}.m) = \exptype{}{\ol{Y} \triangleleft \ol{P}} \ol{U} \to U } \quad \quad 
%   \mathtt{\environmentvdash e_0 : T_0 } \\
%   \mathtt{\textit{bound}(T_0) = \exptype{C}{\ol{Y}}} \quad \quad
%   \mathtt{\environmentvdash \ol{e} : \ol{S} } \quad \quad
%   \mathtt{\Delta \vdash \ol{S} \subeq  [\ol{Y}/\ol{X}]\ol{U}}
% \end{array}}
% {\environmentvdash \mathtt{e_0.\mv{m}(\overline{e}) : [\ol{Y}/\ol{X}]U }}
% }$ & & \rulename{GT-L-INVK}\\

& \\

$\ddfrac{\mathtt{\environmentvdash e_0 : T_0 \quad \quad \Delta \vdash \textit{bound}_\Delta(T_0) \subeq  N}}
{\mathtt{\environmentvdash (N) e_0 : N}}
$ & & \rulename{GT-UCAST} \\

& \\

$ \ddfrac{\begin{array}{c}
  \mathtt{\environmentvdash e_0 : T_0 \quad \quad \Delta \vdash N\ \texttt{ok} \quad \quad \Delta \vdash N \subeq  \textit{bound}_\Delta(T_0) } \\
  \mathtt{N = \exptype{C}{\ol{T}} \quad \quad \textit{bound}_\Delta(T_0) = \exptype{D}{\ol{U}}  \quad \quad \textit{dcast}(C,D)}
\end{array}
}{\mathtt{\environmentvdash (N) e_0 : N}}$ & & \rulename{GT-DCAST} \\

& \\

$\ddfrac{\begin{array}{c}
  \mathtt{\environmentvdash e_0 : T_0 \quad \quad \Delta \vdash N\ \texttt{ok} \quad \quad N = \exptype{C}{\ol{T}}  } \\
  \mathtt{\textit{bound}_\Delta(T_0) = \exptype{D}{\ol{U}} \quad \quad C \ntrianglelefteq D \quad \quad D \ntrianglelefteq C \quad \quad \textit{stupid warning}}
\end{array}}
{\mathtt{\environmentvdash (N) e_0 : N}}
$ & & \rulename{GT-SCAST} 
\end{tabularx}\\[1em]
% \end{small}
% \end{minipage}
% % }
% % \fbox{
% \begin{minipage}{\textwidth}
% \begin{small}
\textbf{Method typing:}\\[1em]
\begin{tabularx}{\textwidth}{X c X r}
  & $\mathtt{\ddfrac{\begin{array}{c}
    % \mathtt{\texttt{class}\ \exptype{C}{\ol{X} \triangleleft \ol{N}}
    %                    \triangleleft N\ \{ \dots \}}\\
                       \mathtt{
  \mathtt{\forall  \ol{T}, T: \exptype{}{} \ol{T} \to T \in \Pi(\exptype{C}{\ol{X}}.m) }\quad
  \textit{override}(m, N, \exptype{}{\ol{Y} \triangleleft
                       \ol{P}}\ol{T} \to T, \Pi) } \\
  \mathtt{\Pi; \Delta ; \ol{x}:\ol{T},\ this : \exptype{C}{\ol{X}} \vdash e_0 : S \quad \quad
  \Delta \vdash S \subeq  T } \\
  \end{array}} {
   \mathtt{\Pi, \Delta \vdash  \texttt{m}(\ol{\mathtt{x}}) \{\texttt{return}\ \mathtt{e}_0;\}
  \texttt{ OK in }\exptype{C}{\ol{X} \triangleleft \ol{N}} \extends N \texttt{
    with } \exptype{}{\ol{Y} \triangleleft  \ol{P}} }
  }}$ & & \rulename{GT-METHOD}\\
\end{tabularx}\\[1em]
\textbf{Class typing:}\\[1em]
  \begin{tabularx}{\textwidth}{X c X r}
  %GT-CLASS: - This rule is modified by us
  & $\ddfrac{
    \begin{array}{c}
      \mathtt{\Pi' = \Pi \cup \set{\exptype{C}{\ol{X}}.m \mapsto \exptype{}{} \ol{T_m} \to T_m \,
      |\, m \in \ol{M}} } \\
      \mathtt{\Pi'' = \Pi \cup \set{\exptype{C}{\ol{X}}.m \mapsto
      \exptype{}{\ol{Y} \triangleleft  \ol{P}} \ol{T_m} \to T_m \,
      |\, m \in \ol{M}} } \\
      \mathtt{\Delta = \ol{X} \subeq  \ol{N}, \ol{Y} \subeq  \ol{P}}
      \qquad \mathtt{\Delta \vdash\ol{P} \
      \texttt{ok}}
      \qquad \forall \mv{m}: \mathtt{\Delta \vdash \ol{T_m}, T_m \ \texttt{ok}}\\
      \mathtt{\ol{X} \subeq  \ol{N} \vdash \ol{N}, N, \ol{T}\ \texttt{ok}
      \quad\quad \mathit{fields}(\mathtt{N}) = \ol{\mathtt{U}} \ \ol{\mathtt{g}}} \\\
      \mathtt{\Pi', \Delta \vdash \ol{M} \ \texttt{OK IN}\
      \exptype{C}{\ol{X} \triangleleft \ol{N}} \extends N  \texttt{
    with } \exptype{}{\ol{Y} \triangleleft  \ol{P}}}\\
      \mathtt{K = C(\overline{U} \ \overline{g}, \overline{T} \ \overline{f}) \{ \texttt{super}(\overline{g}); \ \texttt{this}.\overline{f}=\overline{f}; \} }\\
    %\quad \quad \overline{M} \ \texttt{OK IN C} \\
  \end{array}
    }
  {\mathtt{ \Pi \vdash \texttt{class}\ \exptype{C}{\ol{X}
        \triangleleft \ol{N}} \triangleleft N\ \{ \overline{T} \
      \overline{f}; \ K \ \overline{M} \} \ \texttt{OK} : \Pi''}}
  $ & & \rulename{GT-CLASS} \\
\end{tabularx}\\[1em]
\textbf{Program typing:}\\[1em]
  \begin{tabularx}{\textwidth}{X c X r}
  & $\ddfrac{
    \mathtt{\emptyset \vdash L_1 : \Pi_1} \quad
    \mathtt{\Pi_1 \vdash L_2 : \Pi_2} \quad \dots \quad
    \mathtt{\Pi_{n-1} \vdash L_n : \Pi_n}
  }{
    \mathtt{\vdash \ol L : \Pi_n}
  }$
  && \rulename{GT-PROGRAM}
\end{tabularx}
\end{small}
\end{minipage}
}
\caption{Typing rules}
  \label{fig:expression-typing}
\end{figure}


\begin{figure}[tp]
\fbox{
\begin{minipage}{\textwidth}
  \begin{small}
  \textbf{Field lookup:} \\[1em]
  \begin{tabularx}{\textwidth}{cXr}
    $\mathit{fields}(\mv{Object}) = \bullet$
    && \rulename{F-OBJECT} \\
    && \\
    $\ddfrac{
      \mathtt{\texttt{class}\ \exptype{C}{\ol{X} \triangleleft \ol{N}}\triangleleft
        \ol{N}\ \{ \overline{S} \ \overline{f}; \ K \ \overline{M} \}
      } \qquad
      \mathtt{\mathit{fields} ([\ol T/\ol X]N) = \ol U\ \ol g}
    }{
      \mathit{fields}(\exptype{C}{\ol T}) = \ol U\ \ol g, [\ol T/\ol
      X]\ol S\ \ol f
    }$
    && \rulename{F-CLASS}
  \end{tabularx}\\[1em]
  \textbf{Method type lookup:} \\[1em]
\begin{tabularx}{\textwidth}{cXr}
  $\ddfrac{
    \begin{array}{c}
      \mathtt{\texttt{class}\ \exptype{C}{\ol{X} \triangleleft
      \ol{N}}\ \triangleleft \mv{N}\ \{ \overline{C} \ \overline{f}; \ K \ \overline{M} \} 
      \qquad m \in \ol{M}} \\
  % \mathtt{
  %     \texttt{m}(\ol{\mathtt{x}}) \{\texttt{return}\ \mathtt{e}_0;\} \in \ol{M}
  %     } \\
      \mathtt{
      \exptype{}{\ol{Y} \triangleleft  \ol{P}} \ol{U} \to U \in \Pi (\exptype{C}{\ol{X}}.m) }
\end{array}
   } {\mathtt{
     \textit{mtype}(m, \exptype{C}{\ol{T}}, \Pi) = [\ol{T}/\ol{X}]\exptype{}{\ol{Y} \triangleleft \ol{P}}\ol{U} \to U
    }}$ & & \rulename{MT-CLASS}\\
 & & \\
  $\ddfrac{\mathtt{\texttt{class}\ \exptype{C}{\ol{X} \triangleleft \ol{N}}\triangleleft
  \mv{N}\ \{ \overline{C} \ \overline{f}; \ K \ \overline{M} \} 
  \qquad m \notin \ol{M}} }{
    \mathtt{\textit{mtype}(m, \exptype{C}{\ol{T}}, \Pi) =
      \textit{mtype}(m, [\ol{T}/\ol{X}]N, \Pi)}
    }$ & & \rulename{MT-SUPER} \\
\end{tabularx}\\[1em]
\textbf{Valid method overriding:}\\[1em]
\begin{tabularx}{\textwidth}{c}
  $\ddfrac{
    \mathtt{\textit{mtype}(m, N, \Pi) = \exptype{}{\ol{Y} \triangleleft \ol{Q}} \ol{U} \to \mv U \ 
    \text{implies}\ \ol{P}, \ol{T} = [\ol{Y}/\ol{Z}](\ol{Q},\ol{U}) \ 
   \text{and}\ \ol{Y} \subeq  \ol{P} \vdash T_0 \subeq  [\ol{Y}/\ol{Z}]U_0} 
  } {
    \mathtt{\textit{override}(m, N, \exptype{}{\ol{Y} \triangleleft \ol{P}} \ol{T} \to T_0, \Pi)}
  }$ 
\end{tabularx}
\end{small}
\end{minipage}
}
\caption{Auxiliary functions}
  \label{fig:auxiliary-functions}
\end{figure}





% \fbox{
% \begin{minipage}{\textwidth}
% \begin{small}
% \textbf{Method Typing (broken):}\\[1em]
% \begin{tabularx}{\textwidth}{X c X r}
%   & $\mathtt{\ddfrac{\begin{array}{c}
%     \mathtt{\texttt{class}\ \exptype{C}{\ol{X} \triangleleft \ol{N}} \triangleleft N \{ \ldots\ \ol{M}\ \ldots\}
%     \quad \quad \Delta = \ol{X} \subeq  \ol{N}, \ol{Y} \subeq  \ol{P} }\\
%   \mathtt{\Delta \vdash \ol{T}, T, \ol{P} \ \texttt{ok} \quad \quad
%   \textit{override}(m, N, \exptype{}{\ol{Y} \triangleleft \ol{P}}\ol{T} \to T) } \\
%   \mathtt{\Pi(\exptype{C}{\ol{X}}.m) = \exptype{}{\ol{Y} \triangleleft \ol{P}}\ \ol{T} \to T }\\
%   \mathtt{\Pi; \Delta ; \ol{x}:\ol{T},\ this : \exptype{C}{\ol{X}} \vdash e_0 : S \quad \quad
%   \Delta \vdash S \subeq  T } \\
%   \end{array}} {
%   %{\exptype{}{\ol{Y} \triangleleft \ol{P}}\ T \ m(\ol{T}\ \ol{x}) \{
%   %\texttt{return} \ e_0; \} \ \texttt{OK IN}\ \exptype{C}{\ol{X} \triangleleft
%   %\ol{N}}}
%    \mathtt{\Pi \vdash \exptype{}{\ol{Y} \triangleleft \ol{P}}\ T\ \texttt{meth}(\ol{T}\ \ol{\mathtt{x}}) \{\texttt{return}\ \mathtt{e}_0;\}
%   \texttt{ OK in }\exptype{C}{\ol{X} \triangleleft \ol{N}} }
%   }}$ & & GT-METHOD\\
% \end{tabularx}\\[1em]
% \textbf{Class Typing (broken):}\\[1em]
%   \begin{tabularx}{\textwidth}{X c X r}
%   %GT-CLASS: - This rule is modified by us
%   & $\ddfrac{
%     \begin{array}{c}
%       \mathtt{\Pi = \set{\exptype{C}{\ol{X}}.m : \ol{T_m} \to T_m \, |\, m \in \ol{M}} } \\
%       \mathtt{\ol{X} \subeq  \ol{N} \vdash \ol{N}, N, \ol{T}\ \texttt{ok}
%       \quad\quad fields(\mathtt{N}) = \ol{\mathtt{U}} \ \ol{\mathtt{g}}} \quad \quad
%       \mathtt{\Pi \vdash \ol{M} \ \texttt{OK IN}\ \exptype{C}{\ol{X} \triangleleft \ol{N}}}\\
%       \mathtt{K = C(\overline{U} \ \overline{g}, \overline{T} \ \overline{f}) \{ \texttt{super}(\overline{g}); \ \texttt{this}.\overline{f}=\overline{f}; \} }\\
%     %\quad \quad \overline{M} \ \texttt{OK IN C} \\
%   \end{array}
%     }
%   {\mathtt{ \texttt{class}\ \exptype{C}{\ol{X} \triangleleft \ol{N}} \triangleleft N\ \{ \overline{T} \ \overline{f}; \ K \ \overline{M} \} \ \texttt{OK}}}
%   $ & & GT-CLASS
% \end{tabularx}
% \end{small}
% \end{minipage}
% }

% \fbox{
% \begin{minipage}{\textwidth}
%   \begin{small}
%   \textbf{Method type lookup (broken):} \\[1em]
% \begin{tabularx}{\textwidth}{cXr}
%   $\ddfrac{
% \begin{array}{c}
%   \mathtt{\texttt{class}\ \exptype{C}{\ol{X} \triangleleft \ol{N}}\ \texttt{extends N}\{ \overline{C} \ \overline{f}; \ K \ \overline{M} \} \ \texttt{OK}} \quad \quad \mathtt{m \in \ol{M}} \\
%   \mathtt{
%   \exptype{}{\ol{Y} \triangleleft \ol{P}}\ T\ \texttt{m}(\ol{T}\ \ol{\mathtt{x}}) \{\texttt{return}\ \mathtt{e}_0;\}
%   }
% \end{array}
%    } {\mathtt{
%      \textit{mtype}(m, \exptype{C}{\ol{T}}) = [\ol{T}/\ol{X}]\exptype{}{\ol{Y} \triangleleft \ol{P}}\ol{U} \to U
%     }}$ & & MT-CLASS\\
%  & & \\
%   $\ddfrac{\mathtt{\texttt{class}\ \exptype{C}{\ol{X} \triangleleft \ol{N}}\triangleleft
%   N\{ \overline{C} \ \overline{f}; \ K \ \overline{M} \} \ \mathtt{OK}
%   \quad \quad m \notin \ol{M}} }{
%     %{\exptype{}{\ol{Y} \triangleleft \ol{P}}\ T \ m(\ol{T}\ \ol{x}) \{ \texttt{return} \ e_0; \} \ \texttt{OK IN}\ \exptype{C}{\ol{X} \triangleleft \ol{N}}}
%     \textit{mtype}(m, \exptype{C}{\ol{T}}) = \textit{mtype}(m, [\ol{T}/\ol{X}]N)
%     }$ & & MT-SUPER \\
% \end{tabularx}\\[1em]
% \textbf{Valid method overriding (broken):}\\[1em]
% \begin{tabularx}{\textwidth}{c}
%   $\ddfrac{
%     \textit{mtype}(m, N) = \exptype{}{\ol{Y} \triangleleft \ol{Q}} \ol{U} \to U \ 
%     \text{implies}\ \ol{P}, \ol{T} = [\ol{Y}/\ol{Z}](\ol{Q},\ol{U}) \ 
%    \text{and}\ \ol{Y} \subeq  \ol{P} \vdash T_0 \subeq  [\ol{Y}/\ol{Z}]U_0 
%   } {
%   %{\exptype{}{\ol{Y} \triangleleft \ol{P}}\ T \ m(\ol{T}\ \ol{x}) \{ \texttt{return} \ e_0; \} \ \texttt{OK IN}\ \exptype{C}{\ol{X} \triangleleft \ol{N}}}
%   \textit{override}(m, N, \exptype{}{\ol{Y} \triangleleft \ol{P}} \ol{T} \to T_0)
%   }$ 
% \end{tabularx}
% % PT: this is not used anywhere!
% % \\[1em]
% %   \textbf{Method body lookup:} \\[1em]
% % \begin{tabularx}{\textwidth}{cXr}
% %   $\ddfrac{\begin{array}{c}
% %   \texttt{class}\ \exptype{C}{\ol{X} \triangleleft \ol{N}}\triangleleft
% %              N\{ \overline{S} \ \overline{f}; \ K \ \overline{M} \} \\
% %   \mathtt{\exptype{}{\ol{Y} \triangleleft \ol{P}}\ T\ \texttt{m}(\ol{T}\ \ol{\mathtt{x}}) \{\texttt{return}\ \mathtt{e}_0;\}}
% %   \in \ol{M}
% %   \end{array}} {
% %   %{\exptype{}{\ol{Y} \triangleleft \ol{P}}\ T \ m(\ol{T}\ \ol{x}) \{ \texttt{return} \ e_0; \} \ \texttt{OK IN}\ \exptype{C}{\ol{X} \triangleleft \ol{N}}}
% %   \textit{mbody}(m, \exptype{C}{\ol{Z}}) = [\ol{Z} / \ol{X}](\exptype{}{\ol{Y}} \ol{T} \to T)
% %   }$ & & MB-CLASS \\
% %   & & \\
% %   $\ddfrac{
% %   \mathtt{\texttt{class}\ \exptype{C}{\ol{X} \triangleleft \ol{N}}\triangleleft
% %              \ol{N}\ \{ \overline{S} \ \overline{f}; \ K \ \overline{M} \} \quad \quad m \notin \ol{M}}
% %   } {
% %   %{\exptype{}{\ol{Y} \triangleleft \ol{P}}\ T \ m(\ol{T}\ \ol{x}) \{ \texttt{return} \ e_0; \} \ \texttt{OK IN}\ \exptype{C}{\ol{X} \triangleleft \ol{N}}}
% %   \mathtt{\textit{mbody}(m, \exptype{C}{\ol{T}}) = \textit{mbody}(m, [\ol{T}/\ol{X}]N)}
% %   }$ & & MB-SUPER \\
% % \end{tabularx}
% \end{small}
% \end{minipage}
% }


\todo[inline]{20220128 PJT: at this point it would be good to show
  that any list of classes for which our system infers a type (without
  overloading) can be completed to a valid FGJ program (i.e., that
  type checks) by adding the inferred type annotations. That should
  not be hard to do. }


%%% Local Variables:
%%% mode: latex
%%% TeX-master: "TIforGFJ"
%%% End:

