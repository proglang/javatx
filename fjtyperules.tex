The input for our type inference algorithm is based on Generic Featherweight Java (GFJ).
GFJ is defined by syntax and typing rules.
We already changed the syntax to allow typeless GFJ programs as input for our algorithm.
Additionally we alter the typing rules slightly, which is presented in this chapter.

%Our type inference algorithm takes typeless GFJ classes as input.
%The generated output is correct GFJ, although we have to alter some rules.
%This chapter defines the typing rules for our version of GFJ,
%which our type inference algorithm is able to process.

Most of them stay the same as in the original GFJ language,
except from the following changes:
\begin{itemize}
\item We remove the \texttt{MT-CLASS}, \texttt{D-CAST}, \texttt{U-CAST}, \texttt{S-CAST} rules
\item The \texttt{GT-METHOD} rule is changed
\item Overriding of methods is removed for our typeless GFJ version. Therefore also the rule \texttt{MT-SUPER} is removed.
\item The \texttt{GT-INVK} rule is changed to support overloading
\end{itemize}

\fbox{
\begin{minipage}{\textwidth}
  \textbf{Subtyping:}\\
\begin{tabular}{l l}
%  $
%  \ddfrac{\texttt{class}\ \exptype{C}{\ol{X} \triangleleft \ol{N}} \triangleleft N \{ \ol{S}\ \ol{f};\ K \ \ol{M} \}
%  \quad \quad m \in \ol{M}}
%  {\mathit{mtype}(m, \exptype{C}{\ol{Z}}) = \mathit{mtype}(m, [\ol{T}/\ol{X}]N)}
%  $
%  & MT-SUPER \\
%& \\

$
\triangle \vdash T <: T
$
&   S-REFL \\

& \\
$\ddfrac{
    \triangle \vdash S <: T \quad \quad \triangle \vdash T <: U
}{
    \triangle \vdash S <: U
}$ & S-TRANS \\

& \\

$
\triangle \vdash X <: \triangle(X)
$ & S-VAR \\
& \\
$\ddfrac{
  \texttt{class}\ \exptype{C}{\ol{X} \triangleleft \ol{N}}
  \triangleleft \mv N \set{ \ldots }
}{
  \triangle \vdash \exptype{C}{\ol{T}} <: [\ol{T}/\ol{X}]\mv N
}$ & S-CLASS 
\end{tabular}
\end{minipage}
}

\fbox{
\begin{minipage}{\textwidth}
  \textbf{Well-formed types:}\\
\begin{tabular}{l l}
$\triangle \vdash \texttt{Object}\ \text{ok}
$ & WF-OBJECT\\

& \\
$\ddfrac{
    X \in \textit{dom}(\triangle)
}{
    \triangle \vdash X \ \text{ok}
}
$ & WF-VAR \\
& \\
$\ddfrac{\begin{array}{c}
\texttt{class}\ \exptype{C}{\ol{X} \triangleleft \ol{N}} \triangleleft N \{ \ldots \} \\
\triangle \vdash \ol{T} \ \text{ok} \quad \quad \triangle \vdash \ol{T} <: [\ol{T}/\ol{X}]\ol{N}
\end{array}
}{
\triangle \vdash \exptype{C}{\ol{T}} \ \text{ok}
}
$ & WF-CLASS
\end{tabular}
\end{minipage}
}


\fbox{
\begin{minipage}{\textwidth}
\textbf{Expression Typing:}\\
\begin{tabular}{l l}
$
\triangle ; \Gamma \vdash x : \Gamma(x)
$ & GT-VAR \\
& \\

$\ddfrac{\Gamma \vdash e_0:T_0 \quad \quad \mathit{fields}(\mathit{bound}_\triangle(T_0)) = \overline{T} \ \overline{f}}
{\Gamma \vdash e_0.\mathtt{f}_i : T_i}
$ & GT-FIELD \\
& \\
$ \ddfrac{\triangle \vdash N \ \texttt{ok} \quad \quad \textit{fields}(N) = \ol{T}\ \ol{f} \quad \quad
  \triangle; \Gamma \vdash \ol{e} : \ol{S} \quad \quad \triangle \vdash \ol{S} <: \ol{T}
}{
  \triangle; \Gamma \vdash \texttt{new N}(\ol{e}): N
}$ & GT-NEW \\

& \\

$\ddfrac{\begin{array}{c}
\mathit{mtype}(m, \mathit{bound}_\triangle (T_0)) = (\exptype{}{\ol{Y} \triangleleft \ol{P}} \ol{U} \to U)\\
\triangle; \Gamma \vdash e_0 : T_0 \quad \quad
\triangle \vdash \ol{V} \ \texttt{OK} \quad \quad
\triangle \vdash \ol{V} <: [\ol{V}/\ol{Y}]\ol{P} \\ %\quad \quad
\triangle; \Gamma \vdash \ol{e} : \ol{S} \quad \quad
\triangle \vdash \ol{S} <: [\ol{V}/\ol{Y}]\ol{U}
\end{array}}
{\triangle; \Gamma \vdash \mathtt{e_0.\exptype{m}{\ol{V}}(\overline{e}) : [\ol{V}/\ol{Y}]U }}
$ & GT-INVK
\end{tabular}
\end{minipage}
}



\fbox{
\begin{minipage}{\textwidth}
\begin{tabular}{l l}

  \textbf{Method Typing:} 
  & \\
  $\ddfrac{\begin{array}{c}
  \texttt{class}\ \exptype{C}{\ol{X} \triangleleft \ol{N}} \triangleleft N \{ \ldots\ \ol{M}\ \ldots\} \\
  \textit{mtype}(m, \exptype{C}{\ol{X}}) = \ol{T_m} \to T_m \textrm{ for } m \in \ol{M}\\
  \triangle \vdash \ol{X} <: \ol{N}  \quad \quad 
  \triangle \vdash \ol{T}, T \ \texttt{ok} \\
  \triangle ; \ol{x}:\ol{T_\mathit{meth}},\ this : \exptype{C}{\ol{X}} \vdash e_0 : S \quad \quad
  \triangle \vdash S <: T_\mathit{meth} \\
  \end{array}} {
  %{\exptype{}{\ol{Y} \triangleleft \ol{P}}\ T \ m(\ol{T}\ \ol{x}) \{
  %\texttt{return} \ e_0; \} \ \texttt{OK IN}\ \exptype{C}{\ol{X} \triangleleft
  %\ol{N}}}
   \exptype{}{\ol{Y}} T_\mathit{meth}\ \texttt{meth}(\ol{T_\mathit{meth}}\ \ol{\mathtt{x}}) \{\texttt{return}\ \mathtt{e}_0;\}
  \texttt{ OK in }\exptype{C}{\ol{X} \triangleleft \ol{N}} 
  }$ & GT-METHOD\\
  
  & \\

\textbf{Class Typing:} & \\
& \\
  %GT-CLASS: - This rule is modified by us
  $\ddfrac{
    \begin{array}{c}
      \ol{X} <: \ol{N} \vdash \ol{N}, N, \ol{T}\ \texttt{ok}
      \quad\quad fields(\mathtt{N}) = \ol{\mathtt{U}} \ \ol{\mathtt{g}}\\
      \exptype{}{\ol{Y}} T_\mathit{m}\ \texttt{m}(\ol{T_\mathit{m}}\ \ol{\mathtt{x}}) \{\texttt{return}\ \mathtt{e}_0;\}
  \texttt{ OK in }\exptype{C}{\ol{X} \triangleleft \ol{N}}  \textrm{ for all } m
      \in \ol{\mathtt{M}}\\
    K = C(\overline{D} \ \overline{g}, \overline{C} \ \overline{f}) \{ \texttt{super}(\overline{g}); \ \texttt{this}.\overline{f}=\overline{f}; \} \\
    %\quad \quad \overline{M} \ \texttt{OK IN C} \\
  \end{array}
    }
  {\texttt{class C extends D}\{ \overline{C} \ \overline{f}; \ K \ \overline{M} \} \ \texttt{OK}}
  $ & GT-CLASS
\end{tabular}
\end{minipage}
}
\commentarymargin{
\begin{tabular}{l l}

  \textbf{Method Typing:} & \\
  & \\
  $\ddfrac{\begin{array}{c}
  \texttt{class}\ \exptype{C}{\ol{X} \triangleleft \ol{N}} \triangleleft N \{ \ldots\ \ol{M}\ \ldots\} \\
  \texttt{meth}(\ol{x}) \{\texttt{return}\ e_0;\} \in \ol{M} \\
  \Gamma_C = \set{ (\textit{mtype}(m, \exptype{C}{\ol{X}}) = \ol{T_m} \to T_m) \ |\ m \in \ol{M}}\\
  (\textit{mtype}(\texttt{meth}, \exptype{C}{\ol{X}}) = \ol{T} \to T) \in \Gamma_C\\
  \triangle \vdash \ol{X} <: \ol{N}  \quad \quad 
  \triangle \vdash \ol{T}, T \ \texttt{ok} \\
  \triangle ; \Gamma_C,\ \ol{x}:\ol{T},\ this : \exptype{C}{\ol{X}} \vdash e_0 : S \quad \quad
  \triangle \vdash S <: T \\
  \end{array}} {
  %{\exptype{}{\ol{Y} \triangleleft \ol{P}}\ T \ m(\ol{T}\ \ol{x}) \{ \texttt{return} \ e_0; \} \ \texttt{OK IN}\ \exptype{C}{\ol{X} \triangleleft \ol{N}}}
  \Gamma_C \vdash \textit{mtype}(m, \exptype{C}{\ol{Z}}) = [\ol{Z} / \ol{X}](\exptype{}{\ol{Y}} \ol{T} \to T)
  }$ & GT-METHOD
\end{tabular}
}



\fbox{
\begin{minipage}{\textwidth}
\begin{tabular}{l l}

  \textbf{Method type lookup:} & \\
  & \\
  $\ddfrac{\begin{array}{c}
  \texttt{class}\ \exptype{C}{\ol{X} \triangleleft \ol{N}}\triangleleft
             \ol{N}\{ \overline{C} \ \overline{f}; \ K \ \overline{M} \} \ \mathtt{OK}\\
  \exptype{}{\ol{Y}} T_\mathit{m}\ \texttt{m}(\ol{T_\mathit{m}}\ \ol{\mathtt{x}}) \{\texttt{return}\ \mathtt{e}_0;\}
  \texttt{ OK in }\exptype{C}{\ol{X} \triangleleft \ol{N}}
  \end{array}} {
  %{\exptype{}{\ol{Y} \triangleleft \ol{P}}\ T \ m(\ol{T}\ \ol{x}) \{ \texttt{return} \ e_0; \} \ \texttt{OK IN}\ \exptype{C}{\ol{X} \triangleleft \ol{N}}}
  \textit{mtype}(m, \exptype{C}{\ol{Z}}) = [\ol{Z} / \ol{X}](\exptype{}{\ol{Y}} \ol{T} \to T)
  }$ & MT-CLASS
\end{tabular}
\end{minipage}
}


\commentarymargin{In the typing system of GFJ \textit{mtype} is a global function which gives the type for every method in every class.
It is defined by the \texttt{MT-CLASS} rule.
In our type system it is also a global function but it is defined by the \texttt{GT-METHOD} rule.
The difference to GFJ is that methods are typechecked one after another.
There has to be one method which initially does not call any other method in the input program,
except the ones in the same class.
%Also the resulting \textit{mtype} from the \texttt{GT-METHOD} rule is bound to the local method type context $\Gamma_C$.

The rule \texttt{GT-CLASS} only is valid, if every method in a class $C$ could satisfy the \texttt{GT-METHOD} rule with the same $\Gamma_C$.}


\medskip
In the typing system of GFJ \textit{mtype} is a global function which gives the type for every method in every class.
In our type system it is also a global function but it is differed between
methods declared in the actual class and methods from other classes.

The main difference between the type system of GFJ and our type system is that
in the \texttt{MT-CLASS} rule the correspondig class has to be proved as \texttt{OK}
by the \texttt{GT-CLASS} rule which means that for all methods of the class a type has to
be assumed and proved as correct by the \texttt{GT-METHOD} rule.
In this rule
for all methods in the actual class a type is assumed by the
declaration of the \texttt{mtype} function.
These assumptions have to be proved as correct. Then the assumed type is
\texttt{OK} in the correspondig class. Additionally, the containing type
variables are generalized (marked by $\exptype{}{\ol{Y}}$). The type variables
$\ol{\mathtt{Y}}$ are not generalized during the prove. The reason is the
undecidablity of polymorphic recursion (cp. Sec. \ref{sec:polym-recurs}).



%MT-CLASS: - This rule is not used by us
%\begin{align*}
%\ddfrac{\begin{array}{c}
%\texttt{class} \ \exptype{C}{\ol{X} \triangleleft \ol{N}} \ \{ \ol{S} \ \ol{f}; \ K \ol{M} \}\\
%\exptype{}{\ol{Y} \triangleleft \ol{P}} U \ m(\ol{U} \ \ol{x})\{  \texttt{return} \ e; \} \in \ol{M}
%\end{array}}
%{\mathit{mtype}(m, \exptype{C}{\ol{Z}}) = [\ol{T}/\ol{X}](\exptype{}{\ol{Y} \triangleleft \ol{P}} \ol{U} \to U)}
%\end{align*}

%GT-METHOD:
%\begin{align*}
%\ddfrac{\begin{array}{c}
%\triangle \vdash \ol{X} <: \ol{N}, \ol{Y} <: \ol{P} \quad \quad 
%\triangle \vdash \ol{T}, T, \ol{P} \ \texttt{ok} \\
%\triangle ; \ol{x}:\ol{T}, this : \exptype{C}{\ol{X}} \vdash e_0 : S \quad \quad
%\triangle \vdash S <: T \\
%\texttt{class}\ \exptype{C}{\ol{X} \triangleleft \ol{N}} \triangleleft N \{ \ldots \} \quad \quad
%\textit{override}(m, N, \exptype{}{\ol{Y} \triangleleft \ol{P}} \ol{T} \to T)
%\end{array}}
%{\exptype{}{\ol{Y} \triangleleft \ol{P}}\ T \ m(\ol{T}\ \ol{x}) \{ \texttt{return} \ e_0; \} \ \texttt{OK IN}\ \exptype{C}{\ol{X} \triangleleft \ol{N}}}
%\end{align*}

%%% Local Variables:
%%% mode: latex
%%% TeX-master: "TIforGFJ"
%%% End:
