\section{Featherweight Generic Java with Global Type Inference}
\label{sec:preliminaries}

This section defines the syntax and type system of a modified version
of the language Featherweight Generic Java
(FGJ)~\cite{DBLP:journals/toplas/IgarashiPW01}, which we call \TFGJ
(with Global Type Inference). The main omissions with respect to FGJ are method types specifications
and polymorphic recursion. We finish the section by formally
connecting FGJ and \TFGJ and by establishing some properties about
polymorphic recursion in FGJ.

\subsection{Syntax}\label{chapter:syntax}
% \commentarymargin{Input assumptions}
% No overloaded methods
% No Or-Constraints
\begin{figure}[tp]
\begin{align*}
  \mv T &::= \mv X \mid \mv N \\
  \mv N &::= \exptype{C}{\ol{T}}\\
  \mv L &::= \mathtt{class} \ \exptype{C}{\ol{X} \triangleleft \ol{N}} \triangleleft \ \mv N\ \{ \ol{T} \ \ol{f}; \,\mv K \, \ol{M} \} \\
  \mv K &::= \mv C(\ol{f})\ \{\mathtt{super}(\ol{f}); \ \mathtt{this}.\ol{f}=\ol{f};\} \\
  \mv M &::= \mathtt{m}(\ol{x})\ \{ \mathtt{ return}\ \mv e; \} \\
  \mv e &::= \mv x \mid \mv e.\mv f \mid
             \mv e.\mathtt{m}(\ol{e}) \mid \mathtt{new}\ \mathtt{C}(\ol{e})
             \mid (\mv N)\ \mv e
\end{align*}
  \caption{Syntax of \TFGJ}
  \label{fig:syntax-tfgj}
\end{figure}
Figure~\ref{fig:syntax-tfgj} defines the syntax of \TFGJ.
Compared to FGJ,
type annotations for method parameters and method return types are omitted.
Object creation via \texttt{new} as well as method calls come do not
require instantiation of their generic parameters.
We keep the class constraints ${\ol{X} \triangleleft \ol{N}}$ as well as the types
for fields $\ol{T} \ \ol{f}$ as we consider them as part of the
specification of a class.

We make the following assumptions for the input program:
\begin{itemize}
\item All types $\mv N$ and $\mv T$ are well formed according to the
  rules of FGJ, which carry over to \TFGJ (see Fig.~\ref{fig:well-formedness-and-subtyping}).
\item The methods of a class call each other mutually recursively.
\item The classes in the input are topologically sorted so that later
  classes only call methods in classes that come earlier in the
  sorting order.
\end{itemize}
Our requirements on the method calls do not impose serious
restrictions as any class, say \mv{C}, can be transformed to meet them as
follows. A preliminary dependency analysis determines an 
approximate call graph. We cluster the methods of \mv{C} according to
the $n$ strongly
connected components of the call graph. Then we split the class into a
class hierarchy 
$\mv{C}_1 \extends \dots \extends \mv{C}_n$ such that each class $\mv{C}_i$ contains
exactly the methods of one strongly connected component and assign a
method cluster to $\mv{C}_i$ if all calls to methods of  $\mv{C}$ now
target methods assigned to $\mv{C}_j$, for some $j\ge i$. The class
$\mv{C}_1$ replaces $\mv{C}$ everywhere in the program: in subtype
bounds, in \texttt{new} expressions, and in casts. More precisely, if
\mv{C} is defined by $\mathtt{class\ \exptype{C}{\ol X \extends \ol N}
\extends N \dots} $, then the class headers for the $\mv{C}_i$ are
defined as follows:
\begin{itemize}
\item  $\mathtt{class\ \exptype{C_i}{\ol X \extends \ol N}
\extends \exptype{C_{i+1}}{\ol X} \dots} $, for $1\le i < n$ and
\item  $\mathtt{class\ \exptype{C_n}{\ol X \extends \ol N}
\extends N \dots} $.
\end{itemize}
It follows from this discussion that the resulting classes have to be
processed backwards starting with $\mv{C}_n, \mv{C}_{n-1}, \dots, \mv{C}_1$.
Figure \ref{fig:example-decluster} showcases this process with a short example.

\begin{figure}[tp]
    \begin{subfigure}[t]{0.49\linewidth}
\begin{lstlisting}[style=tfgj]
class C extends Object {
  m1(a){
    return a;
  }
  m2(b){
    return this.id(a);
  }
}
\end{lstlisting}
      \caption{The methods \texttt{m1} and \texttt{m2} can be separated}
    \end{subfigure}
    ~
    \begin{subfigure}[t]{0.49\linewidth}
\begin{lstlisting}[style=tfgj]
class C1 extends C2 {
  m2(b){
    return this.id(a);
  }
}
class C2 extends Object {
  m1(a){
    return a;
  }
}
\end{lstlisting}
      \caption{After the transformation}
    \end{subfigure}
    \caption{Example for splitting a class into its strongly connected components}
    \label{fig:example-decluster}
  \end{figure}

\subsection{Typing}
\label{chapter:type-rules}
The input for our type inference algorithm is based on Generic Featherweight Java (GFJ).
GFJ is defined by syntax and typing rules.
We already changed the syntax to allow typeless GFJ programs as input for our algorithm.
Additionally we alter the typing rules slightly, which is presented in this chapter.

%Our type inference algorithm takes typeless GFJ classes as input.
%The generated output is correct GFJ, although we have to alter some rules.
%This chapter defines the typing rules for our version of GFJ,
%which our type inference algorithm is able to process.

Most of them stay the same as in the original GFJ language,
except from the following changes:
\begin{itemize}
\item We remove the \texttt{MT-CLASS}, \texttt{D-CAST}, \texttt{U-CAST}, \texttt{S-CAST} rules
\item The \texttt{GT-METHOD} rule is changed
\item Overriding of methods is removed for our typeless GFJ version. Therefore also the rule \texttt{MT-SUPER} is removed.
\item The \texttt{GT-INVK} rule is changed to support overloading
\end{itemize}

\fbox{
\begin{minipage}{\textwidth}
  \textbf{Subtyping:}\\
\begin{tabular}{l l}
%  $
%  \ddfrac{\texttt{class}\ \exptype{C}{\ol{X} \triangleleft \ol{N}} \triangleleft N \{ \ol{S}\ \ol{f};\ K \ \ol{M} \}
%  \quad \quad m \in \ol{M}}
%  {\mathit{mtype}(m, \exptype{C}{\ol{Z}}) = \mathit{mtype}(m, [\ol{T}/\ol{X}]N)}
%  $
%  & MT-SUPER \\
%& \\

$
\triangle \vdash T <: T
$
&   S-REFL \\

& \\
$\ddfrac{
    \triangle \vdash S <: T \quad \quad \triangle \vdash T <: U
}{
    \triangle \vdash S <: U
}$ & S-TRANS \\

& \\

$
\triangle \vdash X <: \triangle(X)
$ & S-VAR \\
& \\
$\ddfrac{
  \texttt{class}\ \exptype{C}{\ol{X} \triangleleft \ol{N}} \triangleleft N \set{ \ldots }
}{
  \triangle \vdash \exptype{C}{\ol{T}} <: [\ol{T}/\ol{X}]N
}$ & S-CLASS 
\end{tabular}
\end{minipage}
}

\fbox{
\begin{minipage}{\textwidth}
  \textbf{Well-formed types:}\\
\begin{tabular}{l l}
$\triangle \vdash \texttt{Object}\ \text{ok}
$ & WF-OBJECT\\

& \\
$\ddfrac{
    X \in \textit{dom}(\triangle)
}{
    \triangle \vdash X \ \text{ok}
}
$ & WF-VAR \\
& \\
$\ddfrac{\begin{array}{c}
\texttt{class}\ \exptype{C}{\ol{X} \triangleleft \ol{N}} \triangleleft N \{ \ldots \} \\
\triangle \vdash \ol{T} \ \text{ok} \quad \quad \triangle \vdash \ol{T} <: [\ol{T}/\ol{X}]\ol{N}
\end{array}
}{
\triangle \vdash \exptype{C}{\ol{T}} \ \text{ok}
}
$ & WF-CLASS
\end{tabular}
\end{minipage}
}


\fbox{
\begin{minipage}{\textwidth}
\textbf{Expression Typing:}\\
\begin{tabular}{l l}
$
\triangle ; \Gamma \vdash x : \Gamma(x)
$ & GT-VAR \\
& \\

$\ddfrac{\Gamma \vdash e_0:T_0 \quad \quad \mathit{fields}(\mathit{bound}_\triangle(T_0)) = \overline{T} \ \overline{f}}
{\Gamma \vdash e_0.\mathtt{f}_i : T_i}
$ & GT-FIELD \\
& \\
$ \ddfrac{\triangle \vdash N \ \texttt{ok} \quad \quad \textit{fields}(N) = \ol{T}\ \ol{f} \quad \quad
  \triangle; \Gamma \vdash \ol{e} : \ol{S} \quad \quad \triangle \vdash \ol{S} <: \ol{T}
}{
  \triangle; \Gamma \vdash \texttt{new N}(\ol{e}): N
}$ & GT-NEW \\

& \\

$\ddfrac{\begin{array}{c}
\mathit{mtype}(m, \mathit{bound}_\triangle (T_0)) = (\exptype{}{\ol{Y} \triangleleft \ol{P}} \ol{U} \to U)\\
\triangle; \Gamma \vdash e_0 : T_0 \quad \quad
\triangle \vdash \ol{V} \ \texttt{OK} \quad \quad
\triangle \vdash \ol{V} <: [\ol{V}/\ol{Y}]\ol{P} \\ %\quad \quad
\triangle; \Gamma \vdash \ol{e} : \ol{S} \quad \quad
\triangle \vdash \ol{S} <: [\ol{V}/\ol{Y}]\ol{U}
\end{array}}
{\triangle; \Gamma \vdash \mathtt{e_0.\exptype{m}{\ol{V}}(\overline{e}) : [\ol{V}/\ol{Y}]U }}
$ & GT-INVK
\end{tabular}
\end{minipage}
}



\fbox{
\begin{minipage}{\textwidth}
\begin{tabular}{l l}

  \textbf{Method Typing:} 
  & \\
  $\ddfrac{\begin{array}{c}
  \texttt{class}\ \exptype{C}{\ol{X} \triangleleft \ol{N}} \triangleleft N \{ \ldots\ \ol{M}\ \ldots\} \\
  \textit{mtype}(m, \exptype{C}{\ol{X}}) = \ol{T_m} \to T_m \textrm{ for } m \in \ol{M}\\
  \triangle \vdash \ol{X} <: \ol{N}  \quad \quad 
  \triangle \vdash \ol{T}, T \ \texttt{ok} \\
  \triangle ; \ol{x}:\ol{T_\mathit{meth}},\ this : \exptype{C}{\ol{X}} \vdash e_0 : S \quad \quad
  \triangle \vdash S <: T_\mathit{meth} \\
  \end{array}} {
  %{\exptype{}{\ol{Y} \triangleleft \ol{P}}\ T \ m(\ol{T}\ \ol{x}) \{
  %\texttt{return} \ e_0; \} \ \texttt{OK IN}\ \exptype{C}{\ol{X} \triangleleft
  %\ol{N}}}
   \exptype{}{\ol{Y}} T_\mathit{meth}\ \texttt{meth}(\ol{T_\mathit{meth}}\ \ol{\mathtt{x}}) \{\texttt{return}\ \mathtt{e}_0;\}
  \texttt{ OK in }\exptype{C}{\ol{X} \triangleleft \ol{N}} 
  }$ & GT-METHOD\\
  
  & \\

\textbf{Class Typing:} & \\
& \\
  %GT-CLASS: - This rule is modified by us
  $\ddfrac{
    \begin{array}{c}
      \ol{X} <: \ol{N} \vdash \ol{N}, N, \ol{T}\ \texttt{ok}
      \quad\quad fields(\mathtt{N}) = \ol{\mathtt{U}} \ \ol{\mathtt{g}}\\
      \exptype{}{\ol{Y}} T_\mathit{m}\ \texttt{m}(\ol{T_\mathit{m}}\ \ol{\mathtt{x}}) \{\texttt{return}\ \mathtt{e}_0;\}
  \texttt{ OK in }\exptype{C}{\ol{X} \triangleleft \ol{N}}  \textrm{ for all } m
      \in \ol{\mathtt{M}}\\
    K = C(\overline{D} \ \overline{g}, \overline{C} \ \overline{f}) \{ \texttt{super}(\overline{g}); \ \texttt{this}.\overline{f}=\overline{f}; \} \\
    %\quad \quad \overline{M} \ \texttt{OK IN C} \\
  \end{array}
    }
  {\texttt{class C extends D}\{ \overline{C} \ \overline{f}; \ K \ \overline{M} \} \ \texttt{OK}}
  $ & GT-CLASS
\end{tabular}
\end{minipage}
}
\commentarymargin{
\begin{tabular}{l l}

  \textbf{Method Typing:} & \\
  & \\
  $\ddfrac{\begin{array}{c}
  \texttt{class}\ \exptype{C}{\ol{X} \triangleleft \ol{N}} \triangleleft N \{ \ldots\ \ol{M}\ \ldots\} \\
  \texttt{meth}(\ol{x}) \{\texttt{return}\ e_0;\} \in \ol{M} \\
  \Gamma_C = \set{ (\textit{mtype}(m, \exptype{C}{\ol{X}}) = \ol{T_m} \to T_m) \ |\ m \in \ol{M}}\\
  (\textit{mtype}(\texttt{meth}, \exptype{C}{\ol{X}}) = \ol{T} \to T) \in \Gamma_C\\
  \triangle \vdash \ol{X} <: \ol{N}  \quad \quad 
  \triangle \vdash \ol{T}, T \ \texttt{ok} \\
  \triangle ; \Gamma_C,\ \ol{x}:\ol{T},\ this : \exptype{C}{\ol{X}} \vdash e_0 : S \quad \quad
  \triangle \vdash S <: T \\
  \end{array}} {
  %{\exptype{}{\ol{Y} \triangleleft \ol{P}}\ T \ m(\ol{T}\ \ol{x}) \{ \texttt{return} \ e_0; \} \ \texttt{OK IN}\ \exptype{C}{\ol{X} \triangleleft \ol{N}}}
  \Gamma_C \vdash \textit{mtype}(m, \exptype{C}{\ol{Z}}) = [\ol{Z} / \ol{X}](\exptype{}{\ol{Y}} \ol{T} \to T)
  }$ & GT-METHOD
\end{tabular}
}



\fbox{
\begin{minipage}{\textwidth}
\begin{tabular}{l l}

  \textbf{Method type lookup:} & \\
  & \\
  $\ddfrac{\begin{array}{c}
  \texttt{class}\ \exptype{C}{\ol{X} \triangleleft \ol{N}}\triangleleft
             \ol{N}\{ \overline{C} \ \overline{f}; \ K \ \overline{M} \} \ \mathtt{OK}\\
  \exptype{}{\ol{Y}} T_\mathit{m}\ \texttt{m}(\ol{T_\mathit{m}}\ \ol{\mathtt{x}}) \{\texttt{return}\ \mathtt{e}_0;\}
  \texttt{ OK in }\exptype{C}{\ol{X} \triangleleft \ol{N}}
  \end{array}} {
  %{\exptype{}{\ol{Y} \triangleleft \ol{P}}\ T \ m(\ol{T}\ \ol{x}) \{ \texttt{return} \ e_0; \} \ \texttt{OK IN}\ \exptype{C}{\ol{X} \triangleleft \ol{N}}}
  \textit{mtype}(m, \exptype{C}{\ol{Z}}) = [\ol{Z} / \ol{X}](\exptype{}{\ol{Y}} \ol{T} \to T)
  }$ & MT-CLASS
\end{tabular}
\end{minipage}
}


\commentarymargin{In the typing system of GFJ \textit{mtype} is a global function which gives the type for every method in every class.
It is defined by the \texttt{MT-CLASS} rule.
In our type system it is also a global function but it is defined by the \texttt{GT-METHOD} rule.
The difference to GFJ is that methods are typechecked one after another.
There has to be one method which initially does not call any other method in the input program,
except the ones in the same class.
%Also the resulting \textit{mtype} from the \texttt{GT-METHOD} rule is bound to the local method type context $\Gamma_C$.

The rule \texttt{GT-CLASS} only is valid, if every method in a class $C$ could satisfy the \texttt{GT-METHOD} rule with the same $\Gamma_C$.}


\medskip
In the typing system of GFJ \textit{mtype} is a global function which gives the type for every method in every class.
In our type system it is also a global function but it is differed between
methods declared in the actual class and methods from other classes.

The main difference between the type system of GFJ and our type system is that
in the \texttt{MT-CLASS} rule the correspondig class has to be proved as \texttt{OK}
by the \texttt{GT-CLASS} rule which means that for all methods of the class a type has to
be assumed and proved as correct be the \texttt{GT-METHOD} rule.
In this rule
for all methods in the actual class a type is assumed by the
declaration of the \texttt{mtype} function.
These assumptions have to be proved as correct. Then the assumed type is
\texttt{OK} in the correspondig class. 



%MT-CLASS: - This rule is not used by us
%\begin{align*}
%\ddfrac{\begin{array}{c}
%\texttt{class} \ \exptype{C}{\ol{X} \triangleleft \ol{N}} \ \{ \ol{S} \ \ol{f}; \ K \ol{M} \}\\
%\exptype{}{\ol{Y} \triangleleft \ol{P}} U \ m(\ol{U} \ \ol{x})\{  \texttt{return} \ e; \} \in \ol{M}
%\end{array}}
%{\mathit{mtype}(m, \exptype{C}{\ol{Z}}) = [\ol{T}/\ol{X}](\exptype{}{\ol{Y} \triangleleft \ol{P}} \ol{U} \to U)}
%\end{align*}

%GT-METHOD:
%\begin{align*}
%\ddfrac{\begin{array}{c}
%\triangle \vdash \ol{X} <: \ol{N}, \ol{Y} <: \ol{P} \quad \quad 
%\triangle \vdash \ol{T}, T, \ol{P} \ \texttt{ok} \\
%\triangle ; \ol{x}:\ol{T}, this : \exptype{C}{\ol{X}} \vdash e_0 : S \quad \quad
%\triangle \vdash S <: T \\
%\texttt{class}\ \exptype{C}{\ol{X} \triangleleft \ol{N}} \triangleleft N \{ \ldots \} \quad \quad
%\textit{override}(m, N, \exptype{}{\ol{Y} \triangleleft \ol{P}} \ol{T} \to T)
%\end{array}}
%{\exptype{}{\ol{Y} \triangleleft \ol{P}}\ T \ m(\ol{T}\ \ol{x}) \{ \texttt{return} \ e_0; \} \ \texttt{OK IN}\ \exptype{C}{\ol{X} \triangleleft \ol{N}}}
%\end{align*}


\subsection{Soundness of Typing}
\label{sec:soundness-typing}
\begin{figure}[tp]
    \begin{align*}
      \Erase{\mv x} &= \mv x \\
      \Erase{\mv e.\mv f} &= \Erase{\mv e}.\mv f \\
      \Erase{\exptype{e}{\ol T}.\mathtt{m}(\ol{e})} &= \Erase{\mv e}. \mathtt{m} (\Erase{\ol e}) \\
      \Erase{\mathtt{new}\ \exptype{C}{\ol T}(\ol{e})} & = \mathtt{new}\ \mv{C}(\Erase{\ol{e}}) \\
      \Erase{(\mv N)\ \mv e} & = (\mv N)\ \Erase{\mv e} \\
      \Erase{\exptype{}{\ol{X} \triangleleft \ol{N}}\ \mv{T}\ \mathtt{m}(\ol T\ \ol{x})\ \{ \mathtt{
      return}\ \mv e; \}} & = \mathtt{m}(\ol{x})\ \{ \mathtt{ return}\ \Erase{\mv e}; \} \\
      \Erase{\mv C(\ol{U}\ \ol{g}, \ol{T}\ \ol{f})\ \{\mathtt{super}(\ol{g}); \ \mathtt{this}.\ol{f}=\ol{f};\}} & = \mv C(\ol{g}, \ol{f})\ \{\mathtt{super}(\ol{g}); \ \mathtt{this}.\ol{f}=\ol{f};\} \\
      \Erase{\mathtt{class} \ \exptype{C}{\ol{X} \triangleleft \ol{N}} \triangleleft \ \mv N\ \{ \ol{T} \ \ol{f}; \,\mv K \, \ol{M} \}} & = 
                                                                                                                                          \mathtt{class} \ \exptype{C}{\ol{X} \triangleleft \ol{N}} \triangleleft \ \mv N\ \{ \ol{T} \ \ol{f}; \,\Erase{\mv K} \, \Erase{\ol{M}} \}
    \end{align*}
    \caption{Erasure functions}
    \label{fig:erasure}
  \end{figure}

We show that every typing derived by the \TFGJ rules gives rise to a
completion, that is, a well-typed FGJ program with the same structure.
\begin{definition}[Erasure]\label{def:erasure}
  Let $\mv{e}'$, $\mv{M}'$, $\mv{K}'$, $\mv{L}'$ be expression, method definition, constructor definition, class definition for FGJ. Define erasure functions 
  $\Erase{\mv{e}'}$, $\Erase{\mv{M}'}$, $\Erase{\mv{K}'}$,
  $\Erase{\mv{L}'}$ that map to the corresponding syntactic categories
  of \TFGJ as shown in Figure~\ref{fig:erasure}.
  \end{definition}
\begin{definition}[Completion]\label{def:completion}
  An FGJ expression $\mv{e}'$ is a \emph{completion} of a \TFGJ expression $\mv{e}$ if $\mv{e} = \Erase{\mv{e}'}$. Completions for method definitions, constructor definitions, and class definitions
  are defined analogously.
\end{definition}
\begin{theorem}
  Suppose that $\mathtt{\vdash \ol L : \Pi}$ such that $|\mathtt{\Pi (\exptype{C}{\ol{X} \triangleleft \ol{N}}.m)}| = 1$, for all $\mv{C.m}$ defined in $\ol L$. Then there is a completion $\ol{L}'$ of $\ol L$ such that
  $\ol{L}'\ \mathtt{OK}$ is derivable in FGJ.
\end{theorem}
\begin{proof}
  The proof is by induction on the length of $\ol L$.

  Consider the class typing $\mathtt{ \Pi \vdash \texttt{class}\ \exptype{C}{\ol{X} 
        \triangleleft \ol{N}} \triangleleft N\ \{ \ol{T} \
      \ol{f}; \ K \ \ol{M} \} \ \texttt{OK} : \Pi''}$ for an element of $\ol
    L$.

    We assume that all classes before $\mv{L}$ are completed according to the incoming $\mathtt{\Pi}$:
    If $\mathtt{\Pi (\exptype{D}{\ol{X} \triangleleft \ol{N}}.n)} = \mathtt{\exptype{}{\ol{Y} \triangleleft  \ol{P}}
      \ol{T} \to T}$, then $\mathtt{\exptype{}{\ol{Y} \triangleleft  \ol{P}}\ T\ \mv{n}(\ol{T}\
      \ol{x}) \dots}$ is in the completion of $\mv{D}$.

    Clearly, we can construct a completion for the class, if we can do so for each method. So we
    have to construct $\ol{M}'$ such that $\mathtt{\ol{M}'\ \mathtt{OK\ IN}\ \exptype{C}{\ol{X} 
        \triangleleft \ol{N}}}$. 

    Inversion of \rulename{GT-CLASS} yields
    \begin{gather}
      \label{eq:3}
      \mathtt{\Pi' = \Pi \cup \set{\exptype{C}{\ol{X} \triangleleft \ol{N}}.m \mapsto \exptype{}{} \ol{T_m} \to T_m \mid m \in \ol{M}} } \\
      \mathtt{\Pi'' = \Pi \cup \set{\exptype{C}{\ol{X} \triangleleft \ol{N}}.m \mapsto
          \exptype{}{\ol{Y} \triangleleft  \ol{P}} \ol{T_m} \to T_m \mid m \in \ol{M}} } \\
      \label{eq:5}
      \mathtt{\Pi', \Delta \vdash \ol{M} \ \texttt{OK IN}\
        \exptype{C}{\ol{X} \triangleleft \ol{N}} \extends N  \texttt{
          with } \exptype{}{\ol{Y} \triangleleft  \ol{P}}} \\
      \label{eq:7}
      \mathtt{\Delta = \ol{X} \subeq  \ol{N}, \ol{Y} \subeq  \ol{P}}
    \end{gather}
    Given some $\mv{M} = \texttt{m}(\ol{\mathtt{x}}) \{\texttt{return}\ \mathtt{e}_0;\} \in \ol{M}$,
    we show that
    \begin{gather}
      \label{eq:4}
      \exptype{}{\ol{Y} \triangleleft  \ol{P}}\ \mathtt{T_m}\ \texttt{m}(\ol{T_m}\ \ol{{x}})
      \{\texttt{return}\ \mathtt{e}'_0;\} \texttt{ OK IN }\exptype{C}{\ol{X} \triangleleft \ol{N}}
    \end{gather}
    is derivable for such completion $\mathtt{e_0'}$ of $\mathtt{e_0}$.

    By inversion of \eqref{eq:5} for $\mv{M}$, we obtain
    \begin{gather}
      \label{eq:6}
      \mathtt{\textit{override}(m, N, \exptype{}{\ol{Y} \triangleleft \ol{P}}\ol{T_m} \to T_m,
        \Pi) } \\
      \label{eq:9}
      \mathtt{\Pi; \Delta ; \ol{x}:\ol{T_m},\ this : \exptype{C}{\ol{X}} \vdash e_0 : S} \\
      \label{eq:8}
      \mathtt{\Delta \vdash S \subeq  T_m }
    \end{gather}
    As $\mathtt{\Delta}$ in~\eqref{eq:7} is defined as in \rulename{GT-METHOD'}, the well-formedness
    judgments are all given, the subtyping judgment \eqref{eq:8} is given as well as the override
    \eqref{eq:9}, the rule \rulename{GT-METHOD'} applies if we can establish
    \begin{gather}
      \label{eq:10}
      \mathtt{\Delta ; \ol{x}:\ol{T_m},\ this : \exptype{C}{\ol{X}} \vdash e_0' : S}
    \end{gather}
    for a completion of $\mathtt{e_0}$.

    To see that, we need to consider the rules \rulename{GT-NEW},
    \rulename{GT-CAST}, and \rulename{GT-INVK}. The 
    \rulename{GT-NEW} rule poses the existence of some $\ol{U}$ such that $\mv{N} =
    \exptype{C}{\ol{U}}$ for checking $\mv{e} = \mv{new}\ \mv{C} (\ol{e}) : \mv{N}$. In the completion, we
    define $\mv{e'} = \mv{new}\ \mv{N} (\ol{e}') : \mv{N}$ to apply rule \rulename{GT-NEW'} to the completions
    of the arguments.

    The rule \rulename{GT-CAST} splits into three rules
    \rulename{GT-UCAST'}, \rulename{GT-DCAST'}, and
    \rulename{GT-SCAST'}. These rules are disjoint, so that at most one of
    them applies to each occurrence of a cast. Here we assume a more
    liberal version of \rulename{GT-DCAST'} that admits downcasts that
    are not stable under type erasure semantics.

    For the rule \rulename{GT-INVK}, we first consider calls to methods not defined in the current
    class. By our assumption on previously checked classes $\mv{D}$ and their methods $\mv{n}$,
    $\mathit{mtype} (\mv{n}, \mv{D}, \mv\Pi) \allowbreak = \{\mathit{mtype}' (\mv{n}, \mv{D}')\}$ where the right
    side lookup happens in the completion following the definitions for FGJ (i.e., $\mv{D'}$ is the
    completion for $\mv D$). The \rulename{GT-INVK} rule poses the existence of some $\ol{V}$ that
    satisfies the same conditions as in \rulename{GT-INVK'}. Hence, we define the completion of
    $\mathtt{e_0.\mv{n}(\ol{e}) : [\ol{V}/\ol{Y}]U }$ as
    $\mathtt{e_0'.\exptype{n}{\ol{V}}(\ol{e}') : [\ol{V}/\ol{Y}]U }$.

    Next we consider calls to methods $\mv{n}$ defined in the current class, say, $\mv{C}$. For those methods,
    $\mathit{mtype} (\mv{n}, \mv{C}, \mv\Pi) = \exptype{}{} \ol{U} \to \mv U$, a non-generic
    type. By the definition of $\mathtt{\Pi''}$, we know that the type of this method will be
    published in the completion as $\exptype{}{\ol{Y} \triangleleft  \ol{P}} \ol{U} \to \mv
    U$. Hence, $\mathit{mtype}' (\mv{n}, \mv{C}') = \exptype{}{\ol{Y} \triangleleft  \ol{P}} \ol{U}
    \to \mv U$. As methods in $\mv{C}$ are mutually recursive, the rule must pose that $\ol{V} = \ol{Y}$ (cf.\
    Proposition~\ref{prop:polymorphi-recursion}). This setting fulfills all assumptions:
    \begin{gather}
      \label{eq:11}
      \mathtt{\Delta \vdash \ol{Y} \ \texttt{ok} } \\
      \mathtt{\Delta \vdash \ol{Y} \subeq  [\ol{Y}/\ol{Y}]\ol{P} }       
    \end{gather}
    We set the completion of
    $\mathtt{e_0.\mv{n}(\ol{e}) : [\ol{Y}/\ol{Y}]U }$ to
    $\mathtt{e_0'.\exptype{n}{\ol{Y}}(\ol{e}') : [\ol{Y}/\ol{Y}]U }$, which is derivable in FGJ.

    The remaining expression typing rules are shared between FGJ and \TFGJ, so they do not
    affect completions.
\end{proof}

\subsection{Polymorphic Recursion, Formally}
\label{sec:polym-recurs-form}

Consider an FGJ class $\mv{C}$ with $n$ mutually recursive methods $\mv{m}_i :
\forall\ol{X}_i. \ol{A}_i \to \ol{A}_i$, for $1\le i\le n$. Define the \emph{instantiation
  multigraph $IG(\mv{C})$} as a directed multigraph with vertices $\{1,
\dots, n\}$.
Edges between $i$ and $j$ in this graph are labeled with a
substitution from $\ol{X}_j$ to types in $\mv{m}_i$, which may contain
type variables from $\ol{X}_i$.
In particular, if $\mv{m}_i$ invokes $\mv{m}_j$ where the generic type variables in
the type of $\mv{m}_j$ are instantiated with substitution
$\ol{U}/\ol{X}_j$ (see rule GT-INVK), then 
$
i \stackrel{\ol{U}/\ol{X}_j}{\longrightarrow} j
$
is an edge of $IG (\mv{C})$.

Define the \emph{closure of the instantiation multigraph $IG^*(\mv{C})$} as the multigraph
obtained from $IG(\mv{C})$ by applying the following rule, which
composes the instantiating substitutions, exhaustively:
\begin{gather}\label{eq:1}
  i \stackrel{\ol{U}/\ol{X}_j}{\longrightarrow} j
  \quad\wedge\quad
  j \stackrel{\ol{V}/\ol{X}_k}{\longrightarrow} k
  \qquad\Rightarrow\qquad
  i \stackrel{[\ol{U}/\ol{X}_j]\ol{V}/\ol{X}_k}{\longrightarrow} k
\end{gather}

\begin{definition}\label{def:method-in-poly-rec}
  Method $\mv{m}_i$ is \emph{involved in polymorphic recursion}
  if there is an edge
  \begin{gather}\label{eq:2}
    i \stackrel{\ol{W}/\ol{X}_i}{\longrightarrow} i \quad \in IG^*
    (\mv{C}) \qquad \text{such that} \qquad \ol{W} \ne \ol{X}_i
  \end{gather}
\end{definition}
For the toy example in Figure~\ref{fig:examples-poly-rec}, we obtain
the multigraph $IG^* (\mv{UsePair})$ which indicates that \mv{prc} is
involved in polymorphic recursion:
\begin{gather*}
  \begin{array}{l@{\quad}|@{\quad}l}
    IG(\mv{UsePair})& IG^* (\mv{UsePair}) \\\hline
    \mv{prc} \stackrel{\mv{Y,X}/\mv{X,Y}}{\longrightarrow} \mv{prc} &
    \mv{prc} \stackrel{\mv{Y,X}/\mv{X,Y}}{\longrightarrow} \mv{prc} \qquad
    \mv{prc} \stackrel{\mv{X,Y}/\mv{X,Y}}{\longrightarrow} \mv{prc}
  \end{array}
\end{gather*}
The call to \mv{swap} does not appear in the graph because
\mv{swap} is defined in a different class.

For \mv{UsePair2}, we obtain a multigraph $IG^* (\mv{UsePair2})$ with
infinitely many edges which is also clear indication for polymorphic recursion:
\begin{gather*}
  \begin{array}{l@{\quad}|@{\quad}l}
    IG(\mv{UsePair2})& IG^* (\mv{UsePair2}) \\\hline
    \mv{prc} \stackrel{\mv{Y,Pair<X,Y>}/\mv{XY}}{\longrightarrow} \mv{prc}
    &
    \mv{prc} \stackrel{\mv{Y,Pair<X,Y>}/\mv{XY}}{\longrightarrow}
      \mv{prc} \\
    &
      \mv{prc}
      \stackrel{\mv{Pair<X,Y>,Pair<Y,Pair<X,Y>>}/\mv{XY}}{\longrightarrow}
      \mv{prc}
      \\
                     & \dots
  \end{array}
\end{gather*}


Clearly, $IG (\mv{C})$ is finite and can be constructed effectively by
collecting the instantiating substitutions from all method call
sites.
Repeated application of the propagation rule~\eqref{eq:1} either
results in saturation where no edge of the resulting multigraph satisfies~\eqref{eq:2} or it
detects an instantiating edge as in condition~\eqref{eq:2}. 

The following condition is necessary for the absence of
polymorphic recursion.

\begin{proposition}\label{prop:polymorphi-recursion}
  Suppose an FGJ class \mv{C} has $n$ methods, which are mutually
  recursive.
  If \mv{C} does not exhibit
  polymorphic recursion, then
  \begin{itemize}
  \item all methods quantify over the same number of generic
    variables;
  \item if a method has generic variables $\ol{X}$, then each call to
    a method of \mv{C} instantiates with a permutation of the
    $\ol{X}$;
  \item $IG^* (\mv{C})$ is finite.
  \end{itemize}
\end{proposition}
\begin{proof}
  Suppose for a contradiction that there are two distinct methods $\mv{m}_i$ and
  $\mv{m}_j$ with generic variables $\ol{X}_i$ and $\ol{X}_j$, respectively,
  where $|\ol{X}_i| < |\ol{X}_j|$. By mutual recursion, $\mv{m}_i$ invokes
  $\mv{m}_j$ directly or indirectly and vice versa. Hence, $IG^* (\mv{C})$
  contains edges from $i$ to $j$ and back:
  \begin{gather*}
    i \stackrel{\ol{U}/\ol{X}_j}{\longrightarrow} j
    \qquad
    j \stackrel{\ol{V}/\ol{X}_i}{\longrightarrow} i
  \end{gather*}
  As $IG^*(\mv{C})$ is closed under composition, it must also contain
  the edge
  \begin{gather*}
    j \stackrel{[\ol{V}/\ol{X}_i]\ol{U}/\ol{X}_j}{\longrightarrow} j
    \text{.}
  \end{gather*}
  By assumption $\mv{C}$ does not use polymorphic recursion, so it
  must be that $[\ol{V}/\ol{X}_i]\ol{U}/\ol{X}_j =
  \ol{X}_j/\ol{X}_j$. To fulfill this condition, all components of
  $\ol{U}$ must be variables $\in \ol{X}_i$. As  $|\ol{X}_i| <
  |\ol{X}_j| = |\ol{U}|$, there must be some variable $\mv{X} \in \ol{X}_i$ that
  occurs more than once in $\ol{U}$, say, at positions $j_1$ and $j_2$. 
  But that means the variables at positions $j_1$ and $j_2$ in
  $\ol{X}_j$ are mapped to the same component of $\ol{V}$. This is a
  contradiction because this substitution cannot be the identity
  substitution $\ol{X}_j/\ol{X}_j$.

  Hence, all methods have the same number of generic variables and all
  instantiations must use variables.

  Suppose now that there is a direct call from $\mv{m}_i$ to
  $\mv{m}_j$ where the instantiation $\ol{U}/\ol{X}_j$ is not a permutation. Hence,
  there is a variable that appears more than once in $\ol{U}$, which
  leads to a contradiction using similar reasoning as before.

  Hence, all instantiations must be permutations over a finite set of
  variables, so that $IG^*(\mv{C})$ is finite. 
\end{proof}

Moreover, if a class has only mutually recursive methods without
polymorphic recursion, we can assume that each method uses the same
generic variables, say $\ol{X}$, and each instantiation for
class-internal method calls is the identity $\ol X/\ol X$.

Using the same generic variables is achieved by $\alpha$ conversion.
By Proposition~\ref{prop:polymorphi-recursion}, we already know that
each instantiation is a permutation. Each self-recursive call must use
an identity instantiation already, otherwise it would constitute an
instance of polymorphic recursion. Suppose that method \mv{m} calls
method \mv{n} instantiated with a non-identity permutation, say
$\pi$ so that parameter $\mv{X_i}$ of \mv{n} gets instantiated with
$\mv{X_{\pi(i)}}$ of \mv{m}. In this case, we reorder the generic
parameters of \mv{n}  according to the inverse permutation
$\pi^{-1}$ and propagate this permutation to all call sites of
\mv{n}. For the call in \mv{m}, we obtain the identity permutation
$\pi \cdot \pi^{-1}$, for self-recursive calls inside \mv{n}, the
instantiation remains the identity (for the same reason), for a
call site in another method which instantiates \mv{n} with permutation
$\sigma$, we change that permutation to $\sigma \cdot \pi^{-1}$, which
is again a permutation.
This way, we can eliminate all non-identity instantiations from calls
inside \mv{m}.

We move our attention to \mv{n}. Each self-recursive call and each call to \mv{m} uses the
identity instantiation, the latter by construction. So we only need to
consider calls to $\mv{n'}\notin\{\mv{n}, \mv{m}\}$ with an
instantiation which is not the identity permutation. We can also
assume that $\mv{n'}$ is not called from $\mv{m}$: otherwise, \mv{n'}
would have the generic variables in the same order as \mv{m} and hence
as \mv{n}. But that means we can fix all calls to $\mv{n'}$ by
applying the inverse permutations as for \mv{n} \emph{without disturbing the already
  established identity instantiations}.

Each such step eliminates all non-identity instantiations for at least
one method without disturbing previous identity instantiations. Hence,
the procedure terminates after finitely many steps with a class with
all instantiations being identity permutations.

%%% Local Variables:
%%% mode: latex
%%% TeX-master: "TIforGFJ"
%%% End:
