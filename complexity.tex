
\begin{theoremAndi}
  \label{theo:np-hardness}
  \textbf{(NP-Hard):}
  The type inference algorithm for typeless Featherweight Java is NP-hard.
\end{theoremAndi}

\textit{Proof:} This section will show this by reducing the boolean satisfiability problem (SAT) to the \texttt{FJTYPE} algorithm.

\begin{figure}
\begin{lstlisting}
  class True{}
  class False{}
  class Operations(){
    False nand(True a, True b){ return new True();
    }
    True nand(False a, True b){
      return new False();
    }
    True nand(True a, False b){
      return new False();
    }
    True nand(False a, False b){
      return new False();
    }
  }
  class SATExample{
    True sat(a, b, c){
      return |{\color{red}nand(c, nand(a, b))}|;
    }
  }
\end{lstlisting}

\caption{Representation for a SAT problem in FJ code}
\label{fig:fjSATcode}
\end{figure}

Any given boolean expression $B$ can be transformed to a typeless FJ program.
A type inference algorithm finding a possible typisation of this FJ program also solves the boolean expression $B$.
Figure \ref{fig:fjSATcode} shows an example of this.
The classes \texttt{True}, \texttt{False} and \texttt{Operations} always stay the same.
Here we assume that the boolean expression only consists out of $\neg \land$ (NAND) operators.
Now any boolean expression $B_\text{in} = v_1 \land \neg (v_2 \land v_3) \land \ldots$ can be expressed as a Java method
(see  Example in figure \ref{fig:fjSATcode}).
We force the return type of the \texttt{sat} method to have the type \texttt{True}.
When using the \texttt{FJTYPE} algorithm on the generated FJ code it will
assign each parameter of the \texttt{sat} method with either the type \texttt{True}, \texttt{False}
or \texttt{Object}.
This represents a valid assignment for the expression $B_\text{in}$.
\texttt{Object} means that either true of false can be used for this variable.
If the \texttt{FJTYPE} fails to compute a solution the $B_\text{in}$ also has no possible solution.

Any SAT problem can be transferred in polynomial time to a typeless FJ program.
This reduction of SAT to our type inference algorithm proofs that its complexity is atleast NP-Hard.
\hfill $\square$

\begin{theoremAndi}
  \label{theo:np-completeness}
  \textbf{(NP-Complete):}
  The type inference algorithm for typeless Featherweight Java is NP-Complete.
\end{theoremAndi}

\textit{Proof:} We know the algorithm is NP-hard (see \ref{theo:np-completeness}).
To proof NP-Completeness we have to show that it is possible to verify a solution in polynomial time.
The verification of a type solution is the FJ typecheck.

It is easy to see that the expression typing rules can be checked in polynomial time as long as subtyping between two types is verifiable in polynomial time.

Assume $\exptype{C}{\ol{X}} \leq \exptype{D}{\ol{Y}}$ with the number of generics $\ol{X}$ and $\ol{Y}$ less or equal $n$.
Also the number of classes in the subtyperelation is less or equal to $n$.
With $n$ classes the \texttt{S-TRANS} rule can be applied a maximum of $n$ times.
The \texttt{S-CLASS} rule can be calculated in two steps.

