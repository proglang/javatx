We presented in this paper a global type infrerence algorithm for Featherweight
Generic Java (FGJ).  The extended language we called  TFGJ. FGJ is formal calculus of Java which reduces the method
declarating block to a single expression. In
\cite{DBLP:journals/toplas/IgarashiPW01} syntax, including typing rules, and semantics of FGJ is
declared. We replaced the typing rules which defines type correct TFGJ
programs with method without any type declarations.
%type unification
We reduced the global type inference algorithm to a type unification algorithm
which resolves the generated constraints. The type unification algorithm is not
unitary but finitary. This means that the algorithm terminates but in general
there is more than one solution. The multiplicity of solutions are induced
either by overloading and on the over hand by constraints of the form $a
\lessdot ty$ and $ty \lessdot a$ where $a$ is a type variable und $ty$ is a non
type variable type. In this paper we improved the type unification algorithm
such that constraints of the form $a \lessdot ty$ are not longer resolved but
transferred to bounded type parameters \texttt{a extends ty}. This reduced the
number of solutions enormously, without restricting the generality of the typing
of the FGJ programs. Unfortunately constraints of the form $ty \lessdot a$ have
to resolved as Java allows no lower bounds of type parameters. If we would
transfer our type inference algorithm to Scala then the solution of these
constraints could be disclaimed as Scala allows lower bounds of type
parameters. This would speed-up our type inference algorithm.