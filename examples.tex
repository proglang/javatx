Our unify algorithm processes one or two constraints at a time.
Lets look at some examples of different compositions of constraints
and how our unify algorithm will transform them.

\begin{enumerate}
\item $Eq = \set{a \lessdot \exptype{C}{\ol{X}}, a \lessdot b}$ \\
If this is an end configuration, the constraint set is solved.
We can now put any type for $b$, which is a sub or supertype of $\exptype{C}{\ol{X}}$.
The only exception is, when $b \in \ol{X}$.
In this case $\sigma(b) = \tt{Object}$ is always a correct unifier.
So there is atleast one possible solution.

\item $Eq = \set{a \lessdot \exptype{C}{\ol{X}}, a \lessdot b, b \lessdot c}$
It's the same as the example before.
Replacing $b$ and $c$ with \tt{Object} is also a correct solution. 

\item $Eq = \set{a \lessdot \exptype{C}{\ol{X}}, a \lessdot b, b \lessdot \exptype{D}{\ol{Y}}}$
Here the adopt rule transforms this to:
$\set{a \lessdot \exptype{C}{\ol{X}}, a \lessdot b, b \lessdot \exptype{D}{\ol{Y}}, a \lessdot \exptype{D}{\ol{Y}}}$
and here it depends if $\exptype{C}{\ol{A}} <: \exptype{D}{\ol{B}}$ or $\exptype{D}{\ol{A}} <: \exptype{C}{\ol{B}}$.
Either the $a \lessdot \exptype{C}{\ol{X}}$ or $b \lessdot \exptype{D}{\ol{Y}}$ constraint
gets removed by the adopt rule.

%\item $Eq = \set{a \lessdot \exptype{C}{\ol{X}}, a \lessdot b}$

\end{enumerate}
