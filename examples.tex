Our unify algorithm processes one or two constraints at a time.
Lets look at some examples of different compositions of constraints
and how our unify algorithm will transform them.

\begin{enumerate}
\item $Eq = \set{a \lessdot \exptype{C}{\ol{X}}, a \lessdot b}$ \\
If this is an end configuration, the constraint set is solved.
We can now put any type for $b$, which is a sub or supertype of $\exptype{C}{\ol{X}}$.
The only exception is, when $b \in \ol{X}$.
In this case $\sigma(b) = \tt{Object}$ is always a correct unifier.
So there is atleast one possible solution.

\item $Eq = \set{a \lessdot \exptype{C}{\ol{X}}, a \lessdot b, b \lessdot c}$
It's the same as the example before.
Replacing $b$ and $c$ with \tt{Object} is also a correct solution. 

\item $Eq = \set{a \lessdot \exptype{C}{\ol{X}}, a \lessdot b, b \lessdot \exptype{D}{\ol{Y}}}$
Here the adopt rule transforms this to:
$\set{a \lessdot \exptype{C}{\ol{X}}, a \lessdot b, b \lessdot \exptype{D}{\ol{Y}}, a \lessdot \exptype{D}{\ol{Y}}}$
and here it depends if $\exptype{C}{\ol{A}} <: \exptype{D}{\ol{B}}$ or $\exptype{D}{\ol{A}} <: \exptype{C}{\ol{B}}$.
Either the $a \lessdot \exptype{C}{\ol{X}}$ or $b \lessdot \exptype{D}{\ol{Y}}$ constraint
gets removed by the match rule.

%\item $Eq = \set{a \lessdot \exptype{C}{\ol{X}}, a \lessdot b}$

\end{enumerate}

\textbf{match-rule:}\\
The match-rule takes two constraints of the form $a \lessdot \exptype{C}{\ol{X}}, a \lessdot \exptype{D}{\ol{Y}}$
and checks which one of the two types \texttt{C} and \texttt{D} is a subtype of the other.
We only leave the constraint, which is more restrictive.
If \texttt{C} and \texttt{D} do not have a subtype relation, the match-rule won't apply.
This will lead to the constraint set never reaching solved form.
We cannot remove the $a \lessdot \exptype{D}{\ol{Y}}$ constraint without matching the
$\ol{Y}$ and $\ol{X}$.
We do this by generating the constraint $\exptype{C}{\ol{X}} \lessdot \exptype{D}{\ol{Y}}$.
This constraint will then be processed by the adapt and the reduce rule.

Example:
$Eq = \set{a \lessdot \exptype{C}{b,\tt{String}}, a \lessdot \exptype{D}{b,b}}$
with $\exptype{C}{A,B} <: \exptype{D}{B,B}$.
\begin{displaymath}
\prftree[r]{
    reduce2
        }{
    \prftree[r]{
        adopt
            }{
    \prftree[r]{
        match
            }{
    \set{a \lessdot \exptype{C}{b,\tt{String}}, a \lessdot \exptype{D}{b,b}}
    }{
    a \lessdot \exptype{C}{b,\tt{String}}, \exptype{C}{b,\tt{String}} \lessdot \exptype{D}{b,b}
    }
    }{
         a \lessdot \exptype{C}{b,\tt{String}}, \exptype{D}{\tt{String},\tt{String}} \doteq \exptype{D}{b,b}
    }}
    {
       a \lessdot \exptype{C}{b,\tt{String}}, b \doteq \tt{String}, b \doteq \tt{String}
        }
\end{displaymath}

\textbf{adopt-rule:}\\
\begin{displaymath}
\prftree[r]{
    reduce2
        }{
    \prftree[r]{
        adopt
            }{
    \prftree[r]{
        match
            }{
    \set{a \lessdot \exptype{C}{b,\tt{String}}, a \lessdot \exptype{D}{b,b}}
    }{
    a \lessdot \exptype{C}{b,\tt{String}}, \exptype{C}{b,\tt{String}} \lessdot \exptype{D}{b,b}
    }
    }{
         a \lessdot \exptype{C}{b,\tt{String}}, \exptype{D}{\tt{String},\tt{String}} \doteq \exptype{D}{b,b}
    }}
    {
       a \lessdot \exptype{C}{b,\tt{String}}, b \doteq \tt{String}, b \doteq \tt{String}
        }
\end{displaymath}