
formal models for Java, why based on FGJ

\begin{itemize}
\item Welterweight Java
  \url{https://link.springer.com/chapter/10.1007/978-3-642-13953-6_6}
\item Igarashi, A., Pierce, B.C., Wadler, P.: Featherweight Java: a
  minimal core calculus for Java and GJ. ACM TOPLAS 23(3), 396–450
  (2001)
\item Flatt, M., Krishnamurthi, S., Felleisen, M.: A programmer’s
  reduction semantics for classes and mixins. In: Alves-Foss, J. (ed.)
  Formal Syntax and Semantics of Java. LNCS, vol. 1523,
  p. 241. Springer, Heidelberg (1999)
\item Bierman, G.M., Parkinson, M.J., Pitts, A.M.: MJ: An imperative
  core calculus for Java and Java with effects. Technical report,
  University of Cambridge (2003) 
\item 	Elias Castegren, Tobias Wrigstad:
OOlong: an extensible concurrent object calculus. SAC 2018: 1022-1029
\end{itemize}

Martin: unification

type inference

