
formal models for Java, why based on FGJ

\begin{itemize}
\item Welterweight Java
  \url{https://link.springer.com/chapter/10.1007/978-3-642-13953-6_6}
\item Igarashi, A., Pierce, B.C., Wadler, P.: Featherweight Java: a
  minimal core calculus for Java and GJ. ACM TOPLAS 23(3), 396–450
  (2001)
\item Flatt, M., Krishnamurthi, S., Felleisen, M.: A programmer’s
  reduction semantics for classes and mixins. In: Alves-Foss, J. (ed.)
  Formal Syntax and Semantics of Java. LNCS, vol. 1523,
  p. 241. Springer, Heidelberg (1999)
\item Bierman, G.M., Parkinson, M.J., Pitts, A.M.: MJ: An imperative
  core calculus for Java and Java with effects. Technical report,
  University of Cambridge (2003) 
\item 	Elias Castegren, Tobias Wrigstad:
OOlong: an extensible concurrent object calculus. SAC 2018: 1022-1029
\end{itemize}

Martin: unification
The type unification problem which is addressed in section \ref{sec:unify} is
well-known from polymorphic order-sorted unification which is used in logic
programming languages with order-sorted types. 
\begin{thebibliography}{5}
\bibitem{HiTo92}
Hill, P.M., Topor, R.W.:
\newblock A {S}emantics for {T}yped {L}ogic {P}rograms.
\newblock In Pfenning, F., ed.: Types in Logic Programming.
\newblock MIT Press (1992)  1--62

\bibitem{CB95}
Beierle, C.:
\newblock Type inferencing for polymorphic order-sorted logic programs.
\newblock In: International Conference on Logic Programming. (1995)  765--779

\bibitem{MGS89}
Meseguer, J., Goguen, J.A., Smolka, G.:
\newblock Order-sorted unification.
\newblock Journal of Symbolic Computation \textbf{8} (1989)  383--413

\bibitem{SNGM89}
Smolka, G., Nutt, W., Goguen, J.A., Meseguer, J.:
\newblock Order-sorted equational computation.
\newblock In A{\"\i}t-Kaci, H., Nivat, M., eds.: Resolution of Equations in
  Algebraic Structures, Volume~2.
\newblock Academic Press (1989)  297--367

\bibitem{Wa90}
Waldmann, U.:
\newblock Unitary unification in {O}rder-sorted {S}ignatures.
\newblock Technical Report Forschungsbericht 298, Universit{\"a}t Dortmund
  (1989 (Revised Version 1990))
\end{thebibliography}


type inference
\begin{thebibliography}{99}
\bibitem[Damas and Milner(1982)]{DM82}
L.~Damas and R.~Milner.
\newblock Principal type-schemes for functional programs.
\newblock \emph{Proc. 9th Symposium on Principles of Programming Languages},
  1982.

\bibitem[Fuh and Mishra(1988)]{FM88}
Y.-C. Fuh and P.~Mishra.
\newblock Type inference with subtypes.
\newblock \emph{Proceedings 2nd European Symposium on Programming ({ESOP
  '88})}, pages 94--114, 1988.

\bibitem{Plue06_2}
Pl{\"u}micke, M., B{\"a}uerle, J.:
\newblock Typeless {P}rogramming in {J}ava 5.0.
\newblock In Gitzel, R., Aleksey, M., Schader, M., Krintz, C., eds.: 4th
  {I}nternational {C}onference on {P}rinciples and {P}ractices of {P}rogramming
  in {J}ava. ACM International Conference Proceeding Series, Mannheim
  University Press (August 2006)  175--181

\bibitem[Pierce and Turner(1998)]{PT98}
B.~C. Pierce and D.~N. Turner.
\newblock Local type inference.
\newblock In \emph{Proceedings of the 25th ACM SIGPLAN-SIGACT symposium on
  Principles of programming languages}, POPL '98, pages 252--265, 1998.

\bibitem[Odersky et~al.(2001)Odersky, Zenger, and Zenger]{OZZ01}
M.~Odersky, C.~Zenger, and M.~Zenger.
\newblock Colored local type inference.
\newblock \emph{Proc. 28th ACM Symposium on Principles of Programming
  Languages}, 36\penalty0 (3):\penalty0 41--53, 2001.

\end{thebibliography}
