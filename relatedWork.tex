\section{Related Work}
\label{sec:related-work}


\subsection{Formal models for Java}
\label{sec:formal-models-java}

There is a range of formal models for Java. Flatt et al
\cite{DBLP:conf/java/FlattKF99} define an elaborate model with
interfaces and classes and prove a type soundness result. They do not
address generics. Igarashi et al
\cite{DBLP:journals/toplas/IgarashiPW01} define Featherweight Java
and its generic sibling, Featherweight Generic Java. Their language is
a functional calculus reduced to the bare essentials, they develop the full metatheory, they
support generics, and study the type erasing transformation used by
the Java compiler. MJ \cite{UCAM-CL-TR-563} is a core calculus that
embraces imperative programming as it is targeted towards reasoning
about effects. It does not consider generics. Welterweight Java
\cite{DBLP:conf/tools/OstlundW10} and OOlong
\cite{DBLP:conf/sac/CastegrenW18} are different sketches for a core
language that includes concurrency, which none of the other core
languages considers. 

We chose to base our development on FGJ because it embraces a relevant
subset of Java without including too much complexity (e.g., no imperative
features, no interfaces, no concurrency). It seems that results for
FGJ are easily scalable to full Java. We leave the addition of these
feature to future work, as we see our results on FGJ as a first step
towards a formalized basis for global type inference for Java.

\subsection{Type inference}

Some object-oriented languages like Scala, C\#, and Java perform
\emph{local} type inference \cite{PT98,OZZ01}. Local type 
inference means that missing type annotations are recovered using only
information from adjacent nodes in the syntax tree without long distance
constraints. For instance, the type of a variable initialized with a
non-functional expression or the return type of a method can be
inferred. However, method argument types, in particular for recursive
methods, cannot be inferred by local type inference.

Milner's algorithm $\mathcal{W}$ \cite{DBLP:journals/jcss/Milner78} is
the gold standard for global type inference for languages with 
parametric polymorphism, which is used by ML-style languages. The fundamental idea
of the algorithm is to enforce type equality by many-sorted type
unification \cite{Rob65,MM82}. This approach is effective and results
in so-called principal types because many-sorted unification is
unitary, which means that there is at most one most general result.

Pl\"umicke \cite{Plue07_3} presents a first attempt to adopt Milner's
approach to Java. However, the presence of subtyping means that type
unification is no longer unitary, but still finitary. Thus, there is
no longer a single most general type, but any type is an instance of a
finite set of maximal types. Further work by the same author \cite{plue15_2,plue17_2},
refines this approach by moving to a constraint-based algorithm and by
considering lambda expressions and Scale-like function types.

We rule out polymorphic recursion because its presence makes type
inference (but not type checking: see FGJ) undecidable. Henglein
\cite{DBLP:journals/toplas/Henglein93} as well as Kfoury et al
\cite{DBLP:journals/toplas/KfouryTU93} investigate type inference in
the presence of polymorphic recursion. They show that type inference
is reducible to semi-unification, which is undecidable
\cite{DBLP:journals/iandc/KfouryTU93}. However, the undecidability of
this problem apparently does not matter much in practice
\cite{DBLP:journals/tcs/EmmsL99}. 

\subsection{Unification}

We reduce the type inference problem to constraint solving with
equality and subtype constraints.
The procedure presented in Section~\ref{sec:unify} is inspired by
polymorphic order-sorted unification which is used in logic 
programming languages with polymorphic order-sorted types
\cite{GS89,MH91,HiTo92,CB95}.

Smolka's thesis \cite{GS89} mentions the type unification problem
as an open problem. He gives  an incomplete type inference algorithm
for the logical language \textsf{TEL}. The reason for incompleteness
is the admission of subtype relationships between polymorphic types of
different arities as in  $\texttt{List(a)} \sub
\texttt{myLi(a,b)}$. This liberal stance leads to infinite chains in
the type ordering.
Given  $\texttt{List(a)} \sub \texttt{myLi(a,b)}$, we obtain:
\begin{gather*}
  \texttt{List(a)} \sub \texttt{myLi(a,List(a))} \sub \texttt{myLi(a,myLi(a,List(a)))}  \sub \dots
\end{gather*}
However, this subtyping chain exploits covariant subtyping, which does
not apply to FGJ (but it would apply in the presence of wildcards).

Smolka's algorithm also fails sometimes in the absence of infinite
chains, although there is a unifier. 
For example, given $\texttt{nat} \sub \texttt{int}$ and the set
of subtyping constraints $\set{\mathtt{nat} \lessdot \mathtt{a},
  \mathtt{int} \lessdot \mathtt{a}}$, it returns the substitution
$\set{\mathtt{a} \mapsto \mathtt{nat}}$ generated from the first
constraint encountered. This substitution is not a solution
because $\set{\mathtt{int} \lessdot \mathtt{nat}}$ fails.
However, $\set{\mathtt{a}\mapsto \mathtt{int}}$ is a unifier, which
can be obtained by processing the constraints in a different order: from $\set{\mathtt{int} \lessdot \mathtt{a}, \mathtt{nat} \lessdot
  \mathtt{a}}$ the algorithm calculates the unifier 
$\set{\mathtt{a}\mapsto \mathtt{int}}$.

Hill and Topor  \cite{HiTo92} propose a polymorphically typed logic
programming language with subtyping. They restrict subtyping to type
constructors of the same arity,  which guarantees that all subtyping
chains are finite.
In this approach a \emph{most general type unifier (mgtu)} is
defined as an upper bound of different principal type unifiers. In
general, two type terms need not have an upper bound in the subtype ordering,
which means that there is no mgtu in the sense of Hill and Topor.
For example for  $\texttt{nat} \sub \texttt{int}$, $\texttt{neg} 
\sub \texttt{int}$, and the set of inequations $\set{\mathtt{nat} \lessdot
  \mathtt{a}$, $\mathtt{neg} \lessdot \mathtt{a}}$ the mgtu $\set{\mathtt{a} \mapsto \texttt{int}}$ is
determined. If the subtype ordering is extended by $\mathtt{int} \sub
\mathtt{index}$ and $\mathtt{int} \sub \mathtt{expr}$, then there are three
unifiers $\set{\mathtt{a} \mapsto \texttt{int}}$, $\set{\mathtt{a} \mapsto
  \mathtt{index}}$, and $\set{\mathtt{a} \mapsto 
\mathtt{expr}}$, but none of them is a mgtu \cite{HiTo92}.

The type system of \textsf{PROTOS-L} \cite{CB95} was
derived from \textsf{TEL} by disallowing any explicit subtype relationships
between polymorphic type constructors. 
Beierle \cite{CB95} gives a complete type unification algorithm, which can be extended to the
type system of Hill and Topor.
They also prove that the type unification problem is finitary.

Given the declarations  $\texttt{nat} \sub
\texttt{int}$, $\texttt{neg} \sub \texttt{int}$, $\mathtt{int} \sub
\mathtt{index}$, and $\mathtt{int} \sub \mathtt{expr}$, applying the
type unification algorithm of \textsf{PROTOS-L} to the set of
inequations $\set{\mathtt{nat} \lessdot
  \mathtt{a}$, $\mathtt{neg} \lessdot \mathtt{a}}$ yield three general
unifiers $\set{\mathtt{a} \mapsto \texttt{int}}$, $\set{\mathtt{a} \mapsto
  \mathtt{index}}$, and $\set{\mathtt{a} \mapsto \mathtt{expr}}$. 

Pl\"umicke \cite{plue09_1} realized that the type system of
\textsf{TEL} is related to subtyping in Java.
In contrast to \textsf{TEL}, where infinite chains have a lower bound, 
in Java finite chain have an upper bound. E.g. for $\exptypett{myLi}{b,a} \olsub \exptypett{List}{a}$ holds:

\smallskip
\noindent
$\ldots \olsub
\exptypett{myLi}{\exptypett{?\,extends\,myLi}{\exptypett{\textrm{{\tt ?}}\,extends\,List}{a},a},a}
\ \olsub \exptypett{myLi}{\exptypett{?\,extends\,List}{a},a}$\\
\mbox{ } \hfill $\olsub
\exptypett{List}{a}$\\

Pl\"umicke \cite{plue09_1} solved the open problem of infinite chains
posed by Smolka \cite{GS89}.
He showed that in any infinite chain there is a finite number of elements such that
all other elements of the chain are instances of them. The resulting type
unification algorithm can be used for type inference of Java~5 with
wildcards \cite{Plue07_3}. As FGJ has no wildcards, we based our
algorithm on an earlier work \cite{Plue04_1}.
In contrast to that work, which only infers generic methods with
unbounded types, our algorithm  infers bounded generics.
To this end, we do not expand constraints
of the form $\tv{a} \lessdot \itype{N}$, where $\tv{a}$ is type variable and $\itype{N}$ is is a
non-variable type, but convert them to bounded type parameters of the form
\texttt{X extends N}. This change results in a significant reduction
of the number of solutions of the type
unification algorithm without restricting the generality of typings of
FGJ-programs. Unfortunately, constraints of the form $\itype{N} \lessdot \tv{a}$ have
to be expanded as FGJ (like Java) does not permit lower bounds for
generic parameters. If lower bounds were permitted  (as in Scala), the
number of solutions could be reduced even further.


%%% Local Variables:
%%% mode: latex
%%% TeX-master: "TIforGFJ"
%%% End:
