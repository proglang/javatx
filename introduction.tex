\section{Introduction}
\label{sec:introduction}

Java is one of the most important programming languages. In 2019, Java
was the second most popular language according to a study
based on GitHub
data.\footnote{\url{https://www.businessinsider.de/international/the-10-most-popular-programming-languages-according-to-github-2018-10/}} Estimates
for the number of Java programmers range between 7.6 and 9 million
developers.\footnote{\url{https://www.zdnet.com/article/programming-languages-python-developers-now-outnumber-java-ones/},
\url{http://infomory.com/numbers/number-of-java-developers/}} Java
has been around since 1995 and progressed through 16 versions.

Swarms of programmers have taken their first steps in Java. Many more
have been introduced to object-oriented programming through Java, as
it is one of the first mainsteam languages supporting
object-orientation. Java is a class-based language with static single inheritance among
classes, hence it has nominal types with a specified subtyping
hierarchy. Besides classes there are interfaces to characterize common 
traits independent of the inheritance hierarchy. Since version XXX,
the Java language includes generics that support F-bounded
polymorphism.

The first versions of Java are explicitly typed. That is, 
variables, fields, method parameters, and method returns must be
adorned with their type. Adapting an example from Featherweight Java
\cite{DBLP:journals/toplas/IgarashiPW01} we find
\begin{lstlisting}
  class Pair {
    Object fst;
    Object snd;
    Pair(Object fst, Object snd) {
     this.fst=fst; this.snd=snd;
    }
    Pair setfst(Object newfst) {
      return new Pair(newfst, this.snd);
    }
  }
\end{lstlisting}
Type specifications get even more tedious
with the advent of generics.
\begin{lstlisting}
  class Pair<X, Y> {
    X fst;
    Y snd;
    Pair<X, Y>(X fst, Y snd) {
     this.fst=fst; this.snd=snd;
    }
    Pair<X,Y> setfst(X newfst) {
      return new Pair(newfst, this.snd);
    }
  }  
\end{lstlisting}
It is common to see variable initializations like
these in the wild:\footnote{Taken from
  \url{https://stackoverflow.com/questions/4120216/map-of-maps-how-to-keep-the-inner-maps-as-maps/4120268}.} 
\begin{lstlisting}
  HashMap<String, HashMap<String, Object>> outerMap =
    new HashMap<String, HashMap<String, Object>>();
\end{lstlisting}

Java's \emph{local variable type inference} (since version 10
\cite{java-version-10}) deals
satisfactorily with examples like the initialization of
\lstinline{outerMap}. 
In many initialization scenarios for local variables, Java infers their type
if it is obvious from the context. In the
example, we might write
\begin{lstlisting}
  var outerMap = new HashMap<String, HashMap<String, Object>>();
\end{lstlisting}
because the constructor of the map spells out the type in
full. More specifically,``obvious'' means that the right side of the initialization is
\begin{itemize}
\item a constant of known type (e.g., a string),
\item a constructor call, or
\item a method call (the return type is known from the method
  signature).
\end{itemize}
The \lstinline{var} declaration can also be used for an iteration
variable where the type can be obtained from the elements of the
container or from the initializer.
Alternatively, if the variable is used as the method's return value,
its type can be obtained from the current method's signature.

However, there are still many places where types must be provided. In
particular, \lstinline{var} cannot be used in place of a type for
\begin{itemize}
\item a field of a class,
\item a local variable without initializer or initialized to \lstinline{NULL},
\item a method parameter, or
\item a method return type.
\end{itemize}

In contrast, we study \emph{global type inference} for Java. Our aim
is to write code that omits most type annotations, except for class
headers and field types. Returning to the \lstinline{Pair} example, we
can write
\begin{lstlisting}
  class Pair<X, Y> {
    X fst;
    Y snd;
    Pair(fst, snd) {
     this.fst=fst; this.snd=snd;
    }
    setfst(newfst) {
      return new Pair(newfst, this.snd);
    }
  }  
\end{lstlisting}
and global type inference fills in the rest. Our motivation for global
type inference is threefold.
\begin{itemize}
\item Programmers are relieved from writing down long winded types. 
\item Programmers often write types that leak implementation details. The
  \lstinline{outerMap} example already provides a good example of this
  problem. From a software engineering
  perspective, it would be better to use an abstract type like
  \begin{lstlisting}
    Map<String, Map<String, Object>> outerMap = ...
  \end{lstlisting}
\item Programmers also write types that are more specific than
  necessary instead of using (more verbose) generic types. Here, type
  inference helps programmers to find the most general type.
\todo[inline]{example?}
\item Sometimes, it can be hard to find a correct typing at all.
\todo[inline]{example?}
\end{itemize}

In this paper, we report a theoretical study based on Featherweight
Generic Java \cite{DBLP:journals/toplas/IgarashiPW01} (FGJ), which is backed
by an implementation for full Java.

We proceed to give a bird's eye view of our results, leaving the
details to Section
Our global type inference algorithm is able to infer the types in
method signatures (parameter types and return types) for all
FGJ programs that specify the full class header and all field types,
but do not make use of polymorphic recursion.
Inferred types are generic as much as possible, but do not include bounds.


Java has some features which make global type inference hard:
\begin{itemize}
\item subtyping and overloading combined with a nominal type system;
  we show NP-hardness even without overloading,
  \todo[inline]{correct?}
\item mutable local variables and mutable object state,
\item polymorphic recursion in method calls,
\item every method can potentially call every other method all the
  time,
  \todo[inline]{meaning?}
\item method call graphs can go in circles which is not possible in
  Hindley-Milner type inference,
  \todo[inline]{why not?}
\item methods can have side effects.
\end{itemize}

For Generic Featherweight Java it is easier to decide because there is no state and therefore no wildcard types.
Also we exclude polymorphic recursion.

Example for local type inference in Java:
\begin{lstlisting}{java}
// Java 8 code:
class Local {
  // type inference puts in "ArrayList<String>()"
  List<String> field = new ArrayList<>();
}
\end{lstlisting}
\todo[inline]{Is this a drawback for LVTI? Does GTI put
  \lstinline{List<String>} or \lstinline{Map} in the
  \lstinline{outerMap} example?}

\textbf{Example for global type inference:}
Our global type inference algorithm is able to infer every type annotation.
For the sake of simplicity we will only consider method types in this
paper and assume that field types are specified.
\begin{lstlisting}
class Global{
  method1(a){
    a.add(this);
    return a.get();
  }
}
class List<A> {
  add(A item){...}
  A get() ...
}
\end{lstlisting}
In \texttt{method1} neither the return type nor the type for the parameter \texttt{a} are specified.
The return type of the method depends on the type of \texttt{a}.
If we set in the type \texttt{List<Object>} here, then \texttt{method1} would return \texttt{Object}.
The type \texttt{List<Global>} would also be correct.
Then the return type of the method can also be the type
\texttt{Global}.
\todo[inline]{Not clear what's going on. (Not FGJ code.) If you see
  \lstinline{a.add()}, do you assume that any class with an
  \lstinline{add} method is a candidate for the type of \lstinline{a}?}

\subsection{Contributions}
\label{sec:contributions}

\begin{itemize}
\item Algorithm for global type inference for FGJ (which technical 
  results: soundness, completeness?)
\item Global type inference for FGJ is NP hard.
\item Implementation of global type inference for (full?) Java.
\end{itemize}


%%% Local Variables:
%%% mode: latex
%%% TeX-master: "TIforGFJ"
%%% End:
